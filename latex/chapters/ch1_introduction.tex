% !TeX root = ../main.tex
\chapter{Introduction (TODO)}

\section{Motivation (TODO)}

\section{Research questions}

This thesis provides a quantitative and qualitative comparison between Event Sourcing and traditional CRUD architectures. The primary research question is: \textbf{"How does an Event Sourcing architecture compare to CRUD systems with an independent audit log regarding performance, scalability, flexibility and traceability?"}

To provide a comprehensive answer, the following three sub-research questions (RQ) are addressed:

\begin{enumerate}[label={RQ \arabic*}]
    \item \textbf{Performance and Scalability:} How do \acrshort{crud} and \acrshort{es}-\acrshort{cqrs} implementations perform under increasing load, and what are the resulting implications for system scalability and resource efficiency? \label{rs-performance-scalability}
    \item \textbf{Architectural Complexity and Flexibility:} What are the fundamental structural differences between the two approaches, and how do these impact the long-term flexibility and evolution of the codebase? \label{rs-architecture}
    \item \textbf{Historical Traceability:} To what extent can \acrshort{crud} and \acrshort{es}-\acrshort{cqrs} systems accurately and efficiently reconstruct historical states to satisfy business intent and compliance requirements? \label{rs-traceability}
\end{enumerate}

The individual findings from these research questions are combined in the conclusion to provide a holistic answer to the primary research question.

\section{Goals and non goals}
\label{sec:goals-and-non-goals}

This section defines the scope of the thesis by outlining the specific objectives to be achieved, as well as boundaries that will not be addressed.

\subsection{Project Goals}
\label{sec:goals}

The primary objective of this thesis is to develop two distinct prototype implementations that share an identical interface and follow the same functional contract. One application follows a traditional CRUD-based, layered architecture, while the other utilizes \gls{cqrs} and \gls{es}. Both systems are built using Spring Boot and PostgreSQL, with the \gls{es}/\gls{cqrs} implementation additionally using the Axon Framework and Axon Server. A core goal is to implement both applications according to their respective industry best practices to ensure a fair comparison.

The comparison focuses on providing quantitative measurements regarding performance and scalability under increasing loads. Furthermore, the thesis evaluates architectural flexibility using established static analysis metrics and uses existing literature to form a reasoned opinion on the traceability of both systems. To ensure the results represent typical developer experiences, the performance comparisons rely on out-of-the-box Spring Boot configurations rather than custom-tuned environments. A major focus of the performance evaluation lies on the reproducibility and reliability of these results, which is achieved through reproducible, code-only \gls{vm} configurations, repeatable test scripts, and executing many test iterations to minimize statistical outliers.

\subsection{Non-Goals}
\label{sec:non-goals}

There are several areas that fall outside the scope of this research to keep the focus on the core architectural comparison. While the \gls{es}/\gls{cqrs} application uses domain-driven principles, it is not a goal to achieve perfect \gls{ddd} semantics. Similarly, the thesis does not aim to find the absolute maximum performance of either system through fine-tuning or specialized infrastructure configurations.

From a technical standpoint, the implementations do not include mechanics to support schema evolution or data migration over time. Security is also not a primary focus. While basic authorization is handled through a custom RequestContext, the applications do not follow full-grade production security best practices. Finally, the project aims to create a functional backend for testing purposes rather than a "real" usable application, which means no graphical user interface (GUI) is developed.

\section{Structure of the thesis (TODO)}

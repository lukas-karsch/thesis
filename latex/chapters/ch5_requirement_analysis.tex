% !TeX root = ../main.tex
\chapter{Requirement Analysis}

TODO: source on functional and non-functional requirements.

This thesis aims to provide a fair, quantitative comparison of \acrshort{crud} and \acrshort{cqrs} / \acrshort{es} architectures regarding all three research questions. To achieve this, the architectures should be applied not only to the same domain, but to the exact same requirements. The implementations can then be tested against the same \glspl{contract-test} to ensure their interfaces and behavior is identical.

All requirements which go into achieving this are described here. Functional and non-functional requirements are described. The functional requirements include information about the domain, entities, and business rules. Non-functional requirements to the project are meta: they belong to the whole of the both projects. e.g. identical API, contract test suite...

\section{Functional Requirements}

\subsection{Project requirements}

The applications will implement a course enrollment and grading system which might for example be used in universities. Professors can create courses and lectures which students can enroll to. These lectures can have assignments, which professors enter grades for. Once a lecture is finished, final grades and awarded credits can be calculated. Students are able to view their enrollments, grades and credits.

\subsection{Entities}
\label{sec:entities}

Two types of users exist in the domain: professors and students. Their personal information is not relevant for this thesis, which is why only their first and last name are stored for presentation reasons. The student additionally has a semester.

Professors can create courses. Courses have a name, a description, an amount of credits they yield, a minimum amount of credits required to enroll and can have a set of courses as prerequisites.

Courses are the "blueprints" for lectures. Lectures are the "implementation" of a course for a semester. Each lecture created from a course yields the course's amount of credits and has the requirements specified by the course. Lectures have a lifecycle: they can be in draft state, open for enrollment, in progress, finished or archived. A lecture has a list of time slots and a maximum amount of students that can enroll.

A lecture can have several assessments. Each assessment has a type. The professor can enter grades for a student and an assessment. Grades are integers in the range of 0 to 100. Credits are awarded to a student as soon as they completed all assessments for a lecture with a passing grade (grade higher than 50) and once a lecture's status is set to finished.

\subsection{Business rules}
\label{sec:business-rules}

Relationships and business rules in this system are deliberately chosen complex, involving many relationships between \hyperref[sec:entities]{entities} and intricate validation rules. This approach was adopted in order to be able to make realistic assumptions about the research question by evaluating a project that closely resembles complex, real-world scenarios.

The following list presents a selection of business rules which were implemented.

\begin{itemize}
    \item Existence checks: any requests including references to entities will fail if the references entities do not exist.
    \item Requests leading to conflicts, for example creating a lecture with overlapping time slots, will fail.
    \item When a student tries enrolling to a lecture which is already full, they will be put on a waitlist.
    \item When a student disenrolls from a lecture, the next eligible student (higher semesters are preferred) will be enrolled.
    \item Actions on a lecture can only be performed during the appropriate lifecycle state (enrolling only when the lifecycle is "open for enrollment", grades can only be assigned when the lecture is "finished").
\end{itemize}

\section{Non-functional requirements}

\begin{itemize}
    \item SLOs
    \item Auditing
\end{itemize}

\subsection{Service Level Objectives}
\label{sec:slo}

While \glspl{sla} are agreements with users regarding uptime and performance, \glspl{slo} are the technical targets used by engineers to meet those requirements. \parencite{beyer_site_2016} This thesis attempts to define realistic \acrshortpl{slo} to establish a "breaking point" for each architecture.

Following \textcite[135]{nielsen_usability_1993}, a response time of 100ms is the threshold for human perception of "instant" feedback. This serves as the baseline for the following targets:

\begin{itemize}
    \item \textbf{Latency \acrshort{slo}}: All endpoints must maintain a client-side P95 latency of $\le$100ms to ensure the system feels "instant" for 95\% of requests. \label{slo-latency}
    \item \textbf{Freshness \acrshort{slo}}: In the Event Sourcing implementation, the asynchronous nature of projections introduces a lag. All writes must be reflected in the PostgreSQL read-model within $\le$100ms to ensure eventual consistency remains imperceptible. This \acrshort{slo} applies only to the ES-CQRS implementation. \label{slo-freshness}
    \item \textbf{Reliability \acrshort{slo}}: Both implementations must maintain a failure rate of <0.1\% under stress. \label{slo-reliability}
\end{itemize}

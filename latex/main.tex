%LTeX: enabled=false
\documentclass[11pt,a4paper]{report}
\usepackage[a4paper, margin=3cm]{geometry}
\usepackage{array}
\usepackage[utf8]{inputenc}
\usepackage[T1]{fontenc}
\usepackage{graphicx}
\usepackage[automake, acronym]{glossaries}
\usepackage{xcolor}
\usepackage[export]{adjustbox}
\usepackage{tabularx}
\usepackage{longtable}
\usepackage{booktabs}
\usepackage{float} 
\usepackage{enumitem}
\usepackage{subcaption}
\usepackage{makecell}

\hfuzz=5.6pt 

\usepackage{color}

\definecolor{pblue}{rgb}{0.13,0.13,1}
\definecolor{pgreen}{rgb}{0,0.5,0}
\definecolor{pred}{rgb}{0.9,0,0}
\definecolor{pgrey}{rgb}{0.46,0.45,0.48}
\definecolor{porange}{rgb}{1,0.5,0.1} 

\usepackage{listings}
\lstset{language=Java,
  showspaces=false,
  showtabs=false,
  breaklines=true,
  showstringspaces=false,
  breakatwhitespace=true,
  columns=flexible,
  commentstyle=\color{pgrey},
  keywordstyle=\color{porange},
  stringstyle=\color{pgreen},
  basicstyle={\ttfamily\small},
  moredelim=[il][\textcolor{pgrey}]{\$\$},
  moredelim=[is][\textcolor{pgrey}]{\%\%}{\%\%},
  morekeywords={var, record, yield, sealed, non-sealed, permits}
}

\lstdefinelanguage{JavaScript}{
  keywords={typeof, new, true, false, catch, function, return, null, catch, switch, var, if, in, while, do, else, case, break},
  keywordstyle=\color{porange},
  ndkeywords={class, export, default, boolean, throw, implements, import, this},
  ndkeywordstyle=\color{porange}\bfseries,
  sensitive=false,
  comment=[l]{//},
  morecomment=[s]{/*}{*/},
  commentstyle=\color{pgrey},
  stringstyle=\color{pgreen},
  morestring=[b]',
  morestring=[b]",
  morestring=[b]`
}

\usepackage{caption}
\captionsetup{font=small}
\captionsetup[sub]{font=scriptsize}

\DeclareCaptionFormat{listing}{
  \rule{\linewidth}{0.2pt}\par\vskip1pt #1#2#3
}
\captionsetup[lstlisting]{format=listing, justification=centering, singlelinecheck=false}

\usepackage[
  backend=biber,
  style=ieee, % or authoryear, numeric, ieee, apa, ...
]{biblatex}
\DefineBibliographyStrings{english}{
  and = {\&}
}

\usepackage{setspace}
\usepackage{etoolbox}
% 1.5 Zeilenabstand
\onehalfspacing
% außer für Listings 
\BeforeBeginEnvironment{lstlisting}{\begin{singlespace}}
\AfterEndEnvironment{lstlisting}{\end{singlespace}}

\usepackage{hyperref} % must be loaded last! 

% Adapted from Source - https://tex.stackexchange.com/a
% Posted by Don Hosek
% Retrieved 2026-01-08, License - CC BY-SA 4.0
\ExplSyntaxOn
\NewDocumentCommand{\javaname}{ m }
  {
    \group_begin:
    \tl_set:Nn \l_tmpa_tl { #1 }
    % Replace dots with dot + allowbreak
    \tl_replace_all:Nnn \l_tmpa_tl { . } { .\allowbreak }
    % Replace dashes with dash + allowbreak
    \tl_replace_all:Nnn \l_tmpa_tl { - } { -\allowbreak }
    
    \texttt{ \tl_use:N \l_tmpa_tl }
    \group_end:
  }
\ExplSyntaxOff


% adapted from https://tex.stackexchange.com/a/720461
% Posted by David Carlisle
% Retrieved 2026-01-23, License - CC BY-SA 4.0
\ExplSyntaxOn
% Create a boolean to track the first character
\bool_new:N \l__cschulz_first_char_bool
\NewDocumentCommand{\keyw}{m}
 {
  \texttt{ \cschulz_keyw:n { #1 } }
 }
\cs_new_protected:Nn \cschulz_keyw:n
 {
  % Initialize the flag to true for every new call
  \bool_set_true:N \l__cschulz_first_char_bool
  \tl_map_inline:nn { #1 }
   {
    % Check if uppercase
    \str_if_eq:eeT { ##1 } { \str_uppercase:n { ##1 } } 
     { 
      % ONLY add the hyphen if it's NOT the first character
      \bool_if:NF \l__cschulz_first_char_bool { \- }
     }
    % Print the character
    ##1
    % After the first character is printed, set flag to false
    \bool_set_false:N \l__cschulz_first_char_bool
   }
 }

\ExplSyntaxOff

\addbibresource{references.bib}

\makeglossaries

%LTeX: enabled=true
\newglossaryentry{api}
{
    name=API, 
    description={API stands for \emph{Application Programming Interface}. It describes the public interface of a module or service, often exposed over a network}
}

\newglossaryentry{rest}
{
    name=REST,
    description={stands for \emph{Representational State Transfer}. It is an architectural style for distributed hypermedia systems}
}

\newglossaryentry{http}
{
    name=HTTP,
    description={stands for \emph{Hypertext Transfer Protocol}. It is a protocol used in internet communication and was defined in RFC 2616 \cite{rfc2616}}
}

\newglossaryentry{adm}
{
    name={Anemic Domain Model},
    description={The objects describing the domain only hold data, no logic}
}

\newglossaryentry{rdm}
{
    name={Rich Domain Model},
    description={Objects incorporate both data and the behavior or rules that govern that data}
}

\newglossaryentry{atomicity}
{
    name=Atomicity,
    text=atomicity,
    description={means that an action is either fully executed or not at all. Atomic operations make sure the application is not left in an invalid state \cite[10]{bernstein_principles_2009}}
}

\newglossaryentry{cap}
{
    name={CAP Theorem},
    description={The CAP theorem by Eric Brewer states that a distributed data store can only display at most two of the following three guarantees at the same time: consistency, availability and partition tolerance \cite{gilbert_brewers_2002}}
}

\newglossaryentry{contract-test}
{
    name={Contract Test},
    text={contract test},
    plural={contract tests},
    description={A contract test verifies that services implement a shared interface by testing their interactions against an explicitly defined contract}
}

\newglossaryentry{groovy}
{
    name=Groovy,
    description={A dynamic JVM language that extends Java with concise syntax and powerful features such as closures, making it well suited for scripting, DSLs, and test code \cite{groovy-homepage}}
}

\newglossaryentry{gpath}
{
    name={GPath Expression},
    text={GPath expression},
    plural={GPath expressions},
    description={A Groovy-based path language for navigating and querying nested object graphs (such as JSON or XML) using concise, expressive selectors and closures  \cite{groovy-gpath}}
}

\newglossaryentry{restassured}
{
    name={REST Assured},
    description={A library for testing HTTP servers}
}

\newglossaryentry{dockerfile}
{
    name=Dockerfile,
    description={A text document containing a series of instructions used to assemble a Docker image}
}

\newglossaryentry{docker}
{
    name=Docker,
    description={Open platform for developing, shipping and running containerized applications}
}

\newglossaryentry{testcontainer}
{
    name=Testcontainer,
    description={Testcontainers are a way to declare infrastructure dependencies as code using Docker \cite{testcontainers-homepage}}
}

\newglossaryentry{percentile}
{
    name=Percentile,
    text=percentile,
    description={A statistical measure indicating the value below which a given percentage of observations in a group of data falls. For example, the $n$-th percentile is the threshold where $n$ percent of requests are faster than that specific value}
}

\newglossaryentry{median}
{
    name={Median (P50)},
    text=median,
    description={The middle value in a sorted list of response times, representing the "typical" delay experienced by users. It indicates that exactly 50\% of requests are served faster than this threshold; the other 50\% are slower}
}

\newglossaryentry{tail-latency}
{
    name={Tail Latency},
    text={tail latency},
    plural={tail latencies},
    description={The response times observed at high percentiles (such as P95, P99, or P99.9), representing the slowest requests in a distribution. These "outliers" are critical to monitor because they often affect the most data-intensive operations or represent the worst-case user experience}
}

\newglossaryentry{test-configuration}
{
    name={Test Configuration},
    text={test configuration},
    description={A test configuration is the combination of a test script and the target RPS.}
}

\newglossaryentry{layered-architecture}
{
    name={Layered Architecture},
    description={divides applications into \emph{horizontal layers}, with each layer performing a specific role. Typical layers include Presentation, Business, Persistence}
}

\newglossaryentry{tuple}
{
    name=Tuple,
    description={An ordered, immutable sequence of elements used to store a collection of data.}
}

\newglossaryentry{schema-evolution}
{
    name={Schema Evolution},
    text={schema evolution},
    description={The process of managing how data structures, such as events or database tables, change over time}
}

\newacronym{www}{WWW}{World Wide Web}

\newacronym{hateoas}{HATEOAS}{Hypermedia as the engine of application state}

\newacronym{html}{HTML}{HyperText Markup Language}

\newacronym{json}{JSON}{JavaScript Object Notation}

\newacronym{xml}{XML}{Extensible Markup Language}

\newacronym{dao}{DAO}{Data Access Object}

\newacronym{ddd}{DDD}{Domain Driven Design}

\newacronym[plural=URIs]{uri}{URI}{Uniform Resource Identifier}

\newacronym{crud}{CRUD}{Create Read Update Delete}

\newacronym{dto}{DTO}{Data Transfer Object}

\newacronym{acid}{ACID}{Atomicity, Consistency, Isolation, Durability}

\newacronym{cqrs}{CQRS}{Command Query Responsibility Segregation}

\newacronym{cqs}{CQS}{Command And Query Separation}

\newacronym{base}{BASE}{Basically Available, Soft State, Eventual Consistency}

\newacronym{es}{ES}{Event Sourcing}

\newacronym{dsl}{DSL}{Data Specific Language}

\newacronym{jpa}{JPA}{Jakarta Persistence API}

\newacronym{orm}{ORM}{Object-relational Mapper}

\newacronym{jpql}{JPQL}{Java Persistence Query Language}

\newacronym{pojo}{POJO}{Plain Old Java Object}

\newacronym{jmx}{JMX}{Java Management Extensions}

\newacronym{VU}{VU}{Virtual User}

\newacronym{rps}{RPS}{Requests Per Second}

\newacronym[plural=SLOs,firstplural={Service Level Objectives (SLOs)}]{slo}{SLO}{Service Level Objective}

\newacronym[plural=SLAs,firstplural={Service Level Agreements (SLAs)}]{sla}{SLA}{Service Level Agreement}

\newacronym{vm}{VM}{Virtual Machine}

\newacronym{ssh}{SSH}{Secure Shell}
 
\newacronym{scp}{SCP}{Secure Copy}

\newacronym[text={research question}]{rq}{RQ}{Research Question}

\newacronym{io}{I/O}{Input / Output}

\newacronym{mood}{MOOD}{Metrics for Object-Oriented Design}

\newacronym{mhf}{MHF}{Method Hiding Factor}

\newacronym{ahf}{AHF}{Attribute Hiding Factor}

\newacronym{mif}{MIF}{Method Inheritance Factor}

\newacronym{aif}{AIF}{Attribute Inheritance Factor}

\newacronym{pf}{PF}{Polymorphism factor}

\newacronym{cf}{CF}{Coupling Factor}

\newacronym{clf}{CLF}{Clustering Factor}

\newacronym{rf}{RF}{Reuse Factor}

\newacronym{ck}{CK}{Chidamber and Kemerer Metrics}

\newacronym{wmc}{WMC}{Weighted Methods per Class}

\newacronym{dit}{DIT}{Depth of Inheritance Tree}

\newacronym{noc}{NOC}{Number of Children}

\newacronym{cbo}{CBO}{Coupling Between Objects}

\newacronym{rfc}{RFC}{Response for a Class}

\newacronym{lcom}{LCOM}{Lack of Cohesion in Methods}

\newacronym{ips}{IPS}{Iterations per Second}

\newacronym{l}{L}{Load Test}

\newacronym[
    description={The "middle" of a dataset, representing the 50\% of data between the 25th and the 75th percentile.}
]{iqr}{IQR}{Interquartile Range}

\begin{document}

\title{Performance, flexibility and traceability: Evaluating the impact of Event Sourcing and CQRS compared to traditional systems with an audit log.}
\author{Lukas Karsch}

\begin{titlepage}
    \centering

    \includegraphics[width=0.3\textwidth]{images/HdM_Logo.svg.png}
    \vspace{1cm}

    %LTeX: language=de-DE
    {\large Bachelorarbeit im Studiengang Medieninformatik}
    % {\large Bachelor's Thesis in Computer Science and Media}

    \vspace{1.5cm}

    %LTeX: language=en-US
    % Title
    {\LARGE\bfseries Performance, flexibility and traceability: Evaluating the impact of Event Sourcing and CQRS compared to traditional systems with an audit log.}
    %LTeX: language=de-DE

    \vspace{0.5cm}
    \rule{\linewidth}{0.5pt}
    \vspace{0.5cm}

    {\large\bfseries Vorgelegt von Lukas Karsch}

    {\small mit der Matrikelnummer 45259}

    \vspace{0.3cm}

    %LTeX: language=de-DE
    {\bfseries an der Hochschule der Medien Stuttgart am 02.03.2026}

    % \vspace{0.2cm}

    % to obtain the degree of Bachelor of Science
    zur Erlangung des akademischen Grades eines Bachelor of Science

    \vfill

    \begin{flushleft}
        \begin{tabular}{ll}
            \textbf{Erstprüfer:}  & Prof. Dr. Tobias Jordine \\[0.3cm]
            \textbf{Zweitprüfer:} & Felix Messner            \\[0.3cm]
        \end{tabular}
    \end{flushleft}
    %LTeX: language=en-US

\end{titlepage}

\newpage
\pagenumbering{roman}
%LTeX: language=de-DE
\section*{Ehrenwörtliche Erklärung}
Hiermit versichere ich, Lukas Karsch, ehrenwörtlich, dass ich die vorliegende Bachelorarbeit mit dem Titel: "Performance, flexibility and traceability: Evaluating the impact of Event Sourcing and CQRS compared to traditional systems with an audit log" selbstständig und ohne fremde Hilfe verfasst und keine anderen als die angegebenen Hilfsmittel benutzt habe. Die Stellen der Arbeit, die dem Wortlaut oder dem Sinn nach anderen Werken entnommen wurden, sind in jedem Fall unter Angabe der Quelle kenntlich gemacht. Ebenso sind alle Stellen, die mit Hilfe eines KI-basierten Schreibwerkzeugs erstellt oder überarbeitet wurden, kenntlich gemacht. Die im Anhang aufgelisteten KI-Werkzeuge wurden ausschließlich zum Lektorat verwendet. Anschließend wurde sichergestellt, dass der Inhalt mit den intendierten Aussagen noch übereinstimmt. Die Arbeit ist noch nicht veröffentlicht oder in anderer Form als Prüfungsleistung vorgelegt worden. Ich habe die Bedeutung der ehrenwörtlichen Versicherung und die prüfungsrechtlichen Folgen (§ 24 Abs. 2 Bachelor-SPO) einer unrichtigen oder unvollständigen ehrenwörtlichen Versicherung zur Kenntnis genommen.

\vspace{1.8cm} % Space for signature

\noindent\rule{7cm}{0.4pt} \\
\noindent \small{Lukas Karsch, 02.03.2026}

%LTeX: language=en-US

\newpage
% !TeX root = ../main.tex
\begin{abstract}
    This thesis provides an empirical comparison of traditional Layered CRUD Architectures and applications using Command Query Responsibility Segregation (CQRS) paired with Event Sourcing in terms of performance, scalability, and long-term flexibility. The study addresses these topics by evaluating two functional prototypes of a course enrollment and grading system, one built on a standard Layered Architecture with a relational audit log, and the other on an Event-driven approach using CQRS with Axon Framework. A high degree of comparability is achieved by implementing both applications against an identical contract and requirements.

    Performance metrics taken through load testing reveal that the CRUD application generally maintains higher write performance and resource efficiency, while the ES-CQRS application excels in complex, read-heavy scenarios through its use of denormalized, pre-computed projections. However, this read optimization introduces new challenges, such as eventual consistency. Static analysis was applied to evaluate the architectural flexibility of the applications. Its results further highlight that while CRUD systems may be simpler to implement, they exhibit higher transitive coupling that could hinder the codebase's long-term evolution. In contrast, the isolation provided by CQRS suggests the possibility of a more seamless transition towards a distributed architecture.

    The study also examines how these architectures satisfy requirements for historical traceability. How reliably can a system reconstruct its past while maintaining data integrity? Findings indicate that while CRUD audit logs offer faster retrieval of historic state, they are vulnerable to data divergence. By treating its immutable event log as the primary source of truth, Event Sourcing provides superior integrity and accuracy, additionally preserving business intent. Ultimately, the choice between these patterns depends on the specific domain requirements in terms of read-to-write ratios, its tolerance for consistency delays, and its requirements regarding auditing.

    \vspace{0.2cm}
    \noindent\textbf{Keywords:} CQRS, CRUD, Architecture, Load Testing, Scalability, Traceability, Static Analysis

    \vspace{5em}

    %LTeX: language=de-DE
    \noindent
    Diese Arbeit bietet einen empirischen Vergleich zwischen traditionellen CRUD-Schichtenarchitekturen und Anwendungen, die Command Query Responsibility Segregation (CQRS) mit Event Sourcing nutzen, hinsichtlich Leistung, Skalierbarkeit und langfristiger Flexibilität. Die Thesis befasst sich mit diesen Themen, indem sie zwei funktionale Prototypen eines Kursanmelde- und Benotungssystems bewertet. Von diesen Prototypen basiert einer auf einer traditionellen Schichtenarchitektur mit einem relationalen Audit-Log, und der andere auf einer Event-getriebenen Architektur unter Verwendung von CQRS mit Axon Framework. Beide Anwendungen werden gegen denselben Vertrag und dieselben Anforderungen implementiert, um einen hohen Grad an Vergleichbarkeit zu erreichen.

    Um Leistungsdaten zu sammeln, wurden Stresstests durchgeführt. Die Ergebnisse zeigen, dass die CRUD-Anwendung im Allgemeinen eine höhere Leistung bei Schreibvorgängen sowie bessere Ressourceneffizienz besitzt, während die ES-CQRS-Anwendung durch die Verwendung von denormalisierten, vorberechneten "Projektionen" in intensiven und komplexen Lesevorgängen bessere Leistung zeigt. Diese Optimierung der Lesevorgänge bringt jedoch neue Herausforderungen mit sich, wie z.B. eventuelle Konsistenz. Um die Flexibilität der beiden Architekturen zu bewerten, wurde statische Codeanalyse verwendet. Diese unterstreicht zusätzlich, dass CRUD-Systeme zwar einfacher zu implementieren sind, jedoch eine höhere transitive Kopplung aufweisen, welche die langfristige Weiterentwicklung der Codebasis behindern könnte. Im Gegensatz dazu ermöglicht die erhöhte Isolation durch CQRS einen nahtloseren Übergang zu einer verteilten Architektur.

    Die Arbeit untersucht auch, inwiefern diese Architekturen verschiedene Anforderungen an historische Rückverfolgbarkeit erfüllen. Wie zuverlässig kann ein System seine Vergangenheit rekonstruieren und dabei die Datenintegrität gewährleisten? Die Ergebnisse zeigen, dass Audit-Logs in CRUD zwar einen schnelleren Abruf historischer Zustände ermöglichen, jedoch potenziell anfällig für Datenabweichungen sind. Indem Event Sourcing ein unveränderliches "Ereignisprotokoll" (Event Log) als zentrale Datenquelle verwendet, bietet es eine bessere Integrität und Genauigkeit, und bewahrt zusätzlich die Absicht von Aktionen. Letztendlich hängt die Wahl zwischen den beiden Architekturen von den spezifischen Anforderungen des Anwendungsbereichs in Bezug auf das Verhältnis zwischen Lese- und Schreibvorgängen, seiner Toleranz gegenüber Konsistenzverzögerungen und seinen Anforderungen hinsichtlich Auditing ab.

    \vspace{0.2cm}
    \noindent\textbf{Keywords:} CQRS, CRUD, Architektur, Stresstests, Skalierbarkeit, Rückverfolgbarkeit, Statische Codeanalyse

\end{abstract}

%LTeX: language=en-US


\newpage
\tableofcontents

\newpage
\listoffigures
\listoftables
\lstlistoflistings

\newpage
\printglossaries

\newpage
\pagenumbering{arabic}

% !TeX root = ../main.tex
\chapter{Introduction (TODO)}

\section{Motivation (TODO)}

\section{Research questions}

This thesis provides a quantitative and qualitative comparison between Event Sourcing and traditional CRUD architectures. The primary research question is: \textbf{"How does an Event Sourcing architecture compare to CRUD systems with an independent audit log regarding performance, scalability, flexibility and traceability?"}

To provide a comprehensive answer, the following three sub-research questions (RS) are addressed:

\begin{enumerate}[label={RS \arabic*}]
    \item \textbf{Performance and Scalability:} How do \acrshort{crud} and \acrshort{es}-\acrshort{cqrs} implementations perform under increasing load, and what are the resulting implications for system scalability and resource efficiency? \label{rs-performance-scalability}
    \item \textbf{Architectural Complexity and Flexibility:} What are the fundamental structural differences between the two approaches, and how do these impact the long-term flexibility and evolution of the codebase? \label{rs-architecture}
    \item \textbf{Historical Traceability:} To what extent can \acrshort{crud} and \acrshort{es}-\acrshort{cqrs} systems accurately and efficiently reconstruct historical states to satisfy business intent and compliance requirements? \label{rs-traceability}
\end{enumerate}

The individual findings from these research questions are combined in the conclusion to provide a holistic answer to the primary research question.

\section{Goals and non goals (TODO)}
\label{sec:goals}

This section describes the goals and non-goals of the thesis.

Goals:

\begin{itemize}
    \item Create two implementations adhering to an identical interface
    \item Provide quantitative measurements of performance, architectural flexibility and traceability
    \item Implement both applications "the best they can be": according to principles and best practices for both architectures
    \item Use out-of-the box configurations to provide the performance comparisons
\end{itemize}

Non-goals:

\begin{itemize}
    \item Perfectly implement DDD semantics
    \item Tune applications to get the best performance
    \item Implement mechanics to support schema evolution
    \item Implement full-grade application using security best practices, etc -> custom RequestContext with own "authorization" methods
\end{itemize}

\section{Structure of the thesis (TODO)}

% !TeX root = ../main.tex
\chapter{Basics}

To provide a comprehensive framework for the technical implementation discussed in this thesis, this chapter outlines the foundations of modern web architecture, domain-driven design, and the mechanisms facilitating consistency and auditing in event-driven systems.

\section{WWW, Web APIs, REST}

The \acrfull{www} is a connected information network used to exchange data. Resources are can be accessed via \glspl{uri} which are transferred using formats like \acrshort{json} or \acrshort{html} via protocols like \gls{http}. \gls{http} is a stateless protocol based on a request-response structure. It supports standardized request types, such as \texttt{GET} and \texttt{POST}, which convey a semantic meaning \parencite{jacobs_architecture_2004}.

Web APIs are interfaces that enable applications to communicate. They use \gls{http} as a network-based API \parencite[138]{fielding_architectural_2000}. Modern APIs typically follow \gls{rest} principles. REST stands for "Representational State Transfer" and describes an architectural style for distributed hypermedia systems \parencite[76]{fielding_architectural_2000}.

REST APIs adhere to principles derived from a set of constraints imposed by the \gls{http} protocol, for example. One such constraint is "stateless communication": Communication between clients and the server must be \emph{stateless}, meaning the client must provide all the necessary information for the server to fully understand the request.

Furthermore, every resource in REST applications must be addressable via a unique ID, which can then be used to derive a \acrshort{uri} to access the resource. Below are some examples for resources and \glspl{uri} which could be derived from them:

\begin{itemize}
    \item Book; ID=1; URI=\texttt{http://example.com/books/1}
    \item Book; ID=2; URI=\texttt{http://example.com/books/2}
    \item Author; ID=100; URI=\texttt{http://example.com/authors/100}
\end{itemize}

The "\acrfull{hateoas}" principle states that resources should be linked to each other. Clients should be able to control the application by following a series of links provided by the server \parencite{tilkov_brief_2007}.

Every resource must support the same interface, usually \gls{http} methods (GET, POST, PUT, etc.) where operations on the resource correspond to one method of the interface. For example, a POST operation on a customer might map to the \texttt{createCustomer()} operation on a service.

Resources are decoupled from their representations. Clients can request different representations of a resource, depending on their needs \parencite{tilkov_brief_2007}: a web browser might request \acrshort{html}, while another server or application might request \acrshort{xml} or \acrshort{json}.

%s TODO explain CRUD here or somewhere else? 

\section{Layered Architecture Foundations}
\label{sec:layered}

Layered Architecture is the most common architecture pattern in enterprise applications. Applications following a layered architecture are divided into \emph{horizontal layers}, with each layer performing a specific role. A standard implementation consists of the following layers:

\begin{itemize}
    \item Presentation: Handles requests and displays data in a user interface or by turning it into representations (e.g. \acrshort{json}).
    \item Business: Encapsulates business logic.
    \item Persistence: Persists data by interacting with the underlying persistence technologies (e.g. SQL databases).
    \item Database: Manages the physical storage, retrieval, and integrity of the application's data records.
\end{itemize}

A key concept in this design is layers of isolation, where layers are "closed", meaning a request must pass through the layer directly below it to reach the next, ensuring that changes in one layer do not affect others.

In a layered application, data flows downwards during request handling and upwards during the response: a request arrives in the presentation layer, which delegates to the business layer. The business layer fetches data from the persistence layer which holds logic to retrieve data, e.g. by encapsulating SQL statements.

The database responds with raw data, which is turned into a \acrlong{dao} (\acrshort{dao}) by the persistence layer. The business layer uses this data to execute rules and make decisions. The result will be returned to the presentation layer which can then wrap the response and return it to the caller. \parencite{richards_software_2015}

The data in layered applications is often times modeled in an \emph{anemic} way. In an \gls{adm}, business entities are treated as only data. They are objects which contain no business logic, only getters and setters. Business logic is entirely contained in the business (or "service") layer. \textcite{anemic-fowler-2003} describes this as an object-oriented \emph{antipattern}, as this approach effectively separates data from behavior, resulting in a procedural design that undermines the core principle of object-oriented programming: the encapsulation of state and process within a single unit.

TODO: A figure may placed here

\section{Domain Driven Design}
\label{sec:ddd}

\acrfull{ddd} is a different architectural approach for applications.  It differs from layered architecture primarily in the way the domain is modelled and the responsibilities of application services.

The core idea of \acrshort{ddd} is that the primary focus of a software project should not be the underlying technologies, but the domain. The domain is the topic with which a software concerns itself. The software design should be based on a model that closely matches the domain and reflects a deep understanding of business requirements. \parencite[8, 12]{evans_domain-driven_2004}

This domain model is built from a \emph{ubiquitous language} which is a language shared between domain experts and software experts. This ubiquitous language is built directly from the real domain and must be used in all communications regarding the software. \parencite[24-26]{evans_domain-driven_2004}

%s TODO here, talk about model driven design -> way from the language to the code 

The software must always reflect the way that the domain is talked about. Changes to the domain and the ubiquitous language must result in an immediate change to the domain model.

When modeling the domain model, the aim should not be to create a perfect replica of the real world. While it should carefully be chosen, the domain model is artificial and forms a selective abstraction which should be chosen for its utility. \parencite[12, 13]{evans_domain-driven_2004}

While \hyperref[sec:layered]{Layered Architecture} organizes code into technical tiers and is typically built on \glspl{adm}, often resulting in the \emph{big ball of mud} antipattern \parencite[V]{richards_software_2015}, \acrshort{ddd} demands a \gls{rdm} where objects incorporate both data and the behavior or rules that govern that data. The code is structured semantically into bounded context and modules which are chosen to tell the "story" of a system rather than its technicalities. \parencite[80]{evans_domain-driven_2004}

Entities (also known as reference objects) are domain elements fundamentally defined by a thread of continuity and identity rather than their specific attributes. Entities must be distinguishable from other entities, even if they share the same characteristics. To ensure consistency and identity, a unique identifier is assigned to entities. This identifier is immutable throughout the object's life. \parencite[65-69]{evans_domain-driven_2004}

Value Objects are elements that describe the nature or state of something and have no conceptual identity of their own. They are interesting only for their characteristics. While two entities with the same characteristics are considered as different from each other, the system does not care about "identity" of a value object, since only its characteristics are relevant. Value objects should be used to encapsulate concepts, such as using an "Address" object instead of distinct "Street" and "City" attributes. Value objects should be immutable. They are never modified, instead they are replaced entirely when a new value is required. \parencite[70-72]{evans_domain-driven_2004}

Using a \gls{rdm} does not mean that there should be no layers, the opposite is the case. \textcite{evans_domain-driven_2004} advocates for using layers in domain driven designs. He proposes the following layers: \parencite[53]{evans_domain-driven_2004}

\begin{itemize}
    \item Presentation: Presents information and handles commands
    \item Application Layer: Coordinates app activity. Does not hold business logic, but delegate tasks and hold information about their progress
    \item Domain Layer: Holds information about the domain. Stateful objects (rich domain model) that hold business logic and rules
    \item Infrastructure layer: Supports other layers. Handles concerns like communication and persistence
\end{itemize}

\textcite[75]{evans_domain-driven_2004} points out that in some cases, operations in the domain can not be mapped to one object. For example, transferring money does conceptually not belong to one bank account. In those cases, where operations are important domain concepts, domain services can be introduced as part of model-driven design. To keep the domain model rich and not fall back into procedural style programming like with an \gls{adm}, it is important to use services only when necessary. Services are not allowed to strip the entities and value objects in the domain of behavior. According to Evans, a good domain service has the following characteristics:

\begin{itemize}
    \item The operation relates to a domain concept which would be misplaced on an entity or a value object
    \item The operation performed refers to other objects in the domain
    \item The operation is stateless
\end{itemize}

\section{CRUD architecture}
\label{sec:crud-architecture}

% TODO refine 

\hyperref[sec:layered]{Layered architectures} are the standard for data-oriented enterprise applications. These applications mostly follow a \acrshort{crud} architecture. \acrshort{crud} is an acronym coined by \textcite{martin_managing_1983} that stands for "Create, Read, Update, Delete". These four actions can be applied to any record of data.

The state of domain objects in a \acrshort{crud} architecture is often mapped to normalized tables on a relational database, though other storage mechanisms maybe used. The application acts on the current state of the data, with all actions (reads and writes) acting on the same data. % TODO cite 

\acrshort{acid} (\acrlong{acid}) are an important feature of \acrshort{crud} applications. They can be guaranteed using transactions, ensuring that data stays consistent and operations are \glslink{atomicity}{atomic}. \parencite[10,11]{bernstein_principles_2009} % TODO more explanation: what is a transaction, commit, rollback, etc 

Databases in CRUD systems are typically normalized. Normalization is a process of organizing data into separate tables, removing redundancies and creating relationships through "foreign keys". It is the best practice for relational databases. There are several normal forms that can be achieved, each form building on the previous one: to achieve the second normal form, the first normal form has to be achieved first. \parencite[203]{martin_managing_1983}

\begin{itemize}
    \item 1NF (First Normal Form): Each table cell contains a single (atomic) value, every record is unique
    \item 2NF (Second Normal Form): Remove partial dependencies by requiring that all \emph{non-key} columns are fully dependent on the primary key
    \item 3NF (Third Normal Form): Removes transitive dependencies by requiring that non-key columns depend \emph{only} on the primary key
    \item Further Normal Forms (4NF, 5NF): Require a table can not be broken down into smaller tables without losing data
\end{itemize}

\section{CQRS Architecture}
\label{sec:cqrs}

\acrfull{cqrs} is an architectural pattern based on the fundamental idea that the models used to update information should be separate from the models used to read information. This approach originated as an extension of Bertrand Meyer’s \acrfull{cqs} principle, which states that a method should either perform an action (a command) or return data (a query), but never both. \parencite[148]{meyer_standard_2006}

\acrshort{cqrs} is different from \acrshort{cqs} in the fact that in \acrshort{cqrs}, objects are split into two objects, one containing commands, one containing queries. \parencite[17]{young_cqrs_2010}

\acrshort{cqrs} applications are typically structured by splitting the application into two paths:

\begin{itemize}
    \item Command Side: Deals with data changes and captures user intent. Commands tell the system what needs to be done rather than overwriting previous state. Commands are validated by the system before execution and can be rejected. \parencite[11,12]{young_cqrs_2010}
    \item Read Side: Strictly for reading data. The read side is not allowed to modify anything in the primary data store. The read side typically stores \glspl{dto} in its own data store that can directly be returned to the presentation layer. \parencite[20]{young_cqrs_2010}
\end{itemize}

In a CQRS architecture, the read side typically updates its data asynchronously by consuming notifications or events generated by the write side. Because the models for updating and reading information are strictly separated, a synchronization mechanism is required to ensure the read store eventually reflects the changes made by commands. This usually leads to stale data on the read side.

Each read service independently updates its model by consuming notifications or events published by the write side, allowing the read model to store optimized, denormalized views on the data. \parencite[23]{young_cqrs_2010}

TODO: place a figure / digram here

\section{(Eventual) Consistency}

% TODO have to talk about "read your writes"? 
% TODO maybe move above CRUD and CQRS 

\textcite{gray_dangers_1996} explain that large-scale systems become unstable if they are held consistent at all times according to \acrshort{acid} principles. This is mostly due to the large amount of communication necessary to handle atomic transactions in distributed systems. To address these issues, modern distributed systems often adopt the \acrshort{base} (\acrlong{base}) model which explicitly trades off isolation and strong consistency for availability. Eventually consistent systems are allowed to exist in a so-called "soft state" which eventually converges through the use of synchronization mechanisms over time rather than being strongly consistent at all times. \parencite{braun_tackling_2021, vogels_eventually_2009} This creates an inconsistency window in which data is not consistent across the system. During this window, stale data may be read. \parencite{vogels_eventually_2009}

\section{Event Sourcing and event-driven architectures}
\label{sec:event-sourcing}

Event driven architecture is a design paradigm where systems communicate via the production and consumption of events. Events are records of changes in the system's domain. \parencite{michelson_event-driven_2006} This approach allows for a high degree of loose coupling, as the system publishing an event does not need to know about the recipient(s) or how they will react. These architectures offer excellent horizontal scalability and resilience, as individual system components can fail or be updated without bringing down the entire network. \parencite{fowler_event_2005}

Event Sourcing is an architectural pattern within the landscape of event driven architectures. Event-sourced systems ensure that all changes to a system's state are captured and stored as an ordered sequence of domain events. Unlike traditional persistence models that overwrite data and store only the most recent state, event sourcing maintains an immutable record of every action taken over time. These events are persisted in an append-only event store, which serves as the principal source of truth from which the current system state can be derived. \parencite[457,458]{kleppmann_designing_2017}

The current state of any entity in such a system can be rebuilt by replaying the history of events from the log, starting from an initial blank state. \parencite{fowler_event_2005} To address the performance costs of replaying thousands of events for every request, developers implement projections or materialized views, which are read-only, often denormalized versions of the data optimized for specific queries. \parencites{malyi_developing_2024}[461,462]{kleppmann_designing_2017} This separation of concerns is frequently managed by pairing event sourcing with the \hyperref[sec:cqrs]{\acrlong{cqrs} (\acrshort{cqrs})} pattern, which physically divides the data structures used for reading from those used for writing state changes. \parencite[50]{young_cqrs_2010}

To optimize the reconstruction of state, developers often employ \textit{rolling snapshots}. These snapshots represent a serialized, denormalized point-in-time state of an aggregate, allowing the system to restore the state immediately and only replay the delta of events that occurred after the snapshot was taken. \parencite[20]{young_cqrs_2010} This heuristic effectively caps the maximum number of events to be processed, providing a predictable and significant performance gain during the loading process. It is worth noting that the event stream stays intact and is not impacted by a snapshot.

\section{Traceability and auditing in IT systems}

Traceability and auditing are legal requirements across various sectors, as they are derived from federal laws and regulations intended to protect the integrity and confidentiality of sensitive data. Organizations implement these mechanisms to stay compliant with mandates that require a verifiable, time-sequenced history of system activities to support oversight and forensic reviews. In the U.S. financial sector, for example, 17 CFR § 242.613 requires the establishment of a consolidated audit trail to track the complete lifecycle of securities orders, documenting every stage from origination and routing to final execution. \parencite{us_securities_and_exchange_commission_17_2012}

\subsection{Audit Logs}
\label{sec:audit-log}

An audit log (often called audit trail) is a chronological record which provides evidence of a sequence of activities on an entity. \parencite{committee_on_national_security_systems_national_2010} In information security, the audit log stores a record of system activities, enabling the reconstruction of events. \parencite{atis_committee_atis_2013} A trustworthy audit log in a system can guarantee the principle of traceability which states that actions can be tracked and traced back to the entity who is responsible for them. \parencite[266]{joint_task_force_interagency_working_group_security_2020}

\textcite{fowler_audit_2004} describes an audit log as simple and effective way of storing temporal information. Changes are tracked by writing a record indicating \emph{what} changed \emph{when}. A basic implementation of an audit log can have many forms, for example a text file, database tables or \acrshort{xml} documents. Fowler also mentions that while the audit log is easy to write, it is harder to read and process. While occasional reads can be done by eye, complex processing and reconstruction of historical state can be resource-intensive.

In distributed environments or complex application architectures, the implementation of an audit log often introduces the "dual-write" problem. This occurs when an application is responsible for updating the primary database (the current state) and simultaneously emitting a record to a separate audit log or messaging system. As \textcite[452,453]{kleppmann_designing_2017} notes, ensuring atomicity across these two distinct writes is technically challenging. If the primary database update succeeds, but the audit log write fails, or vice versa, the two systems will diverge, leading to a loss of data integrity where the audit trail no longer reflects the "real" state of the system.

This separation highlights, that in traditional \acrshort{crud} systems, the audit log is simply a secondary source of truth. As it relies on application- or database-level logic, it is nothing more but a passive observer, relying on notifications from the primary process.

\subsection{Event Streams as a Basis for Traceability}

While \hyperref[sec:audit-log]{traditional audit logs} are often implemented as secondary systems that capture state changes, event-driven architectures, such as those utilizing Event Sourcing, turn an event stream into the primary source of truth. In this context, an event stream is not just a diagnostic tool but an exact, chronological sequence of intent-driven records.

As established in \autoref{sec:event-sourcing}, every state change is captured as a discrete event. Because these events are immutable and append-only, they provide a natural foundation for the principle of traceability. Unlike traditional "state-based" auditing, where the system might only record that a value changed from $A$ to $B$, an event stream captures the specific domain context. Tthe \emph{intent} behind a change is semantically conveyed through the event type. For example, while a traditional audit log might simply record a status update to \texttt{CLOSED}, an event-sourced system distinguishes between an \texttt{AccountDeletedByUser} event and an \texttt{AccountTerminatedForInactivity} event. This inherent metadata provides an exhaustive audit trail without the need for additional logging logic.

\textcite{fowler_event_2005} notes that because the event log is complete, the system can perform \emph{Temporal Queries}, effectively "time-traveling" to reconstruct the exact state of the system at any historical checkpoint. This makes event streams particularly robust for forensic reviews and regulatory compliance, as they eliminate the "information loss" associated with traditional database overwrites. In the context of the legal requirements discussed in the previous section, the event stream serves as a sequence of actions that satisfies the need for a verifiable, time-sequenced history.

\subsection{Rebuilding state from an audit log and an event stream}

There is a fundamental difference in how systems built with a secondary audit log versus an event-sourced architecture reconstruct historic state. It lies in the relationship between the operational data and the chronological record. In systems with a secondary audit log, the audit log and the application state are often updated as two separate operations. This decoupling introduces the risk of silent divergence. If a failure occurs during the logging process, but the primary state change succeeds, the audit trail becomes an incomplete reflection of reality. Because the system continues to function using the primary database, these discrepancies may remain undetected until a forensic reconstruction is attempted. In this scenario, the audit log serves as a secondary piece of evidence rather than a definitive blueprint, making it difficult to guarantee that a reconstructed state is perfectly synchronized with the original historical state.

In contrast, rebuilding state from an event stream is a deterministic process. Since the event stream is the primary source of truth, there is no secondary state to diverge from. State is reconstructed by mapping events to objects (aggregates or projections). If an event is not recorded, the state change never occurred. This guarantees that the log and the system state are consistent by design, ensuring traceability.

\section{Scalability of systems (TODO)}

This section describes which factors play a role in the scalability of a system. Architectural concerns (e.g. refactoring systems to be split up into microservices), resource consumption, etc.

We define different angles on scalability:

\begin{itemize}
    \item Throughput Scalability: How the system handles an increasing number of commands (writes) vs. queries (reads).
    \item Data Volume: How the system behaves as the history of events or audit logs grows into the terabytes.
    \item Organizational / architectural Scalability: How easily multiple teams can work on the system without creating bottlenecks (the "microservices" angle).
\end{itemize}

Then address write and read scalability; and differences in ES and CRUD architectures. CRUD: e.g. database contention (locks); CQRS would require sharding based on aggregate ID. Two-phase commits in a distributed system?

ES-CQRS with snapshots as strategy for scaling huge event streams (quicker reconstruction of state); but increased storage size. Data is often duplicated across projections.

% !TeX root = ../main.tex
\chapter{Related Work (TODO)}

After the theoretical foundations of the thesis are covered, this chapter presents work related to the \acrlongpl{rs}. As there are 3 sub-questions to address, this chapter is divided into three sections.

\section{RS1 - Performance and Scalability}

\begin{itemize}
    \item Kleppman, 2017, data intensive systems
          - Chapter 1: scalability, performance, approaches for coping with load
          - Chapter 11 (stream processing). Performance implications of log structured storage vs. B-Trees (db index)
    \item Jogalekar et al, 2000. "Evaluating the scalability of distributed systems" provides a formal mathematical framework for defining "Scalability" ($P = \lambda / T$).
    \item Singh, 2025 presents a performance comparison between DDD and CQRS. While DDD is not a traditional CRUD architecture, clearly the migration to separate read and write paths gave a performance increase
    \item Jayaraman et al, 2024 "Implementing Command Query Responsibility Segregation (CQRS) in Large-Scale Systems"
          includes a performance benchmark (p. 58ff)
\end{itemize}

\section{RS2 - architectural complexity, maintainability, flexibility}

\begin{itemize}
    \item Singh, 2025 "Using CQRS and Event Sourcing in the Architecture of Complex Software Solutions"
          - specific results on code metrics: the transition to CQRS/ES increased the total number of classes (from 47 to 213) but decreased the overall cyclomatic complexity (from 534 to 522)
          - highlights that individual modules become simpler despite higher number of classes
    \item "Object Oriented Coupling based Test Case Prioritization":
          - statistical approach (linear regression, hypothesis testing) to correlate OO metrics with software quality
          - Defines and analyzes the CK Metrics Suite (WMC, DIT, NOC, CBO, RFC, LCOM). It establishes empirical correlations, such as Coupling Between Objects (CBO) having the strongest negative impact on quality
    \item  Comparative Study of the Software Metrics for the complexity and Maintainability of Software Development: can act as "dictionary" for all the code metrics
\end{itemize}

\section{RS3 - traceability}

\begin{itemize}
    \item Maybe Gantz, 2014: Basics of IT Audit
    \item Vamshikrishna Monagari: "Demystifying Event-Driven Microservices in Cloud-Native  FinTech Applications"
          - contains many quantitative results, e.g.40\% lower latency for ES applications; 95\% reduction in compliance audit preparation time; reduced time for root cause analysis.
          - Paper seems to be problematic as I could not find some of the mentioned results in the references
\end{itemize}

Maybe include sources talking about GDPR and the \emph{problems} with a non-erasable event log, and some solutions mitigating the problem? $\rightarrow$ "right to be forgotten". This is not a goal of this thesis but related.

\section{Relevant results}

Not sure if this needs to be its own section, but it will be important to highlight what the key findings are which are relevant to my work. Show what others discovered, things that I build upon, things that may and should be picked up again when discussing results.

% !TeX root = ../main.tex
\chapter{Methodology}

This thesis aims to provide a fair, quantitative comparison of \acrshort{crud} and \acrshort{cqrs} / \acrshort{es} architectures regarding all three research questions. To achieve this, the architectures should be applied not only to the same domain, but to the exact same requirements. The implementations can then be tested against the same \glspl{contract-test} to ensure their interfaces and behavior is identical.

This chapter will first present the requirements and the domain for the actual application, then outline the proposed method used to answer each research question. \acrshort{rs} 1, concerning itself with performance and scalability evaluation, necessitates describing load testing foundations and implications of performance on scalability of a system.

Scalability is also a function of architectural flexibility, which is addressed by \acrshort{rs} 2. This flexibility will be quantified using code quality metrics based on static analysis, which are also presented.

Finally, traceability, the core of \acrshort{rs} 3, is defined and the method to evaluate it will be described. (TODO...)

\section{Project requirements}

The applications will implement a course enrollment and grading system which might for example be used in universities. Core features include:

\begin{itemize}
    \item Professors can create courses and lectures
    \item Students can enroll and disenroll from lectures
    \item Professors can enter grades
    \item Students can view their current and past lectures
    \item Students can view their credits
\end{itemize}

TODO here: create a more precise feature / business rule matrix; consolidate with \autoref{sec:business-rules}.

\subsection{Entities}
\label{sec:entities}

Two types of users exist in the domain: professors and students. Their personal information is not relevant for this thesis, which is why only their first and last name are stored for presentation reasons. The student additionally has a semester.

Professors can create courses. Courses have a name, a description, an amount of credits they yield, a minimum amount of credits required to enroll and can have a set of courses as prerequisites.

Courses are the "blueprints" for lectures. Lectures are the "implementation" of a course for a semester. Each lecture created from a course yields the course's amount of credits and has the requirements specified by the course. Lectures have a lifecycle: they can be in draft state, open for enrollment, in progress, finished or archived. A lecture has a list of time slots and a maximum amount of students that can enroll.

A lecture can have several assessments. Each assessment has a type. The professor can enter grades for a student and an assessment. Grades are integers in the range of 0 to 100. Credits are awarded to a student as soon as they completed all assessments for a lecture with a passing grade (grade higher than 50) and once a lecture's status is set to finished.

\subsection{Business rules}
\label{sec:business-rules}

Relationships and business rules in this system are deliberately chosen complex, involving many relationships between \hyperref[sec:entities]{entities} and intricate validation rules. This approach was adopted in order to be able to make realistic assumptions about the research question by evaluating a project that closely resembles complex, real-world scenarios.

The following list presents a selection of business rules which were implemented.

\begin{itemize}
    \item Existence checks: any requests including references to entities will fail if the references entities do not exist.
    \item Requests leading to conflicts, for example creating a lecture with overlapping time slots, will fail.
    \item When a student tries enrolling to a lecture which is already full, they will be put on a waitlist.
    \item When a student disenrolls from a lecture, the next eligible student (higher semesters are preferred) will be enrolled.
    \item Actions on a lecture can only be performed during the appropriate lifecycle state (enrolling only when the lifecycle is "open for enrollment", grades can only be assigned when the lecture is "finished").
\end{itemize}

\subsection{Contract Tests}
\label{sec:contract-tests}

To ensure both implementations adhere to the business rules, an extensive test suite was set up. While the internals of the implementations are vastly different architecturally and conceptually, they both have the same public \gls{api}. This makes it possible to run the same test suite on both apps by sending \gls{http} requests and verifying their responses. The test suite includes integration tests for all \gls{api} endpoint covering both regular and edge-case (error) scenarios to ensure that the \acrshort{crud} and \acrshort{es}-\acrshort{cqrs} application exhibit identical state transitions and error behaviors. \hyperref[sec:contract-test-implementation]{Section \ref*{sec:contract-test-implementation}} outlines the implementation of those tests in detail.

\section{Performance}

To answer \hyperref[rs-performance-scalability]{\acrshort{rs} 1}, the performance characteristics of both architectural patterns will be evaluated and compared. To achieve this, load tests are executed on selected endpoints. The theoretical foundations of load testing, the environmental constraints, and the data collection methodology are described in this section.

\subsection{Theoretical Load Testing Foundations}
\label{sec:load-test-theory}

To provide accurate performance metrics, load testing is performed on both implementations based on the methodology defined by \textcite[10-17]{kleppmann_designing_2017}. Load is characterized using "load parameters," which vary depending on the system's nature. For this study, the primary load parameter is the number of concurrent requests to the web server. \parencite[11]{kleppmann_designing_2017}

The tests measure \textbf{\acrfull{rps}} and client-side response times. \textcite[15,16]{kleppmann_designing_2017} emphasizes the importance of client-side measurement to account for "queueing delays." As server-side processing is limited by hardware resources (e.g., CPU cores), requests may be stalled before processing begins. Server-side metrics often exclude this wait time, leading to an overly optimistic view of performance. Consequently, the load-generating client must utilize an "open model," sending requests without waiting for previous ones to complete, to simulate realistic concurrent user behavior.

This thesis adopts the approach of keeping system resources constant while measuring performance fluctuations under varying load intensities. \parencite[13]{kleppmann_designing_2017}

To evaluate results, the arithmetic mean is avoided as it obscures the experience of typical users and the impact of outliers. Instead, this thesis uses \glspl{percentile}. The \gls{median} (P50) serves as the metric for "typical" response time. However, to capture the experience of users facing significant delays—often caused by data-intensive operations—it is critical to measure "\glspl{tail-latency}" via the P95 percentile. High percentiles identify the performance of outliers which, despite being a numerical minority, often represent the most valuable or complex operations. \parencite[14-16]{kleppmann_designing_2017}

\subsection{Service Level Objectives}
\label{sec:slo}

While \glspl{sla} are agreements with users regarding uptime and performance, \glspl{slo} are the technical targets used by engineers to meet those requirements. \parencite{beyer_site_2016} This thesis attempts to define realistic \acrshortpl{slo} to establish a "breaking point" for each architecture.

Following \textcite[135]{nielsen_usability_1993}, a response time of 100ms is the threshold for human perception of "instant" feedback. This serves as the baseline for the following targets:

\begin{itemize}
    \item \textbf{Latency \acrshort{slo}}: All endpoints must maintain a client-side P95 latency of $\le$100ms to ensure the system feels "instant" for 95\% of requests. \label{slo-latency}
    \item \textbf{Freshness \acrshort{slo}}: In the Event Sourcing implementation, the asynchronous nature of projections introduces a lag. All writes must be reflected in the PostgreSQL read-model within $\le$100ms to ensure eventual consistency remains imperceptible. This \acrshort{slo} applies only to the ES-CQRS implementation. \label{slo-freshness}
    \item \textbf{Reliability \acrshort{slo}}: Both implementations must maintain a failure rate of <0.1\% under stress. \label{slo-reliability}
\end{itemize}

\subsection{Comparability and Environment}

The test environment and scenarios are defined as code to ensure reproducibility. Tests are executed in an isolated environment with fixed hardware allocations as specified in \autoref{table:hardware-specs}.

\begin{table}[htp!]
    \small
    \centering
    \begin{tabularx}{\linewidth}{lX}
        \toprule
        \textbf{Component} & \textbf{Specification}                                                      \\ \midrule
        CPU                & 13th Gen Intel(R) Core(TM) i7-13700H. 14 Cores, 20 total threads. Max. 5GHz \\
        RAM                & 32GB DDR4 (2x16GB), 3200 MT/s                                               \\
        Hard Drive         & SanDisk Plus SSD 1TB 2.5" SATA 6GB/s                                        \\
        \bottomrule
    \end{tabularx}
    \caption{Hardware specifications for the performance evaluation machine}
    \label{table:hardware-specs}
\end{table}

The physical host provisions two \glspl{vm}: the "client VM" for load generation and the "server VM" for the application and its dependencies (PostgreSQL and Axon Server). While hosting both on one physical machine makes network latency negligible, the "queueing delay" remains measurable at the client level, allowing for the identification of request queues building up on the server, indicating bottlenecks.

\subsection{Test Execution}
\label{sec:test-execution-method}

The tool \hyperref[sec:k6]{k6} is used to generate load. The client follows an "open model" via k6's arrival-rate executors to decouple request rate from response times, as discussed in \autoref{sec:load-test-theory}.

Each test follows a specific pattern:
\begin{itemize}
    \item \textbf{Ramp-up}: Linear increase from 0 to the target \acrshort{rps} over 20 seconds.
    \item \textbf{Steady State}: Constant target \acrshort{rps} for 80 seconds.
    \item \textbf{Ramp-down}: Linear decrease to 0 \acrshort{rps} over 20 seconds.
\end{itemize}

A \emph{test configuration} (the combination of a script and target \acrshort{rps}) is executed at least 25 times. Between runs, the application and its dependencies (PostgreSQL/Axon Server) are restarted to ensure independent results. Client-side metrics are captured in \acrshort{json} format, while server-side resource consumption (CPU \& RAM usage) is collected via \hyperref[sec:actuator]{Spring Boot Actuator}.

\subsection{Collected Metrics}
\label{sec:collected-metrics}

All metrics collected during load testing are listed in \autoref{table:collected-metrics}. The column "Metric" shows the name that will be used to refer to this metric from now on. "Location" describes where the metric is being recorded.

As described in \autoref{sec:load-test-theory}, not just the average latency is measured, but \glspl{percentile} are used to accurately report the number of users experiencing the respective latency.

While internal metrics, like CPU usage, RAM usage, the number of database connections ($hikari\_connections$) and the number of Tomcat worker threads are not measures for user-perceived performance, they can give an indicator about bottlenecks inside the applications.

Even though latencies are measured on both the client and the server, the client latency will be used primarily when visualizing results, as it is the latency users perceive when interacting with the applications. The server latency is recorded because it can help identify queueing delays by exposing differences in server and client latencies which exceed a factor of 1.1x.

$axon\_storage\_size$ is only recorded for the ES-CQRS application. $postgres\_size$ can be recorded for both applications. The CRUD implementation stores its entities and audit log in PostgreSQL, while the ES-CQRS implementation uses PostgreSQL as a secondary data store where denormalized projections and lookup tables live. These metrics are relevant when assessing a system's long-term performance, scalability and maintainability.

\begin{table}[htp!]
    \small
    \centering
    \begin{tabularx}{\linewidth}{lXl}
        \toprule
        \textbf{Metric}       & \textbf{Description}                                 & \textbf{Location} \\ \midrule
        $latency\_avg$        & Average (arithmetic mean) latency                    & Server, Client    \\
        \addlinespace
        $latency\_p50$        & 50th \gls{percentile} Latency (\gls{median})         & Server, Client    \\
        \addlinespace
        $latency\_p95$        & 95th \gls{percentile} Latency                        & Server, Client    \\
        \addlinespace
        $latency\_p99$        & 99th \gls{percentile} Latency                        & Server, Client    \\
        \addlinespace
        $cpu\_usage$          & CPU usage of the Server process                      & Server            \\
        \addlinespace
        $ram\_usage\_heap$    & Usage of Heap memory                                 & Server            \\
        \addlinespace
        $ram\_usage\_total$   & Usage of total memory                                & Server            \\
        \addlinespace
        $hikari\_connections$ & Number of Data Source connections                    & Server            \\
        \addlinespace
        $tomcat\_threads$     & Number of Tomcat worker threads                      & Server            \\
        \addlinespace
        $postgres\_size$      & Size of PostgreSQL database                          & Server            \\
        \addlinespace
        $axon\_storage\_size$ & Size of Axon storage, incl. Event and Snapshot store & Axon Server       \\
        \bottomrule
    \end{tabularx}
    \caption{All metrics collected during load testing}
    \label{table:collected-metrics}
\end{table}

\subsection{Visualizing Results}

Data is visualized using box plots and line graphs. Box plots are used to show the distribution of latencies, while line graphs illustrate performance changes as \acrshort{rps} increases. Per scientific standards, error bars are used to represent the variability of the measurements across the test runs.

\subsection{Performance Implications (TODO)}

The results collected during load testing show how the applications perform under varying load and different situations. Writes and reads are both tested using endpoints which require a varying degree of invariant checking or entity JOINS.

This section will describe the implications of the collected results on the predicted scalability of systems. To do so, literature is analyzed which describes the correlation between performance, resource usage and scalability. We analyze how quickly a system reaches its bottlenecks and in which cases which system is better scalable.

Scalability also depends on the architecture of a software. For example, an application designed to separate reads and writes allows to separate the application into microservices more easily than a monolith. This view on scalability will be described further when presenting the results of architectural metrics.

\subsection{Load Testing Scenarios (TODO)}

This subsection lists every load testing scenario and explains why its results are relevant to the research question.

\begin{itemize}
    \item Create Courses Simple
    \item Create Courses with Prerequisites
    \item Get lectures a student is enrolled in
    \item Enrollment to a lecture
    \item Create lecture and then read (time to consistency)
    \item read grade history (simple time-travel query)
    \item more complex time travel query (TODO)
\end{itemize}

\section{Flexibility - Architectural Metrics (TODO)}

This section describes various architectural metrics which are established in literature and used to assess the flexibility and quality of a software architecture. It serves as a basis to answer \hyperref[rs-architecture]{\acrshort{rs} 2}. The method uses established static analysis methods to compare the two architectures. Describe how the results will be visualized and what they may indicate. Maybe talk about schema evolution.

\textbf{Metrics:}

\begin{itemize}
    \item Martin Metrics, focusing on coupling and instability
    \item MOOD Metrics evaluate the quality of object-oriented designs: method and attribute hiding factor, etc
    \item Complexity metrics, e.g. cyclomatic complexity
    \item Dependency metrics: afferent, efferent; abstractness
\end{itemize}

\textbf{Literature:}

\begin{itemize}
    \item Richards and Ford, "Fundamentals of software architecture: an engineering approach"
    \item Chawla and Kauer, "Comparative Study of the Software Metrics for the complexity and Maintainability of Software Development"
    \item Deshpande et al., "Object Oriented Design Metrics for Software Defect Prediction: An Empirical Study"
    \item Singh, "Object Oriented Coupling based Test Case Prioritization"
    \item Overeem et al., "An empirical characterization of event sourced systems and their schema evolution — Lessons from industry"
\end{itemize}

\section{Traceability (TODO)}

This section describes how traceability will be compared; it serves as a basis to answer \hyperref[rs-traceability]{\acrshort{rs} 3}. Maybe talk about schema evolution here (how does schema evolution affect the ability to make historic queries in the system).

\subsection{Accuracy of reconstruction}

We define qualitative criteria for comparison:

\begin{itemize}
    \item Source of Truth Integrity: How the system guarantees that the history matches the current state. Especially in distributed systems (dual-write problem)
    \item ntent Preservation: The ability to distinguish between different business reasons for the same data change (e.g., "Refund" vs. "Fee")
    \item Schema Resiliency: How the history survives changes to the database structure or business rules.
\end{itemize}

Attempt to define "business metrics":

\begin{itemize}
    \item How ES helps with root cause analysis / time to reproduce and debug bugs in production
    \item Required time and work to prepare for security audits
\end{itemize}

\subsection{Efficiency}

\begin{itemize}
    \item Project already employs load testing
    \item load test time-travel queries on both applications, assuming equal scenarios, equal amount of data etc. then compare performance
    \item Define use cases for time-travel queries:
          \begin{itemize}
              \item grade history (this endpoint already exists).  is rather simple as only the state of one entity is needed
              \item Another, more complex question. Should be a use-case where a broader part of the application's historic state needs to be reconstructed. e.g. which lectures was a student enrolled in at point $T$
          \end{itemize}
    \item measure CPU and RAM usage during reconstruction
    \item assumption: fetching data from audit log tables may be more efficient, because date filters can be used on indexed tables
    \item ES system has to play ALL events. (Snapshots can not be used when replaying events, they are only used when rehydrating aggregates. -> this part probably belongs in result)
\end{itemize}

\section{Technologies}
\label{sec:technologies}

This section describes all technologies used for the implementation and evaluation of the two applications. % TODO dependency matrix in appendix 

\subsection{SpringBoot}

SpringBoot \footnote{\href{https://spring.io/projects/spring-boot}{SpringBoot}} is an open-source, opinionated framework for developing enterprise Java applications. It is based on Spring Framework \footnote{\href{https://spring.io/projects/spring-framework}{Spring Framework}}, which is a platform aiming to make Java development "quicker, easier, and safer for everybody" \parencite{broadcom_inc_why_2026}. At Spring Framework's core is the Inversion of Control (IoC) container. The objects managed by this container are referred to as \textit{Beans}. While the term originates from the JavaBeans specification, a standard for creating reusable software components, Spring extends this concept by taking full responsibility for the lifecycle and configuration of these objects \parencite[Chapter~1.1]{walls_spring_2016}. Instead of a developer manually instantiating classes using the \texttt{new} operator, the container "injects" required dependencies at runtime. This process is known as Dependency Injection. \parencite[Chapter~1]{deinum_spring_2023}. Spring offers support for several programming paradigms: reactive, event-driven, microservices and serverless. \parencite{broadcom_inc_why_2026}

SpringBoot builds on top of the Spring platform by applying a "convention-over-configuration" approach, intended to minimize the need for configuration. In a 2023 survey by JetBrains, SpringBoot was the most popular choice of web framework. \parencite{jetbrains_java_2023}

Spring Boot starters are specialized dependency descriptors designed to simplify dependency management by aggregating commonly used libraries into feature-defined packages. Rather than requiring developers to manually identify and maintain a list of individual group IDs, artifact IDs, and compatible version numbers for every necessary library, starters use transitive dependency resolution to pull in all required components under a single entry. To quickly bootstrap a web application, a developer can simply add the \javaname{spring-boot-starter-web} dependency to their Maven or Gradle build file. By requesting this specific functionality, Spring Boot automatically includes essential dependencies such as Spring MVC, Jackson for JSON processing, and an embedded Tomcat server, ensuring that all included libraries have been tested together for compatibility. This approach shifts the developer's focus from managing individual JAR files to simply defining the high-level capabilities the application requires, minimizing configuration overhead and reducing risk of version mismatches. \parencite[Chapter~1.1.2]{walls_spring_2016}
% TODO code example for IoC container / dependency injection 

\subsection{JPA}
\label{sec:jpa}

\acrfull{jpa} \footnote{\href{https://jakarta.ee/specifications/persistence/}{JPA}}, formerly Java Persistence \gls{api} is a Java specification which provides a mechanism for managing persistence and object-relational mapping (\acrshort{orm}). \glspl{orm} act as a bridge between the relational world of SQL databases and the object-oriented world of Java.

Instead of writing SQL to create the database schema, entities can be described using special Java classes (defined by annotations or \acrshort{xml} configurations) which can be mapped to an SQL schema. \acrshort{jpa} allows querying the database for these entities in a type-safe way by providing a range of helpful query methods on JPA repositories, for example \texttt{findAll()} or \texttt{findById(UUID id)}. This removes the need to write "low-level", database-specific SQL for basic \acrshort{crud} operations. Complex data retrieval is also possible with \acrshort{jpa} using the \acrfull{jpql}, which is an object-oriented, database-agnostic query language.

When using \acrshort{jpa} with SpringBoot by including the \javaname{spring-boot-starter-data-jpa} dependency, \emph{Hibernate} \footnote{\href{https://hibernate.org/orm/}{Hibernate}} is used as implementation of the \acrshort{jpa} standard. \parencite[Chapter~1]{bauer_java_2016}

\subsection{PostgreSQL}
\label{sec:postgresql}

PostgreSQL \footnote{\href{https://www.postgresql.org/}{PostgreSQL}} is an open-source relational database system which has been in active development for over 35 years. Thanks to its reliability, robustness and performance, it has a strong earned reputation. \parencite{postgresql_global_development_group_postgresql_2026} PostgreSQL is designed for a wide range of workloads and can handle many tasks thanks to its extensibility and large suite of extensions, such as the popular PostGIS extension for storing and querying geospatial data. \parencite{postgis_psc_postgis_2023}

\subsection{Jackson}

Jackson \footnote{\href{https://github.com/FasterXML/jackson}{Jackson}} is a high-performance, feature-rich \acrshort{json} processing library for Java. It is the default \acrshort{json} library used within the Spring Boot ecosystem. Its primary purpose is to provide a seamless bridge between Java objects and JSON data through three main processing models: the Streaming API for incremental parsing, the Tree Model for a flexible node-based representation, and the most commonly used Data Binding module. This data binding capability allows developers to automatically convert (\emph{marshal}) Java \glspl{pojo} into JSON and vice versa (\emph{unmarshal}) with minimal configuration. Beyond its speed and efficiency, Jackson is highly extensible, offering modules to handle complex Java types like Java 8 Date/Time and Optional classes. Jackson also supports various other data formats such as XML, YAML and CSV. \parencite{oracle_jackson_nodate, fasterxml_jackson_2025}

\subsection{Axon}
\label{sec:axon}

Axon Framework \footnote{\href{https://www.axoniq.io/framework}{Axon Framework}} is an open-source Java framework for building event-driven applications. Following the \acrshort{cqrs} and event-sourcing pattern, Commands, Events and Queries are the three core message types any Axon application is centered around. Commands are used to describe an intent to change the application's state. Events communicate a change that happened in the application. Queries are used to request information from the application.

Axon also supports \acrlong{ddd} by providing tools to manage entities and domain logic. \parencite{axoniq_introduction_2025,axoniq_messaging_2025}

Axon Server \footnote{\href{https://www.axoniq.io/server}{Axon Server}} is a platform designed specifically for event-driven systems. It functions as both a high-performance Event Store and a dedicated Message Router for commands, queries, and events. By bundling these responsibilities into a single service, Axon Server replaces the need for separate infrastructures such as a relational database for events and a message broker like Kafka or RabbitMQ for communication. Axon Server is designed to seamlessly integrate with Axon Framework. When using the Axon Server Connector, the application automatically finds and connects to the Axon Server. It is then possible to use the Axon server without further configuration. \parencite{axoniq_introduction_2025-1,axoniq_axon_2025} % TODO book source 

\subsubsection*{Command dispatching}

Command dispatching is the starting point for handling a command message in Axon. Axon handles commands by routing them to the appropriate command handler. The command dispatching infrastructure can be interacted with using the low-level \keyw{CommandBus} and a more convenient \keyw{CommandGateway} which is a wrapper around the \keyw{CommandBus}.

\keyw{CommandBus} is the infrastructure mechanism responsible for finding and invoking the correct command handler. At most one handler is invoked for each command; if no handler is found, an exception is thrown.

Using \keyw{CommandGateway} simplifies command dispatching by hiding the manual creation of \keyw{CommandMessages}. The gateway offers two main methods for synchronous and asynchronous patterns. The \keyw{send} method returns a \keyw{CompletableFuture}, which is an asynchronous mechanism in Java. If the thread needs to wait for the command result, the \keyw{sendAndWait} method can be used.

In general, a handled command returns \keyw{null}, if handling was successful. Otherwise, a \keyw{CommandExecutionException} is propagated to the caller. While returning values from a command handler is not forbidden, it is used sparsely as it contradicts with CQRS semantics. One exception: command handlers which \emph{create} an aggregate typically return the aggregate identifier. \parencite{axoniq_command_2025,axoniq_infrastructure_2025}

\subsubsection*{Query Handling}
Before a query is handled, Axon dispatches it through its messaging infrastructure. Just like the command infrastructure, Axon offers a low-level \keyw{QueryBus} which requires manual query message creation and a more high-level \keyw{QueryGateway}.

In contrast to command handling, multiple query handlers can be invoked for a given query. When dispatching a query, callers can decide whether they want a single result or results from all handlers. When no query handler is found, an exception is thrown.

The \keyw{QueryGateway} includes different dispatching methods. For regular "point-to-point" queries, the \keyw{query} method can be used. Subscription queries are queries where callers expect an initial result and continuous updates as data changes. These queries work well with reactive programming. For large result sets, streaming queries should be used. The response returned by the query handler is split into chunks and streamed back to the caller.
All query methods are asynchronous by nature and return Java's \keyw{CompletableFuture}. \parencite{axoniq_query_2025}

\subsubsection*{Aggregates}
\label{sec:aggregates}

An aggregate is a core concept of \acrfull{ddd}. In Axon, an aggregate defines a consistency boundary around domain state and encapsulates business logic. Aggregates are the primary place where domain invariants are enforced and where commands that intend to change domain state are handled.

Aggregates define command handlers using methods or constructors annotated with \keyw{@CommandHandler}. These handlers receive commands and decide whether they are valid according to domain rules. If a command is accepted, the aggregate emits one or more domain events describing \emph{what} happened. Command handlers are responsible only for decision-making; they must not directly mutate the aggregate’s state. Instead, all state changes must occur as a result of applying events.

Every aggregate is typically annotated with \keyw{@Aggregate} and must declare exactly one field annotated with \keyw{@AggregateIdentifier}. This identifier uniquely identifies the aggregate instance. Axon uses it to route incoming commands to the correct aggregate and to load the corresponding event stream when rebuilding aggregate state.

By default, Axon uses event-sourced aggregates. This means that aggregates are not persisted as a snapshot of their fields. Instead, their current state is reconstructed by replaying all previously stored events. Methods annotated with \keyw{@EventSourcingHandler} are called by Axon during this replay process to update the aggregate’s internal state based on event data. Since events represent facts that already occurred, event sourcing handlers must not contain business logic or make decisions.

Axon also supports multi-entity aggregates. In this model, an aggregate may contain child entities that participate in command handling. Such entities are registered using \keyw{@AggregateMember}, and each entity must define a unique identifier annotated with \keyw{@EntityId}. Based on this identifier, Axon is able to route commands to the correct entity instance within the aggregate. \parencite{axoniq_multi-entity_2025}

\subsubsection*{External Command Handlers}
\label{sec:external-command-handlers}

Often, command handling functions are placed directly inside the aggregate. However, this is not required and in some cases it may not be desirable or possible to directly route a command to an aggregate. Thus, any object can be used as a command handler by including methods annotated with \keyw{@CommandHandler}. One instance of this command handling object will be responsible for handling \emph{all} commands of the command types it declares in its methods.

In these external command handlers, aggregates can be loaded manually from Axon's repositories using the aggregate's ID. Afterward, the \keyw{execute} function can be used to execute commands on the loaded aggregate. \parencite{axoniq_command_2025-1}

\subsubsection*{Set-based validation}
\label{sec:set-based-validation}

When receiving a command, aggregates handle it by validating their internal state inside command handlers and either rejecting the command or publishing an event. However, validation across a set of aggregates, called "set-based validation", is not possible inside a single aggregate. A business requirement like "Usernames must be unique" can only be implemented using set-based validation, as the entire set of aggregates must be inspected before making a decision.

Set-based implementation in Axon can be implemented by using lookup tables. This approach utilizes a dedicated command-side projection, often referred to as a lookup projector, to maintain a specialized view of the system state. While projectors are typically associated with the read-side of a \acrshort{cqrs} architecture, a lookup projector is specifically designed to support the command side. It maintains a highly optimized and consistent dataset, such as a registry of unique identifiers, which can be queried during the validation phase of a command.

To ensure that this lookup table remains synchronized and provides the necessary consistency for validation, Axon employs subscribing event processors, which are described in \autoref{sec:axon-events}. Unlike tracking event processors which operate asynchronously and introduce eventual consistency, subscribing event processors execute within the same thread and transaction as the event publication. This mechanism ensures that the lookup table is updated immediately after an event is applied to the aggregate. Consequently, if the update to the lookup table fails due to a constraint violation or database error, the entire transaction is rolled back, preventing the system from reaching an inconsistent state.

In practice, this validation logic is often encapsulated within a domain service or a validator interface that is injected directly into the aggregate's command handler. This service interacts with the lookup table repository to verify global invariants before the aggregate state is modified. By separating the lookup logic from the read-model, the system avoids the latency of eventual consistency while maintaining the architectural integrity of the aggregate as a boundary for consistency. This pattern effectively bridges the gap between the isolated nature of individual aggregates and the necessity for global state verification in complex domain models. \parencite{ceelie_set_2020}

\subsubsection*{Events}
\label{sec:axon-events}

Event handlers are methods annotated with \keyw{@EventHandler} which react to occurrences within the app by handling Axon's event messages. Each event handler specifies the types of events it is interested in. When no handler for a given event type exists in the application, the event is ignored. \parencite{axoniq_event_2025}

Axon's \keyw{@EventBus} is the infrastructure mechanism dispatching events to the subscribed event handlers. Event stores offer these functionalities and additionally persist and retrieve published events. \parencite{axoniq_event_2025-1}

Event processors take care of the technical part aspects of event processing. Axon's \keyw{EventBus} implementations support both subscribing and tracking event processors. \parencite{axoniq_event_2025-1} Subscribing event processors subscribe to a message source, which delivers (pushes) events to the processor. The event is then processed in the same thread that published the event. This makes subscribing event processors suitable for real-time updates of models. However, they can only be used to receive current events and do not support event replay. Additionally, as they run on the same thread, they can not be parallelized. \parencite{axoniq_subscribing_2025}

Tracking event processors, which a type of streaming event processors, read (pull) events to be processed from an event source. They run decoupled from the publishing thread, making them parallelizable. These event processors use tracking tokens track their position in the event stream. Tracking tokens can be reset and events can be replayed and reprocessed. Tracking event processors are the default in Axon and recommended for most ES-CQRS use cases. \parencite{axoniq_streaming_2025}

Subscribing event processors can be configured using SpringBoot's \javaname{application.properties} file or through Java configuration classes.

\subsubsection*{Sagas}
\label{sec:sagas}

In Axon, Sagas are long-running, stateful event handlers which not just react to events, but instead manage and coordinate business transactions. For each transaction being managed, one instance of a Saga exists. A Saga, which is a class annotated with \keyw{@Saga} has a lifecycle that is started by a specific event when a method annotated with \keyw{@StartSaga} is executed. The lifecycle may be ended when a method annotated with \keyw{@EndSaga} is executed; or conditionally using \keyw{SagaLifecycle.end()}. A Saga usually has a clear starting point, but may have many different ways for it to end. Each event handling method in a Saga must additionally have the \keyw{@SagaEventHandler} annotation. \parencite{axoniq_saga_2025}

The way Sagas manage business transactions is by sending commands upon receiving events. They can be used when workflows across several aggregates should be implemented; or to handle long-running processes that may span over any amount of time. \parencite{axoniq_saga_2025} For example, the lifecycle of an order, from being processed, to being shipped and paid, is a process that usually takes multiple days. A use case like this is typically implemented using Sagas.

A Saga is associated with one or more association values, which are key-value pairs used to route events to the correct Saga instance. A \keyw{@StartSaga} method together with the \keyw{@SagaEventHandler(associationProperty="aggregateId")} automatically associates the Saga with that identifier. Additional associations can be made programmatically, by calling \keyw{SagaLifecycle.associateWith()}. Any matching events are then routed to the Saga. \parencite{axoniq_saga_2025-1}

For example, a Saga managing an order's lifecycle may be started by an \keyw{@OrderPlaced} event and associated with the \keyw{orderId}. It can then issue a \keyw{CreateInvoiceCommand} using an \keyw{invoiceId} generated inside the event handler. The Saga then associates itself with this ID to be notified of further events regarding this invoice, such as an \keyw{InvoicePaidEvent}.

% TODO Show command and query gateway and illustrate example flow through an Axon application. 

\subsection{Testing}
\label{sec:testing}

To ensure functionality of the applications, unit and integration tests were implemented using various testing libraries like JUnit as the testing platform, \gls{restassured} for making and asserting \gls{http} calls, Mockito for unit testing and ArchUnit for architecture tests. This section describes all mentioned technologies.

JUnit \footnote{\href{https://docs.junit.org/5.11.0/user-guide/index.html}{JUnit 5}} is an open-source testing framework for Java. It offers a structured way of writing tests, driven by lifecycle methods like \texttt{beforeEach} or \texttt{afterAll}. Tests are annotated with \texttt{@Test}. They can also be parametrized and run repeatedly. Results can be asserted using assertion methods like \texttt{assertTrue()}. \parencite{noauthor_junit_nodate}

REST Assured \footnote{\href{https://rest-assured.io/}{REST Assured}} is a Java library that provides a highly fluent \acrshort{dsl} for testing and validating REST APIs in a readable, chainable style. It allows complex assertions to be written inline using \gls{groovy} expressions, making it easy to deeply verify JSON responses beyond simple field checks. \parencite{restassured-documentation}

The below code example shows how one might use a \gls{groovy} expression to find and validate a path in the returned JSON object:

\begin{lstlisting}[caption={Validating JSON path using Rest Assured},captionpos=b]
RestAssured.when()
    // omitted request 
    .then()
    .body(
        "data.grades.find { it.combinedGrade == 0 }.credits", 
        equalTo(0)
    );
\end{lstlisting}

Here, the path \texttt{data.grades} of the returned JSON object is expected to be an array. The array is filtered using a \gls{gpath} with a closure to find the first entry where \texttt{combinedGrade} equals 0. Then, this entry's \texttt{credits} field is extracted and validated using the \texttt{equalTo(0)} matcher.

% TODO Mockito, ArchUnit 

\subsection{SpringBoot Actuator}
\label{sec:actuator}

Spring Boot Actuator \footnote{\href{https://docs.spring.io/spring-boot/reference/actuator/index.html}{SpringBoot Actuator}} is a tool designed to help monitor and manage Spring Boot applications running in a production environment. It provides several built-in features that allow developers to check the status of the application, gather performance data, and track \gls{http} requests. These features can be accessed using either \gls{http} or \acrshort{jmx} (\acrlong{jmx}), which is a standard Java management technology. By using Actuator, developers can quickly see if an application is running correctly without the need to write custom monitoring code.

The most common way to use Actuator is through its "endpoints", which are specific web addresses that provide different types of information. For example, the health endpoint shows whether the application and its connected services, like databases, are functioning correctly, while the metrics endpoint displays detailed data on memory and CPU usage. Beyond the standard options, developers can also create their own custom endpoints or connect the data to external monitoring software to visualize how an application is performing over time.

Actuator can be enabled in a Spring Boot project by including the \javaname{spring-boot-starter-actuator} dependency. \parencite{broadcom_inc_production-ready_2026}

\subsection{Prometheus}

Prometheus \footnote{\href{https://prometheus.io/docs/introduction/overview/}{Prometheus}} is an open-source systems monitoring toolkit that was originally developed at SoundCloud and is now a project of the Cloud Native Computing Foundation. It is primarily used for collecting and storing multidimensional metrics as time-series data, meaning information is recorded with a timestamp and optional key-value pairs called labels. The system is designed for reliability and is capable of scraping data from instrumented jobs and web servers, storing it in a local time-series database, and triggering alerts based on predefined rules when specific thresholds are met. Through its powerful functional query language, PromQL, developers can aggregate and visualize performance data. \parencite{prometheus_authors_prometheus_2026,prometheus-overview-2026}

To collect and export \hyperref[sec:actuator]{Actuator} metrics specifically for Prometheus, the \javaname{micrometer-registry-prometheus} dependency must be included in the classpath. \parencite{vmware_inc_micrometer_nodate} Access to the metrics is granted by including "prometheus" in the list of exposed web endpoints within the application's configuration properties. Once these components are in place, the metrics are automatically formatted for consumption and can be scraped by a Prometheus server. \parencite{broadcom_inc_metrics_2026}

\subsection{Docker}

\gls{docker} \footnote{\href{https://docs.docker.com/}{Docker}} is a platform used for developing and deploying applications. It is designed to separate software from the underlying infrastructure, allowing for faster delivery and consistent environments.

\gls{docker}'s capabilities are centered around the use of containers, which are lightweight and isolated environments. Each container is packaged with all necessary dependencies required for an application to run, ensuring it operates independently of the host system. These workloads can be executed across different environments, such as local computers, data centers, or cloud providers, ensuring high portability. \parencite{what-is-docker}

A \gls{dockerfile} is a text-based document containing a series of instructions for assembling a Docker image. Each command in this file results in the creation of a layer in the image, making the final template efficient and fast to rebuild. These images serve as read-only blueprints from which runnable instances, or containers, are created. \parencite{writing-a-dockerfile}

Docker Compose is a tool used to define and manage applications consisting of multiple containers. A single configuration file is used to specify the services, networks, and volumes required for the entire application stack. The lifecycle of complex applications can be managed with this tool, enabling all associated services to be started, stopped, and coordinated with a single command. \parencite{what-is-docker-compose}

\subsection{k6}
\label{sec:k6}

Grafana k6 \footnote{\href{https://grafana.com/docs/k6/latest/}{Grafana k6}} is an open-source performance testing tool designed to evaluate the reliability and performance of a system. It simulates various traffic patterns, such as constant load, sudden stress spikes, and long-term soak tests, to identify slow response times and system failures during development and continuous integration. Metrics are collected during execution and can be visualized through platforms like Grafana or exported to various data backends for detailed reporting. \parencite{k6-overview}

k6 allows tests to be written in JavaScript, making it accessible and easy to integrate into existing codebases. Every k6 test follows a common structure. The main component is a function that contains the core logic of the test. This function should be the default export of the JavaScript file. It is executed concurrently for each \acrlong{VU} (\acrshort{VU}), which act as independent execution threads to repeatedly apply the test logic. The tests can be enhanced using k6's lifecycle functions, such as a setup function, which is executed only once and may be utilized to insert seed data into the system. The test execution can be configured using an "options" object, where VUs, test duration and performance thresholds can be set. \parencite{k6-write-your-first-test}

% !TeX root = ../main.tex
\chapter{Requirement Analysis}
\label{ch:requirement-analysis}

This thesis aims to provide a fair, quantitative comparison of \acrshort{crud} and \acrshort{cqrs}-\acrshort{es} architectures regarding all three research questions. To achieve this, the architectures should be applied not only to the same domain, but to the exact same requirements.

\section{Functional Requirements}
\label{sec:functional-requirements}

A functional requirement describes a specific behavior that a product must exhibit under specific circumstances. These requirements specify what the system \emph{does} by detailing the capabilities and functions the solution must possess to allow users to perform their tasks~\cite[4]{wiegers_software_2023}. To ensure clarity regarding exactly how the system should behave, functional requirements are often written using patterns that include the keyword "shall," such as "The system shall let the user do something"~\cite[109]{wiegers_software_2023}.

\subsection{Project Description}
\label{sec:project-description}

The applications will implement a course enrollment and grading system which might for example be used in universities. Professors can create courses and lectures which students can enroll to. These lectures can have assignments, which professors enter grades for. Once a lecture is finished, final grades and awarded credits can be calculated. Students are able to view their enrollments, grades and credits.

\subsection{Entities}
\label{sec:entities}

Two types of users exist in the domain: professors and students. Their personal information is not relevant for this thesis, which is why only their first and last name are stored for presentation reasons. The student additionally has a semester.

Professors can create courses. Courses have a name, a description, an amount of credits they yield, a minimum amount of credits required to enroll and can have a set of courses as prerequisites.

Courses are the "blueprints" for lectures. Lectures are the "implementation" of a course for a semester. Each lecture created from a course yields the course's amount of credits and has the requirements specified by the course. Lectures have a lifecycle: they can be in draft state, open for enrollment, in progress, finished or archived. A lecture has a list of time slots and a maximum amount of students that can enroll.

A lecture can have several assessments. Each assessment has a type. The professor can enter grades for a student and an assessment. Grades are integers in the range of 0 to 100. Credits are awarded to a student as soon as they completed all assessments for a lecture with a passing grade (grade higher than 50) and once a lecture's status is set to finished.

\subsection{Business Rules and System Constraints}
\label{sec:business-rules}

Relationships and business rules in this system are deliberately chosen complex, involving many relationships between \hyperref[sec:entities]{entities} and intricate validation rules. This approach was adopted in order to be able to make realistic assumptions about the research question by evaluating a project that closely resembles complex, real-world scenarios.

Based on the domain described in \autoref{sec:project-description}, the following list presents a selection of constraints and rules which are central to the system. As this thesis focuses on an architectural comparison, not every functional requirement going into the application is listed explicitly.

\begin{itemize}
    \item Referential Integrity: The system shall verify the existence of all referenced entities during request handling. Requests involving non-existent entities shall be rejected.
    \item The system shall prevent conflicts such as time slot overlaps.
    \item When a student tries enrolling to a lecture which is already full, they shall be put on a waitlist.
    \item When a student disenrolls from a lecture, the next eligible student (higher semesters are preferred) shall be enrolled.
    \item Actions on a lecture shall only be performed during the appropriate lifecycle state. For example, enrolling shall only be possible during a lifecycle of "open for enrollment". Grades shall only be assigned when the lecture is "finished".
\end{itemize}

\section{Non-functional Requirements}
\label{sec:non-functional-requirements}

A non-functional requirement, often referred to as a \emph{quality attribute}, describes the quality or performance characteristics of a solution~\cite[4]{wiegers_software_2023}. Rather than defining \emph{what} the product does, these requirements focus on \emph{how well it functions}. They establish specific goals or constraints for the design and implementation, such as targets for security, availability, or response time, to ensure the system satisfies user expectations~\cite[67]{wiegers_software_2023}.

The following non-functional requirements (quality attributes) are defined.

\subsection{Service Level Objectives}
\label{sec:slo}

While \glspl{sla} are agreements with users regarding uptime and performance, \glspl{slo} are the technical targets used by engineers to meet those requirements~\cite[63,65]{beyer_site_2016}. This thesis attempts to define realistic \acrshortpl{slo} to establish a "breaking point" for each architecture.

Following \textcite[135]{nielsen_usability_1993}, a response time of 100ms is the threshold for human perception of "instant" feedback. This serves as the baseline for the following targets:

\begin{enumerate}[label={SLO \arabic*}, ref={SLO \arabic*}, leftmargin=*]
    \item \textbf{Latency \gls{slo}}: All endpoints shall maintain a client-side P95 latency of $\le$100ms to ensure the system feels "instant" for 95\% of requests. \label{slo-latency}
    \item \textbf{Freshness \gls{slo}}: In the Event Sourcing implementation, the asynchronous nature of projections introduces a lag. All writes shall be reflected in the read-model within $\le$100ms to ensure eventual consistency remains imperceptible. While the read-side is eventually consistent, the command-side (write-model) shall maintain immediate consistency to ensure business rules are validated against the latest state. \label{slo-freshness}
    \item \textbf{Reliability \gls{slo}}: Both implementations shall maintain a failure rate of <0.1\% under stress. \label{slo-reliability}
\end{enumerate}

\subsection{Auditing}
\label{sec:req-auditing}

Both systems need to be fully auditable. Every change to an entity must be reflected as a historical record. Historic states should be accurately reconstructible.

\subsection{Observability}
\label{sec:req-observability}

The systems must expose observability endpoints. These should be able to present information about the system internals. Precisely, CPU and threadpool usage, database connections and size of data stores should be available. Additionally, response times should be exposed as histograms.

\subsection{Consistency}
\label{sec:req-consistency}

To ensure the integrity of the operations, the following consistency requirements apply:

\begin{itemize}
    \item \textbf{Write Consistency}: Both architectures shall provide immediate (strong) consistency for write operations. This ensures that any command (e.g., enrolling a student) is validated not only against the individual Aggregate's / entity's state, but also against set-based invariants and global constraints. This prevents violations of business rules that span multiple entities or require a global view of the system state.
    \item \textbf{Read Consistency}: The \acrshort{crud} implementation shall provide immediate consistency for reads. The \acrshort{es}-\acrshort{cqrs} implementation may use eventual consistency for its read-models according to the \hyperref[slo-freshness]{Freshness SLO}.
\end{itemize}

\subsection{Contract}
\label{sec:contract}

Both implementations must follow the same contract regarding endpoints, request and response schemas and state transitions. To ensure this, an extensive test suite shall be set up. While the internals of the implementations will be vastly different architecturally, they will both have the same public \gls{api}, making it possible to send requests and verify the responses. Therefore, one test suite shall be developed which can be executed on both applications. The test suite should include integration tests for all \gls{api} endpoints covering both regular and edge-case (error) scenarios to ensure that both implementations behave identically.


% !TeX root = ../main.tex
\chapter{Implementation}

After defining functional and non-functional requirements, the two applications can be implemented. This will be detailed in this chapter. After describing the utilized technologies, the contract tests ran on both applications are described. Next, implementation details of \acrshort{crud} and \acrshort{es}-\acrshort{cqrs} are given. Finally, it is presented how the load tests were developed.

\section{Endpoints}

Table \autoref{table:endpoints} presents a feature matrix, mapping \acrshort{http} endpoints to their functionality. As this thesis focuses not on the functionality of an application, but instead an architectural comparison, not all implemented endpoints are listed.

\begin{table}[htp!]
    \small
    \centering
    \begin{tabularx}{\linewidth}{lXl}
        \toprule
        \textbf{Endpoint}    & \textbf{Description}                  & \textbf{Response} \\ \midrule
        GET /lectures        & Get lectures a student is enrolled in & 200               \\
        \addlinespace
        POST /courses/create & Used by professors to create a course & 201               \\
        \bottomrule
    \end{tabularx}
    \caption{Selected endpoints implemented in the applications}
    \label{table:endpoints}
\end{table}

\section{Technologies}
\label{sec:technologies}

This section describes all technologies used for the implementation and evaluation of the two applications.

\subsection{SpringBoot}

SpringBoot \footnote{\href{https://spring.io/projects/spring-boot}{SpringBoot}} is an open-source, opinionated framework for developing enterprise Java applications. It is based on Spring Framework \footnote{\href{https://spring.io/projects/spring-framework}{Spring Framework}}, which is a platform aiming to make Java development "quicker, easier, and safer for everybody" \parencite{broadcom_inc_why_2026}. At Spring Framework's core is the Inversion of Control (IoC) container. The objects managed by this container are referred to as \textit{Beans}. While the term originates from the JavaBeans specification, a standard for creating reusable software components, Spring extends this concept by taking full responsibility for the lifecycle and configuration of these objects \parencite[Chapter~1.1]{walls_spring_2016}. Instead of a developer manually instantiating classes using the \texttt{new} operator, the container "injects" required dependencies at runtime. This process is known as Dependency Injection. \parencite[Chapter~1]{deinum_spring_2023}. Spring offers support for several programming paradigms: reactive, event-driven, microservices and serverless. \parencite{broadcom_inc_why_2026}

SpringBoot builds on top of the Spring platform by applying a "convention-over-configuration" approach, intended to minimize the need for configuration. In a 2023 survey by JetBrains, SpringBoot was the most popular choice of web framework. \parencite{jetbrains_java_2023}

Spring Boot starters are specialized dependency descriptors designed to simplify dependency management by aggregating commonly used libraries into feature-defined packages. Rather than requiring developers to manually identify and maintain a list of individual group IDs, artifact IDs, and compatible version numbers for every necessary library, starters use transitive dependency resolution to pull in all required components under a single entry. To quickly bootstrap a web application, a developer can simply add the \javaname{spring-boot-starter-web} dependency to their Maven or Gradle build file. By requesting this specific functionality, Spring Boot automatically includes essential dependencies such as Spring MVC, Jackson for JSON processing, and an embedded Tomcat server, ensuring that all included libraries have been tested together for compatibility. This approach shifts the developer's focus from managing individual JAR files to simply defining the high-level capabilities the application requires, minimizing configuration overhead and reducing risk of version mismatches. \parencite[Chapter~1.1.2]{walls_spring_2016}
% TODO code example for IoC container / dependency injection 

\subsection{PostgreSQL}
\label{sec:postgresql}

PostgreSQL \footnote{\href{https://www.postgresql.org/}{PostgreSQL}} is an open-source relational database system which has been in active development for over 35 years. Thanks to its reliability, robustness and performance, it has a strong earned reputation. \parencite{postgresql_global_development_group_postgresql_2026} PostgreSQL is designed for a wide range of workloads and can handle many tasks thanks to its extensibility and large suite of extensions, such as the popular PostGIS extension for storing and querying geospatial data. \parencite{postgis_psc_postgis_2023}

\subsection{JPA}
\label{sec:jpa}

\acrfull{jpa} \footnote{\href{https://jakarta.ee/specifications/persistence/}{JPA}}, formerly Java Persistence \gls{api} is a Java specification which provides a mechanism for managing persistence and object-relational mapping (\acrshort{orm}). \glspl{orm} act as a bridge between the relational world of SQL databases and the object-oriented world of Java.

Instead of writing SQL to create the database schema, entities can be described using special Java classes, supported by annotations, which can be mapped to an SQL schema. \acrshort{jpa} allows querying the database for these entities in a type-safe way by providing a range of helpful query methods on JPA repositories, for example \texttt{findAll()} or \texttt{findById(UUID id)}. This removes the need to write "low-level", database-specific SQL for basic \acrshort{crud} operations. Complex data retrieval is also possible with \acrshort{jpa} using the \acrfull{jpql}, which is an object-oriented, database-agnostic query language.

When using \acrshort{jpa} with SpringBoot by including the \javaname{spring-boot-starter-data-jpa} dependency, \emph{Hibernate} \footnote{\href{https://hibernate.org/orm/}{Hibernate}} is used as implementation of the \acrshort{jpa} standard. \parencite[Chapter~1]{bauer_java_2016}

\subsection{Hibernate Envers (TODO)}
\label{sec:envers}

Present Hibernate Envers here.

\subsection{Jackson}

Jackson \footnote{\href{https://github.com/FasterXML/jackson}{Jackson}} is a high-performance, feature-rich \acrshort{json} processing library for Java. It is the default \acrshort{json} library used within the Spring Boot ecosystem. Its primary purpose is to provide a seamless bridge between Java objects and JSON data through three main processing models: the Streaming API for incremental parsing, the Tree Model for a flexible node-based representation, and the most commonly used Data Binding module. This data binding capability allows developers to automatically convert (\emph{marshal}) Java \glspl{pojo} into JSON and vice versa (\emph{unmarshal}) with minimal configuration. Beyond its speed and efficiency, Jackson is highly extensible, offering modules to handle complex Java types like Java 8 Date/Time and Optional classes. Jackson also supports various other data formats such as XML, YAML and CSV. \parencite{oracle_jackson_nodate, fasterxml_jackson_2025}

\subsection{Axon}
\label{sec:axon}

Axon Framework \footnote{\href{https://www.axoniq.io/framework}{Axon Framework}} is an open-source Java framework for building event-driven applications. Following the \acrshort{cqrs} and event-sourcing pattern, Commands, Events and Queries are the three core message types any Axon application is centered around. Commands are used to describe an intent to change the application's state. Events communicate a change that happened in the application. Queries are used to request information from the application.

Axon also supports \acrlong{ddd} by providing tools to manage entities and domain logic. \parencite{axoniq_introduction_2025,axoniq_messaging_2025}

Axon Server \footnote{\href{https://www.axoniq.io/server}{Axon Server}} is a platform designed specifically for event-driven systems. It functions as both a high-performance Event Store and a dedicated Message Router for commands, queries, and events. By bundling these responsibilities into a single service, Axon Server replaces the need for separate infrastructures such as a relational database for events and a message broker like Kafka or RabbitMQ for communication. Axon Server is designed to seamlessly integrate with Axon Framework. When using the Axon Server Connector, the application automatically finds and connects to the Axon Server. It is then possible to use the Axon server without further configuration. \parencite{axoniq_introduction_2025-1,axoniq_axon_2025} % TODO book source 

\subsubsection*{Command dispatching}

Command dispatching is the starting point for handling a command message in Axon. Axon handles commands by routing them to the appropriate command handler. The command dispatching infrastructure can be interacted with using the low-level \keyw{CommandBus} and a more convenient \keyw{CommandGateway} which is a wrapper around the \keyw{CommandBus}.

\keyw{CommandBus} is the infrastructure mechanism responsible for finding and invoking the correct command handler. At most one handler is invoked for each command; if no handler is found, an exception is thrown.

Using \keyw{CommandGateway} simplifies command dispatching by hiding the manual creation of \keyw{CommandMessages}. The gateway offers two main methods for synchronous and asynchronous patterns. The \keyw{send} method returns a \keyw{CompletableFuture}, which is an asynchronous mechanism in Java. If the thread needs to wait for the command result, the \keyw{sendAndWait} method can be used.

In general, a handled command returns \keyw{null}, if handling was successful. Otherwise, a \keyw{CommandExecutionException} is propagated to the caller. While returning values from a command handler is not forbidden, it is used sparsely as it contradicts with CQRS semantics. One exception: command handlers which \emph{create} an aggregate typically return the aggregate identifier. \parencite{axoniq_command_2025,axoniq_infrastructure_2025}

\subsubsection*{Query Handling}
Before a query is handled, Axon dispatches it through its messaging infrastructure. Just like the command infrastructure, Axon offers a low-level \keyw{QueryBus} which requires manual query message creation and a more high-level \keyw{QueryGateway}.

In contrast to command handling, multiple query handlers can be invoked for a given query. When dispatching a query, callers can decide whether they want a single result or results from all handlers. When no query handler is found, an exception is thrown.

The \keyw{QueryGateway} includes different dispatching methods. For regular "point-to-point" queries, the \keyw{query} method can be used. Subscription queries are queries where callers expect an initial result and continuous updates as data changes. These queries work well with reactive programming. For large result sets, streaming queries should be used. The response returned by the query handler is split into chunks and streamed back to the caller.
All query methods are asynchronous by nature and return Java's \keyw{CompletableFuture}. \parencite{axoniq_query_2025}

\subsubsection*{Aggregates}
\label{sec:aggregates}

An aggregate is a core concept of \acrfull{ddd}. In Axon, an aggregate defines a consistency boundary around domain state and encapsulates business logic. Aggregates are the primary place where domain invariants are enforced and where commands that intend to change domain state are handled.

Aggregates define command handlers using methods or constructors annotated with \keyw{@CommandHandler}. These handlers receive commands and decide whether they are valid according to domain rules. If a command is accepted, the aggregate emits one or more domain events describing \emph{what} happened. Command handlers are responsible only for decision-making; they must not directly mutate the aggregate’s state. Instead, all state changes must occur as a result of applying events.

Every aggregate is typically annotated with \keyw{@Aggregate} and must declare exactly one field annotated with \keyw{@AggregateIdentifier}. This identifier uniquely identifies the aggregate instance. Axon uses it to route incoming commands to the correct aggregate and to load the corresponding event stream when rebuilding aggregate state.

By default, Axon uses event-sourced aggregates. This means that aggregates are not persisted as a snapshot of their fields. Instead, their current state is reconstructed by replaying all previously stored events. Methods annotated with \keyw{@EventSourcingHandler} are called by Axon during this replay process to update the aggregate’s internal state based on event data. Since events represent facts that already occurred, event sourcing handlers must not contain business logic or make decisions.

Axon also supports multi-entity aggregates. In this model, an aggregate may contain child entities that participate in command handling. Such entities are registered using \keyw{@AggregateMember}, and each entity must define a unique identifier annotated with \keyw{@EntityId}. Based on this identifier, Axon is able to route commands to the correct entity instance within the aggregate. \parencite{axoniq_multi-entity_2025}

\subsubsection*{External Command Handlers}
\label{sec:external-command-handlers}

Often, command handling functions are placed directly inside the aggregate. However, this is not required and in some cases it may not be desirable or possible to directly route a command to an aggregate. Thus, any object can be used as a command handler by including methods annotated with \keyw{@CommandHandler}. One instance of this command handling object will be responsible for handling \emph{all} commands of the command types it declares in its methods.

In these external command handlers, aggregates can be loaded manually from Axon's repositories using the aggregate's ID. Afterward, the \keyw{execute} function can be used to execute commands on the loaded aggregate. \parencite{axoniq_command_2025-1}

\subsubsection*{Set-based validation}
\label{sec:set-based-validation}

When receiving a command, aggregates handle it by validating their internal state inside command handlers and either rejecting the command or publishing an event. However, validation across a set of aggregates, called "set-based validation", is not possible inside a single aggregate. A business requirement like "Usernames must be unique" can only be implemented using set-based validation, as the entire set of aggregates must be inspected before making a decision.

Set-based implementation in Axon can be implemented by using lookup tables. This approach utilizes a dedicated command-side projection, often referred to as a lookup projector, to maintain a specialized view of the system state. While projectors are typically associated with the read-side of a \acrshort{cqrs} architecture, a lookup projector is specifically designed to support the command side. It maintains a highly optimized and consistent dataset, such as a registry of unique identifiers, which can be queried during the validation phase of a command.

To ensure that this lookup table remains synchronized and provides the necessary consistency for validation, Axon employs subscribing event processors, which are described in \autoref{sec:axon-events}. Unlike tracking event processors which operate asynchronously and introduce eventual consistency, subscribing event processors execute within the same thread and transaction as the event publication. This mechanism ensures that the lookup table is updated immediately after an event is applied to the aggregate. Consequently, if the update to the lookup table fails due to a constraint violation or database error, the entire transaction is rolled back, preventing the system from reaching an inconsistent state.

In practice, this validation logic is often encapsulated within a domain service or a validator interface that is injected directly into the aggregate's command handler. This service interacts with the lookup table repository to verify global invariants before the aggregate state is modified. By separating the lookup logic from the read-model, the system avoids the latency of eventual consistency while maintaining the architectural integrity of the aggregate as a boundary for consistency. This pattern effectively bridges the gap between the isolated nature of individual aggregates and the necessity for global state verification in complex domain models. \parencite{ceelie_set_2020}

\subsubsection*{Events}
\label{sec:axon-events}

Event handlers are methods annotated with \keyw{@EventHandler} which react to occurrences within the app by handling Axon's event messages. Each event handler specifies the types of events it is interested in. When no handler for a given event type exists in the application, the event is ignored. \parencite{axoniq_event_2025}

Axon's \keyw{@EventBus} is the infrastructure mechanism dispatching events to the subscribed event handlers. Event stores offer these functionalities and additionally persist and retrieve published events. \parencite{axoniq_event_2025-1}

Event processors take care of the technical part aspects of event processing. Axon's \keyw{EventBus} implementations support both subscribing and tracking event processors. \parencite{axoniq_event_2025-1} Subscribing event processors subscribe to a message source, which delivers (pushes) events to the processor. The event is then processed in the same thread that published the event. This makes subscribing event processors suitable for real-time updates of models. However, they can only be used to receive current events and do not support event replay. Additionally, as they run on the same thread, they can not be parallelized. \parencite{axoniq_subscribing_2025}

Tracking event processors, which a type of streaming event processors, read (pull) events to be processed from an event source. They run decoupled from the publishing thread, making them parallelizable. These event processors use tracking tokens track their position in the event stream. Tracking tokens can be reset and events can be replayed and reprocessed. Tracking event processors are the default in Axon and recommended for most ES-CQRS use cases. \parencite{axoniq_streaming_2025}

Subscribing event processors can be configured using SpringBoot's \javaname{application.properties} file or through Java configuration classes.

\subsubsection*{Sagas}
\label{sec:sagas}

In Axon, Sagas are long-running, stateful event handlers which not just react to events, but instead manage and coordinate business transactions. For each transaction being managed, one instance of a Saga exists. A Saga, which is a class annotated with \keyw{@Saga} has a lifecycle that is started by a specific event when a method annotated with \keyw{@StartSaga} is executed. The lifecycle may be ended when a method annotated with \keyw{@EndSaga} is executed; or conditionally using \keyw{SagaLifecycle.end()}. A Saga usually has a clear starting point, but may have many different ways for it to end. Each event handling method in a Saga must additionally have the \keyw{@SagaEventHandler} annotation. \parencite{axoniq_saga_2025}

The way Sagas manage business transactions is by sending commands upon receiving events. They can be used when workflows across several aggregates should be implemented; or to handle long-running processes that may span over any amount of time. \parencite{axoniq_saga_2025} For example, the lifecycle of an order, from being processed, to being shipped and paid, is a process that usually takes multiple days. A use case like this is typically implemented using Sagas.

A Saga is associated with one or more association values, which are key-value pairs used to route events to the correct Saga instance. A \keyw{@StartSaga} method together with the \keyw{@SagaEventHandler(associationProperty="aggregateId")} automatically associates the Saga with that identifier. Additional associations can be made programmatically, by calling \keyw{SagaLifecycle.associateWith()}. Any matching events are then routed to the Saga. \parencite{axoniq_saga_2025-1}

For example, a Saga managing an order's lifecycle may be started by an \keyw{@OrderPlaced} event and associated with the \keyw{orderId}. It can then issue a \keyw{CreateInvoiceCommand} using an \keyw{invoiceId} generated inside the event handler. The Saga then associates itself with this ID to be notified of further events regarding this invoice, such as an \keyw{InvoicePaidEvent}.

% TODO Show command and query gateway and illustrate example flow through an Axon application. 

\subsection{Testing}
\label{sec:testing}

To ensure functionality of the applications, unit and integration tests were implemented using various testing libraries like JUnit as the testing platform, \gls{restassured} for making and asserting \gls{http} calls, Mockito for unit testing and ArchUnit for architecture tests. This section describes all mentioned technologies.

JUnit \footnote{\href{https://docs.junit.org/5.11.0/user-guide/index.html}{JUnit 5}} is an open-source testing framework for Java. It offers a structured way of writing tests, driven by lifecycle methods like \texttt{beforeEach} or \texttt{afterAll}. Tests are annotated with \texttt{@Test}. They can also be parametrized and run repeatedly. Results can be asserted using assertion methods like \texttt{assertTrue()}. \parencite{noauthor_junit_nodate}

REST Assured \footnote{\href{https://rest-assured.io/}{REST Assured}} is a Java library that provides a highly fluent \acrshort{dsl} for testing and validating REST APIs in a readable, chainable style. It allows complex assertions to be written inline using \gls{groovy} expressions, making it easy to deeply verify JSON responses beyond simple field checks. \parencite{restassured-documentation}

The below code example shows how one might use a \gls{groovy} expression to find and validate a path in the returned JSON object:

\begin{lstlisting}[caption={Validating JSON path using Rest Assured},captionpos=b]
RestAssured.when()
    // omitted request 
    .then()
    .body(
        "data.grades.find { it.combinedGrade == 0 }.credits", 
        equalTo(0)
    );
\end{lstlisting}

Here, the path \texttt{data.grades} of the returned JSON object is expected to be an array. The array is filtered using a \gls{gpath} with a closure to find the first entry where \texttt{combinedGrade} equals 0. Then, this entry's \texttt{credits} field is extracted and validated using the \texttt{equalTo(0)} matcher.

% TODO Mockito, ArchUnit 

\subsection{SpringBoot Actuator}
\label{sec:actuator}

Spring Boot Actuator \footnote{\href{https://docs.spring.io/spring-boot/reference/actuator/index.html}{SpringBoot Actuator}} is a tool designed to help monitor and manage Spring Boot applications running in a production environment. It provides several built-in features that allow developers to check the status of the application, gather performance data, and track \gls{http} requests. These features can be accessed using either \gls{http} or \acrshort{jmx} (\acrlong{jmx}), which is a standard Java management technology. By using Actuator, developers can quickly see if an application is running correctly without the need to write custom monitoring code.

The most common way to use Actuator is through its "endpoints", which are specific web addresses that provide different types of information. For example, the health endpoint shows whether the application and its connected services, like databases, are functioning correctly, while the metrics endpoint displays detailed data on memory and CPU usage. Beyond the standard options, developers can also create their own custom endpoints or connect the data to external monitoring software to visualize how an application is performing over time.

Actuator can be enabled in a Spring Boot project by including the \javaname{spring-boot-starter-actuator} dependency. \parencite{broadcom_inc_production-ready_2026}

\subsection{Prometheus}

Prometheus \footnote{\href{https://prometheus.io/docs/introduction/overview/}{Prometheus}} is an open-source systems monitoring toolkit that was originally developed at SoundCloud and is now a project of the Cloud Native Computing Foundation. It is primarily used for collecting and storing multidimensional metrics as time-series data, meaning information is recorded with a timestamp and optional key-value pairs called labels. The system is designed for reliability and is capable of scraping data from instrumented jobs and web servers, storing it in a local time-series database, and triggering alerts based on predefined rules when specific thresholds are met. Through its powerful functional query language, PromQL, developers can aggregate and visualize performance data. \parencite{prometheus_authors_prometheus_2026,prometheus-overview-2026}

To collect and export \hyperref[sec:actuator]{Actuator} metrics specifically for Prometheus, the \javaname{micrometer-registry-prometheus} dependency must be included in the classpath. \parencite{vmware_inc_micrometer_nodate} Access to the metrics is granted by including "prometheus" in the list of exposed web endpoints within the application's configuration properties. Once these components are in place, the metrics are automatically formatted for consumption and can be scraped by a Prometheus server. \parencite{broadcom_inc_metrics_2026}

\subsection{Docker}

\gls{docker} \footnote{\href{https://docs.docker.com/}{Docker}} is a platform used for developing and deploying applications. It is designed to separate software from the underlying infrastructure, allowing for faster delivery and consistent environments.

\gls{docker}'s capabilities are centered around the use of containers, which are lightweight and isolated environments. Each container is packaged with all necessary dependencies required for an application to run, ensuring it operates independently of the host system. These workloads can be executed across different environments, such as local computers, data centers, or cloud providers, ensuring high portability. \parencite{what-is-docker}

A \gls{dockerfile} is a text-based document containing a series of instructions for assembling a Docker image. Each command in this file results in the creation of a layer in the image, making the final template efficient and fast to rebuild. These images serve as read-only blueprints from which runnable instances, or containers, are created. \parencite{writing-a-dockerfile}

Docker Compose is a tool used to define and manage applications consisting of multiple containers. A single configuration file is used to specify the services, networks, and volumes required for the entire application stack. The lifecycle of complex applications can be managed with this tool, enabling all associated services to be started, stopped, and coordinated with a single command. \parencite{what-is-docker-compose}

\subsection{k6}
\label{sec:k6}

Grafana k6 \footnote{\href{https://grafana.com/docs/k6/latest/}{Grafana k6}} is an open-source performance testing tool designed to evaluate the reliability and performance of a system. It simulates various traffic patterns, such as constant load, sudden stress spikes, and long-term soak tests, to identify slow response times and system failures during development and continuous integration. Metrics are collected during execution and can be visualized through platforms like Grafana or exported to various data backends for detailed reporting. \parencite{k6-overview}

k6 allows tests to be written in JavaScript, making it accessible and easy to integrate into existing codebases. Every k6 test follows a common structure. The main component is a function that contains the core logic of the test. This function should be the default export of the JavaScript file. It is executed concurrently for each \acrlong{VU} (\acrshort{VU}), which act as independent execution threads to repeatedly apply the test logic. The tests can be enhanced using k6's lifecycle functions, such as a setup function, which is executed only once and may be utilized to insert seed data into the system. The test execution can be configured using an "options" object, where VUs, test duration and performance thresholds can be set. \parencite{k6-write-your-first-test}

\section{Contract Test Implementation}
\label{sec:contract-test-implementation}

The \hyperref[sec:contract-tests]{contract tests} are implemented in a separate maven module called \texttt{test-{\allowbreak}suite}\footnote{\javaname{test-suite/src/test/java/karsch.lukas}}. The test classes use the \texttt{JUnit 5} testing framework and \texttt{REST Assured} to send and assert \gls{http} requests. A basic test might look like this:

\begin{lstlisting}
%%@DisplayName%%("GET /lectures should return 200 and include 2 dates")
%%@Test%%
void getLectureDetails_shouldReturn200_returnTwoDates() {
    // First, create seed data
    var lectureSeedData = createLectureSeedData();

    RestAssured.given()
            .when() 
            .get("/lectures/{lectureId}", lectureSeedData.lectureId())
            .then()
            .statusCode(200)
            .body("data.dates", hasSize(2));
}
\end{lstlisting}
{
\captionof{lstlisting}[Contract test example]{Contract test example; adapted from \javaname{test-suite/src/test/karsch.lukas.lectures.AbstractLecturesE2ETest}}
\label{lst:e2e-test}
}

All contract tests follow a consistent pattern as shown in \autoref{lst:e2e-test}. First, a test method is annotated with \javaname{@DisplayName} to provide a descriptive, human-readable name. The test method itself is precisely named after the behavior it asserts. In the example above, the test verifies that the response status code is \javaname{200} and that the response body contains a field called \javaname{dates} consisting of an array of size two.

Before making these assertions, each test creates "seed data". Seed data is prerequisite data that must exist on the system under test for the execution to be valid. For instance, a professor, a course, and a lecture must be created before the endpoint to \javaname{GET} that specific lecture can be tested. Tests that assert invariants, such as the business rule preventing lecture from having overlapping timeslots, typically set the system time via a Spring Boot Actuator endpoint first.

Once the prerequisites are met, the request is executed and assertions are made using \gls{restassured}. The \texttt{given()} block sets up the request requirements like headers, parameters, or body content; the \texttt{when()} block defines the action, such as the \gls{http} method (GET, POST) and the endpoint URL. Finally, the \texttt{then()} block is used to verify the response, allowing the developer to assert status codes and validate the data returned in the response body.

The test classes in \texttt{test-suite} are all \texttt{abstract}, meaning they can not be run directly. Instead, they are intended to be subclassed by the modules implementing the concrete applications (\texttt{impl-crud} \& \texttt{impl-es-cqrs}). The subclasses must implement a set of abstract methods which are implementation specific, for example a method to reset the database in between each test, a method to set the application's time and methods to create seed data for tests.

Necessary infrastructure is spun up by the subclasses using \glspl{testcontainer}. \glspl{testcontainer} is a way to declare infrastructure dependencies as code and is an open-source library available for many programming languages. \parencite{testcontainers-homepage}

\begin{lstlisting}
%%@TestConfiguration%%
public class PostgresTestcontainerConfiguration {
    %%@Bean%%
    %%@ServiceConnection%%
    %%@RestartScope%%
    PostgreSQLContainer<?> postgreSQLContainer() {
        return new PostgreSQLContainer<>(
                DockerImageName.parse("postgres:latest"));
    }
}
\end{lstlisting}
{
\captionof{lstlisting}[\keyw{PostgresTestcontainerConfiguration}]{\javaname{impl-crud/src/test/karsch.lukas.}\keyw{PostgresTestcontainerConfiguration}}
\label{lst:testcontainer-configuration}
}

\autoref{lst:testcontainer-configuration} starts a \hyperref[sec:postgresql]{PostgreSQL} container using the latest available image. \texttt{@ServiceConnection} makes sure the Spring application can connect to the container. This configuration can then be imported as shown in \autoref{lst:import-testcontainer}.

\begin{lstlisting}
%%@SpringBootTest%%
%%@Import%%(PostgresTestcontainerConfiguration.class)
public class CrudLecturesE2ETest extends AbstractLecturesE2ETest { }
\end{lstlisting}
{
\captionof{lstlisting}[\keyw{CrudLecturesE2ETest}]{\javaname{impl-crud/src/test/karsch.lukas.e2e.lectures.CrudLecturesE2ETest}}
\label{lst:import-testcontainer}
}

\section{CRUD implementation}

% TODO explain architecture / layout 

This section presents the relevant aspects of the CRUD implementation\footnote{\javaname{impl-crud/src/main/java/karsch.lukas}}, mainly focusing on relational modeling using \hyperref[sec:jpa]{\acrshort{jpa}} and the audit log implementation.

\subsection{Relational Modeling}

The CRUD implementation uses a \hyperref[sec:crud-architecture]{normalized database} in the Third Normal Form. TODO!! this still shows auditing class. Create a new diagram; also note that Envers' created auditing tables are not shown here.

\begin{figure}[h]
    \includegraphics[width=\textwidth, inner]{../vault/Thesis/images/CRUD_ER_Diagram_3.png}
    \caption{Entity Relationship Diagram for the CRUD App}
    \label{fig:crud-er-diagram}
\end{figure}

Figure \ref{fig:crud-er-diagram} shows the Entity Relationship Diagram for the CRUD app. It includes nine entities and a value object for the app's relational database schema. Each box corresponds to an entity or value object, with the bold text being the name. Below the table's name, all attributes of the entity are listed with their type and name.

Arrows represent an association. The numbers at the end of the arrows convey the multiplicity. An arrow pointing in only one direction stands for a unidirectional association, while an arrow pointing in both directions conveys a bidirectional association. For example, an arrow pointing between entity \texttt{A} and entity \texttt{B} like so: $1 \longleftrightarrow 0..1$ shows that one \texttt{A} can be associated with any number of \texttt{B}'s, and a \texttt{B} is always associated with exactly one \texttt{A}. % TODO kinda unreadable 

Arrows with a filled diamond represent a composition. Compositions are used when an entity has a reference to a value object. This value object has no identity and is directly embedded into the entity. The only value object in figure \ref{fig:crud-er-diagram} is the \keyw{TimeSlotValueObject}.

In the app's ER diagram, the \keyw{LectureEntity} serves as core of the schema, having several key associations. The $0..* \longrightarrow 1$ association to \keyw{CourseEntity} shows that many lectures can be created from a course and a lecture is always associated with a course. The $0..* \longrightarrow  1$ association to \keyw{ProfessorEntity} shows that a professor can hold many lectures (or none), and that a lecture is always associated with a professor. From the lecture's side, these relationships are called "Many to One" relationships.

\keyw{LectureEntity} also has "One to Many" relationships to \keyw{LectureWaitlistEntryEntity}, \keyw{EnrollmentEntity} and \keyw{LectureAssessmentEntity}. \keyw{LectureWaitlistEntryEntity} is a table which stores students who are waitlisted for a lecture. It is effectively a join table (with one extra column to track when the student was waitlisted) and represents a Many to Many relationship between lectures and students. The same applies to \keyw{EnrollmentEntity} which is a table storing which students are enrolled to which lecture. \keyw{LectureAssessmentEntity} represents the fact that a lecture can have many assessments (which may be an exam, a paper or a project). Each assessment in turn has many \keyw{AssessmentGradeEntity}s associated with it. This table stores which student scored which grade on an assessment. % TODO fix spacing 

These entities are implemented using SpringBoot's \acrshort{jpa} integration. For example, an entity with a "One to Many" relationship can be implemented as presented in \autoref{lst:simple-one-to-many}.

\begin{lstlisting}[caption={Simple JPA entity with a "One to Many" relationship},captionpos=b,label={lst:simple-one-to-many}]
%%@Entity%% 
class LectureEntity {
    %%@Id%% 
    private UUID id; 

    %%@OneToMany(fetch=FetchType.LAZY)%%
    private List<EnrollmentEntity> enrollments; 
}
\end{lstlisting}

The \javaname{@Entity} annotation informs \acrshort{jpa} that the class should be mapped to a database table. If the schema generation feature is enabled, \acrshort{jpa} automatically creates a table structure that mirrors the class definition. In production environments where this feature is typically disabled, developers must provide SQL scripts to manually define the expected structure. This is commonly achieved either by including a basic initialization script or by utilizing dedicated database migration tools such as Flyway or Liquibase to manage versioned schema changes.

Each entity must include a field annotated with \javaname{@Id}, which serves as the unique primary key for the corresponding database record.

The \keyw{@OneToMany} annotation defines a relational link between two entities. While the collection is accessed in Java as a standard list via \keyw{lecture.getEnrollments()}, \acrshort{jpa} manages this behind the scenes using a foreign key relationship. The \texttt{fetch} parameter determines when this data is retrieved: \texttt{LAZY} loading defers the database query until the collection is explicitly accessed in the code, whereas \texttt{EAGER} loading fetches the related entities immediately alongside the parent object.

\subsection{Audit Log implementation}
\label{sec:audit-log-implementation}

There are several strategies to implement an audit log, each with its own trade-offs:

\begin{enumerate}
    \item \textbf{Manual Logging}: Developers explicitly call a logging service in every service method that modifies data. While simple, this can lead to code duplication and is prone to human error, such as developers forgetting to add a log statement. A simple code example is presented in \autoref{lst:audit-log-code-example}.

          \begin{lstlisting}[caption={Code example for manual audit logging},captionpos=b,label={lst:audit-log-code-example}]
public void updatePhoneNumber(User user, int newNumber) {
    logChange(Date.now(), user, user.getPhoneNumber(), newNumber, "UserRequestedNumberChange");
    user.setPhoneNumber(newNumber);
}

void logChange(
    Date date, User user, Object oldValue, Object newValue, String context
) {
    LogEntry logEntry = new LogEntry(date, user, oldValue, newValue);
    logRepository.persist(logEntry);
}
\end{lstlisting}
    \item \textbf{Database Triggers or Stored Procedures} can capture changes automatically and directly on the database. This guarantees that no change is missed, even if made outside the application. \textcite[515]{ingram_design_2009} mentions that database triggers run on a "per-record" basis, meaning the logic is run for each changed record individually. This may lead to degraded performance during batch operations, which is why stored procedures should be preferred over triggers for auditing concerns. It is also worth noting that this approach ties the auditing logic to a specific database, making it less portable.
    \item \textbf{JPA Entity Listeners}: JPA's lifecycle events (\texttt{@PrePersist}, \texttt{@PreUpdate}, etc.) can be used to intercept changes. Inside event handling functions designed for those events, it is possible to capture the changes and persist them in separate auditing tables. This approach is database-independent and keeps the logic within the Java application, allowing access to application internals like beans and Spring's security context. In full-grade applications built using Spring Security, the security context lets developers access the current user, making it possible to attach them to the new audit log entry. Additional context can also be added through thread-local or request-scoped variables. \parencite[Section 13.2]{bauer_java_2016} \label{item:jpa-entity-listener}
    \item  \textbf{Hibernate Envers} is an auditing solution for JPA-based applications which automatically versions entities by using the concept of revisions. Envers creates an auditing table for each entity. This table stores historical data whenever a transaction is committed. It builds on top of JPA entity listeners and avoids the need for developers to build a custom auditing solution. Custom revision entities and change listeners can be implemented to capture additional context. \parencite{hibernate_envers_nodate} (TODO: is it true that Envers builds on JPA listeners?)
    \item TODO: talk about CDC (change data capture) here or in \autoref{sec:audit-log}
\end{enumerate}

\subsection{Audit Log with Hibernate Envers}
\label{sec:chosen-implementation-hibernate-envers}

The audit log in the developed CRUD application is implemented using Hibernate Envers. This solution was chosen because it seamlessly integrates with existing JPA entities to manage historical versions of data in dedicated audit tables.

\subsubsection*{Enabling Auditing on Entities}

To track changes for a specific entity, it must be annotated with \keyw{@Audited}. In this implementation, a common base class \texttt{AuditableEntity}\footnote{\javaname{impl-crud/src/main/java/karsch.lukas.audit.}\keyw{AuditableEntity}} is used to handle basic auditing metadata such as creation and modification timestamps using Spring Data JPA annotations. \autoref{lst:audited-entity} presents the state of an entity after enabling Envers auditing. Apart from the \keyw{@Audited} annotation, no changes are necessary, unless developers wish to exclude certain fields from auditing, in which case \keyw{@NotAudited} can be used.

\begin{lstlisting}
%%@Entity%%
%%@Audited%%
public class CourseEntity extends AuditableEntity { 
    @Id @GeneratedUuidV7
    private UUID id;
    // all fields remain unchanged 
} 
\end{lstlisting}
{
\captionof{lstlisting}{Auditing configuration for the Course entity (\javaname{impl-crud/src/main/java/karsch.lukas.courses.}\keyw{CourseEntity}})
\label{lst:audited-entity}
}

\subsubsection*{Custom Revision Entity and Listener}

While Envers provides a default revision table (storing only a revision ID and timestamp), a custom implementation is required to capture application-specific context, such as the user responsible for the change and a descriptive, optional context which allows capturing additional information about a change.

As shown in \autoref{lst:custom-revision-entity}, the \keyw{CustomRevisionEntity} extends Envers' \keyw{DefaultRevisionEntity} to include the fields \keyw{revisionMadeBy} and \keyw{additionalContext}.

\begin{lstlisting}
%%@Entity%%
%%@RevisionEntity%%(UserRevisionListener.class)
public class CustomRevisionEntity extends DefaultRevisionEntity {
    private String revisionMadeBy; 
    private String additionalContext;
}
\end{lstlisting}
{
\captionof{lstlisting}[Custom revision entity]{Custom Envers revision entity (\javaname{impl-crud/src/main/java/karsch.lukas.audit.}\keyw{CustomRevisionEntity})}
\label{lst:custom-revision-entity}
}

The association between a transaction and this metadata is handled by the \keyw{UserRevisionListener}. This listener intercepts the creation of a new revision and populates the fields by accessing the current request scope and a custom \keyw{AuditContext} bean. Its implementation is detailed in \autoref{sec:capturing-request-scoped-context}.

\subsubsection*{Capturing request-scoped context}
\label{sec:capturing-request-scoped-context}

To ensure the audit log contains meaningful information about why or by whom a change was made, the implementation utilizes Spring's \keyw{@RequestScope}. This annotation can be placed on beans, which will then be request-scoped, meaning they are re-created for each request. This annotation is used on two beans: \keyw{RequestContext}, holding information about the current user, and \keyw{AuditContext}, which is a bean able to capture additional context for auditing purposes. As \keyw{UserRevisionListener} is a Hibernate specific class living outside of Spring's managed environment, a static \keyw{getBean} method is used to access the relevant Spring beans.

\begin{lstlisting}
public class UserRevisionListener implements RevisionListener {
    %%@Override%%
    public void newRevision(Object revisionEntity) {
        CustomRevisionEntity rev = (CustomRevisionEntity) revisionEntity;

        if (isInsideRequestScope()) {
            RequestContext ctx = SpringContext.getBean(RequestContext.class);
            AuditContext audit = SpringContext.getBean(AuditContext.class);
            
            rev.setRevisionMadeBy(ctx.getUserType() + "_" + ctx.getUserId());
            rev.setAdditionalContext(audit.getAdditionalContext());
        } else {
            rev.setRevisionMadeBy("SYSTEM");
        }
    }
}
\end{lstlisting}
{
\captionof{lstlisting}[Implementation of the Revision Listener]{Implementation of the Revision Listener (\javaname{impl-crud/src/main/java/karsch.lukas.audit.}\keyw{UserRevisionListener})}
\label{lst:revision-listener}
}

\subsubsection*{Global Auditing Configuration}

Finally, the \texttt{AuditingConfig}\footnote{\javaname{impl-crud/src/main/java/karsch.lukas.audit.}\keyw{AuditingConfig}} configuration class connects the application's custom time provider to the JPA auditing infrastructure. This ensures that both the standard \texttt{createdAt} fields and the Envers revision timestamps are synchronized with the application's internal clock, which is essential for consistent testing. Additionally, the configuration connects the application's request context to the auditing infrastructure, providing information about the current user. In a full-grade application, Spring security would provide the user context, though for this project, a simpler solution was preferred, as described in (TODO).

\subsubsection{Reconstructing historic state (TODO refine language)}

Envers stores its revision data and historic records in specific auditing tables. While those hold the information necessary to reconstruct historic state, it is interesting to examine how well this state can \emph{actually} be recreated. Imagine a service method with the purpose of returning the history of grade changes made to one grade. Envers provides a specific \acrshort{api} which can be queried to reconstruct historical state. The mentioned service method is implemented in \keyw{StatsService}, as shown in \autoref{lst:envers-historical-query}.

First, \acrshort{jpa}'s entity manager is used to obtain an instance of the AuditReader class, which provides methods to create historic queries. Using the \keyw{reader.createQuery()} method, it is possible to create a query instance by matching a specific class for which revisions shall be fetched, as well as adding a filter to match the relevant entity using its ID.

In this use-case, a date filter is part of the \acrshort{api}. Envers enables developers to add additional matchers based on revision properties. In this case, the revision property \keyw{timestamp} is used to define lower and upper date bounds, inside which the changes are relevant.

Once the query is built, the result list can be fetched, which is a list containing arrays of objects. More precisely, though not reflected by the type, each list entry is a tuple. Its first value is the historic entity, its second value is the revision entity which was created for this specific revision. Because a custom revision entity is registered, the type of this revision entity is \keyw{CustomRevisionEntity}.

\begin{lstlisting}
public GradeHistoryResponse getGradeHistory(
    UUID studentId, UUID assessmentId) {
    var assessment = fetchAssessment(assessmentId);
    var grade = fetchGrade(assessmentId, studentId);

    AuditReader reader = AuditReaderFactory.get(entityManager);

    AuditQuery query = reader.createQuery()
            .forRevisionsOfEntity(AssessmentGradeEntity.class, false, true)
            .add(AuditEntity.id().eq(grade.getId())); // match by entity ID

    if (startDate != null) {
        query.add(AuditEntity.revisionProperty("timestamp").gt(
                startDate.toEpochMilli())
        );
    }
    if (endDate != null) {
        query.add(AuditEntity.revisionProperty("timestamp").le(
                endDate.toEpochMilli())
        );
    }

    List<Object[]> results = query.getResultList();

    var gradeChanges = results.stream()
            .map(result -> {
                AssessmentGradeEntity entity = (AssessmentGradeEntity) result[0];
                CustomRevisionEntity revision = (CustomRevisionEntity) result[1];

                return new GradeChangeDTO(
                        lectureAssessmentId,
                        entity.getGrade(),
                        revision.getTimestamp()
                );
            })
            .toList();
    
    return new GradeHistoryResponse(gradeChanges);
}
\end{lstlisting}
{
\captionof{lstlisting}[Reconstructing historic state using Envers]{Reconstructing historic state using Envers, simplified code example adapted from \keyw{impl-crud/src/main/java/karsch.lukas.stats.StatsService}}
\label{lst:envers-historical-query}
}

\section{ES/CQRS implementation}
\label{sec:es-cqrs-implementation}

\subsection{Architecture Overview}
\label{sec:architecture-overview}

The architecture of the \texttt{impl-es-cqrs} application \footnote{\javaname{impl-es-cqrs/src/main/java/karsch.lukas}} differs from the traditional layered architecture seen in the \texttt{impl-crud} application. While the CRUD implementation also has some vertical slicing, the ES-CQRS implementation is much more explicit about it. The code is organized into "features", each representing a vertical slice of the application's functionality (e.g., \texttt{course}, \texttt{enrollment}, \texttt{lectures}). Each feature is self-contained and includes its own command handlers, event sourcing handlers, query handlers, and its own web controller, if needed.

A "feature slice" architecture is descriptive and able to communicate the features of a project at a glance. As clean architecture is not in the scope of this thesis, the separation into features with clear naming conventions for command and query components is sufficient, however introducing completely separate modules for the command and read sides would have increased the project structure's readability even more by clearly showing how command and read side have no access to each other. % TODO reference for "feature slicing" 

\subsection{The API Layer}
\label{sec:the-api-layer}

The \texttt{api} package in each feature slice is shared between web controllers, command side and read side, containing the public interface of the application. It defines the Commands, Events, and Queries that are dispatched and handled by the \javaname{impl-es-cqrs} application. Keeping the public API in a separate package ensures that the internal implementation details of the \javaname{impl-es-cqrs} application are not exposed to its clients.

\subsection{Command Side}
\label{sec:command-side}

The command side is responsible for handling state changes in the application. It is implemented using Axon's Aggregates, Command Handlers, and Sagas. This section goes in detail about the implementation aspects, using the courses feature as an example.

\subsubsection{Aggregates and Set-Based Validation}
\label{sec:aggregates-and-set-based-validation}

Aggregates are the core components of the command side. They represent a consistency boundary for state changes. In this implementation, an example of an aggregate is the \keyw{CourseAggregate} \footnote{\javaname{impl-es-cqrs/src/main/java/karsch.lukas.features.course.commands.CourseAggregate}}. It handles the \keyw{CreateCourseCommand}, validates it, and if successful, emits a \keyw{CourseCreatedEvent}.

Before creating a course, the system must verify that all the specified prerequisite courses actually exist. This is handled by the \keyw{ICourseValidator},\footnote{\javaname{impl-es-cqrs/src/main/java/karsch.lukas.features.course.commands.ICourseValidator}} which is injected into the aggregate's command handler. The validator employs set-based validation as described in \autoref{sec:set-based-validation}. Once the prerequisite courses are validated, a \keyw{CourseCreatedEvent} is emitted. Otherwise, a specific \keyw{MissingCoursesException} is thrown, indicating that command handling was rejected.

\subsubsection{External Command Handlers}
\label{sec:impl-external-command-handlers}

Not all commands can be handled by a single aggregate. For instance, assigning a grade to a student for a specific lecture involves the \texttt{EnrollmentAggregate} and the \texttt{LectureAggregate}. In such cases, a dedicated command handler, \texttt{Enrollment{\allowbreak}Command{\allowbreak}Handler}, is used. This handler coordinates the interaction between the aggregates. It loads the \texttt{EnrollmentAggregate} from the event sourcing repository, validates the command (e.g., checking if the professor is allowed to assign a grade for the lecture), and then executes the command on the aggregate.

\subsubsection{Sagas for Process Management}
\label{sec:sagas-for-process-management}

Sagas are used to manage long-running business processes that span multiple aggregates. The \texttt{AwardCreditsSaga} is a prime example. It is initiated when an \keyw{EnrollmentCreatedEvent} occurs. The saga then waits for a \keyw{LectureLifecycleAdvancedEvent} with the status \texttt{FINISHED}. Once this event is received, the saga sends an \keyw{AwardCreditsCommand} to the \keyw{EnrollmentAggregate}. The saga ends when it receives a \keyw{CreditsAwardedEvent}. This ensures that credits are only awarded after a lecture is finished and all assessments have been graded. It is interesting to note that while the CRUD application calculates awarded credits based on the current state of a lecture, in the ES-CQRS implementation, the fact that credits are awarded after finishing a lecture is explicit. Even when changing the Saga later on, credits which have already been awarded will not be revoked, unless additional, explicit logic is implemented (e.g. by applying a \keyw{CreditsRevokedEvent}). % TODO keep elaborating on traceability here, OR move it to the end / Fazit.

\subsection{Read Side}
\label{sec:read-side}

The read side listens to events asynchronously and builds read models, called "projections", which are views of the system. A component that listens for events and maintains projections is called a "projector". Projections are designed to answer specific questions about the system: each projector saves exactly the necessary information. This is achieved by using denormalized data models, a contrast to typical CRUD systems that follow normalization rules.

When the system is queried, the queries are routed to the read side. The read side can efficiently fetch data from the projections, usually without \texttt{JOINs}. This makes reads fast. It is important to keep in mind that projections are built asynchronously, meaning they are eventually consistent and may not always reflect the latest changes applied by the command side.

In the context of the ES-CQRS implementation, a good example of a projector that stores denormalized data for efficient querying is the \keyw{LectureProjector}. It demonstrates the fact that each projector maintains its own view of the system. Projectors must not query the system using Axon's \keyw{QueryGateway} to get access to any data needed for the projection. One reason for that is the fact that when \emph{rebuilding} projections, a common use case in event sourcing, the projectors should be able to run in parallel. If projectors depend on each other, this can result in one projection attempting to query data from another projection that is not yet up to date. This is why the \keyw{LectureProjector} not only maintains a view of lectures, but also of courses, professors and students, which are then used when building the lecture's projection.

The projector also illustrates how the projection's database entities are designed: they are built in the same way as the DTO which is returned from the query handler. Arrays and associated objects are not stored via foreign keys but are instead serialized to \acrshort{json}. This allows the retrieval of all the necessary data to respond to a query with a simple \texttt{SELECT} statement. The same concepts apply to all other projectors in the ES-CQRS implementation.

\subsection{Synchronous Responses with Subscription Queries}
\label{sec:synchronous-responses-with-subscription-queries}

A common challenge in \acrshort{cqrs} and event-driven architectures is providing synchronous feedback to users. For example, when a student enrolls in a lecture, they expect an immediate response indicating whether they were successfully enrolled or placed on a waitlist. However, commands are usually handled asynchronously. In \acrshort{cqrs}, commands are also not intended to return data.

To solve this, the \keyw{LecturesController} uses Axon's subscription queries. When an enrollment request is received, it sends the \keyw{EnrollStudentCommand} and simultaneously opens a subscription query (\keyw{EnrollmentStatusQuery}). This query waits for an \keyw{EnrollmentStatusUpdate} event. The read-side projector responsible for processing enrollments publishes this update after processing the respective \keyw{StudentEnrolledEvent} or \keyw{StudentWaitlistedEvent}. The controller blocks for a short period, waiting for this update to be published, and then returns the result to the user. This approach makes the user interface synchronous, while not contradicting with the asynchronous nature of \acrshort{cqrs} systems, as the command handling process is unchanged. While this approach provides the desired synchronous user experience, it has the downside of coupling the client to the event processing flow. In a typical scenario, developers might employ WebSockets or other client-side notification mechanisms to inform the user about the result of their action. However, for the context of this thesis, where the primary goal is to implement two applications with an identical interface, this solution is a pragmatic compromise. % TODO improve this section

\subsection{Encapsulation and API Boundaries}
\label{sec:encapsulation-and-api-boundaries}

To enforce the separation of concerns and maintain a clean architecture, the internal components of the command and read sides are package-private. For example, the \keyw{CourseAggregate} and \keyw{CourseProjector} are not accessible from outside their respective feature packages. The public API of the application is exposed through the controllers, which only interact with the \keyw{CommandGateway} and \keyw{QueryGateway}. This ensures that all interactions with the system go through the proper channels and that internal implementations can be changed without affecting the clients.

\subsection{Tracing Request Flow}
\label{sec:tracing-request-flow}

This section illustrates the flow of commands and queries through the system. Axon's \keyw{CommandGateway} and \keyw{QueryGateway} are used in controllers to decouple them from the internals of the application. The gateways create location transparency: a controller does not need to know where its commands and queries are being routed to. % TODO reference 

\subsubsection{Command Request: CreateCourseCommand}
\label{sec:command-request-createcoursecommand}

Figure \ref{fig:es-cqrs-command-flow} illustrates the flow of a command through the system using the example of the \texttt{POST} \texttt{/courses} endpoint. Upon receiving a request, the controller constructs a \keyw{CreateCourseCommand} containing the request data and dispatches it through the \keyw{CommandGateway}. This gateway is responsible for routing the command to the appropriate destination, which in this case is the constructor of the \keyw{CourseAggregate}. This constructor is annotated with \keyw{@CommandHandler}. The command handler verifies that the command is allowed to be executed by performing validation logic. When creating courses, it has to be made sure that all prerequisite courses actually exist. This check is done using set-based validation. If the validation is successful, the aggregate triggers a state change by applying a \keyw{CourseCreatedEvent} via the \keyw{AggregateLifecycle.apply()} method. This action notifies the system of the change and persists the event by recording it in the event store.

After being applied, Axon routes the event to all subscribed handlers. The \keyw{CourseAggregate}'s \keyw{@EventSourcingHandler} is executed, changing the aggregate's internal state. What is worth noting here is that in the case of \keyw{CourseAggregate}, only the \texttt{id} of the course is set as other properties of the event, like name or description of the newly created course, are not relevant to the command side. Any read-side projectors with \keyw{@EventHandlers} for the \keyw{CourseCreatedEvent} are also executed after the event is applied.

\begin{figure}[H]
    \includegraphics[width=\textwidth, inner]{images/es-cqrs-command-flow.png}
    \caption{Sequence Diagram: Command Flow inside the ES-CQRS application}
    \label{fig:es-cqrs-command-flow}
\end{figure}

\subsubsection{Query Request: FindAllCoursesQuery}
\label{sec:query-request-findallcoursesquery}

\hyperref[fig:es-cqrs-query-flow]{Figure \ref*{fig:es-cqrs-query-flow}} illustrates the flow of a query through the application using the \texttt{GET} \texttt{/courses} request as an example. The request is received by \keyw{CoursesController}. It creates a \keyw{FindAllCoursesQuery} instance and sends it to Axon's \keyw{QueryGateway}, which routes the query to the appropriate \keyw{@QueryHandler} method responsible for \keyw{FindAllCoursesQuery}. The query handler method then accesses its JPA repository to get all courses, maps them to a list of \keyw{CourseDTOs} and returns this list. The \keyw{QueryGateway} hands this result over to the web controller which reads the data and sends it back to the client.

\begin{figure}[H]
    \includegraphics[width=\textwidth, inner]{images/es-cqrs-query-flow.png}
    \caption{Sequence Diagram: Query Flow inside the ES-CQRS application}
    \label{fig:es-cqrs-query-flow}
\end{figure}

\section{Infrastructure}
\label{sec:infrastructure}

The project's infrastructure is designed for consistency and reproducibility across development and testing environments. It is composed of a containerized environment for running the applications and their dependencies, an automated \gls{vm} provisioning setup for performance testing, as well as an integration testing strategy using Testcontainers, described in \autoref{sec:contract-test-implementation}.

\subsection{Containerized Services}
\label{sec:containerized-services}

The core of the infrastructure is defined in a \gls{docker} compose file at the root of the project, which orchestrates the deployment of the two primary applications and their external dependencies: a \hyperref[sec:postgresql]{PostgreSQL} database, used by both applications, and an \hyperref[sec:axon]{Axon Server} instance, used by the ES-CQRS application.

A \keyw{postgres:18-alpine} container provides the relational database used by both applications. The database schema, user, and credentials are configured through environment variables. A volume is used to persist data across container restarts.

An \keyw{axoniq/axonserver} container provides the necessary infrastructure for the Event Sourcing and CQRS implementation, handling event storage and message routing. It is configured to run in development mode.

The CRUD and ES-CQRS applications are containerized using \keyw{Dockerfile}s. Both use \keyw{amazoncorretto:25} as the base image, and the compiled Java application (\keyw{.jar} file) is copied into the container and executed.

Configuration details, such as database connection strings and server hostnames, are externalized from the \keyw{application.properties} files. They are injected into the application containers at runtime as environment variables via the \keyw{docker-compose.yml} file, allowing for flexible configuration without modifying the application code.

\subsection{Local Development and Integration Testing}
\label{sec:local-development-and-integration-testing}

For local development and integration testing, the project uses the \glspl{testcontainer} library. This approach allows developers to programmatically define and manage the lifecycle of throwaway Docker containers for dependencies like PostgreSQL and Axon Server directly from the test code. (TODO duplicate?)

By integrating with Spring Boot's \glspl{testcontainer} support, running the application or its tests automatically starts the required containers. This eliminates the need to manually install and manage these services on their local machines, ensuring a consistent and isolated testing environment. The configuration for this is found in the test resources, where a special JDBC URL prefix signals Spring Boot to manage the database container.

\subsection{VM Provisioning for Performance Testing}
\label{sec:vm-provisioning}

To ensure a stable and isolated environment for performance benchmarks, a dedicated \acrshort{vm} setup is used. The process of creating and provisioning these \acrshortpl{vm} on a Proxmox host is fully automated.

A shell script, \keyw{create-vm.sh},\footnote{\javaname{performance-tests/vm/scripts/create-vm.sh}} orchestrates the creation of a \acrshort{vm} template from an Ubuntu 24.04 cloud image. Cloud images are pre-configured, lightweight variants of operating systems. This script works in conjunction with a CloudInit \footnote{\url{https://cloudinit.readthedocs.io/en/latest/}} configuration file that handles the provisioning of the \acrshort{vm} upon its first boot. \footnote{\javaname{performance-tests/vm/scripts/cloud-init.yml}}

During the provisioning process, a number of steps are executed. First, it is made sure that the system is up-to-date by installing any available software updates. Next, a `thesis` user is created for which the environment is configured. Afterward, the script installs all necessary software, including \gls{docker}, git, Conda, Python, k6, Maven, and Java 25. Once all necessary software is installed, the project's git repository is cloned and a Maven build is triggered. Finally, the \gls{docker} images are built. After these steps are completed, the provisioned \acrshort{vm} is ready to run the applications and load tests.

Instead of starting the \acrshort{vm} directly, the script shuts the \acrshort{vm} down and converts it into a Proxmox template, which can be re-created efficiently. This template is used to create the client and server \glspl{vm}.

The test environment and scenarios are defined as code to ensure reproducibility. Tests are executed in an isolated environment with fixed hardware allocations as specified in \autoref{table:hardware-specs}.

\begin{table}[htp!]
    \small
    \centering
    \begin{tabularx}{\linewidth}{lX}
        \toprule
        \textbf{Component} & \textbf{Specification}                                                      \\ \midrule
        CPU                & 13th Gen Intel(R) Core(TM) i7-13700H. 14 Cores, 20 total threads. Max. 5GHz \\
        RAM                & 32GB DDR4 (2x16GB), 3200 MT/s                                               \\
        Hard Drive         & SanDisk Plus SSD 1TB 2.5" SATA 6GB/s                                        \\
        \bottomrule
    \end{tabularx}
    \caption{Hardware specifications for the performance evaluation machine}
    \label{table:hardware-specs}
\end{table}

The physical host provisions two \glspl{vm}: the "client VM" for load generation and the "server VM" for the application and its dependencies (PostgreSQL and Axon Server). While hosting both on one physical machine makes network latency negligible, the "queueing delay" remains measurable at the client level, allowing for the identification of request queues building up on the server, indicating bottlenecks.

\section{Load Tests}

This section describes the implementation of load tests.

\subsection{k6 Scripts}
\label{sec:k6-implementation}

The core of the load testing suite are the load-generating scripts developed using \hyperref[sec:k6]{k6}. \autoref{lst:create-course-k6-script} illustrates the implementation of a typical k6 script using the creation of courses with prerequisites as an example. \footnote{\javaname{performance-tests/k6/writes/create-course-prerequisites/create-course-prerequisites.js}}

After defining necessary imports, the test script extracts execution parameters from the \keyw{\_\_ENV} object which is injected by the k6 test runner. Most k6 scripts written for this project rely on \acrshort{rps}, representing the target iteration rate, and \texttt{TARGET\_HOST}, which is the URL the application under test is reachable at.

The value of \acrshort{rps} is used to define test options. Namely, a scenario, optional thresholds and the statistics to collect are defined. A test may have several scenarios, however in the k6 scripts used in this project, only one scenario per test is defined. Each scenario has a specific executor. In this case, the "ramping-arrival-rate" executor is used, as opposed to the "ramping-vus" executor. While the "ramping-vus" executor defines the number of virtual users interacting with the application (closed model), "ramping-arrival-rate" executors define the number of iterations per second (open model). This important distinction is described in more detail in \autoref{sec:load-test-theory}. Stages in a scenario define the "timeline" of \acrshort{rps}. In the given example, \acrshort{rps} are increased from 0 to the target \acrshort{rps} over a duration of 20 seconds. This \acrshort{rps} is then held for a duration of 80 seconds, before decreasing \acrshort{rps} back to 0 over a span of 20 seconds.

After defining test options, an optional setup function is implemented. It is executed once by k6, before running the load-generating "export default" function. In the setup function, seed data can be created. The given code example uses the setup function to create 10 prerequisite courses. Their IDs are returned from the setup function.

Data returned from the setup function can be passed to the "export default" function, which is the core of any load test. This is the function that is executed repeatedly to generate load. The implementation of this function in the given example is rather simple. One POST request is sent to the server. This request includes a payload which references a random number of prerequisite courses, as well as other required parameters for course creation.

\begin{lstlisting}[language=JavaScript]
// Imports omitted
const {TARGET_HOST, RPS} = __ENV;

export const options = {
    scenarios: {
        createCourses: {
            executor: "ramping-arrival-rate",
            timeUnit: "1s",
            preAllocatedVUs: RPS,
            stages: [
                {target: RPS, duration: "20s"},
                {target: RPS, duration: "80s"},
                {target: 0, duration: "20s"}
            ]
        }
    },
    thresholds: {
        'http_req_failed': ['rate<0.01'], // Error rate must be <1%
    },
    summaryTrendStats: ["med", "p(99)", "p(95)", "avg"],
};

export function setup() {
    const prerequisiteIds = createPrerequisites(10);
    return { prerequisiteIds };
}

export default function (data) {
    const {prerequisiteIds} = data; 

    const url = `${TARGET_HOST}/courses`;
    const prerequisiteCourseIds = selectRandomPrerequisiteIds();
    const payload = createPayload(prerequisiteCourseIds);
    const res = http.post(url, payload);
    checkResponseIs201(res);
}
\end{lstlisting}
{\captionof{lstlisting}[k6 script, simplified code example]{Simplified code example of a k6 script to test course creation. Adapted from \javaname{performance-tests/k6/writes/create-course-prerequisites/create-course-prerequisites.js}}}
\label{lst:create-course-k6-script}

\subsection{Load Test Lifecycle}

The k6 scripts alone are not enough to execute a large, repeated load test. While they can generate load on a running application and are capable of collecting client-side metrics, external lifecycle management is needed to control the infrastructure and ensure a clean environment in between each test run.

The lifecycle of repeated load tests is managed using python scripts. The core scripts are \keyw{perf\_runner.py} \footnote{\keyw{performance-tests/perf\_runner.py}} and \keyw{many\_runs.py} \footnote{\keyw{performance-tests/many\_runs.py}}. These scripts instrument the entire lifecycle of the application and k6 runs. They are responsible for starting the application using \gls{docker}, collecting server-side metrics using Prometheus and post-processing results.

The core logic within \keyw{perf\_runner.py} follows a defined flow for every single test run. It begins by determining the execution context. If a remote configuration is provided, it establishes a \gls{docker} Remote Context via \gls{ssh} to interact with the target \acrshort{vm}. It then deploys the application using \texttt{docker compose up}. Before directing any traffic towards the application, the Actuator's health endpoint is polled to ensure the application is running properly.

Once the application is healthy, the script sets up Prometheus for server-side monitoring. After dynamically generating a \javaname{prometheus.yml} configuration file, a Prometheus container is started, targeted to scrape the application under test. To ensure short-term spikes in latency or resource consumption can be captured, the configuration defines a polling interval of 2 seconds.

With the environment and monitoring active, the script invokes k6. Configuration parameters for the test run are expected to be defined in \javaname{metric.json}, which is a file placed alongside a test script. It includes metadata and parameters such as the number of \acrshortpl{VU} and the target host URL. These parameters are passed directly to the k6 engine via environment variables. Inside the k6 scripts, the \acrshortpl{VU} environment variable is used to define the arrival rate within the ramping-arrival-rate executor rather than a fixed number of concurrent users. Because k6 is configured to trigger a specific number of iterations per second, this parameter effectively acts as a control for Requests Per Second (RPS), ensuring the load remains consistent regardless of how long the individual HTTP calls take to complete. % TODO unverständlicher satz wahrscheinlich

After k6 completed its load generation, the script enters a data-extraction phase. It queries the Prometheus API to retrieve system-level metrics. Next, it parses the k6-summary.json file, which is a file generated by k6 that includes all metrics recorded during the run. The collected data is processed and merged into standardized CSV files (client\_metrics.csv and server\_metrics.csv).

Once all data is extracted, the system is ready for the next run. To prepare the environment, all containers need to be stopped first. That is done by running \javaname{docker compose down -v} inside the \gls{docker} remote context, with the \texttt{-v} argument explicitly removing all docker volumes. This makes sure PostgreSQL's and Axon Server's data stores are emptied out before the next test iteration.

While \javaname{perf\_runner.py} manages the lifecycle of a single test, \javaname{many\_runs.py} acts as a high-level orchestrator, designed to automate large-scale comparative benchmarks by executing multiple iterations across both implementations by running a single command. The script can be configured to run an arbitrary number of tests, which will be executed for both applications. The script accepts the metric configuration files and passes them on to \javaname{perf\_runner.py}.

\subsection{Post Processing Test Results}

After extracting data from the k6 output and Prometheus, it is consolidated into a unified CSV format. This is necessary because the two systems use differing naming conventions and units: while k6 might report the 95th percentile latency as $p(95)$ in milliseconds, Prometheus might expose it through a complex PromQL query resulting in a label like $latency\_p95$, measured in seconds. Precisely, k6's $med$, $avg$ and percentile latency metrics are mapped to the Prometheus equivalent, laid out in \autoref{table:collected-metrics}. Performing this normalization step immediately after the test run means the collected data can easily be compared and visualized later.

\subsection{Testing "Freshness": Time to Consistency}

To assess the eventual consistency of the ES-CQRS architecture, a specialized test for the \hyperref[slo-freshness]{Freshness \acrshort{slo}} was developed.\footnote{\keyw{performance-tests/k6/time-to-consistency/create-lecture/create-lecture.js}} Unlike standard performance scripts, which measure the speed of isolated requests, this script is specifically designed to measure the synchronization delay between the command and query sides of the application. This delay, called eventual consistency, occurs because the write-side (Command) and read-side (Query) are strictly separated in \acrshort{cqrs}.

The primary difference from a typical k6 test lies in the execution flow within the default function. Rather than executing a single \acrshort{http} call, this test executes two calls to the application. First, it performs a POST request to create a lecture and captures the resulting ID. After creating the lecture, the script performs sleeps for exactly 0.1 seconds, the threshold defined in the Freshness \acrshort{slo}. After this threshold, the application is expected to have synchronized the write- and read-side. Once the script wakes up from its sleep, it performs a GET request, attempting to fetch the newly created lecture.

To track the success rate of this request, the script introduces a custom Rate metric named $read\_visible\_rate$. By manually adding true or false to this metric based on whether the lecture was found, indicated by a response status of 200, the script generates a percentage of "fresh" requests inside the required threshold of 100ms. This provides a clear statistical view of how reliably the ES-CQRS system maintains its "fresh" data under varying levels of load.

% !TeX root = ../main.tex
\chapter{Results}

\section{Performance}
\label{sec:performance-results}

This section describes results of load testing. Each \gls{l} is presented in order, describing the kind of endpoint and test that was employed.

For all load tests, the primary results are visualized through diagrams like latency-vs-load line plots. The raw numerical data, including confidence intervals and significance levels for all tested load levels, are provided in \autoref{appendix:results}.

\subsection{Significance}

The significance levels indicated in the tables inside the appendix are calculated using the \emph{Mann-Whitney U} significance test. This test is used because it doesn't require normally distributed samples and compares the \emph{rankings} of results rather than the averages, making it more resistant to outliers\cite[Chapter 13.3]{triola_elementary_2012}. In load testing, outliers (\glspl{tail-latency}) can create \emph{skewing} that would incorrectly bias a standard average-based test.

\subsection{Steady-state performance}

To capture the system's median CPU usage under load, the initial usage spike ("transient"), exhibited across all runs, was excluded. This approach mitigates the influence of startup spikes and resource initialization overhead, instead focusing on the steady-state performance. This process is called \emph{transient removal}~\cite{jain_art_1991}.

\subsection{Ratios}

When comparing the CRUD and ES-CQRS applications' performance, a "speedup" factor or ratio is calculated per metric and \gls{rps}. This ratio may be mentioned for textual descriptions of the resulting graphs, and is always present in the tables inside the appendix. The term "speedup" is used when comparing latencies, while the term "ratio" is used for other metrics. The given values are calculated like this: $\frac{metric_{crud}}{metric_{es\_cqrs}}$. Therefore, a $latency\_p95$ of 100ms for the CRUD application and a $latency\_p95$ of 50ms for the ES-CQRS application would result in a speedup of 2x for that specific metric at the respective \gls{rps}. Ratios involving the value zero for either application are marked as \emph{N/A}.

\subsection{Dropped Iterations}

In some tests, iterations were dropped, meaning that k6 skipped iterations and did not send requests to the server. This occurs if an iteration takes longer than 1 second and no more \glspl{VU}, which act as request processors, are available to k6. The fact that iterations were dropped during testing is mentioned when describing results, and the rate of dropped iterations per second ($dropped\_iterations\_rate$) is available in the results tables inside the appendix.

\subsection{Threadpool}

Each load test shows the threadpool usage ($tomcat\_threads$) of the application under test. The maximum value this metric can take in these tests is 200, as it SpringBoot's default configuration.

\subsection{Database Connections}

Each load test presents the number of used database connections of the application under test --- $hikari\_connections$. The maximum value this metric can take in these tests is 10, which can again be attributed to SpringBoot's default configuration.

It is important to clarify that a median value of zero active database connections does not imply a lack of database activity. Instead, this result is an artifact of the metric collection frequency and the statistical properties of the median. Resource consumption metrics were sampled at two-second intervals. In scenarios where database interactions completed within milliseconds, the majority of these snapshots captured the connection pool in an idle state. Consequently, while the database was used during request handling, a high rate of zero-value samples drives the median to zero.

\subsection{Data Store Size}

The size of the data store is calculated differently for both applications. The CRUD application uses PostgreSQL as its only data store. Therefore, $postgres\_size$ equals the application's data store size.

The total data store size for the ES-CQRS application is defined as the sum of the PostgreSQL projection size and the allocated Axon storage: $postgres\_size + axon\_storage\_size$. It is important to note that Axon Server allocates storage in fixed-size segments (or "pages") of 4MB. Because these segments are allocated eagerly, the total storage includes a minimum overhead of 8MB (one 4MB segment each for events and snapshots), regardless of the actual data density within those blocks. Consequently, the measured size represents the allocated capacity rather than the literal byte-count of the stored records.

\subsection{Graphs}

In graphs that show latencies, the shaded areas in the line graphs represent the \emph{confidence interval (CI)} of the latency measurements. The confidence interval is the range of values that is likely to contain the true answer. It estimates the level of uncertainty by providing a margin of error around a specific result.

Graphs showing resource consumption typically show the area between the 25th and 75th percentile in the shaded area, also called \emph{\gls{iqr}}. This shows the typical range of resource usage across runs.

The lines between the measured data points are interpolated, representing an estimation of performance trends across the full range of measurements. These lines help visualize the transition between different loads, though the actual values were only recorded at the marked intervals on the x-axis. In some cases, the x-axis omits intermediate labels for improved readability. However, all data points remain connected by interpolated lines to illustrate performance trends between measurements. For a complete view of every value, refer to the full results provided in \autoref{appendix:results}.

\subsection{Write Performance}

\subsubsection{L1: Create Courses Simple}
\label{sec:l1}

Professors can create courses using the \texttt{POST} \texttt{/courses} endpoint. The "simple" test case tests creation of courses \emph{without prerequisites}. This means that no validation is necessary to create a new entity, making this test the raw insertion performance.

\autoref{fig:l1_combined} presents line graphs comparing the endpoint's latency under increasing load using a logarithmic y-axis. The graphs describe the observed client-side and server-side latencies under varying loads of both applications by showing their $latency\_p50$ (median) and $latency\_p95$ (tail latency). \ref{slo-latency} defined the threshold for $latency\_p95$ at 100ms, meaning the ES-CQRS application failed to satisfy the \gls{slo} between 500 and 1000 \gls{rps}. Additionally, its $dropped\_iterations\_rate$ reaches a value of 51 at 1000 RPS. Meanwhile, the \gls{crud} application achieves a sub 10ms $latency\_p95$ until at least 1000 \gls{rps}.

It can be noted that the latency curve follows the same pattern in both the measurements made on the client and on the server, except for the ES-CQRS $latency\_p95$ which exhibits a value of over 1000ms on the client, but only around 300ms on the server.

\begin{figure}[H]
    \centering
    \begin{subfigure}[b]{0.49\textwidth}
        \centering
        \includegraphics[width=\textwidth]{images/results/create-course-simple/create_course_simple_latency-vs-load__client.png}
        \caption{Latency vs. Load measured on the client}
        \label{fig:l1_latency_vs_load_client}
    \end{subfigure}
    \hfill
    \begin{subfigure}[b]{0.49\textwidth}
        \centering
        \includegraphics[width=\textwidth]{images/results/create-course-simple/create_course_simple_latency-vs-load__server.png}
        \caption{Latency vs. Load measured on the server}
        \label{fig:l1_latency_vs_load_server}
    \end{subfigure}
    \caption[L1 Performance Metrics, load measured in \gls{rps}]{L1 Performance Metrics: Load measured in \gls{rps}. Detailed results in \ref{results:l1}.}
    \label{fig:l1_combined}
\end{figure}

\autoref{fig:l1_cpu_usage_vs_load} presents median CPU usage of the endpoint under increasing load. The ES-CQRS application generally has a higher CPU usage, exhibiting a value of $\approx$45\% at 1000 \gls{rps}, while the CRUD application uses $\approx$10\%. The CRUD application's CPU usage rises linearly, while the ES-CQRS application's CPU usage rises slower past 500 \gls{rps}.

The threadpool usage of both applications is visualized in \autoref{fig:l1_threadpool_usage}. At 500 \gls{rps}, ES-CQRS starts using more threads, finally reaching threadpool saturation --- meaning that all 200 possible threads of the threadpool are used --- at 1000 \gls{rps}. The CRUD application uses at most around 25 threads at 1000 \gls{rps}.

\begin{figure}[H]
    \centering
    \begin{subfigure}[t]{0.49\textwidth}
        \centering
        \includegraphics[width=\textwidth]{images/results/create-course-simple/CPU_Usage_vs_Load.png}
        \caption{CPU Usage (\%) vs. Load}
        \label{fig:l1_cpu_usage_vs_load}
    \end{subfigure}
    \begin{subfigure}[t]{0.49\textwidth}
        \centering
        \includegraphics[width=\textwidth]{images/results/create-course-simple/create-course-threadpool-usage.png}
        \caption{Threadpool Usage vs. Load}
        \label{fig:l1_threadpool_usage}
    \end{subfigure}
    \caption[L1 Resource Usage, load measured in \gls{rps}]{L1 Resource Usage: Load measured in \gls{rps}. Detailed results in \ref{results:l1}.}
    \label{fig:l1_resource_vs_load}
\end{figure}

In \autoref{fig:l1_data_store_vs_load}, the size of the data store under increasing load is presented. The CRUD application's data store, consisting only of the PostgreSQL database, exhibits a linear growth, reaching a size of $\approx$70MB at 1000 \gls{rps}. With 100,000 requests sent at 1000 \gls{rps}, this comes out to around 0.7kB per persisted course. The ES-CQRS graph also grows linearly. At 500 \gls{rps}, the size of the data store is about 60MB, a 1.5x higher storage consumption than the CRUD application. This equals around 1kB per persisted course. However, at 1000 \gls{rps}, the storage size decreases. This can be attributed to a projection lag caused by the eventual consistency present in the application and is explained further in \autoref{ch:discussion}. The true storage consumption of the ES-CQRS application can be assumed to be around 120MB, once all projections are updated.

\begin{figure}[H]
    \centering
    \begin{subfigure}[b]{0.49\textwidth}
        \centering
        \includegraphics[width=\textwidth]{images/results/create-course-simple/create-course-simple_db-connections.png}
        \caption{Database Connections vs. Load}
        \label{fig:l1_db_connections}
    \end{subfigure}
    \hfill
    \begin{subfigure}[b]{0.49\textwidth}
        \centering
        \includegraphics[width=\textwidth]{images/results/create-course-simple/data_store_size.png}
        \caption{Data Store Size vs. Load}
        \label{fig:l1_data_store_vs_load}
    \end{subfigure}
    \caption[L1 Database Usage, load measured in \gls{rps}]{L1 Database Usage: Load measured in \gls{rps}. Detailed results in \ref{results:l1}.}
    \label{fig:l1_database}
\end{figure}

These values correspond to the following scalability metrics for the CRUD application: $\psi(200, 500)\approx 5.4$, $\psi(500, 1000)\approx 1.4$; and the following scalability metrics for the ES-CQRS application: $\psi(200, 500)\approx 0.6$, $\psi(500, 1000)\approx 0.01$.

\subsubsection{L2: Create Courses Prerequisites}
\label{sec:l2}

This test also evaluates the performance of the endpoint described in \autoref{sec:l1}. However, before executing the load generation, a set of "prerequisite" courses are generated. Each iteration, a random number of these courses (0 to 5) are selected which are referenced during load generation. This creates the necessity to do additional checks on existing data, verifying whether the referenced courses actually exist.

The endpoint's performance, presented in \autoref{fig:l2_combined} using a logarithmic y-axis, is similar to the observations made in \hyperref[sec:l1]{L1}. After exceeding 500 \gls{rps}, the ES-CQRS application fails to satisfy \ref{slo-latency} with a $latency\_p95$ exceeding the threshold of 100ms. Additionally, its $dropped\_iterations\_rate$ reaches a value of 61. The CRUD application, on the other hand, is able to fulfill the latency threshold up to at least 1000 \gls{rps} without dropping iterations.

It can be noted that the latency curve follows the same pattern in both the measurements made on the client and on the server, except for the ES-CQRS $latency\_p95$ which exhibits a value of over 1000ms on the client, but only $\approx$360ms on the server at 1000 \gls{rps}.

\begin{figure}[H]
    \centering
    \begin{subfigure}[b]{0.49\textwidth}
        \centering
        \includegraphics[width=\textwidth]{images/results/create-course-prerequisites/course-prerequisites-latency-load-CLIENT.png}
        \caption{Latency vs. Load measured on the client}
        \label{fig:l2_latency_vs_load_client}
    \end{subfigure}
    \hfill
    \begin{subfigure}[b]{0.49\textwidth}
        \centering
        \includegraphics[width=\textwidth]{images/results/create-course-prerequisites/course-prerequisites-latency-load-SERVER.png}
        \caption{Latency vs. Load measured on the server}
        \label{fig:l2_latency_vs_load_server}
    \end{subfigure}
    \caption[L2 Performance Metrics, load measured in \gls{rps}]{L2 Performance Metrics: Load measured in \gls{rps}. Detailed results in \ref{results:l2}.}
    \label{fig:l2_combined}
\end{figure}

\autoref{fig:l2_cpu_usage} presents the median $cpu\_usage$ of the endpoint under increasing load. The ES-CQRS application has a higher CPU usage, exhibiting a value of $\approx$45\% at 1000 \gls{rps}, while the CRUD application uses $\approx$10\%. The CRUD application's CPU usage rises linearly, while the ES-CQRS application's CPU usage rises slower past 500 \gls{rps}.

The threadpool usage of both applications is visualized in \autoref{fig:l2_threadpool_usage}. At 200 \gls{rps}, ES-CQRS starts using more threads, finally reaching threadpool saturation at 1000 \gls{rps}. The CRUD application uses at most around 25 threads at 1000 \gls{rps}.

\begin{figure}[H]
    \centering
    \begin{subfigure}[b]{0.49\textwidth}
        \centering
        \includegraphics[width=\textwidth]{images/results/create-course-prerequisites/courses-prerequisites-CPU-usage.png}
        \caption{Resource Usage (\%) vs. Load}
        \label{fig:l2_cpu_usage}
    \end{subfigure}
    \hfill
    \begin{subfigure}[b]{0.49\textwidth}
        \centering
        \includegraphics[width=\textwidth]{images/results/create-course-prerequisites/courses-prerequisites-threadpool-usage.png}
        \caption{Threadpool Usage vs. Load}
        \label{fig:l2_threadpool_usage}
    \end{subfigure}
    \caption[L2 Resource Usage, load measured in \gls{rps}]{L2 Resource Usage: Load measured in \gls{rps}. Detailed results in \ref{results:l2}.}
    \label{fig:l2_resource_usage_cpu}
\end{figure}

\autoref{fig:l2_database} presents database usage statistics of both applications under increasing load.

The median used database connections are shown in \autoref{fig:l2_db_connections}. At 1000 \gls{rps}, the ES-CQRS application uses all available connections, while the CRUD application utilizes around 2.

In \autoref{fig:l2_data_size}, the size of the data store under increasing load is presented. The CRUD application's data store exhibits a linear growth, reaching a size of $\approx$107MB at 1000 \gls{rps}. With 100,000 requests sent at 1000 \gls{rps}, this comes out to roughly 1.1kB per persisted course --- around 60\% higher than when creating courses without prerequisites.

The ES-CQRS application's storage consumption initially follows a linear growth pattern. At 200 \gls{rps}, the data store occupies approximately 45MB, representing a 1.6x increase compared to the CRUD application—roughly 2.3kB per persisted course. However, as the load increases beyond 500 \gls{rps}, the observed storage growth decelerates, eventually showing a decline at 1000 \gls{rps}. Again, this can be attributed to projection lag. Once the system updates all projections, the true storage consumption of the ES-CQRS application is estimated to be approximately 225MB.

\begin{figure}[H]
    \centering
    \begin{subfigure}[b]{0.49\textwidth}
        \centering
        \includegraphics[width=\textwidth]{images/results/create-course-prerequisites/courses-prerequisites-db_connections-usage.png}
        \caption{Database Connections vs. Load}
        \label{fig:l2_db_connections}
    \end{subfigure}
    \hfill
    \begin{subfigure}[b]{0.49\textwidth}
        \centering
        \includegraphics[width=\textwidth]{images/results/create-course-prerequisites/courses-prerequisites-data_size.png}
        \caption{Data Store Size vs. Load}
        \label{fig:l2_data_size}
    \end{subfigure}
    \caption[L2 Database Usage, load measured in \gls{rps}]{L2 Database Usage: Load measured in \gls{rps}. Detailed results in \ref{results:l2}.}
    \label{fig:l2_database}
\end{figure}

These values correspond to the following scalability metrics for the CRUD application: $\psi(200, 500)\approx 2.5$, $\psi(500, 1000)\approx 1.1$; and the following scalability metrics for the ES-CQRS application: $\psi(200, 500)\approx 0.4$, $\psi(500, 1000)\approx 0.02$.

\subsubsection{L3: Enrollment}

The endpoint \texttt{POST /lectures/\{lectureId\}/enroll} enrolls students to a lecture. Its performance under load up to 500 \gls{rps} is visualized in \autoref{fig:l3_combined} using a logarithmic y-axis.

The ES-CQRS application's $latency\_p95$ rises above 100ms between 50 and 100 \gls{rps}, violating \autoref{slo-latency}. The CRUD application's $latency\_p95$ and $latency\_p50$ remain below 10ms until 200 \gls{rps}, representing more than a 20x slowdown for the ES-CQRS application. Between 200 and 500 \gls{rps}, CRUD's P95 latency rises to over 1000ms, violating \ref{slo-latency}. At 500 \gls{rps}, ES-CQRS' $latency\_p95$ exceeds 10,000ms.

It is worth noting that starting from 200 \gls{rps}, the ES-CQRS's load tests started dropping iterations. $dropped\_iterations\_rate$ reached a value of around 240 at 500 \gls{rps}, at which point the CRUD application also exhibited dropped iterations with a rate of around 36.

\begin{figure}[H]
    \centering
    \begin{subfigure}[b]{0.49\textwidth}
        \centering
        \includegraphics[width=\textwidth]{images/results/enrollment/enrollment_latency-vs-load_client.png}
        \caption{Latency vs. Load measured on the client}
        \label{fig:l3_latency_vs_load_client}
    \end{subfigure}
    \hfill
    \begin{subfigure}[b]{0.49\textwidth}
        \centering
        \includegraphics[width=\textwidth]{images/results/enrollment/enrollment_latency-vs-load_server.png}
        \caption{Latency vs. Load measured on the server}
        \label{fig:l3_latency_vs_load_server}
    \end{subfigure}
    \caption[L3 Performance Metrics, load measured in \gls{rps}]{L3 Performance Metrics: Load measured in \gls{rps}. Detailed results in \ref{results:l3}.}
    \label{fig:l3_combined}
\end{figure}

\autoref{fig:l3_cpu_usage} compares CPU utilization by both applications as the load increases. The ES-CQRS application consistently requires more CPU than the CRUD application across all tested loads. At lower intensities (below 200 RPS), the ES-CQRS CPU usage grows, before peaking at around 42\% at 200 \gls{rps} and remaining at that level for 500 RPS. The CRUD application shows a steady increase in CPU consumption as load increases, reaching aroud 36\% at 500 RPS.

\autoref{fig:l3_threadpool_usage} illustrates the number of active threads utilized by each application to process the incoming load. At 25 and 50 RPS, both applications maintain an identical, stable baseline of 10 threads. Beyond 50 RPS, the utilized threads for ES-CQRS begin to increase. It reaches a maximum capacity of 200 threads at in between 100 and 200 \gls{rps} and remains saturated at that limit through 500 RPS. In contrast, the CRUD application maintains its baseline of 10 threads until 200 RPS. Even at the maximum load of 500 RPS, its median thread usage remains at around 17 threads. However, it should be noted that the \gls{iqr} at 500 RPS spans up to 200 threads, indicating a high variance in the measured results.

\begin{figure}[H]
    \centering
    \begin{subfigure}[b]{0.49\textwidth}
        \centering
        \includegraphics[width=\textwidth]{images/results/enrollment/enrollment-cpu-usage-vs-load.png}
        \caption{Resource Usage (\%) vs. Load}
        \label{fig:l3_cpu_usage}
    \end{subfigure}
    \hfill
    \begin{subfigure}[b]{0.49\textwidth}
        \centering
        \includegraphics[width=\textwidth]{images/results/enrollment/enrollment-threadpool_usage.png}
        \caption{Threadpool Usage vs. Load}
        \label{fig:l3_threadpool_usage}
    \end{subfigure}
    \caption[L3 Resource Usage, load measured in \gls{rps}]{L3 Resource Usage: Load measured in \gls{rps}. Detailed results in \ref{results:l3}.}
    \label{fig:l3_resource_usage_cpu}
\end{figure}

Database usage under increasing load is visualized in \autoref{fig:l3_database}. \autoref{fig:l3_db_connections} shows the median active database connections. It can be seen that the ES-CQRS application reaches its peak at around 4 connections simultaneously at 200 \gls{rps}, plateauing for the final measured load of 500 \gls{rps}. On the other hand, the CRUD application uses a median of 0 connections until 100 \gls{rps}. At this point, the graph starts rising, reaching a median of 4 at 500 \gls{rps} with a wide \gls{iqr} spanning from 2 to 10.

Size of the data store is presented in \autoref{fig:l3_data_size}. The CRUD application shows a linear growth, reaching a final size of 41MB. In contrast, ES-CQRS shows a sharper growth. At 100 RPS, its size is already around 56MB, at 200 RPS, its size is 227MB. Doubling the amount of requests increased the data store's size by a factor of almost 4! At 500 RPS, the size grows only slightly, reaching a final value of 232MB. Once again, this indicates a projection lag.

\begin{figure}[H]
    \centering
    \begin{subfigure}[b]{0.49\textwidth}
        \centering
        \includegraphics[width=\textwidth]{images/results/enrollment/enrollment-database-connections.png}
        \caption{Database Connections vs. Load}
        \label{fig:l3_db_connections}
    \end{subfigure}
    \hfill
    \begin{subfigure}[b]{0.49\textwidth}
        \centering
        \includegraphics[width=\textwidth]{images/results/enrollment/enrollment-data-size.png}
        \caption{Data Store Size vs. Load}
        \label{fig:l3_data_size}
    \end{subfigure}
    \caption[L3 Database Usage, load measured in \gls{rps}]{L3 Database Usage: Load measured in \gls{rps}. Detailed results in \ref{results:l3}.}
    \label{fig:l3_database}
\end{figure}

These values correspond to the following scalability metric for the CRUD application: $\psi(50, 100)\approx 2.7$; and the following scalability metric for the ES-CQRS application: $\psi(50, 100)\approx 0.2$.

\subsection{Read Performance}

When measuring read performance, no data is created during load generation. Therefore, the visualizations of data store size compared to load are replaced by simple tables, as each test creates the exact same amount of data.

\subsubsection{L4: Read Lectures for Student}
\label{sec:l4}

\texttt{GET} \texttt{/lectures} returns all lectures a student is enrolled or waitlisted in. \autoref{fig:l4_combined} presents client-side and server-side latencies for this endpoint using a logarithmic y-axis.

In the client-side graph (\autoref{fig:l4_latency_vs_load_client}), both applications maintain $latency\_p50$ and $latency\_p95$ of below 10ms up to 3000 \gls{rps}. At 3000 \gls{rps}, the ES-CQRS application exhibits an around 6x higher $latency\_p95$ than the CRUD app. Beyond 3000 \gls{rps}, though, the CRUD latencies overtake the ES-CQRS latencies. At 4000 \gls{rps}, the ES-CQRS application exhibits a $latency\_p95$ of around 370ms, which is around 1.7x faster than  the CRUD application at 620ms and a $latency\_p50$ of 20ms, around a 6x speedup compared to CRUD. Both applications violate \ref{slo-latency} beyond 3000 \gls{rps}.

It should be noted that at 4000 RPS, tests on both applications dropped some iterations with a $dropped\_iterations\_rate$ of roughly 2, meaning about 2 iterations, or about 0.05\% of iterations, were dropped per second. However, this metric exhibits high variance. Because the confidence intervals exceed the mean values, the results for this specific metric are not statistically significant (see \autoref{table:run-read-lectures_client}).

The server-side graph, presented in \autoref{fig:l6_latency_vs_load_server}, initially shows a similar pattern. Once exceeding 3000 \gls{rps}, the latencies also start increasing, however, the observed increase is not as strong as in the client-side latencies. At 4000 \gls{rps}, the ES-CQRS application shows a $latency\_p95$ of around 60ms, the CRUD application has a $latency\_p95$ of around 130ms.

\begin{figure}[H]
    \centering
    \begin{subfigure}[t]{0.49\textwidth}
        \centering
        \includegraphics[width=\textwidth]{images/results/read-lectures/read-lectures_latency-vs-load_client.png}
        \caption{Latency vs. Load measured on the client}
        \label{fig:l4_latency_vs_load_client}
    \end{subfigure}
    \hfill
    \begin{subfigure}[t]{0.49\textwidth}
        \centering
        \includegraphics[width=\textwidth]{images/results/read-lectures/read-lectures_latency-vs-load_server.png}
        \caption{Latency vs. Load measured on the server}
        \label{fig:l4_latency_vs_load_server}
    \end{subfigure}
    \caption[L4 Performance Metrics, load measured in \gls{rps}]{L4 Performance Metrics: Load measured in \gls{rps}. Detailed results in \ref{results:l4}.}
    \label{fig:l4_combined}
\end{figure}

Regarding CPU usage, pictured in \autoref{fig:l4_cpu_usage}, both applications exhibit a linear increase as the load intensifies. Between 25 and 3000 \gls{rps}, the ES-CQRS application consistently maintains a higher median CPU usage compared to the CRUD application, reaching a peak relative difference at 100 \gls{rps} where it is 1.9x higher. As the load reaches 4000 \gls{rps}, the two applications converge at approximately 60\% usage, with the CRUD application showing a slightly higher median of 61\% and a wider confidence interval.

\autoref{fig:l4_threadpool_usage} visualizes the application's threadpool usage. Both applications remain stable at a baseline of 10 threads for loads between 25 and 500 \gls{rps}. As the load increases to 1000 \gls{rps}, the ES-CQRS application begins using more threads, climbing to 47 threads at 2000 \gls{rps} and 158 threads at 3000 \gls{rps}, which is 5.1x higher than the CRUD application at that same point. The CRUD application maintains a lower threadpool usage until it reaches 3000 \gls{rps}. Finally, at the maximum load of 4000 \gls{rps}, both applications reach an identical ceiling of 200 threads.

\begin{figure}[H]
    \centering
    \begin{subfigure}[b]{0.49\textwidth}
        \centering
        \includegraphics[width=\textwidth]{images/results/read-lectures/read-lectures_cpu_usage.png}
        \caption{Resource Usage (\%) vs. Load}
        \label{fig:l4_cpu_usage}
    \end{subfigure}
    \hfill
    \begin{subfigure}[b]{0.49\textwidth}
        \centering
        \includegraphics[width=\textwidth]{images/results/read-lectures/read-lectures_threadpool_usage.png}
        \caption{Threadpool Usage vs. Load}
        \label{fig:l4_threadpool_usage}
    \end{subfigure}
    \caption[L4 Resource Usage, load measured in \gls{rps}]{L4 Resource Usage: Load measured in \gls{rps}. Detailed results in \ref{results:l4}.}
    \label{fig:l4_resource_usage_cpu}
\end{figure}

Active database connections are visualized in \autoref{fig:l4_db_connections}. Both the CRUD and ES-CQRS applications maintain a median of zero active connections for loads ranging from 25 to 1000 RPS. The CRUD application begins to utilize more connections first, reaching a median of 1 at 2000 RPS and 2 at 3000 RPS. During this same interval, the ES-CQRS application remains at a median of zero connections until 3000 RPS, where it records a median of 1, making it 2x lower than the CRUD application. At the maximum load of 4000 RPS, database connection usage increases for both, with the CRUD application reaching a median of 8 and the ES-CQRS application reaching a median of 6.

In terms of data store size for seed data, shown in \autoref{fig:l4_data_size}, the ES-CQRS application requires significantly more storage than the CRUD application. Specifically, the CRUD application uses around 9MB, while the ES-CQRS application uses around 17MB, a 2x higher storage value.

\begin{figure}[H]
    \centering
    \begin{subfigure}[c]{0.49\textwidth}
        \centering
        \includegraphics[width=\textwidth]{images/results/read-lectures/read-lectures_db_connections.png}
        \caption{Database Connections vs. Load}
        \label{fig:l4_db_connections}
    \end{subfigure}
    \hfill
    \begin{subfigure}[c]{0.49\textwidth}
        \centering
        \small
        \resizebox{\textwidth}{!}{
            \begin{tabular}{rrrcrrcc}
    \toprule
                 & \multicolumn{2}{c}{\textbf{CRUD (ms)}} &                   & \multicolumn{2}{c}{\textbf{ES-CQRS (ms)}} &                 &                                                            \\
    \cmidrule{2-3} \cmidrule{5-6}
    \textbf{RPS} & \textbf{Median}                        & \textbf{CI $\pm$} &                                           & \textbf{Median} & \textbf{CI $\pm$} & \textbf{Ratio} & \textbf{Significance} \\
    \midrule

    All          & 8.67MB                                 & 0.01              &                                           & 17.31MB         & 0.01              & 2.0x Higher    & ***                   \\

    \bottomrule
\end{tabular}
        }
        \caption{Data Store Size}
        \label{fig:l4_data_size}
    \end{subfigure}
    \caption[L4 Database Usage, load measured in \gls{rps}]{L4 Database Usage: Load measured in \gls{rps}. Detailed results in \ref{results:l4}.}
    \label{fig:l4_database}
\end{figure}

These values correspond to the following scalability metric for the CRUD application: $\psi(2000, 3000)\approx 0.6$; and the following scalability metric for the ES-CQRS application: $\psi(2000, 3000)\approx 0.3$. TODO: Add 3000 → 4000 RPS.

\subsubsection{L5: Read All Lectures (TODO)}
\label{sec:l5}

The \texttt{GET} \texttt{/lectures/all} endpoint returns a list of all lectures, including details.

Latency here. Most likely, the ES-CQRS app will be faster, which i have shown in local testing.

CPU Usage here.

Database connections here.

\subsubsection{L6: Get Credits}
\label{sec:l6}

The \texttt{GET} \texttt{/stats/credits} endpoint retrieves a student's total collected credits. \autoref{fig:l6_combined} presents the endpoint's client-side and server-side latencies using a logarithmic y-axis.

\autoref{fig:l6_latency_vs_load_client} presents the client-side latency under increasing load. It can be seen that the $latency\_p50$ remains below 5ms for both applications until around 2000 \gls{rps}.

Between 2000 \gls{rps} and 3000 \gls{rps}, a performance divergence occurs: the CRUD application's $latency\_p50$ and $latency\_p95$ increase to more than 1000ms, violating \ref{slo-latency} beyond 2000 \gls{rps}. On the other hand, the ES-CQRS application still exhibits a $latency\_p95$ of less than 10ms at 3000 \gls{rps}. This represents a speedup of around 200x. At 4000 \gls{rps}, the ES-CQRS application exhibits a $latency\_p95$ of around 80ms, a 20x speedup. Beyond 4000 \gls{rps}, the ES-CQRS app's latencies also begin to violate \ref{slo-latency}, reaching latencies around 1000ms.

It can be noted that starting at 3000 \gls{rps}, the CRUD application's latencies seem to reach a plateau, with identical observed latencies at 3000, 4000 and 5000 \gls{rps}. Meanwhile, the ES-CQRS application's latencies keep increasing up to a load of 5000 \gls{rps}. At this load, the ES-CQRS application's $latency\_p95$ is about 2.4x lower than the CRUD application's.

At 3000 RPS, the CRUD application has a $dropped\_iterations\_rate$ of around 360. Until 5000 RPS, the value reaches 1581. Dropping iterations artificially keeps queues on the server shorter, which could explain the observed latency plateau in CRUD. It should also be noted that at 5000 RPS, ES-CQRS begins to drop some iterations with $dropped\_iterations\_rate\approx38$.

The server latencies, displayed in \autoref{fig:l6_latency_vs_load_server}, differ from client latencies at higher loads. Starting at 3000 \gls{rps}, the measured latencies do not increase as strongly as observed on the client. At 5000 \gls{rps}, the $latency\_p95$ of the CRUD app resides around 260ms, while the ES-CQRS app has a $latency\_p95$ of around 50ms.

\begin{figure}[H]
    \centering
    \begin{subfigure}[t]{0.49\textwidth}
        \centering
        \includegraphics[width=\textwidth]{images/results/get-credits/get-credits-latency-vs-load__client.png}
        \caption{Latency vs. Load measured on the client}
        \label{fig:l6_latency_vs_load_client}
    \end{subfigure}
    \hfill
    \begin{subfigure}[t]{0.49\textwidth}
        \centering
        \includegraphics[width=\textwidth]{images/results/get-credits/get-credits-latency-vs-load__server.png}
        \caption{Latency vs. Load measured on the server}
        \label{fig:l6_latency_vs_load_server}
    \end{subfigure}
    \caption[L6 Performance Metrics, load measured in \gls{rps}]{L6 Performance Metrics: Load measured in \gls{rps}. Detailed results in \ref{results:l6}.}
    \label{fig:l6_combined}
\end{figure}

\autoref{fig:l6_cpu_usage} shows CPU usage of both applications from 25 to 4000 \gls{rps}. At the lowest load of 25 \gls{rps}, both applications exhibit a median $cpu\_usage$ close to 0. As the load increases to 1000 \gls{rps}, the CRUD median reaches around 21\% compared to $\approx$17\% for ES-CQRS. At 2000 \gls{rps}, the gap widens with CRUD at 45\% and ES-CQRS at 35\%, a 1.3x lower value for ES-CQRS. Both applications peak near 4000 \gls{rps}, where CRUD records 56\% and ES-CQRS records 57\%. The \gls{iqr} remains narrow for both applications across most data points.

Threadpool usage is visualized in \autoref{fig:l6_threadpool_usage}. From 25 to 500 \gls{rps}, both CRUD and ES-CQRS maintain a constant median threadpool usage of 10. At 1000 \gls{rps}, the values begin to diverge, with CRUD at 11 and ES-CQRS at 16. At 2000 \gls{rps}, CRUD usage rises to 36 while ES-CQRS is at 31. An increase occurs at 3000 \gls{rps}, where CRUD reaches a median of 200 utilized threads, while ES-CQRS reaches 123. By 4000 \gls{rps}, both applications reach a maximum median value of 200. The graph shows a wide \gls{iqr} for ES-CQRS at 3000 \gls{rps}. In contrast, CRUD has a very tight \gls{iqr} beyond 3000 \gls{rps}, indicating that the median of 200 utilized threads remains constant across most runs.

\begin{figure}[H]
    \centering
    \begin{subfigure}[b]{0.49\textwidth}
        \centering
        \includegraphics[width=\textwidth]{images/results/get-credits/get-credits-cpu-usage.png}
        \caption{Resource Usage (\%) vs. Load}
        \label{fig:l6_cpu_usage}
    \end{subfigure}
    \hfill
    \begin{subfigure}[b]{0.49\textwidth}
        \centering
        \includegraphics[width=\textwidth]{images/results/get-credits/get-credits-threadpool-usage.png}
        \caption{Threadpool Usage vs. Load}
        \label{fig:l6_threadpool_usage}
    \end{subfigure}
    \caption[L6 Resource Usage, load measured in \gls{rps}]{L6 Resource Usage: Load measured in \gls{rps}. Detailed results in \ref{results:l6}.}
    \label{fig:l6_resource_usage_cpu}
\end{figure}

\autoref{fig:l6_database} illustrates the median number of active database connections under increasing load. Until 500 \gls{rps}, both applications maintain a value of 0. Beyond 500 \gls{rps}, the CRUD application's value increases, reaching a ceiling of 10 active connections at 3000 \gls{rps}. In contrast, the ES-CQRS application's value starts rising linearly only at 2000 \gls{rps}, reaching its maximum of 2 active connections at 4000 \gls{rps}, with an \gls{iqr} spanning from 0 to 6.

The size of the data store is presented in \autoref{fig:l6_data_size}. At 19.6MB, the ES-CQRS application uses twice as much storage to store the seed data than the CRUD application, which uses around 9.6MB.

\begin{figure}[H]
    \centering
    \begin{subfigure}[c]{0.49\textwidth}
        \centering
        \includegraphics[width=\textwidth]{images/results/get-credits/get-credits-db-connections.png}
        \caption{Database Connections vs. Load}
        \label{fig:l6_db_connections}
    \end{subfigure}
    \hfill
    \begin{subfigure}[c]{0.49\textwidth}
        \centering
        \small
        \resizebox{\textwidth}{!}{
            \begin{tabular}{rrrcrrcc}
    \toprule
                 & \multicolumn{2}{c}{\textbf{CRUD (ms)}} &                   & \multicolumn{2}{c}{\textbf{ES-CQRS (ms)}} &                 &                                                            \\
    \cmidrule{2-3} \cmidrule{5-6}
    \textbf{RPS} & \textbf{Median}                        & \textbf{CI $\pm$} &                                           & \textbf{Median} & \textbf{CI $\pm$} & \textbf{Ratio} & \textbf{Significance} \\
    \midrule
    All          & 9.55                                   & 0.0               &                                           & 19.58           & 0.03              & 2.1x Higher    & ***                   \\
    \bottomrule
\end{tabular}

        }
        \caption{Data Store Size}
        \label{fig:l6_data_size}
    \end{subfigure}
    \caption[L6 Database Usage, load measured in \gls{rps}]{L6 Database Usage: Load measured in \gls{rps}. Detailed results in \ref{results:l6}.}
    \label{fig:l6_database}
\end{figure}

These values correspond to the following scalability metrics for the CRUD application: $\psi(1000, 2000)\approx 0.4$, $\psi(2000, 3000)\approx 0.00$; and the following scalability metrics for the ES-CQRS application: $\psi(1000, 2000)\approx 1.1$, $\psi(2000, 3000)\approx 0.3$.

\subsection{Time to Consistency / Freshness}

\ref{slo-freshness} defined a threshold of 100ms inside which all writes shall be reflected on the read-side. This "freshness" is measured in the following test.

\subsubsection{L7: Create Lecture, then Read}
\label{sec:l7}

This load test differs from others in the fact that each iteration executes 2 \gls{http} requests. The first request creates a lecture. After sleeping for 100ms --- the consistency threshold defined in \ref{slo-freshness} --- the script executes a request to \texttt{GET} the created lecture. If status code $404$ is returned, the write was not reflected in the read model in time. The rate of successful reads is recorded as a metric called $read\_visible\_rate$, presented in \autoref{fig:l7_read_visible_rate}. It can be seen that once exceeding 200 \gls{ips}, the ES-CQRS application failed to synchronize the read-side fast enough. \autoref{fig:l7_latency_vs_load} shows the latencies under varying loads using a logarithmic y-axis. $latency\_p50$ remains similar for both applications, however the ES-CQRS application's $latency\_p95$ increases with rising \gls{ips}, finally violating the threshold of 100ms defined in \ref{slo-latency} once exceeding 400 \gls{ips}.

The CRUD application consistently maintains a 100\% $read\_visible\_rate$.

\begin{figure}[H]
    \centering
    \begin{subfigure}[t]{0.49\textwidth}
        \centering
        \includegraphics[width=\textwidth]{images/results/time-to-consistency/get_lecture_latency-vs-load__client.png}
        \caption{Latency vs. Load, visualized for the subsequent GET request}
        \label{fig:l7_latency_vs_load}
    \end{subfigure}
    \hfill
    \begin{subfigure}[t]{0.49\textwidth}
        \centering
        \includegraphics[width=\textwidth]{images/results/time-to-consistency/get_lecture-read-visible-rate.png}
        \caption{Rate of visible reads}
        \label{fig:l7_read_visible_rate}
    \end{subfigure}
    \caption{L7 Performance Metrics, load measured in \gls{ips} with 2 requests per iteration. Detailed results in \ref{results:l7}}
    \label{fig:l7_combined}
\end{figure}

$cpu\_usage$ of both applications is plotted in \autoref{fig:l6_cpu_usage}. A linear increase in usage can be observed in both cases, with the ES-CQRS application consistenly utilizing more CPU. At the final tested load of 500 \gls{rps}, ES-CQRS reaches a value above 40\%, while CRUD utilizes around 16\%.

Threadpool usage of both applications is visualized in \autoref{fig:l6_threadpool_usage}. In both cases, the value remains very close to the base value of 10. Beyond 200 \gls{rps}, ES-CQRS starts to use more threads, reaching a utilization of 100 threads at 400 \gls{rps}, and reaching the ceiling of 200 threads at the final tested load of 500 \gls{rps}. CRUD remains below 20 utilized threads consistently.

\begin{figure}[H]
    \centering
    \begin{subfigure}[b]{0.49\textwidth}
        \centering
        \includegraphics[width=\textwidth]{images/results/time-to-consistency/time-to-consistency-CPU-Usage.png}
        \caption{Resource Usage (\%) vs. Load}
        \label{fig:l7_cpu_usage}
    \end{subfigure}
    \hfill
    \begin{subfigure}[b]{0.49\textwidth}
        \centering
        \includegraphics[width=\textwidth]{images/results/time-to-consistency/time-to-consistency-Threadpool-Usage.png}
        \caption{Threadpool Usage vs. Load}
        \label{fig:l7_threadpool_usage}
    \end{subfigure}
    \caption[L7 Resource Usage, load measured in \gls{rps}]{L7 Resource Usage: Load measured in \gls{rps}. Detailed results in \ref{results:l7}.}
    \label{fig:l7_resource_usage_cpu}
\end{figure}

Active database connections are visualized in \autoref{fig:l6_db_connections}. Both the CRUD and ES-CQRS applications maintain a median of zero active connections for loads up to 100 \gls{rps}, though the \gls{iqr} shows that ES-CQRS uses some database connections. At 200 RPS, both applications use a median of 1 database connections. CRUD stays at this level until 500 RPS, where it utilizes 2 connections. ES-CQRS, in contrast, begins claiming more connections, reaching a final value of 5 at 500 RPS.

In terms of data store size, shown in \autoref{fig:l6_data_size}, ES-CQRS again utilizes more storage. Both applications show a linear increase. The CRUD application shows a final data store size of around 46MB. This comes out to $\approx$1kB per created lecture. ES-CQRS reaches its peak at 400 \gls{rps} with a value of 60MB, around 1.5kB per lecture --- 1.6x higher than CRUD. However, at 500 \gls{rps}, the storage size drops slightly, indicating a projection lag.

\begin{figure}[H]
    \centering
    \begin{subfigure}[b]{0.49\textwidth}
        \centering
        \includegraphics[width=\textwidth]{images/results/time-to-consistency/time-to-consistency-DB-Connections.png}
        \caption{Database Connections vs. Load}
        \label{fig:l7_db_connections}
    \end{subfigure}
    \hfill
    \begin{subfigure}[b]{0.49\textwidth}
        \centering
        \includegraphics[width=\textwidth]{images/results/time-to-consistency/time-to-consistency-data_size.png}
        \caption{Data Store Size vs. Load}
        \label{fig:l7_data_size}
    \end{subfigure}
    \caption[L7 Database Usage, load measured in \gls{rps}]{L7 Database Usage: Load measured in \gls{rps}. Detailed results in \ref{results:l7}.}
    \label{fig:l7_database}
\end{figure}

These values correspond to the following scalability metric for the CRUD application: $\psi(300, 400)\approx 1.2$; and the following scalability metric for the ES-CQRS application: $\psi(300, 400)\approx 0.3$. It should be noted that the latency used for calculation was the sum of both requests' $latency\_p95$.

\subsection{Historic Reconstruction}
\label{sec:results-historic-reconstruction}

This subsection presents load tests which evaluate the performance of endpoints reconstructing historical state.

\subsubsection{L8: Grade History}
\label{sec:l8}

The endpoint \texttt{GET} \texttt{/stats/grades/history} returns the historical states of a grade. Its client-side and server-side latencies are presented in \autoref{fig:l6_combined}. Up to a load of 1000 \gls{rps}, both applications maintain a sub-10ms $latency\_p50$ and $latency\_p95$. Beyond that point, though, the ES-CQRS application's latencies increase sharply, reaching values above 1000ms. This marks the point at which the ES-CQRS application fails \ref{slo-latency}. The CRUD application, on the other hand, remains at a $latency\_p95$ of just around 2ms at the final tested load of 2000\gls{rps}.

The server latencies, visualized in \autoref{fig:l8_latency_vs_load_server}, paint a similar picture. The CRUD application's server-side response times stay at around 1ms, while the ES-CQRS application's response time increases to over 100ms at 2000 \gls{rps}.

\begin{figure}[H]
    \centering
    \begin{subfigure}[t]{0.49\textwidth}
        \centering
        \includegraphics[width=\textwidth]{images/results/grade-history/grade-history-latency-vs-load-CLIENT.png}
        \caption{Latency vs. Load measured on the client}
        \label{fig:l8_latency_vs_load_client}
    \end{subfigure}
    \hfill
    \begin{subfigure}[t]{0.49\textwidth}
        \centering
        \includegraphics[width=\textwidth]{images/results/grade-history/grade-history-latency-vs-load-SERVER.png}
        \caption{Latency vs. Load measured on the server}
        \label{fig:l8_latency_vs_load_server}
    \end{subfigure}
    \caption{L8 Performance Metrics, load measured in \gls{rps}. Detailed results in \ref{results:l8}}
    \label{fig:l8_combined}
\end{figure}

The applications' $cpu\_usage$ is depicted in \autoref{fig:l8_cpu_usage}. Both implementations exhibit a linear growth, with the CRUD application reaching its maximum at around 23\% and ES-CQRS reaching a maximum $cpu\_usage$ of 51\%, which is a 2.2x increase.

$tomcat\_threads$ is visualized in \autoref{fig:l8_threadpool_usage}. Both applications show a similar thread usage up to 1000 \gls{rps}, staying below a median value of 25. At 2000 \gls{rps}, the ES-CQRS application uses a median of 200 threads, with a wide \gls{iqr} reaching from around 40 to 200. The CRUD application stays at a median thread utilization of 25.

\begin{figure}[H]
    \centering
    \begin{subfigure}[b]{0.49\textwidth}
        \centering
        \includegraphics[width=\textwidth]{images/results/grade-history/grade-history-cpu-usage.png}
        \caption{Resource Usage (\%) vs. Load}
        \label{fig:l8_cpu_usage}
    \end{subfigure}
    \hfill
    \begin{subfigure}[b]{0.49\textwidth}
        \centering
        \includegraphics[width=\textwidth]{images/results/grade-history/grade-history-threadpool-usage.png}
        \caption{Threadpool Usage vs. Load}
        \label{fig:l8_threadpool_usage}
    \end{subfigure}
    \caption[L8 Resource Usage, load measured in \gls{rps}]{L8 Resource Usage: Load measured in \gls{rps}. Detailed results in \ref{results:l8}.}
    \label{fig:l8_resource_usage_cpu}
\end{figure}

\autoref{fig:l8_db_connections} shows the median number of utilized database connections --- $hikari\_connections$ --- under increasing load. Up to 1000 \gls{rps}, both applications show identical behavior, staying at a median of 0 until 500 \gls{rps}, before increasing to 1 at 1000 \gls{rps}. Afterward, however, the ES-CQRS application shows a sharp increase, reaching a median of 10 at 2000 \gls{rps} with a wide \gls{iqr} spannin from 4 to 10. At this load, the CRUD application still remains at a median of 1.

\autoref{tab:l8_data_size} shows the storage taken up by the test's seed data. The ES-CQRS application takes up twice the amount of storage at $\approx$19.7MB, while CRUD uses around 9.8MB.

\begin{figure}[H]
    \centering
    \begin{subfigure}[c]{0.49\textwidth}
        \centering
        \includegraphics[width=\textwidth]{images/results/grade-history/grade-history-db-connections.png}
        \caption{Database Connections vs. Load}
        \label{fig:l8_db_connections}
    \end{subfigure}
    \hfill
    \begin{subfigure}[c]{0.49\textwidth}
        \centering
        \small
        \resizebox{\columnwidth}{!}{
    \begin{tabular}{rrrcrrcc}
        \toprule
                     & \multicolumn{2}{c}{\textbf{CRUD (ms)}} &                   & \multicolumn{2}{c}{\textbf{ES-CQRS (ms)}} &                 &                                                            \\
        \cmidrule{2-3} \cmidrule{5-6}
        \textbf{RPS} & \textbf{Median}                        & \textbf{CI $\pm$} &                                           & \textbf{Median} & \textbf{CI $\pm$} & \textbf{Ratio} & \textbf{Significance} \\
        \midrule

        All          & 9.82                                   & 0.0               &                                           & 19.65           & 0.02              & 2.0x Higher    & ***                   \\

        \bottomrule
    \end{tabular}
}

        \caption{Data Store Size}
        \label{tab:l8_data_size}
    \end{subfigure}
    \caption[L8 Database Usage, load measured in \gls{rps}]{L8 Database Usage: Load measured in \gls{rps}. Detailed results in \ref{results:l8}.}
    \label{fig:l8_database}
\end{figure}

These values correspond to the following scalability metrics for the CRUD application: $\psi(500, 1000)\approx 2.0$, $\psi(1000, 2000) \approx 1.3$; and the following scalability metrics for the ES-CQRS application: $\psi(500, 1000) \approx 1.2$, $\psi(1000, 2000) \approx 0.00$.

\section{Static Analysis}
\label{sec:statc-analysis-results}

\acrshort{rq} 2 attempts to evaluate the architectural flexibility of the architectural styles \gls{crud} and \gls{es}-\gls{cqrs}. In \autoref{sec:flexibility-architectural-metrics}, several static analysis methods were established. This section presents results and visualizations for these metrics.

\subsection{Graphs}
\label{sec:static-graphs}

Boxplots are used to visualize the results of static analysis. They are a standardized way of displaying the distribution of data based on a five-number summary: minimum, first quartile (25th \gls{percentile}), \gls{median}, third quartile (75th \gls{percentile}), and maximum. The central box represents the \gls{iqr}, which encompasses the middle 50\% of data points. The \emph{whiskers} extend from the edges of the box to indicate the variability outside the upper and lower quartiles, showing the full range of the data excluding extreme values. Finally, individual points plotted beyond the whiskers are outliers, representing data points that fall further from the median than the rest of the population.

\subsection{Coupling Metrics}
\label{sec:results-coupling}

Afferent coupling ($C_a$, describing incoming connections) and Efferent coupling ($C_e$, describing outgoing connections), which were described in \autoref{sec:coupling-metrics}, are calculated by MetricsReloaded on a package-basis. The results are visualized in a boxplot. Package size influences the value of $C_a$ and $C_e$, which is why the results were normalized by class count. Therefore, the plots show this data:

\begin{equation}
    C_a\_norm = \frac{C_a}{Class\_count}
    \label{eq:c_a_normalized}
\end{equation}


\begin{equation}
    C_e\_norm = \frac{C_e}{Class\_count}
    \label{eq:c_e_normalized}
\end{equation}

\autoref{fig:combined_ca} presents the normalized $C_a$ per application. The CRUD application generally has higher Afferent coupling across its packages (a 33\% higher median and higher 75th \gls{percentile}). Most packages in the ES-CQRS architecture have low Afferent coupling, but some packages areas are more coupled than any package found in the CRUD app.

\begin{figure}[H]
    \centering
    \begin{subfigure}[c]{0.48\textwidth}
        \centering
        \includegraphics[width=\textwidth]{images/static-analysis/Ca_normalized_Afferent_Coupling_per_package.png}
        \caption{Boxplot distribution}
    \end{subfigure}
    \hfill
    \begin{subfigure}[c]{0.48\textwidth}
        \centering
        \small
        \resizebox{\textwidth}{!}{
            \begin{tabular}{lrrrrrr}
    \toprule
    \textbf{Application} & \textbf{Min} & \textbf{P25} & \textbf{Median} & \textbf{P75} & \textbf{Max} & \textbf{Outliers} \\
    \midrule
    CRUD                 & 0.0          & 0.0          & 3.0             & 8.3          & 14.4         & 0                 \\
    ES-CQRS              & 0.0          & 0.0          & 2.0             & 5.0          & 17.0         & 4                 \\
    \bottomrule
\end{tabular}
        }
        \caption{Descriptive statistics}
    \end{subfigure}
    \caption{Comparison of $C_a$ (Afferent coupling) by Application.}
    \label{fig:combined_ca}
\end{figure}

Regarding Efferent coupling, the two architectures show more similarity. The medians are almost equal at values of 2.1 (CRUD) and 2.5 (ES-CQRS). CRUD's \gls{iqr} is wider, ranging from 0 to around 13, while the ES-CQRS \gls{iqr} ranges from 0 to around 9. However, the ES-CQRS architecture displays more outliers (7, versus 2 in CRUD) reaching a value of 148, while the maximum value of the CRUD architecture is 92.

\begin{figure}[H]
    \centering
    \begin{subfigure}[c]{0.48\textwidth}
        \centering
        \includegraphics[width=\textwidth]{images/static-analysis/Ce_normalized_Efferent_Coupling_per_package.png}
        \caption{Boxplot distribution}
    \end{subfigure}
    \hfill
    \begin{subfigure}[c]{0.48\textwidth}
        \centering
        \small
        \resizebox{\textwidth}{!}{
            \begin{tabular}{lrrrrrr}
    \toprule
    \textbf{Application} & \textbf{Min} & \textbf{P25} & \textbf{Median} & \textbf{P75} & \textbf{Max} & \textbf{Outliers} \\
    \midrule
    CRUD                 & 0.0          & 0.0          & 2.1             & 12.9         & 92.0         & 2                 \\
    ES-CQRS              & 0.0          & 0.0          & 2.5             & 8.6          & 148.0        & 7                 \\
    \bottomrule
\end{tabular}
        }
        \caption{Descriptive statistics}
    \end{subfigure}
    \caption{Comparison of $C_e$ (Efferent coupling) by Application.}
    \label{fig:combined_ce}
\end{figure}

\subsection{Instability and Abstractness}

Instability $I$ is defined as the ratio of Efferent coupling to the total coupling of a package, as shown in \autoref{eq:instability}. Therefore, it can take values between 0 and 1. It is visualized using a boxplot in \autoref{fig:instability}. The plot highlights a wide \gls{iqr} and a median value around 0.5 to 0.6 for both architectures.

Abstractness $A$ measures the ratio of abstract classes and interfaces to the total number of classes in a package, as presented in \autoref{eq:abstractness}. Therefore, it can take values between 0 and 1. Its visualization in \autoref{fig:abstractness} shows that the typical package in both applications has an Abstractness of 0. The \gls{iqr} of the CRUD architecture reaches from 0 to 0.2; the \gls{iqr} of the ES-CQRS architecture reaches from 0 to $0.4$.

\autoref{fig:main-sequence} shows a scatter plot which visualizes $A$ and $I$, as well as the "Main Sequence" (gray diagonal line). The concept of the "Main Sequence" was explained in \autoref{sec:instability}. The distribution shows two prominent large clusters of the ES-CQRS architecture located at the corners of (0, 0) and (1, 0), while the remaining smaller points are scattered primarily in the lower half of the graph below the diagonal line. Generally, it can be seen that ES-CQRS has more packages than CRUD and is more abstract. The ES-CQRS architecture exhibits a slightly lower median Distance from the Main Sequence $D$ (\autoref{fig:distance-main-sequence-per-package}) with a value of around 0.4, opposed to around 0.5 for the CRUD architecture. However, both \glspl{iqr} for $D$ have a wide spread, with the ES-CQRS architecture spanning from 0.1 to almost~1.

\begin{figure}[H]
    \centering
    \begin{subfigure}[t]{0.3\textwidth}
        \centering
        \includegraphics[width=\textwidth]{images/static-analysis/Instability_per_package.png}
        \caption{Instability per package. Descriptive statistics in \autoref{table:instability}.}
        \label{fig:instability}
    \end{subfigure}
    \hfill
    \begin{subfigure}[t]{0.3\textwidth}
        \centering
        \includegraphics[width=\textwidth]{images/static-analysis/Abstractness_per_package.png}
        \caption{Abstractness per package. Descriptive statistics in \autoref{table:abstractness}.}
        \label{fig:abstractness}
    \end{subfigure}
    \hfill
    \begin{subfigure}[t]{0.3\textwidth}
        \centering
        \includegraphics[width=\textwidth]{images/static-analysis/Distance_main_sequene_per_package.png}
        \caption{Distance from the main sequence per package. Descriptive statistics in \autoref{table:distance-from-the-main-sequence}.}
        \label{fig:distance-main-sequence-per-package}
    \end{subfigure}
    \par\medskip
    \begin{subfigure}[t]{0.9\textwidth}
        \centering
        \includegraphics[width=\textwidth]{images/static-analysis/Martin_main_sequence_graph.png}
        \caption{Visualization of the Main Sequence}
        \label{fig:main-sequence}
    \end{subfigure}
    \caption[Comparison of Instability $I$ and Abstractness $A$.]{Comparison of Instability $I$ and Abstractness $A$. Detailed results in \autoref{appendix:instability-abstractness}.}
    \label{fig:combined_instability_abstractness}
\end{figure}

\subsection{Dependency Metrics}
\label{sec:results-dependency}

\autoref{table:dependency-metrics} outlined all dependency metrics recorded in the applications. In this section, these metrics are visualized using boxplots and descriptive statistics. Per-class results for the described dependency metrics are available in \autoref{appendix:dependency}.

\autoref{fig:combined_dpt} illustrates the distribution of $Dpt$, the number of classes depending directly on a class (\emph{dependents}), for both architectures. Both architectures have an equal median at 2. This indicates that in both cases, a typical class has two dependents. The range between the 25th to 75th \gls{percentile}, also called \gls{iqr} is indicated by the shaded areas. It differs for the two architectures. The CRUD architecture shows that 50\% of classes have 1 to 3 dependents, while the ES-CQRS architecture exhibits higher variability. Its IQR spans from 0 to 4. Notably, since the 25th percentile aligns with the minimum value of 0, at least 75\% of the classes in this architecture have 4 or fewer dependents.

\begin{figure}[H]
    \centering
    \begin{subfigure}[c]{0.48\textwidth}
        \centering
        \includegraphics[width=\textwidth]{images/static-analysis/class_dependencies_DPT.png}
        \caption{Boxplot distribution}
    \end{subfigure}
    \hfill
    \begin{subfigure}[c]{0.48\textwidth}
        \centering
        \small
        \resizebox{\textwidth}{!}{ % Resizes table to fit subfigure width
            \begin{tabular}{lrrrrrr}
    \toprule
    \textbf{Application} & \textbf{Min} & \textbf{P25} & \textbf{Median} & \textbf{P75} & \textbf{Max} & \textbf{Outliers} \\
    \midrule
    CRUD                 & 0.0          & 1.0          & 2.0             & 3.0          & 18.0         & 7                 \\
    ES-CQRS              & 0.0          & 0.0          & 2.0             & 4.0          & 10.0         & 0                 \\
    \bottomrule
\end{tabular}
        }
        \caption{Descriptive statistics}
    \end{subfigure}
    \caption{Comparison of $Dpt$ (direct dependants) by Application.}
    \label{fig:combined_dpt}
\end{figure}

$Dpt^*$, the transitive dependent count, is presented in \autoref{fig:combined_dpt_transitive}. Compared to $Dpt$, a larger difference between the architectures is visible. While the median values of both applications are still similar, with values between 2 and 4, the CRUD architecture's \gls{iqr} has a much wider range than the ES-CQRS architecture. 50\% of classes in the CRUD architecture have between 2 and 33 transitive dependents, while at least 75\% of the ES-CQRS architecture's classes have between 0 and 7 transitive dependents.

\begin{figure}[H]
    \centering
    \begin{subfigure}[c]{0.48\textwidth}
        \centering
        \includegraphics[width=\textwidth]{images/static-analysis/class_dependencies_DPT-transitive.png}
        \caption{Boxplot distribution}
    \end{subfigure}
    \hfill
    \begin{subfigure}[c]{0.48\textwidth}
        \centering
        \small
        \resizebox{\textwidth}{!}{ % Resizes table to fit subfigure width
            \begin{tabular}{lrrrrrr}
    \toprule
    \textbf{Application} & \textbf{Min} & \textbf{P25} & \textbf{Median} & \textbf{P75} & \textbf{Max} & \textbf{Outliers} \\
    \midrule
    CRUD                 & 0.0          & 1.8          & 4.0             & 33.0         & 45.0         & 0                 \\
    ES-CQRS              & 0.0          & 0.0          & 2.0             & 7.0          & 31.0         & 2                 \\
    \bottomrule
\end{tabular}
        }
        \caption{Descriptive statistics}
    \end{subfigure}
    \caption{Comparison of $Dpt^*$ (transitive dependants) by Application.}
    \label{fig:combined_dpt_transitive}
\end{figure}

\autoref{fig:combined_dcy} illustrates the distribution of $Dcy$, a metric representing the number of classes a given class directly depends on (\emph{dependencies}). Similar to the results for $Dpt$, the direct dependencies across both architectures exhibit similar values. CRUD and ES-CQRS architecture exhibit a median $Dcy$ of 2, respectively 1, with both \glspl{iqr} situated between 0 and 4. Both architectures feature several outliers, with individual classes reaching direct dependency counts of approximately 40.

\begin{figure}[H]
    \centering
    \begin{subfigure}[c]{0.48\textwidth}
        \centering
        \includegraphics[width=\textwidth]{images/static-analysis/class_dependencies_DCY.png}
        \caption{Boxplot distribution}
    \end{subfigure}
    \hfill
    \begin{subfigure}[c]{0.48\textwidth}
        \centering
        \small
        \resizebox{\textwidth}{!}{
            \begin{tabular}{lrrrrrr}
    \toprule
    \textbf{Application} & \textbf{Min} & \textbf{P25} & \textbf{Median} & \textbf{P75} & \textbf{Max} & \textbf{Outliers} \\
    \midrule
    CRUD                 & 0.0          & 0.8          & 2.0             & 4.0          & 41.0         & 7                 \\
    ES-CQRS              & 0.0          & 0.0          & 1.0             & 3.0          & 39.0         & 13                \\
    \bottomrule
\end{tabular}
        }
        \caption{Descriptive statistics}
    \end{subfigure}
    \caption{Comparison of $Dcy$ (direct dependencies) by Application.}
    \label{fig:combined_dcy}
\end{figure}

A more pronounced divergence is visible in the transitive dependencies, $Dcy^*$, presented in \autoref{fig:combined_dcy_transitive}. In the CRUD architecture, this distribution shows a much wider range, with the \gls{iqr} sitting between 1 and 26 transitive dependencies.

The ES-CQRS architecture exhibits a more concentrated distribution. Its \gls{iqr} is narrower, with at least 75\% of classes having between 0 and 3 transitive dependencies. While there are numerous outliers reaching more than 40 transitive dependencies, the primary distribution remains lower than that of the CRUD application.

\begin{figure}[H]
    \centering
    \begin{subfigure}[c]{0.48\textwidth}
        \centering
        \includegraphics[width=\textwidth]{images/static-analysis/class_dependencies_DCY-transitive.png}
        \caption{Boxplot distribution}
    \end{subfigure}
    \hfill
    \begin{subfigure}[c]{0.48\textwidth}
        \centering
        \small
        \resizebox{\textwidth}{!}{
            \begin{tabular}{lrrrrrr}
    \toprule
    \textbf{Application} & \textbf{Min} & \textbf{P25} & \textbf{Median} & \textbf{P75} & \textbf{Max} & \textbf{Outliers} \\
    \midrule
    CRUD                 & 0.0          & 0.8          & 12.5            & 26.0         & 84.0         & 2                 \\
    ES-CQRS              & 0.0          & 0.0          & 1.0             & 3.0          & 44.0         & 20                \\
    \bottomrule
\end{tabular}
        }
        \caption{Descriptive statistics}
    \end{subfigure}
    \caption{Comparison of $Dcy^*$ (transitive dependencies) by Application.}
    \label{fig:combined_dcy_transitive}
\end{figure}

$PDpt$, presented in \autoref{fig:combined_PDpt}, is a metric describing the number of packages transitively depending on a class. Both architectures have a median value of 1 and a P75 of 2, indicating that at least 75\% of classes have less than 2 transitive package dependents.

\begin{figure}[H]
    \centering
    \begin{subfigure}[c]{0.48\textwidth}
        \centering
        \includegraphics[width=\textwidth]{images/static-analysis/class_dependencies_PDpt.png}
        \caption{Boxplot distribution}
    \end{subfigure}
    \hfill
    \begin{subfigure}[c]{0.48\textwidth}
        \centering
        \small
        \resizebox{\textwidth}{!}{
            \begin{tabular}{lrrrrrr}
    \toprule
    \textbf{Application} & \textbf{Min} & \textbf{P25} & \textbf{Median} & \textbf{P75} & \textbf{Max} & \textbf{Outliers} \\
    \midrule
    CRUD                 & 0.0          & 1.0          & 1.0             & 2.0          & 7.0          & 9                 \\
    ES-CQRS              & 0.0          & 0.0          & 1.0             & 2.0          & 6.0          & 1                 \\
    \bottomrule
\end{tabular}
        }
        \caption{Descriptive statistics}
    \end{subfigure}
    \caption{Comparison of $PDpt$ (transitive package dependents) by Application.}
    \label{fig:combined_PDpt}
\end{figure}

The $PDcy$ metric describes the number of packages a class transitively depends on. It is presented in \autoref{fig:combined_PDcy}. Again, both architectures show similar values with a median of 1.5 for the CRUD architecture and 1 for ES-CQRS. The \gls{iqr} ranges from $\approx1$ to $\approx3$ for CRUD, and from 0 to 2 for ES-CQRS. The ES-CQRS architecture exhibits more outliers at 7 with a maximum value of 13, while the CRUD application shows one outlier depending on 8 packages.

\begin{figure}[H]
    \centering
    \begin{subfigure}[c]{0.48\textwidth}
        \centering
        \includegraphics[width=\textwidth]{images/static-analysis/class_dependencies_PDcy.png}
        \caption{Boxplot distribution}
    \end{subfigure}
    \hfill
    \begin{subfigure}[c]{0.48\textwidth}
        \centering
        \small
        \resizebox{\textwidth}{!}{
            \begin{tabular}{lrrrrrr}
    \toprule
    \textbf{Application} & \textbf{Min} & \textbf{P25} & \textbf{Median} & \textbf{P75} & \textbf{Max} & \textbf{Outliers} \\
    \midrule
    CRUD                 & 0.0          & 0.8          & 1.5             & 3.2          & 8.0          & 1                 \\
    ES-CQRS              & 0.0          & 0.0          & 1.0             & 2.0          & 13.0         & 7                 \\
    \bottomrule
\end{tabular}

        }
        \caption{Descriptive statistics}
    \end{subfigure}
    \caption{Comparison of $PDcy$ (transitive package dependencies) by Application.}
    \label{fig:combined_PDcy}
\end{figure}

\subsection{MOOD Metrics}
\label{sec:results-mood}

Results of the \acrfull{mood} suite, outlined in \autoref{sec:mood}, are presented in \autoref{fig:mood-results}. \gls{ahf} is the highest value for both architectures. Both architectures sit at around 95\%.

The CRUD architecture exhibits a \gls{mif} of around 16\%, higher than the ES-CQRS architecture at 1\%. With a value of 45\%, the ES-CQRS architecture's \gls{mhf} is higher than the 29\% for the CRUD architecture.

The CRUD architecture's \gls{cf} sits at around 12\%, while the ES-CQRS's architecture's \gls{cf} has a value of around 3\%.

The \gls{aif} of the CRUD architecture is at around 28\%, the ES-CQRS application reaches 8\%.

Generally, the CRUD Architecture covers a larger total surface area on the diagram, specifically showing higher values on the \gls{aif} and \gls{cf} axes compared to the ES-CQRS Architecture.

It is worth noting that the \gls{pf} of both architectures is not present in the diagram. This is due to the fact that the values exceed 100\%, with the CRUD architecture having a \gls{pf} of 360\%, and the ES-CQRS architecture having a \gls{pf} of 185\%. The \gls{pf} calculates the ratio of polymorphic situations to the maximum possible number of polymorphic situations, but the static analysis tool used does not take classes from libraries or external modules into consideration when computing the total possible number, which is why the values exceed 100\%.

\begin{figure}
    \centering
    \includegraphics[width=0.7\textwidth]{images/static-analysis/MOOD_spider_diagram.png}
    \caption{MOOD metrics presented in a spider diagram. Results in \ref{table:mood}}
    \label{fig:mood-results}
\end{figure}

% \subsection{Complexity Metrics}
% \label{sec:results-complexity}

% Commented out: Ergebnisse sind sehr nichtssagend. In Methodik ebenfalls auskommentiert. 

\subsection{Chidamber Kemerer Metrics}

In \autoref{sec:ck-metrics}, the \acrlong{ck} suite was described. Its results are presented in \autoref{fig:ck-results} using a spider diagram. As the metrics are calculated on a per-class basis, the values were normalized and aggregated using a median and a mean for visualization purposes.

\autoref{fig:ck_median} shows the median values of the metrics. The CRUD architecture's plot (blue) forms the larger shape. It reaches the highest point on the \gls{cbo} axis, with a value around 0.1, and the \gls{dit} axis with a value around 0.13. It also shows a distinct outward point on the \gls{rfc} axis.

The ES-CQRS architecture's plot forms a smaller shape nested mostly inside the blue area. It shows lower values than the CRUD architecture on the \gls{cbo} (0.07), \gls{wmc} (0), and \gls{rfc} (< 0.05) axes. It sits close to the CRUD architecture on the \gls{lcom} and \gls{dit} axes. While the median \gls{wmc} of ES-CQRS sits at a value of 0, the mean (\autoref{fig:ck_mean}) shows a value of about 0.02.

Both architectures exhibit a median \gls{noc} of 0. However, \autoref{fig:ck_mean} reveals that some polymorphism exists in the CRUD application.

\begin{figure}[H]
    \centering
    \begin{subfigure}[b]{0.49\textwidth}
        \centering
        \includegraphics[width=\textwidth]{images/static-analysis/Median-c-k-metrics.png}
        \caption{Median CK-Metrics}
        \label{fig:ck_median}
    \end{subfigure}
    \hfill
    \begin{subfigure}[b]{0.49\textwidth}
        \centering
        \includegraphics[width=\textwidth]{images/static-analysis/Mean-c-k-metrics.png}
        \caption{Mean CK Metrics}
        \label{fig:ck_mean}
    \end{subfigure}
    \caption[CK-Metrics presented in spider diagrams]{CK-Metrics presented in spider diagrams. Detailed results in \ref{appendix:ck-results}}
    \label{fig:ck-results}
\end{figure}


% !TeX root = ../main.tex
\chapter{Discussion}
\label{ch:discussion}

\section{Interpretation of Results}

TODO: explain this section. First, interpret results for each research question, finally give "recommendations" --- when is which architecture better suited?

\subsection{Performance and Scalability}
\label{sec:interpretation-performance}

The results obtained during load testing reveal a significant tradeoff between the two architectural patterns. The CRUD application excels in write throughput and resource efficiency, while the ES-CQRS application offers specialized advantages for complex read queries at the cost of higher overhead. In most write-heavy scenarios like \hyperref[sec:l1]{L1}, \hyperref[sec:l2]{L2}, and \hyperref[sec:l2]{L3}, the CRUD system maintains significantly lower latency and consumes fewer resources. In contrast, the ES-CQRS system frequently hits resource saturation --- reaching 200 threads and 10 database connections --- and violates latency thresholds as load increases. This divergence is especially visible in the enrollment test (L3), where the CRUD application manages much higher request volumes before performance degrades. Across nearly all tests, the ES-CQRS architecture requires more CPU and storage, often using double the disk space due to the overhead of storing both events and projections.

These differences are reflected in the calculated scalability metric $\psi$. While the CRUD application often exhibits values above 1.0, indicating efficient scaling where throughput increases outpace cost, the ES-CQRS application frequently shows values near zero under high load (e.g., $\psi \approx 0.01$ in L1). This metric serves as a useful tool for evaluating "productivity" by balancing throughput against resource costs like CPU and thread saturation. However, its usability is sensitive to how the "cost" function is weighted. For instance, the inclusion of a high penalty for projection lag (weighted at 3.0 in the model) heavily penalizes the ES-CQRS application's scalability score when consistency thresholds are missed.

However, the ES-CQRS application demonstrates its strength in specific read-heavy scenarios such as retrieving student credits (L6), where it provides a significant speedup compared to the CRUD application. While the CRUD application struggles with complex aggregations at high loads, reaching a scalability of $\psi \approx 0.00$ as its threadpool saturates, ES-CQRS benefits from pre-calculated read models that keep latencies low even at 3000 \gls{rps}. A similar advantage can be observed in L5 (reading a list of all lectures), because \gls{cqrs} is specifically designed to optimize these types of data retrievals. In these read-intensive contexts, the ES-CQRS application maintains a higher scalability factor (e.g., $\psi \approx 1.1$ at 2000 RPS) because the low latency and lack of complex DB joins compensate for its higher idle resource usage.

A critical drawback of \acrlong{es} and \gls{cqrs} is the projection lag observed in the freshness test (L7), where the application fails to reflect new writes on the read side within the required 100ms window (\ref{slo-freshness}) under heavy load. This eventual consistency means that even if reads have low latencies, they are not guaranteed to be up-to-date. Represented by the $read\_visible\_rate$ metric, this \emph{freshness} drops as low as $0.02$ for ES-CQRS at 400 \gls{ips}. This highlights a tradeoff for systems that require immediate data freshness: the mechanism that enables fast and scalable reads in ES-CQRS can also lead to stale data being served when the system is under pressure.

It should be noted that the results of L8 --- the historic reconstruction load test presented in \autoref{sec:l8} --- yielded a rather surprising result in terms of active database connections. Despite only using the database for two indexed ID lookups before streaming events from Axon's Event Store, the ES-CQRS application saturated the connection pool with a median of 10 active connections. This behavior is counter-intuitive compared to the CRUD implementation, which consistently maintains fewer active connections even though it performs more complex data fetching directly from the relational database through Envers.

TODO: higher Overhead in ES-CQRS in general.

TODO: Bottlenecks / headroom.

\subsection{Architectural Flexibility}
\label{sec:architectural-flexibility}

The static analysis results reveal distinct characteristics of the \gls{crud} and \gls{es}-\gls{cqrs} implementations. While metric suites like \gls{mood} and \acrlong{ck} provide a broad overview of code quality, the most insightful data for evaluating architectural flexibility results from the coupling and dependency metrics. These metrics provide a direct representation of how components interact and the extent to which a change in one area of the system propagates through others.

\subsubsection{Evaluation of Metric Utility}

The analysis indicates that transitive dependency metrics ($Dpt^*$ and $Dcy^*$), described in \autoref{sec:results-dependency}, are more useful than their direct counterparts ($Dpt$ and $Dcy$) for assessing the quality and coupling of an architecture. While the median values for direct dependencies are similar across both architectures, the \gls{es}-\gls{cqrs} approach shows a significantly narrower \gls{iqr} for transitive dependencies. In the \gls{crud} application, 50\% of classes have between 2 and 33 transitive dependents, whereas at least 75\% of \gls{es}-\gls{cqrs} classes have 7 or fewer. This suggests that the \gls{es}-\gls{cqrs} architecture more effectively isolates components, preventing a "rippling effect" of changes common in highly coupled architectures.

In contrast, the Distance from the Main Sequence ($D$), composed of Abstractness ($A$) and Instability ($I$), appears to be less accurate in this specific context. It is especially worth noting that a high Abstractness score does not inherently equate to a high level of \emph{functional abstraction}. For instance, the \gls{es}-\gls{cqrs} architecture utilizes many \gls{jpa} repositories on its Query side. These components are technically interfaces. This inflates the Abstractness score without necessarily having an architectural impact. Furthermore, these metrics are highly sensitive to the chosen package layout rather than the logic itself. A prime example is the \texttt{api} package in the \gls{es}-\gls{cqrs} application. It serves as a vital decoupling point between the Command and Query side, but because the classes are not abstract and highly depended upon, they appear "poor" according to traditional instability metrics despite their arguably high architectural value. Similarly, colocating Controller, Service and Repository classes in the same package in the CRUD application results in improved Instability values, even though the dependencies between classes are still present in the same way.

\subsubsection{Architectural Impact on Evolution and Scalability}

The structural differences between the two approaches have an impact regarding long-term flexibility and evolution. The \gls{crud} architecture displays higher \gls{cbo}, higher coupling ($C_a$, $C_e$) and more dependencies ($Dcy^*$, $Dpt^*$), and a larger "surface area" in the \gls{mood} metrics. This increased coupling implies that as the codebase grows, the complexity of making changes increases non-linearly, as each class is transitively linked to a larger portion of the system.

In contrast, the \gls{es}-\gls{cqrs} architecture demonstrates a structural advantage for scalability and evolution. The primary difference lies in the separation of concerns between writes and reads. By decoupling the Command and Query sides, the architecture maintains lower transitive coupling across the majority of the system.

Consequently, the \gls{es}-\gls{cqrs} approach likely provides a more seamless transition to horizontal scaling. Because the dependencies are already logically and physically partitioned (shown by the lower $Dcy^*$ and $PDcy$ values), the effort required to split the monolith into independent microservices is significantly reduced. This allows for specific parts of the system, such as high-traffic read models, to be scaled horizontally on separate infrastructure without requiring the entire application to be replicated. Additionally, Axon Framework provides location transparency, reducing the need to write additional boilerplate code when attempting to split the system into separate services. Therefore, while the \gls{es}-\gls{cqrs} architecture may have more outliers in package coupling, its fundamental structure provides the necessary isolation for the long-term scalability and independent evolution of system components.

\subsection{Traceability (TODO)}

TODO: schema evolution (maybe remove?), incorporate some "related work" findings.

The findings presented in this section are primarily derived from a synthesis of existing literature regarding system architecture and data integrity that was previously presented in \autoref{sec:basics-traceability-and-auditing} and \autoref{sec:traceability-related-work}. The performance data used to evaluate reconstruction efficiency was collected in L8 (\autoref{sec:l8}), but the qualitative comparison of reconstruction accuracy relies on the established literature.

The comparison between \gls{crud} and \gls{es}-\gls{cqrs} systems reveals a trade-off between the reliability of historical data and the speed of accessing it. In traditional \gls{crud} systems, the audit log acts as a secondary observer that records changes to a database. This architecture may suffer from the dual-write problem where the database update succeeds but the log entry fails. Consequently, the audit trail can diverge from reality, which may undermine the integrity required for strict legal compliance. Furthermore, \gls{crud} audit logs often capture the fact that data changed without preserving the specific business intent behind that change. Capturing and reconstructing intent requires additional measures in \gls{crud} systems, while in \gls{es}-\gls{cqrs}, events carry inherent metadata which contain intent.

Event-sourced systems address these accuracy concerns by treating the event stream as the sole source of truth. Because every state change is recorded as an immutable event, the system preserves the exact domain context and intent. This makes the reconstruction process deterministic and eliminates the risk of silent data divergence. Even if read-side projections do get out of sync, it is trivial to fix any bugs in their implementation and rebuild them afterward. However, the efficiency of historic reconstruction remains a challenge. While \gls{crud} systems can use indexed database tables and date filters to quickly retrieve specific historical snapshots, an \gls{es}-\gls{cqrs} system typically needs to replay the entire sequence of relevant events to reach a desired point in time.

Load testing results (L8) confirm that \gls{crud} architectures are significantly faster for simple history queries. Reconstructing a student's grade history, for example, proved more efficient in the \gls{crud} model, likely because it avoided the computational overhead of replaying event logs. While snapshots in \gls{es} can speed up the rehydration of Aggregates (current state), they are not applicable when replaying the full history for a time-travel query. Therefore, while \gls{es}-\gls{cqrs} provides superior accuracy and schema resilience for forensic auditing, the \gls{crud} approach remains more performant for frequent or high-volume historical lookups.

It should also be mentioned that frequent lookups of historic state are likely uncommon in typical applications. If a high volume of time-travel queries was a real use-case in an application, architects would most likely adopt an entirely different approach in storing current and historic state, such as time-series databases or creating specific database tables with validity ranges.

\subsection{Architectural Trade-offs and Recommendations (TODO)}

Abcdefg.

\section{Limitations of the Study}
\label{sec:limitations-of-the-study}

This research provides an empirical, quantifiable comparison between \gls{crud} and \gls{es}-\gls{cqrs} architectures. However, several technical and methodological limitations must be acknowledged when interpreting the results. One of these limitations is the infrastructure environment used for testing. Because both the load generation and the server-side components were hosted on \glspl{vm} running on the same physical machine, the observed network latency does not accurately reflect a distributed production environment. In a real-world scenario, services typically communicate across a physical network rather than on a single host. Particularly the ES-CQRS architecture involves a more complex chain of communication, including the command and query bus, aggregate handling, projections and event storage. In a distributed environment, the cumulative network "round trips" required for these steps would likely result in higher latencies than those recorded in this study's test setup.

Except for Axon Server's storage size, the server-side metrics were only collected from the SpringBoot application. Because the \gls{vm} also hosted PostgreSQL and Axon Server, the collected CPU metric ($process\_cpu\_usage$) does not capture the total system load. It is likely that the overall system CPU was fully saturated even when the server's CPU usage suggested remaining overhead. This is a threat to the validity of the performance data.

The accuracy of the benchmarking results was further affected by the testing configuration. In some load tests, a significant number of iterations was dropped at high RPS because the maximum number of \glspl{VU} in the k6 configuration was insufficient for the load. This means the results do not reflect the real performance under the respective load, as the actual load on the system was lower in these cases. Additionally, a more granular analysis would require measuring garbage collection pauses, memory allocation rates, and other system metrics. Furthermore, without profiling, the "root causes" of specific bottlenecks remain theoretical, as the collected data shows the \emph{results} of system stress rather than a detailed breakdown of internal execution delays. The performance analysis is also limited because the tests only varied the number of requests per second. Other factors, such as the size of the data being sent in POST requests or the volume of data fetched in queries, remained constant per test and were not evaluated.

The accuracy of server-side metrics collected through Actuator and Prometheus may be reduced under very high load because the metric collection in itself is bound to system resources and may start slowing down under high load. Also, the actuator endpoints themselves experience the same queueing delay as other requests.

Furthermore, although both applications were developed following industry standards and common best practices, it cannot be formally guaranteed that the implementations are entirely free of defects or suboptimal coding practices. Consequently, the observed performance metrics may be influenced by unidentified implementation errors rather than being representative of the underlying architecture.

Beyond the technical measurements, the given architectural evaluation is limited as it is simply a synthesis of static analysis. This study did not investigate the topic of schema evolution, which is the process of managing how data structures, such as events or database tables, change over time. In a long-running production system, the difficulty of evolving an event store's schema is an additional factor in the total cost of ownership and flexibility of an \gls{es}-\gls{cqrs} system. Similarly, managing complex database migrations in coupled \gls{crud} architectures without data-loss may introduce additional challenges.

Because the use case and the load patterns used in this study were artificially generated, they may not capture the unpredictable nature of real-world user behavior or specific business requirements.

While a load test was developed and executed for a simple history query (L8), the potential use-cases for time-travel queries are far wider. More complex historical queries involving several entities would have been a different challenge, and could have shown different results for the two architectures, both in implementation complexity and latencies.

Finally, the emphasis placed on certain architectural benefits, such as traceability or scalability, is inherently subjective. Different organizations or developers might prioritize \glspl{slo} or compliance requirements differently, which would alter the perceived value of one architecture over the other.

\section{Answering the Research Questions}

This section will provide a conclusive answer to the three sub-research questions, before providing a holistic answer to the main \glslink{rq}{research question}.

\subsection{RQ 1: Performance and Scalability (TODO)}
\label{sec:answering-rq-1}

\begin{quote}
    \textit{How do CRUD and ES-CQRS implementations perform under increasing load, and what are the resulting implications for system scalability and resource efficiency?}
\end{quote}

\subsection{RQ 2: Architectural Complexity and Flexibility (TODO)}
\label{sec:answering-rq-2}

\begin{quote}
    \textit{What are the fundamental structural differences between the two approaches, and how do these impact the long-term flexibility and evolution of the codebase?}
\end{quote}

\subsection{RQ 3: Historical Traceability (TODO)}
\label{sec:answering-rq-3}

\begin{quote}
    \textit{To what extent can CRUD and ES-CQRS systems accurately and efficiently reconstruct historical states to satisfy business intent and compliance requirements?}
\end{quote}

\subsection{Conclusion (TODO)}

Here, provide a final, holistic answer to the main research question.

\section{Further Work}
\label{sec:further-work}

\subsection{Improvements to the Method --- TODO remove this section?}

Fix limitations, basically

\begin{itemize}
    \item Run on the network
    \item Fix code: e.g. potential N+1 queries in CRUD, moving blocking lookups in ES-CQRS to external command handlers / interceptors
\end{itemize}

\subsection{Optimizations and Further Work}
\label{sec:optimizations-and-further-work}

This subsection outlines potential technical refinements for the existing implementations and identifies opportunities for future work. While the current results establish a baseline for both architectures, several optimization strategies could be applied to reduce the performance bottlenecks in both applications.

Beyond architectural changes, the underlying infrastructure and framework configurations offer room for improvement. While no tuning was done for this thesis on purpose to make "out-of-the-box" performance comparable, \autoref{tab:system-tuning} summarizes parameters that could be tuned to resolve the resource saturation and latency bottlenecks observed in the ES-CQRS and CRUD applications. Before doing any tuning, further profiling would be necessary to identify bottlenecks in function calls or queries.

\begin{table}[h!]
    \centering
    \small
    \begin{tabularx}{\linewidth}{lXX}
        \toprule
        \textbf{Keyword}          & \textbf{Action}                                                                                                                                               & \textbf{Goal}                                                                       \\ \midrule
        \textbf{Pool Tuning}      & Adjust Tomcat thread counts and HikariDB connection limits.                                                                                                   & Reduce queueing delays.                                                             \\ \addlinespace
        \textbf{Event Store}      & Tune Axon page sizes and set different snapshot thresholds.                                                                                                   & Reduce the number of expensive \gls{io} operations when reading long event streams. \\ \addlinespace
        \textbf{Reactivity}       & Implement Spring WebFlux\footnotemark[1] and reative JPA repositories\footnotemark[2].                                                                        & Free up worker threads while waiting for database or network responses.             \\ \addlinespace
        \textbf{Query Refinement} & Replace auto-generated Hibernate queries with custom JPQL queries, optimize further using \texttt{@BatchSize} and EAGER fetching options.                     & Eliminate inefficient queries and reduce the total number of database round-trips.  \\ \addlinespace
        \textbf{Serialization}    & Tune Jackson's serialization options. In ES-CQRS, JSON projections could be replaced by binary formats like Protobuf\footnotemark[3] or Kryo\footnotemark[4]. & Minimize CPU overhead, reduce projection storage size.                              \\ \addlinespace
        \textbf{Projections}      & Transition from JSON-based projections to flat, denormalized SQL-native tables.                                                                               & Eliminate serialization overhead and enable high-performance JDBC mapping.          \\
        \bottomrule
    \end{tabularx}
    \caption{Proposed System Tuning and Optimization Strategies}
    \label{tab:system-tuning}
\end{table}

\footnotetext[1]{\url{https://docs.spring.io/spring-framework/reference/web/webflux.html}}
\footnotetext[2]{\url{https://docs.spring.io/spring-data/jpa/reference/data-commons/api/java/org/springframework/data/repository/reactive/ReactiveCrudRepository.html}}
\footnotetext[3]{\url{https://protobuf.dev/}}
\footnotetext[4]{\url{https://github.com/EsotericSoftware/kryo}}

Additional performance gains could be achieved through the implementation of caching. Introducing a cache for frequently accessed student data could provide the CRUD application with read speeds comparable to \gls{cqrs} without the structural complexity of \acrlong{es}, though this introduces its own challenges regarding cache invalidation.

Furthermore, as this study evaluated both applications on a single machine, the assessment of scalability and architectural flexibility remains primarily theoretical. To provide empirical evidence for these scalability claims, it would be necessary to physically transition the systems into microservices and perform horizontal scaling. Only by implementing these structural changes can the practical performance limits and the true decoupling of the ES-CQRS pattern be fully validated. Additionally, the effort required to transform the monolithic structure into microservices would serve as a practical benchmark to validate the accuracy of the static analysis results.


\newpage
\printbibliography

\appendix

\chapter{Source Code}

The full source code for this thesis, including both applications, performance tests and markdown notes, is available at the following locations:

\begin{itemize}
    \item \url{https://gitlab.mi.hdm-stuttgart.de/lk224/thesis}
    \item \url{https://github.com/lukas-karsch/thesis}
\end{itemize}

The repository contains \texttt{README} files with instructions on how to launch the applications, execute load tests and how to reproduce the \acrshort{vm} environments.

\chapter{Load Testing Results}
\label{appendix:results}

This chapter of the appendix presents tables generated from the results during load testing. The \emph{speedup} represents the factor by which the ES-CQRS application differed from the CRUD application. The statistical significance of the performance differences was evaluated using the \emph{Mann-Whitney U} test. If its result is significant ($p \le 0.05$), that means that the difference in values are unlikely to be due to noise or randomness.

The results are reported using the probability thresholds defined in \autoref{table:significance}, where a lower p-value indicates a higher level of statistical significance.

\begin{table}[htp!]
    \small
    \centering
    \begin{tabularx}{0.7\linewidth}{cll}
        \textbf{Significance} & \textbf{$p$-value} & \textbf{Interpretation} \\
        \midrule
        ***                   & $p \le 0.001$      & Highly significant      \\
        **                    & $p \le 0.01$       & Very significant        \\
        *                     & $p \le 0.05$       & Significant             \\
        n.s.                  & $p > 0.05$         & Not significant         \\
        \bottomrule
    \end{tabularx}
    \caption{Significance thresholds}
    \label{table:significance}
\end{table}

\newpage
\section{L1: Create Course Simple}
\label{results:l1}

\begin{table}[H]
    \small \centering
    \resizebox{\columnwidth}{!}{
        \begin{tabular}{lcrrcrrcc}
            \toprule
                            &              & \multicolumn{2}{c}{\textbf{CRUD (ms)}} &                   & \multicolumn{2}{c}{\textbf{ES-CQRS (ms)}} &               &                                                              \\
            \cmidrule{3-4} \cmidrule{6-7}
            \textbf{Metric} & \textbf{RPS} & \textbf{Mean}                          & \textbf{CI $\pm$} &                                           & \textbf{Mean} & \textbf{CI $\pm$} & \textbf{Speedup} & \textbf{Significance} \\
            \midrule

            $latency\_avg$  & 25           & 4.47                                   & 0.0               &                                           & 7.84          & 0.0               & 1.8x Slower      & ***                   \\

            $latency\_p50$  & 25           & 4.17                                   & 0.0               &                                           & 7.02          & 0.0               & 1.7x Slower      & ***                   \\

            $latency\_p95$  & 25           & 6.0                                    & 0.0               &                                           & 12.22         & 0.0               & 2.0x Slower      & ***                   \\

            $latency\_p99$  & 25           & 14.44                                  & 0.0               &                                           & 19.94         & 0.0               & 1.4x Slower      & ***                   \\

            $latency\_avg$  & 50           & 3.43                                   & 0.0               &                                           & 6.3           & 0.0               & 1.8x Slower      & ***                   \\

            $latency\_p50$  & 50           & 3.15                                   & 0.0               &                                           & 5.48          & 0.0               & 1.7x Slower      & ***                   \\

            $latency\_p95$  & 50           & 5.26                                   & 0.0               &                                           & 10.7          & 0.0               & 2.0x Slower      & ***                   \\

            $latency\_p99$  & 50           & 6.86                                   & 0.0               &                                           & 14.83         & 0.0               & 2.2x Slower      & ***                   \\

            $latency\_avg$  & 100          & 2.66                                   & 0.0               &                                           & 4.87          & 0.0               & 1.8x Slower      & ***                   \\

            $latency\_p50$  & 100          & 2.41                                   & 0.0               &                                           & 4.18          & 0.0               & 1.7x Slower      & ***                   \\

            $latency\_p95$  & 100          & 4.27                                   & 0.0               &                                           & 9.17          & 0.0               & 2.1x Slower      & ***                   \\

            $latency\_p99$  & 100          & 6.08                                   & 0.0               &                                           & 13.8          & 0.0               & 2.3x Slower      & ***                   \\

            $latency\_avg$  & 200          & 2.12                                   & 0.0               &                                           & 4.09          & 0.0               & 1.9x Slower      & ***                   \\

            $latency\_p50$  & 200          & 1.91                                   & 0.0               &                                           & 3.32          & 0.0               & 1.7x Slower      & ***                   \\

            $latency\_p95$  & 200          & 3.38                                   & 0.0               &                                           & 8.69          & 0.0               & 2.6x Slower      & ***                   \\

            $latency\_p99$  & 200          & 5.3                                    & 0.0               &                                           & 14.43         & 0.0               & 2.7x Slower      & ***                   \\

            $latency\_avg$  & 500          & 1.96                                   & 0.0               &                                           & 7.46          & 0.0               & 3.8x Slower      & ***                   \\

            $latency\_p50$  & 500          & 1.82                                   & 0.0               &                                           & 3.87          & 0.0               & 2.1x Slower      & ***                   \\

            $latency\_p95$  & 500          & 2.67                                   & 0.0               &                                           & 25.78         & 0.0               & 9.6x Slower      & ***                   \\

            $latency\_p99$  & 500          & 5.6                                    & 0.0               &                                           & 48.48         & 0.01              & 8.7x Slower      & ***                   \\

            $latency\_avg$  & 1000         & 2.19                                   & 0.0               &                                           & 317.14        & 0.01              & 144.8x Slower    & ***                   \\

            $latency\_p50$  & 1000         & 1.99                                   & 0.0               &                                           & 6.61          & 0.0               & 3.3x Slower      & ***                   \\

            $latency\_p95$  & 1000         & 2.98                                   & 0.0               &                                           & 1582.87       & 0.03              & 530.6x Slower    & ***                   \\

            $latency\_p99$  & 1000         & 6.88                                   & 0.0               &                                           & 1890.43       & 0.03              & 274.7x Slower    & ***                   \\

            \bottomrule
        \end{tabular}
    }
    \caption{Statistical comparison for POST /courses simple (client), averaged out over at least 25 runs}
    \label{table:run-create-course-simple_client}
\end{table}
\begin{table}[H]
    \small \centering
    \resizebox{\columnwidth}{!}{
        \begin{tabular}{lcrrcrrcc}
            \toprule
                            &              & \multicolumn{2}{c}{\textbf{CRUD (ms)}} &                   & \multicolumn{2}{c}{\textbf{ES-CQRS (ms)}} &               &                                                              \\
            \cmidrule{3-4} \cmidrule{6-7}
            \textbf{Metric} & \textbf{RPS} & \textbf{Mean}                          & \textbf{CI $\pm$} &                                           & \textbf{Mean} & \textbf{CI $\pm$} & \textbf{Speedup} & \textbf{Significance} \\
            \midrule

            $latency\_avg$  & 25           & 3.46                                   & 0.0               &                                           & 6.69          & 0.0               & 1.9x Slower      & ***                   \\

            $latency\_p50$  & 25           & 3.25                                   & 0.0               &                                           & 6.09          & 0.0               & 1.9x Slower      & ***                   \\

            $latency\_p95$  & 25           & 5.29                                   & 0.0               &                                           & 11.03         & 0.0               & 2.1x Slower      & ***                   \\

            $latency\_p99$  & 25           & 12.7                                   & 0.0               &                                           & 17.39         & 0.0               & 1.4x Slower      & ***                   \\

            $latency\_avg$  & 50           & 2.65                                   & 0.0               &                                           & 5.34          & 0.0               & 2.0x Slower      & ***                   \\

            $latency\_p50$  & 50           & 2.45                                   & 0.0               &                                           & 4.87          & 0.0               & 2.0x Slower      & ***                   \\

            $latency\_p95$  & 50           & 4.27                                   & 0.0               &                                           & 9.63          & 0.0               & 2.3x Slower      & ***                   \\

            $latency\_p99$  & 50           & 5.63                                   & 0.0               &                                           & 13.29         & 0.0               & 2.4x Slower      & ***                   \\

            $latency\_avg$  & 100          & 2.05                                   & 0.0               &                                           & 4.09          & 0.0               & 2.0x Slower      & ***                   \\

            $latency\_p50$  & 100          & 1.88                                   & 0.0               &                                           & 3.46          & 0.0               & 1.8x Slower      & ***                   \\

            $latency\_p95$  & 100          & 3.46                                   & 0.0               &                                           & 8.22          & 0.0               & 2.4x Slower      & ***                   \\

            $latency\_p99$  & 100          & 5.15                                   & 0.0               &                                           & 12.34         & 0.0               & 2.4x Slower      & ***                   \\

            $latency\_avg$  & 200          & 1.74                                   & 0.0               &                                           & 3.57          & 0.0               & 2.1x Slower      & ***                   \\

            $latency\_p50$  & 200          & 1.62                                   & 0.0               &                                           & 2.9           & 0.0               & 1.8x Slower      & ***                   \\

            $latency\_p95$  & 200          & 2.79                                   & 0.0               &                                           & 7.91          & 0.0               & 2.8x Slower      & ***                   \\

            $latency\_p99$  & 200          & 4.67                                   & 0.0               &                                           & 13.04         & 0.0               & 2.8x Slower      & ***                   \\

            $latency\_avg$  & 500          & 1.66                                   & 0.0               &                                           & 6.84          & 0.0               & 4.1x Slower      & ***                   \\

            $latency\_p50$  & 500          & 1.57                                   & 0.0               &                                           & 3.52          & 0.0               & 2.2x Slower      & ***                   \\

            $latency\_p95$  & 500          & 2.29                                   & 0.0               &                                           & 24.19         & 0.0               & 10.6x Slower     & ***                   \\

            $latency\_p99$  & 500          & 5.14                                   & 0.0               &                                           & 46.07         & 0.01              & 9.0x Slower      & ***                   \\

            $latency\_avg$  & 1000         & 1.94                                   & 0.0               &                                           & 76.88         & 0.0               & 39.6x Slower     & ***                   \\

            $latency\_p50$  & 1000         & 1.82                                   & 0.0               &                                           & 6.17          & 0.0               & 3.4x Slower      & ***                   \\

            $latency\_p95$  & 1000         & 2.68                                   & 0.0               &                                           & 342.92        & 0.0               & 127.9x Slower    & ***                   \\

            $latency\_p99$  & 1000         & 6.31                                   & 0.0               &                                           & 436.11        & 0.0               & 69.2x Slower     & ***                   \\

            \bottomrule
        \end{tabular}
    }
    \caption{Statistical comparison for POST /courses without prerequisites (server), averaged out over at least 25 runs}
    \label{table:run-create-course-simple_server}
\end{table}
\begin{table}[H]
    \small
    \centering
    \begin{tabular}{rrrcrrcc}
        \toprule
                     & \multicolumn{2}{c}{\textbf{CRUD (ms)}} &                   & \multicolumn{2}{c}{\textbf{ES-CQRS (ms)}} &                 &                                                            \\
        \cmidrule{2-3} \cmidrule{5-6}
        \textbf{RPS} & \textbf{Median}                        & \textbf{CI $\pm$} &                                           & \textbf{Median} & \textbf{CI $\pm$} & \textbf{Ratio} & \textbf{Significance} \\
        \midrule

        25           & 0.0163                                 & 0.0012            &                                           & 0.0404          & 0.0019            & 2.5x Higher    & ***                   \\

        50           & 0.0182                                 & 0.0019            &                                           & 0.0506          & 0.0033            & 2.8x Higher    & ***                   \\

        100          & 0.0252                                 & 0.0013            &                                           & 0.0818          & 0.0062            & 3.2x Higher    & ***                   \\

        200          & 0.0322                                 & 0.0039            &                                           & 0.1368          & 0.0062            & 4.3x Higher    & ***                   \\

        500          & 0.0533                                 & 0.0007            &                                           & 0.2894          & 0.0028            & 5.4x Higher    & ***                   \\

        1000         & 0.1066                                 & 0.0013            &                                           & 0.4563          & 0.0025            & 4.3x Higher    & ***                   \\

        \bottomrule
    \end{tabular}
    \caption[Comparison of $cpu\_usage$ for POST /courses simple]{Statistical comparison of $cpu\_usage$ for POST /courses simple, aggregated over at least 25 runs}
    \label{table:run-create-course-simple-aggregated-CPU_Usage}
\end{table}

\begin{table}[H]
    \small
    \centering
    \resizebox{\columnwidth}{!}{
        \begin{tabular}{rrrcrrcc}
            \toprule
                         & \multicolumn{2}{c}{\textbf{CRUD (ms)}} &                   & \multicolumn{2}{c}{\textbf{ES-CQRS (ms)}} &                 &                                                            \\
            \cmidrule{2-3} \cmidrule{5-6}
            \textbf{RPS} & \textbf{Median}                        & \textbf{CI $\pm$} &                                           & \textbf{Median} & \textbf{CI $\pm$} & \textbf{Ratio} & \textbf{Significance} \\
            \midrule

            25           & 10.0                                   & 0.0               &                                           & 10.0            & 0.0               & 1.0x Lower     & n.s.                  \\

            50           & 10.0                                   & 0.0               &                                           & 10.0            & 0.0               & 1.0x Lower     & n.s.                  \\

            100          & 10.0                                   & 0.0               &                                           & 10.0            & 0.0               & 1.0x Lower     & n.s.                  \\

            200          & 10.0                                   & 0.0               &                                           & 10.0            & 0.0               & 1.0x Lower     & ***                   \\

            500          & 11.0                                   & 0.0               &                                           & 38.0            & 3.5               & 3.5x Higher    & ***                   \\

            1000         & 23.0                                   & 0.5               &                                           & 200.0           & 0.0               & 8.7x Higher    & ***                   \\

            \bottomrule
        \end{tabular}
    }
    \caption{Statistical comparison of $tomcat\_threads$ for GET /lectures, aggregated over at least 25 runs}
    \label{table:run-create-course-simple-aggregated-Threadpool_Usage}
\end{table}
\begin{table}[H]
    \small
    \centering
    \resizebox{\columnwidth}{!}{
        \begin{tabular}{rrrcrrcc}
            \toprule
                         & \multicolumn{2}{c}{\textbf{CRUD (ms)}} &                   & \multicolumn{2}{c}{\textbf{ES-CQRS (ms)}} &                 &                                                            \\
            \cmidrule{2-3} \cmidrule{5-6}
            \textbf{RPS} & \textbf{Median}                        & \textbf{CI $\pm$} &                                           & \textbf{Median} & \textbf{CI $\pm$} & \textbf{Ratio} & \textbf{Significance} \\
            \midrule

            25           & 0.0                                    & 0.0               &                                           & 0.0             & 0.0               & $N/A$          & ***                   \\

            50           & 0.0                                    & 0.0               &                                           & 0.0             & 0.0               & $N/A$          & ***                   \\

            100          & 0.0                                    & 0.0               &                                           & 0.0             & 0.0               & $N/A$          & ***                   \\

            200          & 0.0                                    & 0.0               &                                           & 1.0             & 0.0               & $N/A$          & ***                   \\

            500          & 1.0                                    & 0.0               &                                           & 4.0             & 0.5               & 4.0x Higher    & ***                   \\

            1000         & 2.0                                    & 0.0               &                                           & 9.0             & 0.5               & 4.5x Higher    & ***                   \\

            \bottomrule
        \end{tabular}
    }
    \caption{Statistical comparison of $hikari\_connections$ for GET /lectures, aggregated over at least 25 runs}
    \label{table:run-create-course-simple-aggregated-Database_Connections}
\end{table}
\begin{table}[H]
    \small
    \centering

    \begin{tabular}{rrrcrrcc}
        \toprule
                     & \multicolumn{2}{c}{\textbf{CRUD (ms)}} &                   & \multicolumn{2}{c}{\textbf{ES-CQRS (ms)}} &                 &                                                            \\
        \cmidrule{2-3} \cmidrule{5-6}
        \textbf{RPS} & \textbf{Median}                        & \textbf{CI $\pm$} &                                           & \textbf{Median} & \textbf{CI $\pm$} & \textbf{Ratio} & \textbf{Significance} \\
        \midrule

        25           & 9.51                                   & 0.0               &                                           & 18.95           & 0.01              & 2.0x Higher    & ***                   \\

        50           & 11.05                                  & 0.0               &                                           & 21.28           & 0.0               & 1.9x Higher    & ***                   \\

        100          & 14.12                                  & 0.0               &                                           & 25.87           & 0.01              & 1.8x Higher    & ***                   \\

        200          & 20.26                                  & 0.0               &                                           & 35.14           & 0.01              & 1.7x Higher    & ***                   \\

        500          & 38.69                                  & 0.01              &                                           & 60.57           & 0.19              & 1.6x Higher    & ***                   \\

        1000         & 69.33                                  & 0.02              &                                           & 52.91           & 0.32              & 1.3x Lower     & ***                   \\

        \bottomrule
    \end{tabular}

    \caption{Statistical comparison of Data Store Size (MB) for run-create-course-simple, aggregated over at least 25 runs}
    \label{table:run-create-course-simple-aggregated-datastore-size}
\end{table}


\newpage
\section{L2: Create Course Prerequisites}
\label{results:l2}

\begin{table}[H]
\small \centering
\resizebox{\columnwidth}{!}{
\begin{tabular}{lcrrcrrcc}
\toprule
& & \multicolumn{2}{c}{\textbf{CRUD (ms)}} & & \multicolumn{2}{c}{\textbf{ES-CQRS (ms)}} & & \\
\cmidrule{3-4} \cmidrule{6-7}
\textbf{Metric} & \textbf{RPS} & \textbf{Mean} & \textbf{CI $\pm$} & & \textbf{Mean} & \textbf{CI $\pm$} & \textbf{Speedup} & \textbf{Significance} \\
\midrule

    $dropped\_iterations\_rate$ & 25 & 0.0 & 0.0 & & 0.0 & 0.0 &
    NaN & n.s. \\

    $latency\_avg$ & 25 & 5.12 & 0.0 & & 7.99 & 0.0 &
    1.6x Slower & *** \\

    $latency\_p50$ & 25 & 4.71 & 0.0 & & 7.09 & 0.0 &
    1.5x Slower & *** \\

    $latency\_p95$ & 25 & 7.39 & 0.0 & & 12.91 & 0.0 &
    1.7x Slower & *** \\

    $latency\_p99$ & 25 & 16.3 & 0.0 & & 20.38 & 0.0 &
    1.3x Slower & *** \\

    $dropped\_iterations\_rate$ & 50 & 0.0 & 0.0 & & 0.0 & 0.0 &
    NaN & n.s. \\

    $latency\_avg$ & 50 & 3.95 & 0.0 & & 6.44 & 0.0 &
    1.6x Slower & *** \\

    $latency\_p50$ & 50 & 3.59 & 0.0 & & 5.54 & 0.0 &
    1.5x Slower & *** \\

    $latency\_p95$ & 50 & 6.35 & 0.0 & & 11.45 & 0.0 &
    1.8x Slower & *** \\

    $latency\_p99$ & 50 & 8.5 & 0.0 & & 15.91 & 0.0 &
    1.9x Slower & *** \\

    $dropped\_iterations\_rate$ & 100 & 0.0 & 0.0 & & 0.0 & 0.0 &
    NaN & n.s. \\

    $latency\_avg$ & 100 & 3.11 & 0.0 & & 5.02 & 0.0 &
    1.6x Slower & *** \\

    $latency\_p50$ & 100 & 2.8 & 0.0 & & 4.26 & 0.0 &
    1.5x Slower & *** \\

    $latency\_p95$ & 100 & 5.27 & 0.0 & & 9.89 & 0.0 &
    1.9x Slower & *** \\

    $latency\_p99$ & 100 & 7.48 & 0.0 & & 14.97 & 0.0 &
    2.0x Slower & *** \\

    $dropped\_iterations\_rate$ & 200 & 0.0 & 0.0 & & 0.0 & 0.0 &
    NaN & n.s. \\

    $latency\_avg$ & 200 & 2.54 & 0.0 & & 4.33 & 0.0 &
    1.7x Slower & *** \\

    $latency\_p50$ & 200 & 2.3 & 0.0 & & 3.4 & 0.0 &
    1.5x Slower & *** \\

    $latency\_p95$ & 200 & 4.16 & 0.0 & & 9.84 & 0.0 &
    2.4x Slower & *** \\

    $latency\_p99$ & 200 & 6.43 & 0.0 & & 15.98 & 0.0 &
    2.5x Slower & *** \\

    $dropped\_iterations\_rate$ & 500 & 0.0 & 0.0 & & 0.0 & 0.0 &
    NaN & n.s. \\

    $latency\_avg$ & 500 & 2.38 & 0.0 & & 10.46 & 0.0 &
    4.4x Slower & *** \\

    $latency\_p50$ & 500 & 2.13 & 0.0 & & 4.16 & 0.0 &
    2.0x Slower & *** \\

    $latency\_p95$ & 500 & 3.73 & 0.0 & & 38.78 & 0.0 &
    10.4x Slower & *** \\

    $latency\_p99$ & 500 & 6.61 & 0.0 & & 79.76 & 0.01 &
    12.1x Slower & *** \\

    $dropped\_iterations\_rate$ & 1000 & 0.0 & 0.0 & & 61.09 & 1.8 &
    NaN & *** \\

    $latency\_avg$ & 1000 & 2.74 & 0.0 & & 375.46 & 0.01 &
    137.1x Slower & *** \\

    $latency\_p50$ & 1000 & 2.38 & 0.0 & & 9.54 & 0.0 &
    4.0x Slower & *** \\

    $latency\_p95$ & 1000 & 5.14 & 0.0 & & 1651.79 & 0.02 &
    321.5x Slower & *** \\

    $latency\_p99$ & 1000 & 8.46 & 0.0 & & 1989.37 & 0.02 &
    235.1x Slower & *** \\

\bottomrule
\end{tabular}
}
\caption{Statistical comparison for POST /courses (client), averaged out over at least 25 runs}
\label{table:run-create-course-prerequisites_client}
\end{table}
\begin{table}[!ht]
    \small \centering
    \resizebox{\columnwidth}{!}{
        \begin{tabular}{lcrrcrrcc}
            \toprule
                            &              & \multicolumn{2}{c}{\textbf{CRUD (ms)}} &                   & \multicolumn{2}{c}{\textbf{ES-CQRS (ms)}} &               &                                                              \\
            \cmidrule{3-4} \cmidrule{6-7}
            \textbf{Metric} & \textbf{RPS} & \textbf{Mean}                          & \textbf{CI $\pm$} &                                           & \textbf{Mean} & \textbf{CI $\pm$} & \textbf{Speedup} & \textbf{Significance} \\
            \midrule

            $latency\_avg$  & 25           & 4.14                                   & 0.0               &                                           & 6.62          & 0.0               & 1.6x Slower      & ***                   \\

            $latency\_p50$  & 25           & 3.85                                   & 0.0               &                                           & 6.03          & 0.0               & 1.6x Slower      & ***                   \\

            $latency\_p95$  & 25           & 6.53                                   & 0.0               &                                           & 10.98         & 0.0               & 1.7x Slower      & ***                   \\

            $latency\_p99$  & 25           & 14.04                                  & 0.0               &                                           & 17.9          & 0.0               & 1.3x Slower      & ***                   \\

            $latency\_avg$  & 50           & 3.27                                   & 0.0               &                                           & 5.29          & 0.0               & 1.6x Slower      & ***                   \\

            $latency\_p50$  & 50           & 3.05                                   & 0.0               &                                           & 4.83          & 0.0               & 1.6x Slower      & ***                   \\

            $latency\_p95$  & 50           & 5.5                                    & 0.0               &                                           & 9.57          & 0.0               & 1.7x Slower      & ***                   \\

            $latency\_p99$  & 50           & 7.22                                   & 0.0               &                                           & 13.31         & 0.0               & 1.8x Slower      & ***                   \\

            $latency\_avg$  & 100          & 2.69                                   & 0.0               &                                           & 4.01          & 0.0               & 1.5x Slower      & ***                   \\

            $latency\_p50$  & 100          & 2.56                                   & 0.0               &                                           & 3.42          & 0.0               & 1.3x Slower      & ***                   \\

            $latency\_p95$  & 100          & 4.58                                   & 0.0               &                                           & 8.05          & 0.0               & 1.8x Slower      & ***                   \\

            $latency\_p99$  & 100          & 6.56                                   & 0.0               &                                           & 12.14         & 0.0               & 1.8x Slower      & ***                   \\

            $latency\_avg$  & 200          & 2.29                                   & 0.0               &                                           & 3.48          & 0.0               & 1.5x Slower      & ***                   \\

            $latency\_p50$  & 200          & 2.03                                   & 0.0               &                                           & 2.83          & 0.0               & 1.4x Slower      & ***                   \\

            $latency\_p95$  & 200          & 3.88                                   & 0.0               &                                           & 7.66          & 0.0               & 2.0x Slower      & ***                   \\

            $latency\_p99$  & 200          & 5.72                                   & 0.0               &                                           & 12.55         & 0.0               & 2.2x Slower      & ***                   \\

            $latency\_avg$  & 500          & 2.44                                   & 0.0               &                                           & 6.23          & 0.0               & 2.6x Slower      & ***                   \\

            $latency\_p50$  & 500          & 1.99                                   & 0.0               &                                           & 3.53          & 0.0               & 1.8x Slower      & ***                   \\

            $latency\_p95$  & 500          & 5.55                                   & 0.0               &                                           & 21.29         & 0.0               & 3.8x Slower      & ***                   \\

            $latency\_p99$  & 500          & 7.88                                   & 0.0               &                                           & 38.05         & 0.0               & 4.8x Slower      & ***                   \\

            $latency\_avg$  & 1000         & 3.08                                   & 0.0               &                                           & 75.82         & 0.0               & 24.6x Slower     & ***                   \\

            $latency\_p50$  & 1000         & 2.27                                   & 0.0               &                                           & 6.12          & 0.0               & 2.7x Slower      & ***                   \\

            $latency\_p95$  & 1000         & 8.66                                   & 0.0               &                                           & 329.44        & 0.0               & 38.0x Slower     & ***                   \\

            $latency\_p99$  & 1000         & 13.28                                  & 0.0               &                                           & 422.97        & 0.0               & 31.9x Slower     & ***                   \\

            \bottomrule
        \end{tabular}
    }
    \caption{Statistical comparison for POST /courses with prerequisites (server), averaged out over at least 25 runs}
    \label{table:run-create-course-prerequisites_server}
\end{table}
\begin{table}[H]
    \small
    \centering
    \begin{tabular}{rrrcrrcc}
        \toprule
                     & \multicolumn{2}{c}{\textbf{CRUD (ms)}} &                   & \multicolumn{2}{c}{\textbf{ES-CQRS (ms)}} &                 &                                                            \\
        \cmidrule{2-3} \cmidrule{5-6}
        \textbf{RPS} & \textbf{Median}                        & \textbf{CI $\pm$} &                                           & \textbf{Median} & \textbf{CI $\pm$} & \textbf{Ratio} & \textbf{Significance} \\
        \midrule

        25           & 0.02                                   & 0.0019            &                                           & 0.0391          & 0.0024            & 2.0x Higher    & ***                   \\

        50           & 0.0237                                 & 0.0019            &                                           & 0.0544          & 0.0032            & 2.3x Higher    & ***                   \\

        100          & 0.0328                                 & 0.0031            &                                           & 0.0913          & 0.0046            & 2.8x Higher    & ***                   \\

        200          & 0.0404                                 & 0.0051            &                                           & 0.1531          & 0.0061            & 3.8x Higher    & ***                   \\

        500          & 0.0677                                 & 0.0006            &                                           & 0.3424          & 0.0036            & 5.1x Higher    & ***                   \\

        1000         & 0.1423                                 & 0.0012            &                                           & 0.4594          & 0.0025            & 3.2x Higher    & ***                   \\

        \bottomrule
    \end{tabular}
    \caption[Comparison of $cpu\_usage$ for POST /courses with prerequisites]{Statistical comparison of $cpu\_usage$ for POST /courses with prerequisites, aggregated over at least 25 runs}
    \label{table:run-create-course-prerequisites-aggregated-CPU_Usage}
\end{table}

\begin{table}[H]
    \small
    \centering
    \begin{tabular}{rrrcrrcc}
        \toprule
                     & \multicolumn{2}{c}{\textbf{CRUD (ms)}} &                   & \multicolumn{2}{c}{\textbf{ES-CQRS (ms)}} &                 &                                                            \\
        \cmidrule{2-3} \cmidrule{5-6}
        \textbf{RPS} & \textbf{Median}                        & \textbf{CI $\pm$} &                                           & \textbf{Median} & \textbf{CI $\pm$} & \textbf{Ratio} & \textbf{Significance} \\
        \midrule

        25           & 10.0                                   & 0.0               &                                           & 10.0            & 0.0               & 1.0x Lower     & n.s.                  \\

        50           & 10.0                                   & 0.0               &                                           & 10.0            & 0.0               & 1.0x Lower     & n.s.                  \\

        100          & 10.0                                   & 0.0               &                                           & 10.0            & 0.0               & 1.0x Lower     & n.s.                  \\

        200          & 10.0                                   & 0.0               &                                           & 10.0            & 0.0               & 1.0x Lower     & n.s.                  \\

        500          & 12.0                                   & 0.0               &                                           & 61.0            & 2.5               & 5.1x Higher    & ***                   \\

        1000         & 25.0                                   & 0.5               &                                           & 200.0           & 0.0               & 8.0x Higher    & ***                   \\

        \bottomrule
    \end{tabular}
    \caption[Comparison of $tomcat\_threads$ for POST /courses with prerequisites]{Statistical comparison of $tomcat\_threads$ for POST /courses with prerequisites, aggregated over at least 25 runs}
    \label{table:run-create-course-prerequisites-aggregated-Threadpool_Usage}
\end{table}
\begin{table}[H]
    \small
    \centering
    \begin{tabular}{rrrcrrcc}
        \toprule
                     & \multicolumn{2}{c}{\textbf{CRUD (ms)}} &                   & \multicolumn{2}{c}{\textbf{ES-CQRS (ms)}} &                 &                                                            \\
        \cmidrule{2-3} \cmidrule{5-6}
        \textbf{RPS} & \textbf{Median}                        & \textbf{CI $\pm$} &                                           & \textbf{Median} & \textbf{CI $\pm$} & \textbf{Ratio} & \textbf{Significance} \\
        \midrule

        25           & 0.0                                    & 0.0               &                                           & 0.0             & 0.0               & $N/A$          & ***                   \\

        50           & 0.0                                    & 0.0               &                                           & 0.0             & 0.0               & $N/A$          & ***                   \\

        100          & 0.0                                    & 0.0               &                                           & 0.0             & 0.0               & $N/A$          & ***                   \\

        200          & 0.0                                    & 0.0               &                                           & 1.0             & 0.5               & $N/A$          & ***                   \\

        500          & 1.0                                    & 0.0               &                                           & 6.0             & 0.5               & 6.0x Higher    & ***                   \\

        1000         & 2.0                                    & 0.0               &                                           & 10.0            & 0.0               & 5.0x Higher    & ***                   \\

        \bottomrule
    \end{tabular}
    \caption[Comparison of $hikari\_connections$ for POST /courses with prerequisites]{Statistical comparison of $hikari\_connections$ for POST /courses with prerequisites, aggregated over at least 25 runs}
    \label{table:run-create-course-prerequisites-aggregated-Database_Connections}
\end{table}
\begin{table}[H]
    \small
    \centering
    \resizebox{\columnwidth}{!}{
        \begin{tabular}{rrrcrrcc}
            \toprule
                         & \multicolumn{2}{c}{\textbf{CRUD (ms)}} &                   & \multicolumn{2}{c}{\textbf{ES-CQRS (ms)}} &                 &                                                            \\
            \cmidrule{2-3} \cmidrule{5-6}
            \textbf{RPS} & \textbf{Median}                        & \textbf{CI $\pm$} &                                           & \textbf{Median} & \textbf{CI $\pm$} & \textbf{Ratio} & \textbf{Significance} \\
            \midrule

            25           & 10.94                                  & 0.01              &                                           & 20.3            & 0.01              &
            1.9x Higher  & ***                                                                                                                                                                                   \\

            50           & 13.41                                  & 0.01              &                                           & 23.86           & 0.01              &
            1.8x Higher  & ***                                                                                                                                                                                   \\

            100          & 18.39                                  & 0.02              &                                           & 30.91           & 0.01              &
            1.7x Higher  & ***                                                                                                                                                                                   \\

            200          & 28.33                                  & 0.02              &                                           & 45.01           & 0.01              &
            1.6x Higher  & ***                                                                                                                                                                                   \\

            500          & 58.13                                  & 0.03              &                                           & 76.99           & 0.37              &
            1.3x Higher  & ***                                                                                                                                                                                   \\

            1000         & 107.64                                 & 0.05              &                                           & 64.26           & 0.36              &
            1.7x Lower   & ***                                                                                                                                                                                   \\

            \bottomrule
        \end{tabular}
    }
    \caption{Statistical comparison of Data Store Size (MB) for POST /courses with prerequisites, aggregated over at least 25 runs}
    \label{table:run-create-course-prerequisites-aggregated-datastore-size}
\end{table}


\newpage
\section{L3: Enroll to Lecture}
\label{results:l3}

\begin{table}[H]
    \small \centering
    \resizebox{\columnwidth}{!}{
        \begin{tabular}{lcrrcrrcc}
            \toprule
                                        &              & \multicolumn{2}{c}{\textbf{CRUD (ms)}} &                   & \multicolumn{2}{c}{\textbf{ES-CQRS (ms)}} &               &                                                            \\
            \cmidrule{3-4} \cmidrule{6-7}
            \textbf{Metric}             & \textbf{RPS} & \textbf{Mean}                          & \textbf{CI $\pm$} &                                           & \textbf{Mean} & \textbf{CI $\pm$} & \textbf{Ratio} & \textbf{Significance} \\
            \midrule

            $dropped\_iterations\_rate$ & 25           & 0.0                                    & 0.0               &                                           & 0.0           & 0.0               &
            NaN                         & n.s.                                                                                                                                                                                               \\

            $latency\_avg$              & 25           & 7.04                                   & 0.0               &                                           & 17.2          & 0.0               &
            2.4x Slower                 & ***                                                                                                                                                                                                \\

            $latency\_p50$              & 25           & 6.47                                   & 0.0               &                                           & 15.63         & 0.0               &
            2.4x Slower                 & ***                                                                                                                                                                                                \\

            $latency\_p95$              & 25           & 9.97                                   & 0.0               &                                           & 26.13         & 0.0               &
            2.6x Slower                 & ***                                                                                                                                                                                                \\

            $latency\_p99$              & 25           & 19.04                                  & 0.0               &                                           & 35.19         & 0.0               &
            1.8x Slower                 & ***                                                                                                                                                                                                \\

            $dropped\_iterations\_rate$ & 50           & 0.0                                    & 0.0               &                                           & 0.0           & 0.0               &
            NaN                         & n.s.                                                                                                                                                                                               \\

            $latency\_avg$              & 50           & 5.98                                   & 0.0               &                                           & 15.48         & 0.0               &
            2.6x Slower                 & ***                                                                                                                                                                                                \\

            $latency\_p50$              & 50           & 5.57                                   & 0.0               &                                           & 13.89         & 0.0               &
            2.5x Slower                 & ***                                                                                                                                                                                                \\

            $latency\_p95$              & 50           & 8.61                                   & 0.0               &                                           & 24.73         & 0.0               &
            2.9x Slower                 & ***                                                                                                                                                                                                \\

            $latency\_p99$              & 50           & 11.39                                  & 0.0               &                                           & 35.99         & 0.0               &
            3.2x Slower                 & ***                                                                                                                                                                                                \\

            $dropped\_iterations\_rate$ & 100          & 0.0                                    & 0.0               &                                           & 0.0           & 0.0               &
            NaN                         & n.s.                                                                                                                                                                                               \\

            $latency\_avg$              & 100          & 5.38                                   & 0.0               &                                           & 41.32         & 0.0               &
            7.7x Slower                 & ***                                                                                                                                                                                                \\

            $latency\_p50$              & 100          & 5.14                                   & 0.0               &                                           & 15.11         & 0.0               &
            2.9x Slower                 & ***                                                                                                                                                                                                \\

            $latency\_p95$              & 100          & 7.28                                   & 0.0               &                                           & 169.83        & 0.03              &
            23.3x Slower                & ***                                                                                                                                                                                                \\

            $latency\_p99$              & 100          & 10.12                                  & 0.0               &                                           & 578.7         & 0.07              &
            57.2x Slower                & ***                                                                                                                                                                                                \\

            $dropped\_iterations\_rate$ & 200          & 0.0                                    & 0.0               &                                           & 36.29         & 0.44              &
            NaN                         & ***                                                                                                                                                                                                \\

            $latency\_avg$              & 200          & 5.7                                    & 0.0               &                                           & 1262.75       & 0.01              &
            221.4x Slower               & ***                                                                                                                                                                                                \\

            $latency\_p50$              & 200          & 5.56                                   & 0.0               &                                           & 683.72        & 0.01              &
            123.0x Slower               & ***                                                                                                                                                                                                \\

            $latency\_p95$              & 200          & 7.75                                   & 0.0               &                                           & 2611.33       & 0.04              &
            336.9x Slower               & ***                                                                                                                                                                                                \\

            $latency\_p99$              & 200          & 9.57                                   & 0.0               &                                           & 10014.17      & 0.0               &
            1046.2x Slower              & ***                                                                                                                                                                                                \\

            $dropped\_iterations\_rate$ & 500          & 36.12                                  & 0.29              &                                           & 238.56        & 0.34              &
            6.6x Higher                 & ***                                                                                                                                                                                                \\

            $latency\_avg$              & 500          & 555.24                                 & 0.0               &                                           & 3347.53       & 0.01              &
            6.0x Slower                 & ***                                                                                                                                                                                                \\

            $latency\_p50$              & 500          & 15.7                                   & 0.0               &                                           & 2564.43       & 0.03              &
            163.4x Slower               & ***                                                                                                                                                                                                \\

            $latency\_p95$              & 500          & 1993.44                                & 0.01              &                                           & 10537.1       & 0.79              &
            5.3x Slower                 & ***                                                                                                                                                                                                \\

            $latency\_p99$              & 500          & 2595.67                                & 0.01              &                                           & 11933.24      & 0.01              &
            4.6x Slower                 & ***                                                                                                                                                                                                \\

            \bottomrule
        \end{tabular}
    }
    \caption{Statistical comparison for POST /lectures/{lectureId}/enroll (client), averaged out over at least 25 runs}
    \label{table:run-enroll-to-lecture_client}
\end{table}

\begin{table}[H]
    \small \centering
    \resizebox{\columnwidth}{!}{
        \begin{tabular}{lcrrcrrcc}
            \toprule
                            &              & \multicolumn{2}{c}{\textbf{CRUD (ms)}} &                   & \multicolumn{2}{c}{\textbf{ES-CQRS (ms)}} &                 &                                                            \\
            \cmidrule{3-4} \cmidrule{6-7}
            \textbf{Metric} & \textbf{RPS} & \textbf{Median}                        & \textbf{CI $\pm$} &                                           & \textbf{Median} & \textbf{CI $\pm$} & \textbf{Ratio} & \textbf{Significance} \\
            \midrule

            $latency\_avg$  & 25           & 6.07                                   & 0.0               &                                           & 16.18           & 0.0               & 2.7x Slower    & ***                   \\

            $latency\_p50$  & 25           & 5.49                                   & 0.0               &                                           & 14.81           & 0.0               & 2.7x Slower    & ***                   \\

            $latency\_p95$  & 25           & 8.96                                   & 0.0               &                                           & 25.54           & 0.0               & 2.8x Slower    & ***                   \\

            $latency\_p99$  & 25           & 18.27                                  & 0.0               &                                           & 33.55           & 0.0               & 1.8x Slower    & ***                   \\

            $latency\_avg$  & 50           & 5.21                                   & 0.0               &                                           & 14.56           & 0.0               & 2.8x Slower    & ***                   \\

            $latency\_p50$  & 50           & 5.02                                   & 0.0               &                                           & 13.21           & 0.0               & 2.6x Slower    & ***                   \\

            $latency\_p95$  & 50           & 7.69                                   & 0.0               &                                           & 23.89           & 0.0               & 3.1x Slower    & ***                   \\

            $latency\_p99$  & 50           & 9.85                                   & 0.0               &                                           & 35.31           & 0.0               & 3.6x Slower    & ***                   \\

            $latency\_avg$  & 100          & 4.75                                   & 0.0               &                                           & 41.94           & 0.0               & 8.8x Slower    & ***                   \\

            $latency\_p50$  & 100          & 4.7                                    & 0.0               &                                           & 14.41           & 0.0               & 3.1x Slower    & ***                   \\

            $latency\_p95$  & 100          & 6.76                                   & 0.0               &                                           & 151.88          & 0.04              & 22.5x Slower   & ***                   \\

            $latency\_p99$  & 100          & 8.91                                   & 0.0               &                                           & 573.93          & 0.09              & 64.4x Slower   & ***                   \\

            $latency\_avg$  & 200          & 5.2                                    & 0.0               &                                           & 1247.57         & 0.02              & 240.1x Slower  & ***                   \\

            $latency\_p50$  & 200          & 5.13                                   & 0.0               &                                           & 689.99          & 0.01              & 134.5x Slower  & ***                   \\

            $latency\_p95$  & 200          & 7.06                                   & 0.0               &                                           & 2647.03         & 0.05              & 375.1x Slower  & ***                   \\

            $latency\_p99$  & 200          & 8.75                                   & 0.0               &                                           & 9698.48         & 0.01              & 1108.4x Slower & ***                   \\

            $latency\_avg$  & 500          & 244.63                                 & 0.0               &                                           & 1283.39         & 0.01              & 5.2x Slower    & ***                   \\

            $latency\_p50$  & 500          & 13.82                                  & 0.0               &                                           & 633.12          & 0.03              & 45.8x Slower   & ***                   \\

            $latency\_p95$  & 500          & 1289.29                                & 0.0               &                                           & 8969.55         & 0.06              & 7.0x Slower    & ***                   \\

            $latency\_p99$  & 500          & 1776.48                                & 0.01              &                                           & 9823.63         & 0.01              & 5.5x Slower    & ***                   \\

            \bottomrule
        \end{tabular}
    }
    \caption{Statistical comparison for POST /lectures/\{lectureId\}/enroll (server), aggregated over at least 25 runs}
    \label{table:run-enroll-to-lecture_server}
\end{table}
\begin{table}[H]
    \small
    \centering
    \resizebox{\columnwidth}{!}{
        \begin{tabular}{rrrcrrcc}
            \toprule
                         & \multicolumn{2}{c}{\textbf{CRUD (ms)}} &                   & \multicolumn{2}{c}{\textbf{ES-CQRS (ms)}} &                 &                                                            \\
            \cmidrule{2-3} \cmidrule{5-6}
            \textbf{RPS} & \textbf{Median}                        & \textbf{CI $\pm$} &                                           & \textbf{Median} & \textbf{CI $\pm$} & \textbf{Ratio} & \textbf{Significance} \\
            \midrule

            25           & 0.0338                                 & 0.0017            &                                           & 0.0914          & 0.0035            &
            2.5x Higher  & ***                                                                                                                                                                                   \\

            50           & 0.0416                                 & 0.002             &                                           & 0.1481          & 0.005             &
            3.2x Higher  & ***                                                                                                                                                                                   \\

            100          & 0.0639                                 & 0.0022            &                                           & 0.284           & 0.0087            &
            4.2x Higher  & ***                                                                                                                                                                                   \\

            200          & 0.1146                                 & 0.0032            &                                           & 0.4198          & 0.0113            &
            3.5x Higher  & ***                                                                                                                                                                                   \\

            500          & 0.3566                                 & 0.0066            &                                           & 0.4033          & 0.0124            &
            1.1x Higher  & ***                                                                                                                                                                                   \\

            \bottomrule
        \end{tabular}
    }
    \caption{Statistical comparison of $cpu\_usage$ for POST /lectures/\{lectureId\}/enroll, aggregated over at least 25 runs}
    \label{table:run-enroll-to-lecture-aggregated-cpu-usage}
\end{table}
\begin{table}[H]
    \small
    \centering
    \resizebox{\columnwidth}{!}{
        \begin{tabular}{rrrcrrcc}
            \toprule
                         & \multicolumn{2}{c}{\textbf{CRUD (ms)}} &                   & \multicolumn{2}{c}{\textbf{ES-CQRS (ms)}} &                 &                                                            \\
            \cmidrule{2-3} \cmidrule{5-6}
            \textbf{RPS} & \textbf{Median}                        & \textbf{CI $\pm$} &                                           & \textbf{Median} & \textbf{CI $\pm$} & \textbf{Ratio} & \textbf{Significance} \\
            \midrule

            25           & 10.0                                   & 0.0               &                                           & 10.0            & 0.0               &
            1.0x         & n.s.                                                                                                                                                                                  \\

            50           & 10.0                                   & 0.0               &                                           & 10.0            & 0.0               &
            1.0x         & n.s.                                                                                                                                                                                  \\

            100          & 10.0                                   & 0.0               &                                           & 65.0            & 2.1559            &
            5.8x Higher  & ***                                                                                                                                                                                   \\

            200          & 10.0                                   & 0.0               &                                           & 200.0           & 4.2754            &
            18.2x Higher & ***                                                                                                                                                                                   \\

            500          & 17.0                                   & 7.9176            &                                           & 200.0           & 0.0               &
            2.1x Higher  & ***                                                                                                                                                                                   \\

            \bottomrule
        \end{tabular}
    }
    \caption{Statistical comparison of $tomcat\_threads$ for POST /lectures/\{lectureId\}/enroll, averaged out over at least 25 runs}
    \label{table:run-enroll-to-lecture-aggregated-threadpool-usage}
\end{table}
\begin{table}[H]
    \small
    \centering
    \begin{tabular}{rrrcrrcc}
        \toprule
                     & \multicolumn{2}{c}{\textbf{CRUD (ms)}} &                   & \multicolumn{2}{c}{\textbf{ES-CQRS (ms)}} &                 &                                                            \\
        \cmidrule{2-3} \cmidrule{5-6}
        \textbf{RPS} & \textbf{Median}                        & \textbf{CI $\pm$} &                                           & \textbf{Median} & \textbf{CI $\pm$} & \textbf{Ratio} & \textbf{Significance} \\
        \midrule

        25           & 0.0                                    & 0.0312            &                                           & 0.0             & 0.081             & 3.3x Higher    & ***                   \\

        50           & 0.0                                    & 0.0352            &                                           & 1.0             & 0.0898            & 3.6x Higher    & ***                   \\

        100          & 0.0                                    & 0.0385            &                                           & 2.0             & 0.1316            & 5.3x Higher    & ***                   \\

        200          & 1.0                                    & 0.0414            &                                           & 4.0             & 0.1711            & 4.7x Higher    & ***                   \\

        500          & 4.0                                    & 0.3303            &                                           & 4.0             & 0.1692            & 1.4x Lower     & ***                   \\

        \bottomrule
    \end{tabular}
    \caption[Comparison of $hikari\_connections$ for POST /lectures/\{lectureId\}/enroll]{Statistical comparison of $hikari\_connections$ for POST /lectures/\{lectureId\}/enroll, aggregated over at least 25 runs}
    \label{table:run-enroll-to-lecture-aggregated-database-connections}
\end{table}
\begin{table}[H]
    \small
    \centering
    \resizebox{\columnwidth}{!}{
        \begin{tabular}{rrrcrrcc}
            \toprule
                         & \multicolumn{2}{c}{\textbf{CRUD (ms)}} &                   & \multicolumn{2}{c}{\textbf{ES-CQRS (ms)}} &                 &                                                            \\
            \cmidrule{2-3} \cmidrule{5-6}
            \textbf{RPS} & \textbf{Median}                        & \textbf{CI $\pm$} &                                           & \textbf{Median} & \textbf{CI $\pm$} & \textbf{Ratio} & \textbf{Significance} \\
            \midrule

            25           & 10.54                                  & 0.0               &                                           & 24.11           & 0.02              & 2.3x Higher    & ***                   \\

            50           & 12.32                                  & 0.0               &                                           & 31.64           & 0.02              & 2.6x Higher    & ***                   \\

            100          & 15.86                                  & 0.0               &                                           & 55.94           & 0.05              & 3.5x Higher    & ***                   \\

            200          & 22.97                                  & 0.01              &                                           & 227.27          & 11.74             & 9.7x Higher    & ***                   \\

            500          & 40.77                                  & 0.03              &                                           & 232.4           & 0.55              & 5.7x Higher    & ***                   \\

            \bottomrule
        \end{tabular}
    }
    \caption{Statistical comparison of Data Store Size (MB) for POST /lectures/\{lectureId\}/enroll, aggregated over at least 25 runs}
    \label{table:run-enroll-to-lecture-aggregated-datastore-size}
\end{table}

\newpage
\section{L4: Read Lectures for Student}
\label{results:l4}

\begin{table}[H]
    \small \centering
    \resizebox{\columnwidth}{!}{
        \begin{tabular}{lcrrcrrcc}
            \toprule
                            &              & \multicolumn{2}{c}{\textbf{CRUD (ms)}} &                   & \multicolumn{2}{c}{\textbf{ES-CQRS (ms)}} &               &                                                              \\
            \cmidrule{3-4} \cmidrule{6-7}
            \textbf{Metric} & \textbf{RPS} & \textbf{Mean}                          & \textbf{CI $\pm$} &                                           & \textbf{Mean} & \textbf{CI $\pm$} & \textbf{Speedup} & \textbf{Significance} \\
            \midrule

            $latency\_avg$  & 25           & 2.52                                   & 0.0               &                                           & 3.16          & 0.0               &
            1.3x Slower     & ***                                                                                                                                                                                                  \\

            $latency\_p50$  & 25           & 2.37                                   & 0.0               &                                           & 3.03          & 0.0               &
            1.3x Slower     & ***                                                                                                                                                                                                  \\

            $latency\_p95$  & 25           & 3.73                                   & 0.0               &                                           & 4.38          & 0.0               &
            1.2x Slower     & ***                                                                                                                                                                                                  \\

            $latency\_p99$  & 25           & 5.18                                   & 0.0               &                                           & 5.41          & 0.0               &
            1.0x Slower     & ***                                                                                                                                                                                                  \\

            $latency\_avg$  & 50           & 1.88                                   & 0.0               &                                           & 2.49          & 0.0               &
            1.3x Slower     & ***                                                                                                                                                                                                  \\

            $latency\_p50$  & 50           & 1.7                                    & 0.0               &                                           & 2.32          & 0.0               &
            1.4x Slower     & ***                                                                                                                                                                                                  \\

            $latency\_p95$  & 50           & 3.12                                   & 0.0               &                                           & 3.78          & 0.0               &
            1.2x Slower     & ***                                                                                                                                                                                                  \\

            $latency\_p99$  & 50           & 4.68                                   & 0.0               &                                           & 5.04          & 0.0               &
            1.1x Slower     & ***                                                                                                                                                                                                  \\

            $latency\_avg$  & 100          & 1.4                                    & 0.0               &                                           & 1.94          & 0.0               &
            1.4x Slower     & ***                                                                                                                                                                                                  \\

            $latency\_p50$  & 100          & 1.23                                   & 0.0               &                                           & 1.75          & 0.0               &
            1.4x Slower     & ***                                                                                                                                                                                                  \\

            $latency\_p95$  & 100          & 2.41                                   & 0.0               &                                           & 3.17          & 0.0               &
            1.3x Slower     & ***                                                                                                                                                                                                  \\

            $latency\_p99$  & 100          & 4.15                                   & 0.0               &                                           & 4.7           & 0.0               &
            1.1x Slower     & ***                                                                                                                                                                                                  \\

            $latency\_avg$  & 200          & 1.02                                   & 0.0               &                                           & 1.45          & 0.0               &
            1.4x Slower     & ***                                                                                                                                                                                                  \\

            $latency\_p50$  & 200          & 0.88                                   & 0.0               &                                           & 1.26          & 0.0               &
            1.4x Slower     & ***                                                                                                                                                                                                  \\

            $latency\_p95$  & 200          & 1.83                                   & 0.0               &                                           & 2.59          & 0.0               &
            1.4x Slower     & ***                                                                                                                                                                                                  \\

            $latency\_p99$  & 200          & 3.32                                   & 0.0               &                                           & 4.22          & 0.0               &
            1.3x Slower     & ***                                                                                                                                                                                                  \\

            $latency\_avg$  & 500          & 0.88                                   & 0.0               &                                           & 1.2           & 0.0               &
            1.4x Slower     & ***                                                                                                                                                                                                  \\

            $latency\_p50$  & 500          & 0.82                                   & 0.0               &                                           & 1.06          & 0.0               &
            1.3x Slower     & ***                                                                                                                                                                                                  \\

            $latency\_p95$  & 500          & 1.26                                   & 0.0               &                                           & 1.93          & 0.0               &
            1.5x Slower     & ***                                                                                                                                                                                                  \\

            $latency\_p99$  & 500          & 2.28                                   & 0.0               &                                           & 3.56          & 0.0               &
            1.6x Slower     & ***                                                                                                                                                                                                  \\

            $latency\_avg$  & 1000         & 0.84                                   & 0.0               &                                           & 1.14          & 0.0               &
            1.4x Slower     & ***                                                                                                                                                                                                  \\

            $latency\_p50$  & 1000         & 0.79                                   & 0.0               &                                           & 0.99          & 0.0               &
            1.3x Slower     & ***                                                                                                                                                                                                  \\

            $latency\_p95$  & 1000         & 1.12                                   & 0.0               &                                           & 1.73          & 0.0               &
            1.5x Slower     & ***                                                                                                                                                                                                  \\

            $latency\_p99$  & 1000         & 2.13                                   & 0.0               &                                           & 3.89          & 0.0               &
            1.8x Slower     & ***                                                                                                                                                                                                  \\

            $latency\_avg$  & 2000         & 0.82                                   & 0.0               &                                           & 1.51          & 0.0               &
            1.8x Slower     & ***                                                                                                                                                                                                  \\

            $latency\_p50$  & 2000         & 0.76                                   & 0.0               &                                           & 1.18          & 0.0               &
            1.5x Slower     & ***                                                                                                                                                                                                  \\

            $latency\_p95$  & 2000         & 1.09                                   & 0.0               &                                           & 3.21          & 0.0               &
            2.9x Slower     & ***                                                                                                                                                                                                  \\

            $latency\_p99$  & 2000         & 2.13                                   & 0.0               &                                           & 6.72          & 0.0               &
            3.2x Slower     & ***                                                                                                                                                                                                  \\

            $latency\_avg$  & 3000         & 1.12                                   & 0.0               &                                           & 4.17          & 0.0               &
            3.7x Slower     & ***                                                                                                                                                                                                  \\

            $latency\_p50$  & 3000         & 0.92                                   & 0.0               &                                           & 2.61          & 0.0               &
            2.8x Slower     & ***                                                                                                                                                                                                  \\

            $latency\_p95$  & 3000         & 1.96                                   & 0.0               &                                           & 11.19         & 0.0               &
            5.7x Slower     & ***                                                                                                                                                                                                  \\

            $latency\_p99$  & 3000         & 4.81                                   & 0.0               &                                           & 27.54         & 0.0               &
            5.7x Slower     & ***                                                                                                                                                                                                  \\

            $latency\_avg$  & 4000         & 208.97                                 & 0.08              &                                           & 80.26         & 0.03              &
            2.6x Faster     & **                                                                                                                                                                                                   \\

            $latency\_p50$  & 4000         & 124.24                                 & 0.11              &                                           & 20.16         & 0.01              &
            6.2x Faster     & *                                                                                                                                                                                                    \\

            $latency\_p95$  & 4000         & 624.77                                 & 0.13              &                                           & 369.65        & 0.15              &
            1.7x Faster     & **                                                                                                                                                                                                   \\

            $latency\_p99$  & 4000         & 732.74                                 & 0.13              &                                           & 457.09        & 0.16              &
            1.6x Faster     & **                                                                                                                                                                                                   \\

            \bottomrule
        \end{tabular}
    }
    \caption{Statistical comparison for GET /lectures (client), averaged out over at least 25 runs}
    \label{table:run-read-lectures_client}
\end{table}
\begin{table}[H]
    \small \centering
    \resizebox{\columnwidth}{!}{
        \begin{tabular}{lcrrcrrcc}
            \toprule
                            &              & \multicolumn{2}{c}{\textbf{CRUD (ms)}} &                   & \multicolumn{2}{c}{\textbf{ES-CQRS (ms)}} &               &                                                              \\
            \cmidrule{3-4} \cmidrule{6-7}
            \textbf{Metric} & \textbf{RPS} & \textbf{Mean}                          & \textbf{CI $\pm$} &                                           & \textbf{Mean} & \textbf{CI $\pm$} & \textbf{Speedup} & \textbf{Significance} \\
            \midrule

            $latency\_avg$  & 25           & 1.6                                    & 0.0               &                                           & 2.23          & 0.0               &
            1.4x Slower     & ***                                                                                                                                                                                                  \\

            $latency\_p50$  & 25           & 1.48                                   & 0.0               &                                           & 2.12          & 0.0               &
            1.4x Slower     & ***                                                                                                                                                                                                  \\

            $latency\_p95$  & 25           & 2.69                                   & 0.0               &                                           & 3.26          & 0.0               &
            1.2x Slower     & ***                                                                                                                                                                                                  \\

            $latency\_p99$  & 25           & 3.62                                   & 0.0               &                                           & 3.94          & 0.0               &
            1.1x Slower     & ***                                                                                                                                                                                                  \\

            $latency\_avg$  & 50           & 1.14                                   & 0.0               &                                           & 1.7           & 0.0               &
            1.5x Slower     & ***                                                                                                                                                                                                  \\

            $latency\_p50$  & 50           & 0.98                                   & 0.0               &                                           & 1.58          & 0.0               &
            1.6x Slower     & ***                                                                                                                                                                                                  \\

            $latency\_p95$  & 50           & 2.19                                   & 0.0               &                                           & 2.79          & 0.0               &
            1.3x Slower     & ***                                                                                                                                                                                                  \\

            $latency\_p99$  & 50           & 3.21                                   & 0.0               &                                           & 3.61          & 0.0               &
            1.1x Slower     & ***                                                                                                                                                                                                  \\

            $latency\_avg$  & 100          & 0.8                                    & 0.0               &                                           & 1.26          & 0.0               &
            1.6x Slower     & ***                                                                                                                                                                                                  \\

            $latency\_p50$  & 100          & 0.61                                   & 0.0               &                                           & 1.11          & 0.0               &
            1.8x Slower     & ***                                                                                                                                                                                                  \\

            $latency\_p95$  & 100          & 1.65                                   & 0.0               &                                           & 2.33          & 0.0               &
            1.4x Slower     & ***                                                                                                                                                                                                  \\

            $latency\_p99$  & 100          & 2.73                                   & 0.0               &                                           & 3.29          & 0.0               &
            1.2x Slower     & ***                                                                                                                                                                                                  \\

            $latency\_avg$  & 200          & 0.64                                   & 0.0               &                                           & 1.03          & 0.0               &
            1.6x Slower     & ***                                                                                                                                                                                                  \\

            $latency\_p50$  & 200          & 0.55                                   & 0.0               &                                           & 0.75          & 0.0               &
            1.4x Slower     & ***                                                                                                                                                                                                  \\

            $latency\_p95$  & 200          & 1.28                                   & 0.0               &                                           & 1.95          & 0.0               &
            1.5x Slower     & ***                                                                                                                                                                                                  \\

            $latency\_p99$  & 200          & 2.2                                    & 0.0               &                                           & 2.93          & 0.0               &
            1.3x Slower     & ***                                                                                                                                                                                                  \\

            $latency\_avg$  & 500          & 0.57                                   & 0.0               &                                           & 0.88          & 0.0               &
            1.5x Slower     & ***                                                                                                                                                                                                  \\

            $latency\_p50$  & 500          & 0.52                                   & 0.0               &                                           & 0.59          & 0.0               &
            1.1x Slower     & ***                                                                                                                                                                                                  \\

            $latency\_p95$  & 500          & 0.99                                   & 0.0               &                                           & 1.52          & 0.0               &
            1.5x Slower     & ***                                                                                                                                                                                                  \\

            $latency\_p99$  & 500          & 1.68                                   & 0.0               &                                           & 2.7           & 0.0               &
            1.6x Slower     & ***                                                                                                                                                                                                  \\

            $latency\_avg$  & 1000         & 0.57                                   & 0.0               &                                           & 0.86          & 0.0               &
            1.5x Slower     & ***                                                                                                                                                                                                  \\

            $latency\_p50$  & 1000         & 0.51                                   & 0.0               &                                           & 0.56          & 0.0               &
            1.1x Slower     & ***                                                                                                                                                                                                  \\

            $latency\_p95$  & 1000         & 0.97                                   & 0.0               &                                           & 1.37          & 0.0               &
            1.4x Slower     & ***                                                                                                                                                                                                  \\

            $latency\_p99$  & 1000         & 1.51                                   & 0.0               &                                           & 3.0           & 0.0               &
            2.0x Slower     & ***                                                                                                                                                                                                  \\

            $latency\_avg$  & 2000         & 0.57                                   & 0.0               &                                           & 1.21          & 0.0               &
            2.1x Slower     & ***                                                                                                                                                                                                  \\

            $latency\_p50$  & 2000         & 0.51                                   & 0.0               &                                           & 0.81          & 0.0               &
            1.6x Slower     & ***                                                                                                                                                                                                  \\

            $latency\_p95$  & 2000         & 0.97                                   & 0.0               &                                           & 2.63          & 0.0               &
            2.7x Slower     & ***                                                                                                                                                                                                  \\

            $latency\_p99$  & 2000         & 1.45                                   & 0.0               &                                           & 5.57          & 0.0               &
            3.8x Slower     & ***                                                                                                                                                                                                  \\

            $latency\_avg$  & 3000         & 0.82                                   & 0.0               &                                           & 3.55          & 0.0               &
            4.3x Slower     & ***                                                                                                                                                                                                  \\

            $latency\_p50$  & 3000         & 0.56                                   & 0.0               &                                           & 2.22          & 0.0               &
            4.0x Slower     & ***                                                                                                                                                                                                  \\

            $latency\_p95$  & 3000         & 1.47                                   & 0.0               &                                           & 9.8           & 0.0               &
            6.7x Slower     & ***                                                                                                                                                                                                  \\

            $latency\_p99$  & 3000         & 3.57                                   & 0.0               &                                           & 24.42         & 0.0               &
            6.8x Slower     & ***                                                                                                                                                                                                  \\

            $latency\_avg$  & 4000         & 23.38                                  & 0.0               &                                           & 22.17         & 0.0               &
            1.1x Faster     & n.s.                                                                                                                                                                                                 \\

            $latency\_p50$  & 4000         & 2.33                                   & 0.0               &                                           & 16.98         & 0.0               &
            7.3x Slower     & ***                                                                                                                                                                                                  \\

            $latency\_p95$  & 4000         & 126.04                                 & 0.01              &                                           & 57.58         & 0.0               &
            2.2x Faster     & ***                                                                                                                                                                                                  \\

            $latency\_p99$  & 4000         & 187.99                                 & 0.01              &                                           & 74.92         & 0.0               &
            2.5x Faster     & ***                                                                                                                                                                                                  \\

            \bottomrule
        \end{tabular}
    }
    \caption{Statistical comparison for GET /lectures (server), averaged out over at least 25 runs}
    \label{table:run-read-lectures_server}
\end{table}
\begin{table}[H]
    \small
    \centering
    \begin{tabular}{rrrcrrcc}
        \toprule
                     & \multicolumn{2}{c}{\textbf{CRUD (ms)}} &                   & \multicolumn{2}{c}{\textbf{ES-CQRS (ms)}} &                 &                                                            \\
        \cmidrule{2-3} \cmidrule{5-6}
        \textbf{RPS} & \textbf{Median}                        & \textbf{CI $\pm$} &                                           & \textbf{Median} & \textbf{CI $\pm$} & \textbf{Ratio} & \textbf{Significance} \\
        \midrule

        25           & 0.0138                                 & 0.0012            &                                           & 0.0201          & 0.0012            & 1.5x Higher    & ***                   \\

        50           & 0.0163                                 & 0.0013            &                                           & 0.0277          & 0.0013            & 1.7x Higher    & ***                   \\

        100          & 0.0214                                 & 0.0013            &                                           & 0.0416          & 0.002             & 1.9x Higher    & ***                   \\

        200          & 0.0251                                 & 0.0008            &                                           & 0.0446          & 0.002             & 1.8x Higher    & ***                   \\

        500          & 0.0536                                 & 0.0006            &                                           & 0.0941          & 0.0015            & 1.8x Higher    & ***                   \\

        1000         & 0.1121                                 & 0.0011            &                                           & 0.1835          & 0.0013            & 1.6x Higher    & ***                   \\

        2000         & 0.2324                                 & 0.0016            &                                           & 0.3869          & 0.0025            & 1.7x Higher    & ***                   \\

        3000         & 0.4149                                 & 0.0051            &                                           & 0.5437          & 0.0024            & 1.3x Higher    & ***                   \\

        4000         & 0.6112                                 & 0.0219            &                                           & 0.6015          & 0.0062            & 1.0x Lower     & n.s.                  \\

        \bottomrule
    \end{tabular}
    \caption[Comparison of $cpu\_usage$ for GET /lectures]{Statistical comparison of $cpu\_usage$ for GET /lectures, aggregated over at least 25 runs}
    \label{table:run-read-lectures-aggregated-CPU_Usage}
\end{table}
\begin{table}[H]
    \small
    \centering
    \resizebox{\columnwidth}{!}{
        \begin{tabular}{rrrcrrcc}
            \toprule
                         & \multicolumn{2}{c}{\textbf{CRUD (ms)}} &                   & \multicolumn{2}{c}{\textbf{ES-CQRS (ms)}} &                 &                                                            \\
            \cmidrule{2-3} \cmidrule{5-6}
            \textbf{RPS} & \textbf{Median}                        & \textbf{CI $\pm$} &                                           & \textbf{Median} & \textbf{CI $\pm$} & \textbf{Ratio} & \textbf{Significance} \\
            \midrule

            25           & 10.0                                   & 0.0               &                                           & 10.0            & 0.0               & 1.0x Lower     & n.s.                  \\

            50           & 10.0                                   & 0.0               &                                           & 10.0            & 0.0               & 1.0x Lower     & n.s.                  \\

            100          & 10.0                                   & 0.0               &                                           & 10.0            & 0.0               & 1.0x Lower     & n.s.                  \\

            200          & 10.0                                   & 0.0               &                                           & 10.0            & 0.0               & 1.0x Lower     & n.s.                  \\

            500          & 10.0                                   & 0.0               &                                           & 10.0            & 0.0               & 1.0x Lower     & ***                   \\

            1000         & 15.0                                   & 0.5               &                                           & 20.0            & 0.5               & 1.3x Higher    & ***                   \\

            2000         & 24.0                                   & 0.0               &                                           & 47.0            & 0.5               & 2.0x Higher    & ***                   \\

            3000         & 31.0                                   & 0.5               &                                           & 158.0           & 14.0              & 5.1x Higher    & ***                   \\

            4000         & 200.0                                  & 0.0               &                                           & 200.0           & 0.0               & 1.0x Lower     & ***                   \\

            \bottomrule
        \end{tabular}
    }
    \caption{Statistical comparison of $tomcat\_threads$ for GET /lectures, aggregated over at least 25 runs}
    \label{table:run-read-lectures-aggregated-Threadpool_Usage}
\end{table}
\begin{table}[H]
    \small
    \centering
    \begin{tabular}{rrrcrrcc}
        \toprule
                     & \multicolumn{2}{c}{\textbf{CRUD (ms)}} &                   & \multicolumn{2}{c}{\textbf{ES-CQRS (ms)}} &                 &                                                            \\
        \cmidrule{2-3} \cmidrule{5-6}
        \textbf{RPS} & \textbf{Median}                        & \textbf{CI $\pm$} &                                           & \textbf{Median} & \textbf{CI $\pm$} & \textbf{Ratio} & \textbf{Significance} \\
        \midrule

        25           & 0.0                                    & 0.0               &                                           & 0.0             & 0.0               & $N/A$          & ***                   \\

        50           & 0.0                                    & 0.0               &                                           & 0.0             & 0.0               & $N/A$          & n.s.                  \\

        100          & 0.0                                    & 0.0               &                                           & 0.0             & 0.0               & $N/A$          & n.s.                  \\

        200          & 0.0                                    & 0.0               &                                           & 0.0             & 0.0               & $N/A$          & n.s.                  \\

        500          & 0.0                                    & 0.0               &                                           & 0.0             & 0.0               & $N/A$          & ***                   \\

        1000         & 0.0                                    & 0.0               &                                           & 0.0             & 0.0               & $N/A$          & ***                   \\

        2000         & 1.0                                    & 0.0               &                                           & 0.0             & 0.5               & $N/A$          & ***                   \\

        3000         & 2.0                                    & 0.0               &                                           & 1.0             & 0.5               & 2.0x Lower     & **                    \\

        4000         & 8.0                                    & 0.5               &                                           & 6.0             & 1.0               & 1.3x Lower     & ***                   \\

        \bottomrule
    \end{tabular}
    \caption[Comparison of $hikari\_connections$ for GET /lectures]{Statistical comparison of $hikari\_connections$ for GET /lectures, aggregated over at least 25 runs}
    \label{table:run-read-lectures-aggregated-Database_Connections}
\end{table}


\newpage
\section{L5: Read All Lectures}
\label{results:l5}

\begin{table}[H]
    \small \centering
    \resizebox{\columnwidth}{!}{
        \begin{tabular}{lcrrcrrcc}
            \toprule
                                        &              & \multicolumn{2}{c}{\textbf{CRUD (ms)}} &                   & \multicolumn{2}{c}{\textbf{ES-CQRS (ms)}} &                 &                                                             \\
            \cmidrule{3-4} \cmidrule{6-7}
            \textbf{Metric}             & \textbf{RPS} & \textbf{Median}                        & \textbf{CI $\pm$} &                                           & \textbf{Median} & \textbf{CI $\pm$} & \textbf{Ratio}  & \textbf{Significance} \\
            \midrule

            $dropped\_iterations\_rate$ & 25           & 0.0                                    & 0.0               &                                           & 0.0             & 0.0               & N/A             & n.s.                  \\

            $latency\_avg$              & 25           & 26.25                                  & 0.0               &                                           & 4.43            & 0.0               & 5.9x Faster     & ***                   \\

            $latency\_p50$              & 25           & 25.75                                  & 0.0               &                                           & 4.03            & 0.0               & 6.4x Faster     & ***                   \\

            $latency\_p95$              & 25           & 29.31                                  & 0.0               &                                           & 6.96            &
            0.0                         &
            4.2x Faster                 & ***                                                                                                                                                                                                   \\

            $latency\_p99$              & 25           & 33.76                                  & 0.0               &                                           & 9.73            &
            0.0                         &
            3.5x Faster                 & ***                                                                                                                                                                                                   \\

            $dropped\_iterations\_rate$ & 50           & 0.0                                    & 0.0               &                                           & 0.0             &
            0.0                         &
            N/A                         & n.s.                                                                                                                                                                                                  \\

            $latency\_avg$              & 50           & 27.13                                  & 0.0               &                                           & 3.56            &
            0.0                         &
            7.6x Faster                 & ***                                                                                                                                                                                                   \\

            $latency\_p50$              & 50           & 26.68                                  & 0.0               &                                           & 3.21            &
            0.0                         &
            8.3x Faster                 & ***                                                                                                                                                                                                   \\

            $latency\_p95$              & 50           & 30.22                                  & 0.0               &                                           & 6.0             &
            0.0                         &
            5.0x Faster                 & ***                                                                                                                                                                                                   \\

            $latency\_p99$              & 50           & 33.76                                  & 0.0               &                                           & 8.67            &
            0.0                         &
            3.9x Faster                 & ***                                                                                                                                                                                                   \\

            $dropped\_iterations\_rate$ & 100          & 0.0                                    & 0.0               &                                           & 0.0             &
            0.0                         &
            N/A                         & n.s.                                                                                                                                                                                                  \\

            $latency\_avg$              & 100          & 36.75                                  & 0.0               &                                           & 2.75            &
            0.0                         &
            13.4x Faster                & ***                                                                                                                                                                                                   \\

            $latency\_p50$              & 100          & 32.73                                  & 0.0               &                                           & 2.44            &
            0.0                         &
            13.4x Faster                & ***                                                                                                                                                                                                   \\

            $latency\_p95$              & 100          & 65.72                                  & 0.03              &                                           & 4.87            &
            0.0                         &
            13.5x Faster                & ***                                                                                                                                                                                                   \\

            $latency\_p99$              & 100          & 124.88                                 & 0.03              &                                           & 7.77            &
            0.0                         &
            16.1x Faster                & ***                                                                                                                                                                                                   \\

            $dropped\_iterations\_rate$ & 150          & 25.16                                  & 0.14              &                                           & 0.0             &
            0.0                         &
            N/A                         & ***                                                                                                                                                                                                   \\

            $latency\_avg$              & 150          & 2256.08                                & 0.01              &                                           & 2.16            &
            0.0                         &
            1042.7x Faster              & ***                                                                                                                                                                                                   \\

            $latency\_p50$              & 150          & 2733.07                                & 0.01              &                                           & 1.83            &
            0.0                         &
            1495.0x Faster              & ***                                                                                                                                                                                                   \\

            $latency\_p95$              & 150          & 4778.46                                & 0.02              &                                           & 4.24            &
            0.0                         &
            1127.1x Faster              & ***                                                                                                                                                                                                   \\

            $latency\_p99$              & 150          & 6548.91                                & 0.02              &                                           & 7.33            &
            0.0                         &
            893.2x Faster               & ***                                                                                                                                                                                                   \\

            $dropped\_iterations\_rate$ & 200          & 59.19                                  & 0.21              &                                           & 0.0             &
            0.0                         &
            N/A                         & ***                                                                                                                                                                                                   \\

            $latency\_avg$              & 200          & 3210.72                                & 0.01              &                                           & 1.98            &
            0.0                         &
            1618.7x Faster              & ***                                                                                                                                                                                                   \\

            $latency\_p50$              & 200          & 3697.29                                & 0.01              &                                           & 1.63            &
            0.0                         &
            2273.8x Faster              & ***                                                                                                                                                                                                   \\

            $latency\_p95$              & 200          & 5815.3                                 & 0.04              &                                           & 3.79            &
            0.0                         &
            1534.3x Faster              & ***                                                                                                                                                                                                   \\

            $latency\_p99$              & 200          & 7568.92                                & 0.04              &                                           & 7.13            &
            0.0                         &
            1061.2x Faster              & ***                                                                                                                                                                                                   \\

            $dropped\_iterations\_rate$ & 250          & 94.27                                  & 0.17              &                                           & 0.0             &
            0.0                         &
            N/A                         & ***                                                                                                                                                                                                   \\

            $latency\_avg$              & 250          & 4113.37                                & 0.01              &                                           & 1.89            &
            0.0                         &
            2180.6x Faster              & ***                                                                                                                                                                                                   \\

            $latency\_p50$              & 250          & 4649.66                                & 0.01              &                                           & 1.55            &
            0.0                         &
            3003.4x Faster              & ***                                                                                                                                                                                                   \\

            $latency\_p95$              & 250          & 6864.04                                & 0.03              &                                           & 3.53            &
            0.0                         &
            1943.9x Faster              & ***                                                                                                                                                                                                   \\

            $latency\_p99$              & 250          & 8594.06                                & 0.04              &                                           & 6.98            &
            0.0                         &
            1232.0x Faster              & ***                                                                                                                                                                                                   \\

            $dropped\_iterations\_rate$ & 500          & 260.24                                 & 0.21              &                                           & 0.0             &
            0.0                         &
            N/A                         & ***                                                                                                                                                                                                   \\

            $latency\_avg$              & 500          & 8357.45                                & 0.02              &                                           & 1.69            &
            0.0                         &
            4935.9x Faster              & ***                                                                                                                                                                                                   \\

            $latency\_p50$              & 500          & 9253.31                                & 0.02              &                                           & 1.38            &
            0.0                         &
            6701.6x Faster              & ***                                                                                                                                                                                                   \\

            $latency\_p95$              & 500          & 11820.67                               & 0.04              &                                           & 3.21            &
            0.0                         &
            3688.1x Faster              & ***                                                                                                                                                                                                   \\

            $latency\_p99$              & 500          & 13540.36                               & 0.07              &                                           & 7.09            &
            0.0                         &
            1909.2x Faster              & ***                                                                                                                                                                                                   \\

            $dropped\_iterations\_rate$ & 1000         & 566.0                                  & 0.14              &                                           & 0.0             &
            0.0                         &
            N/A                         & ***                                                                                                                                                                                                   \\

            $latency\_avg$              & 1000         & 16657.54                               & 0.02              &                                           & 1.83            &
            0.0                         &
            9115.5x Faster              & ***                                                                                                                                                                                                   \\

            $latency\_p50$              & 1000         & 18600.29                               & 0.04              &                                           & 1.4             & 0.0               & 13272.3x Faster & ***                   \\

            $latency\_p95$              & 1000         & 21746.72                               & 0.08              &                                           & 3.98            & 0.0               & 5464.2x Faster  & ***                   \\

            $latency\_p99$              & 1000         & 23531.71                               & 0.1               &                                           & 9.67            & 0.0               & 2433.1x Faster  & ***                   \\

            \bottomrule
        \end{tabular}
    }
    \caption{Statistical comparison for GET /lectures/all (client), aggregated over at least 20 runs}
    \label{table:run-read-all-lectures_client}
\end{table}

\begin{table}[H]
    \small \centering
    \resizebox{\columnwidth}{!}{
        \begin{tabular}{lcrrcrrcc}
            \toprule
                            &              & \multicolumn{2}{c}{\textbf{CRUD (ms)}} &                   & \multicolumn{2}{c}{\textbf{ES-CQRS (ms)}} &                 &                                                            \\
            \cmidrule{3-4} \cmidrule{6-7}
            \textbf{Metric} & \textbf{RPS} & \textbf{Median}                        & \textbf{CI $\pm$} &                                           & \textbf{Median} & \textbf{CI $\pm$} & \textbf{Ratio} & \textbf{Significance} \\
            \midrule

            $latency\_avg$  & 25           & 25.07                                  & 0.0               &                                           & 3.11            &
            0.0             &
            8.1x Faster     & ***                                                                                                                                                                                                  \\

            $latency\_p50$  & 25           & 25.31                                  & 0.0               &                                           & 2.75            &
            0.0             &
            9.2x Faster     & ***                                                                                                                                                                                                  \\

            $latency\_p95$  & 25           & 27.96                                  & 0.0               &                                           & 5.47            &
            0.0             &
            5.1x Faster     & ***                                                                                                                                                                                                  \\

            $latency\_p99$  & 25           & 33.16                                  & 0.0               &                                           & 8.06            &
            0.0             &
            4.1x Faster     & ***                                                                                                                                                                                                  \\

            $latency\_avg$  & 50           & 25.87                                  & 0.0               &                                           & 2.44            &
            0.0             &
            10.6x Faster    & ***                                                                                                                                                                                                  \\

            $latency\_p50$  & 50           & 25.43                                  & 0.0               &                                           & 2.11            &
            0.0             &
            12.0x Faster    & ***                                                                                                                                                                                                  \\

            $latency\_p95$  & 50           & 30.44                                  & 0.0               &                                           & 4.65            &
            0.0             &
            6.5x Faster     & ***                                                                                                                                                                                                  \\

            $latency\_p99$  & 50           & 33.27                                  & 0.0               &                                           & 6.89            &
            0.0             &
            4.8x Faster     & ***                                                                                                                                                                                                  \\

            $latency\_avg$  & 100          & 35.54                                  & 0.0               &                                           & 1.83            &
            0.0             &
            19.4x Faster    & ***                                                                                                                                                                                                  \\

            $latency\_p50$  & 100          & 31.3                                   & 0.0               &                                           & 1.58            &
            0.0             &
            19.9x Faster    & ***                                                                                                                                                                                                  \\

            $latency\_p95$  & 100          & 63.78                                  & 0.03              &                                           & 3.69            &
            0.0             &
            17.3x Faster    & ***                                                                                                                                                                                                  \\

            $latency\_p99$  & 100          & 125.91                                 & 0.03              &                                           & 6.18            &
            0.0             &
            20.4x Faster    & ***                                                                                                                                                                                                  \\

            $latency\_avg$  & 150          & 1565.62                                & 0.0               &                                           & 1.62            &
            0.0             &
            963.9x Faster   & ***                                                                                                                                                                                                  \\

            $latency\_p50$  & 150          & 1838.09                                & 0.0               &                                           & 1.36            &
            0.0             &
            1354.3x Faster  & ***                                                                                                                                                                                                  \\

            $latency\_p95$  & 150          & 3905.5                                 & 0.0               &                                           & 3.31            &
            0.0             &
            1178.3x Faster  & ***                                                                                                                                                                                                  \\

            $latency\_p99$  & 150          & 5609.12                                & 0.03              &                                           & 5.7             &
            0.0             &
            984.4x Faster   & ***                                                                                                                                                                                                  \\

            $latency\_avg$  & 200          & 1670.39                                & 0.01              &                                           & 1.51            &
            0.0             &
            1108.4x Faster  & ***                                                                                                                                                                                                  \\

            $latency\_p50$  & 200          & 1869.72                                & 0.0               &                                           & 1.27            &
            0.0             &
            1473.8x Faster  & ***                                                                                                                                                                                                  \\

            $latency\_p95$  & 200          & 3921.77                                & 0.0               &                                           & 3.07            &
            0.0             &
            1276.6x Faster  & ***                                                                                                                                                                                                  \\

            $latency\_p99$  & 200          & 5654.95                                & 0.01              &                                           & 5.59            &
            0.0             &
            1012.1x Faster  & ***                                                                                                                                                                                                  \\

            $latency\_avg$  & 250          & 1719.74                                & 0.0               &                                           & 1.43            &
            0.0             &
            1198.8x Faster  & ***                                                                                                                                                                                                  \\

            $latency\_p50$  & 250          & 1884.6                                 & 0.0               &                                           & 1.21            &
            0.0             &
            1551.2x Faster  & ***                                                                                                                                                                                                  \\

            $latency\_p95$  & 250          & 3931.56                                & 0.0               &                                           & 2.94            &
            0.0             &
            1338.6x Faster  & ***                                                                                                                                                                                                  \\

            $latency\_p99$  & 250          & 5688.65                                & 0.02              &                                           & 5.54            &
            0.0             &
            1025.9x Faster  & ***                                                                                                                                                                                                  \\

            $latency\_avg$  & 500          & 2014.02                                & NaN               &                                           & 1.33            &
            0.0             &
            1515.5x Faster  & n.s.                                                                                                                                                                                                 \\

            $latency\_p50$  & 500          & 1946.13                                & NaN               &                                           & 1.07            &
            0.0             &
            1823.6x Faster  & n.s.                                                                                                                                                                                                 \\

            $latency\_p95$  & 500          & 3543.35                                & NaN               &                                           & 2.69            &
            0.0             &
            1317.9x Faster  & n.s.                                                                                                                                                                                                 \\

            $latency\_p99$  & 500          & 3571.98                                & NaN               &                                           & 6.09            &
            0.0             &
            586.6x Faster   & n.s.                                                                                                                                                                                                 \\

            $latency\_avg$  & 1000         & 2504.97                                & 0.12              &                                           & 1.47            &
            0.0             &
            1701.2x Faster  & ***                                                                                                                                                                                                  \\

            $latency\_p50$  & 1000         & 2039.8                                 & 0.04              &                                           & 1.19            &
            0.0             &
            1720.2x Faster  & ***                                                                                                                                                                                                  \\

            $latency\_p95$  & 1000         & 4359.39                                & 0.58              &                                           & 3.35            &
            0.0             &
            1302.7x Faster  & ***                                                                                                                                                                                                  \\

            $latency\_p99$  & 1000         & 5471.07                                & 0.59              &                                           & 8.31            &
            0.0             &
            658.6x Faster   & ***                                                                                                                                                                                                  \\

            \bottomrule
        \end{tabular}
    }
    \caption[Statistical comparison of latencies for GET /lectures/all (server)]{Statistical comparison of latencies for GET /lectures/all (server), aggregated over at least 20 runs}
    \label{table:run-read-all-lectures_server}
\end{table}

\begin{table}[H]
    \small
    \centering
    \begin{tabular}{rrrcrrcc}
        \toprule
                     & \multicolumn{2}{c}{\textbf{CRUD (ms)}} &                   & \multicolumn{2}{c}{\textbf{ES-CQRS (ms)}} &                 &                                                            \\
        \cmidrule{2-3} \cmidrule{5-6}
        \textbf{RPS} & \textbf{Median}                        & \textbf{CI $\pm$} &                                           & \textbf{Median} & \textbf{CI $\pm$} & \textbf{Ratio} & \textbf{Significance} \\
        \midrule

        25           & 0.0188                                 & 0.0012            &                                           & 0.0302          & 0.0013            & 1.6x Higher    & ***                   \\

        50           & 0.0288                                 & 0.0008            &                                           & 0.0416          & 0.0014            & 1.4x Higher    & ***                   \\

        100          & 0.0567                                 & 0.001             &                                           & 0.0566          & 0.0033            & 1.0x           & n.s.                  \\

        200          & 0.0706                                 & 0.0059            &                                           & 0.0712          & 0.0043            & 1.0x           & n.s.                  \\

        250          & ---                                    & ---               &                                           & 0.0827          & 0.0019            & N/A            & n.s.                  \\

        500          & ---                                    & ---               &                                           & 0.1536          & 0.0008            & N/A            & n.s.                  \\

        1000         & ---                                    & ---               &                                           & 0.3043          & 0.0014            & N/A            & n.s.                  \\

        \bottomrule
    \end{tabular}
    \caption{Statistical comparison of $cpu\_usage$ for GET /lectures/all, aggregated over at least 20 runs. No data could be collected for CRUD at 250, 500, 1000 RPS}
    \label{table:run-read-all-lectures-aggregated-CPU_Usage}
\end{table}
\begin{table}[H]
    \small
    \centering
        \begin{tabular}{rrrcrrcc}
            \toprule
                         & \multicolumn{2}{c}{\textbf{CRUD (ms)}} &                   & \multicolumn{2}{c}{\textbf{ES-CQRS (ms)}} &                 &                                                            \\
            \cmidrule{2-3} \cmidrule{5-6}
            \textbf{RPS} & \textbf{Median}                        & \textbf{CI $\pm$} &                                           & \textbf{Median} & \textbf{CI $\pm$} & \textbf{Ratio} & \textbf{Significance} \\
            \midrule

            25           & 10.0                                   & 0.0               &                                           & 10.0            & 0.0               & 1.0x           & n.s.                  \\

            50           & 10.0                                   & 0.0               &                                           & 10.0            & 0.0               & 1.0x           & n.s.                  \\

            100          & 15.0                                   & 0.5               &                                           & 10.0            & 0.0               & 1.5x Lower     & ***                   \\

            200          & 96.0                                   & 87.0              &                                           & 10.0            & 0.0               & 9.6x Lower     & ***                   \\

            250          & 10.0                                   & 6.5               &                                           & 10.0            & 0.0               & 1.0x           & ***                   \\

            500          & 200.0                                  & 31.0              &                                           & 10.0            & 0.0               & 20.0x Lower    & ***                   \\

            1000         & 200.0                                  & 0.0               &                                           & 20.0            & 1.0               & 10.0x Lower    & ***                   \\

            \bottomrule
        \end{tabular}
    \caption{Statistical comparison of $tomcat\_threads$ for GET /lectures/all, aggregated over at least 20 runs}
    \label{table:run-read-all-lectures-aggregated-Threadpool_Usage}
\end{table}

\begin{table}[H]
    \small
    \centering
    \begin{tabular}{rrrcrrcc}
        \toprule
                     & \multicolumn{2}{c}{\textbf{CRUD (ms)}} &                   & \multicolumn{2}{c}{\textbf{ES-CQRS (ms)}} &                 &                                                            \\
        \cmidrule{2-3} \cmidrule{5-6}
        \textbf{RPS} & \textbf{Median}                        & \textbf{CI $\pm$} &                                           & \textbf{Median} & \textbf{CI $\pm$} & \textbf{Ratio} & \textbf{Significance} \\
        \midrule

        25           & 1.0                                    & 0.0               &                                           & 0.0             & 0.0               & NaN            & ***                   \\

        50           & 1.0                                    & 0.0               &                                           & 0.0             & 0.0               & NaN            & ***                   \\

        100          & 3.0                                    & 0.0               &                                           & 0.0             & 0.0               & NaN            & ***                   \\

        200          & 10.0                                   & 0.5               &                                           & 0.0             & 0.0               & NaN            & ***                   \\

        250          & 3.0                                    & 1.0               &                                           & 0.0             & 0.0               & NaN            & ***                   \\

        500          & 10.0                                   & 0.5               &                                           & 0.0             & 0.0               & NaN            & ***                   \\

        1000         & 10.0                                   & 5.0               &                                           & 0.0             & 0.0               & NaN            & ***                   \\

        \bottomrule
    \end{tabular}
    \caption[Comparison of $hikari\_connections$ for GET /lectures/all]{Statistical comparison of $hikari\_connections$ for GET /lectures/all, aggregated over at least 20 runs}
    \label{table:run-read-all-lectures-aggregated-Database_Connections}
\end{table}


\newpage
\section{L6: Get Credits}
\label{results:l6}

\begin{table}[H]
\small \centering
\resizebox{\columnwidth}{!}{
\begin{tabular}{lcrrcrrcc}
\toprule
& & \multicolumn{2}{c}{\textbf{CRUD (ms)}} & & \multicolumn{2}{c}{\textbf{ES-CQRS (ms)}} & & \\
\cmidrule{3-4} \cmidrule{6-7}
\textbf{Metric} & \textbf{RPS} & \textbf{Mean} & \textbf{CI $\pm$} & & \textbf{Mean} & \textbf{CI $\pm$} & \textbf{Speedup} & \textbf{Significance} \\
\midrule

    $dropped\_iterations\_rate$ & 25 & 0.0 & 0.0 & & 0.0 & 0.0 &
    NaN & n.s. \\

    $latency\_avg$ & 25 & 2.52 & 0.0 & & 2.35 & 0.0 &
    1.1x Faster & *** \\

    $latency\_p50$ & 25 & 2.42 & 0.0 & & 2.29 & 0.0 &
    1.1x Faster & *** \\

    $latency\_p95$ & 25 & 3.34 & 0.0 & & 3.08 & 0.0 &
    1.1x Faster & *** \\

    $latency\_p99$ & 25 & 4.47 & 0.0 & & 3.76 & 0.0 &
    1.2x Faster & *** \\

    $dropped\_iterations\_rate$ & 50 & 0.0 & 0.0 & & 0.0 & 0.0 &
    NaN & n.s. \\

    $latency\_avg$ & 50 & 2.09 & 0.0 & & 1.94 & 0.0 &
    1.1x Faster & *** \\

    $latency\_p50$ & 50 & 1.98 & 0.0 & & 1.85 & 0.0 &
    1.1x Faster & *** \\

    $latency\_p95$ & 50 & 2.86 & 0.0 & & 2.67 & 0.0 &
    1.1x Faster & *** \\

    $latency\_p99$ & 50 & 3.94 & 0.0 & & 3.43 & 0.0 &
    1.1x Faster & *** \\

    $dropped\_iterations\_rate$ & 100 & 0.0 & 0.0 & & 0.0 & 0.0 &
    NaN & n.s. \\

    $latency\_avg$ & 100 & 1.77 & 0.0 & & 1.59 & 0.0 &
    1.1x Faster & *** \\

    $latency\_p50$ & 100 & 1.67 & 0.0 & & 1.51 & 0.0 &
    1.1x Faster & *** \\

    $latency\_p95$ & 100 & 2.42 & 0.0 & & 2.24 & 0.0 &
    1.1x Faster & *** \\

    $latency\_p99$ & 100 & 3.49 & 0.0 & & 3.17 & 0.0 &
    1.1x Faster & *** \\

    $dropped\_iterations\_rate$ & 200 & 0.0 & 0.0 & & 0.0 & 0.0 &
    NaN & n.s. \\

    $latency\_avg$ & 200 & 1.46 & 0.0 & & 1.25 & 0.0 &
    1.2x Faster & *** \\

    $latency\_p50$ & 200 & 1.37 & 0.0 & & 1.15 & 0.0 &
    1.2x Faster & *** \\

    $latency\_p95$ & 200 & 2.04 & 0.0 & & 1.84 & 0.0 &
    1.1x Faster & *** \\

    $latency\_p99$ & 200 & 3.01 & 0.0 & & 2.79 & 0.0 &
    1.1x Faster & *** \\

    $dropped\_iterations\_rate$ & 500 & 0.0 & 0.0 & & 0.0 & 0.0 &
    NaN & n.s. \\

    $latency\_avg$ & 500 & 1.37 & 0.0 & & 1.07 & 0.0 &
    1.3x Faster & *** \\

    $latency\_p50$ & 500 & 1.3 & 0.0 & & 0.99 & 0.0 &
    1.3x Faster & *** \\

    $latency\_p95$ & 500 & 1.86 & 0.0 & & 1.44 & 0.0 &
    1.3x Faster & *** \\

    $latency\_p99$ & 500 & 2.54 & 0.0 & & 2.28 & 0.0 &
    1.1x Faster & *** \\

    $dropped\_iterations\_rate$ & 1000 & 0.0 & 0.0 & & 0.0 & 0.0 &
    NaN & n.s. \\

    $latency\_avg$ & 1000 & 1.4 & 0.0 & & 1.04 & 0.0 &
    1.3x Faster & *** \\

    $latency\_p50$ & 1000 & 1.3 & 0.0 & & 0.97 & 0.0 &
    1.3x Faster & *** \\

    $latency\_p95$ & 1000 & 1.96 & 0.0 & & 1.32 & 0.0 &
    1.5x Faster & *** \\

    $latency\_p99$ & 1000 & 3.02 & 0.0 & & 2.36 & 0.0 &
    1.3x Faster & *** \\

    $dropped\_iterations\_rate$ & 2000 & 0.0 & 0.0 & & 0.0 & 0.0 &
    NaN & n.s. \\

    $latency\_avg$ & 2000 & 2.78 & 0.0 & & 1.23 & 0.0 &
    2.3x Faster & *** \\

    $latency\_p50$ & 2000 & 1.98 & 0.0 & & 1.05 & 0.0 &
    1.9x Faster & *** \\

    $latency\_p95$ & 2000 & 5.2 & 0.0 & & 2.01 & 0.0 &
    2.6x Faster & *** \\

    $latency\_p99$ & 2000 & 17.9 & 0.0 & & 4.54 & 0.0 &
    3.9x Faster & *** \\

    $dropped\_iterations\_rate$ & 3000 & 354.74 & 1.81 & & 0.0 & 0.0 &
    infx Faster & *** \\

    $latency\_avg$ & 3000 & 959.0 & 0.01 & & 2.86 & 0.0 &
    334.8x Faster & *** \\

    $latency\_p50$ & 3000 & 1248.93 & 0.0 & & 1.85 & 0.0 &
    674.6x Faster & *** \\

    $latency\_p95$ & 3000 & 1508.36 & 0.01 & & 7.1 & 0.0 &
    212.3x Faster & *** \\

    $latency\_p99$ & 3000 & 1606.26 & 0.01 & & 19.67 & 0.0 &
    81.7x Faster & *** \\

    $dropped\_iterations\_rate$ & 4000 & 977.73 & 5.85 & & 0.0 & 0.0 &
    infx Faster & *** \\

    $latency\_avg$ & 4000 & 1476.68 & 0.01 & & 11.29 & 0.0 &
    130.8x Faster & *** \\

    $latency\_p50$ & 4000 & 1729.38 & 0.01 & & 4.5 & 0.0 &
    384.3x Faster & *** \\

    $latency\_p95$ & 4000 & 2024.84 & 0.01 & & 51.3 & 0.01 &
    39.5x Faster & *** \\

    $latency\_p99$ & 4000 & 2170.06 & 0.02 & & 108.37 & 0.01 &
    20.0x Faster & *** \\

    $dropped\_iterations\_rate$ & 5000 & 1583.63 & 5.56 & & 38.26 & 12.42 &
    41.4x Faster & *** \\

    $latency\_avg$ & 5000 & 1860.04 & 0.01 & & 604.23 & 0.1 &
    3.1x Faster & *** \\

    $latency\_p50$ & 5000 & 2152.05 & 0.01 & & 674.29 & 0.15 &
    3.2x Faster & *** \\

    $latency\_p95$ & 5000 & 2465.92 & 0.01 & & 1035.72 & 0.02 &
    2.4x Faster & *** \\

    $latency\_p99$ & 5000 & 2658.04 & 0.03 & & 1120.06 & 0.02 &
    2.4x Faster & *** \\

\bottomrule
\end{tabular}
}
\caption{Statistical comparison for POST /courses (client), averaged out over at least 25 runs}
\label{table:run-get-credits_client}
\end{table}
\begin{table}[H]
    \small \centering
    \resizebox{\columnwidth}{!}{
        \begin{tabular}{lcrrcrrcc}
            \toprule
                            &              & \multicolumn{2}{c}{\textbf{CRUD (ms)}} &                   & \multicolumn{2}{c}{\textbf{ES-CQRS (ms)}} &                 &                                                            \\
            \cmidrule{3-4} \cmidrule{6-7}
            \textbf{Metric} & \textbf{RPS} & \textbf{Median}                        & \textbf{CI $\pm$} &                                           & \textbf{Median} & \textbf{CI $\pm$} & \textbf{Ratio} & \textbf{Significance} \\
            \midrule

            $latency\_avg$  & 25           & 1.64                                   & 0.0               &                                           & 1.49            & 0.0               & 1.1x Faster    & ***                   \\

            $latency\_p50$  & 25           & 1.57                                   & 0.0               &                                           & 1.44            & 0.0               & 1.1x Faster    & ***                   \\

            $latency\_p95$  & 25           & 2.37                                   & 0.0               &                                           & 2.09            & 0.0               & 1.1x Faster    & ***                   \\

            $latency\_p99$  & 25           & 3.01                                   & 0.0               &                                           & 2.5             & 0.0               & 1.2x Faster    & ***                   \\

            \addlinespace

            $latency\_avg$  & 50           & 1.37                                   & 0.0               &                                           & 1.19            & 0.0               & 1.1x Faster    & ***                   \\

            $latency\_p50$  & 50           & 1.29                                   & 0.0               &                                           & 1.14            & 0.0               & 1.1x Faster    & ***                   \\

            $latency\_p95$  & 50           & 2.03                                   & 0.0               &                                           & 1.79            & 0.0               & 1.1x Faster    & ***                   \\

            $latency\_p99$  & 50           & 2.66                                   & 0.0               &                                           & 2.27            & 0.0               & 1.2x Faster    & ***                   \\

            \addlinespace

            $latency\_avg$  & 100          & 1.16                                   & 0.0               &                                           & 0.94            & 0.0               & 1.2x Faster    & ***                   \\

            $latency\_p50$  & 100          & 1.1                                    & 0.0               &                                           & 0.69            & 0.0               & 1.6x Faster    & ***                   \\

            $latency\_p95$  & 100          & 1.72                                   & 0.0               &                                           & 1.47            & 0.0               & 1.2x Faster    & ***                   \\

            $latency\_p99$  & 100          & 2.32                                   & 0.0               &                                           & 1.98            & 0.0               & 1.2x Faster    & ***                   \\

            \addlinespace

            $latency\_avg$  & 200          & 1.08                                   & 0.0               &                                           & 0.82            & 0.0               & 1.3x Faster    & ***                   \\

            $latency\_p50$  & 200          & 1.0                                    & 0.0               &                                           & 0.58            & 0.0               & 1.7x Faster    & ***                   \\

            $latency\_p95$  & 200          & 1.6                                    & 0.0               &                                           & 1.29            & 0.0               & 1.2x Faster    & ***                   \\

            $latency\_p99$  & 200          & 2.06                                   & 0.0               &                                           & 1.71            & 0.0               & 1.2x Faster    & ***                   \\

            \addlinespace

            $latency\_avg$  & 500          & 1.08                                   & 0.0               &                                           & 0.74            & 0.0               & 1.5x Faster    & ***                   \\

            $latency\_p50$  & 500          & 1.02                                   & 0.0               &                                           & 0.53            & 0.0               & 1.9x Faster    & ***                   \\

            $latency\_p95$  & 500          & 1.58                                   & 0.0               &                                           & 1.02            & 0.0               & 1.5x Faster    & ***                   \\

            $latency\_p99$  & 500          & 1.99                                   & 0.0               &                                           & 1.58            & 0.0               & 1.3x Faster    & ***                   \\

            \addlinespace

            $latency\_avg$  & 1000         & 1.13                                   & 0.0               &                                           & 0.74            & 0.0               & 1.5x Faster    & ***                   \\

            $latency\_p50$  & 1000         & 1.05                                   & 0.0               &                                           & 0.52            & 0.0               & 2.0x Faster    & ***                   \\

            $latency\_p95$  & 1000         & 1.67                                   & 0.0               &                                           & 0.99            & 0.0               & 1.7x Faster    & ***                   \\

            $latency\_p99$  & 1000         & 2.38                                   & 0.0               &                                           & 1.66            & 0.0               & 1.4x Faster    & ***                   \\

            \addlinespace

            $latency\_avg$  & 2000         & 2.15                                   & 0.0               &                                           & 0.92            & 0.0               & 2.3x Faster    & ***                   \\

            $latency\_p50$  & 2000         & 1.54                                   & 0.0               &                                           & 0.61            & 0.0               & 2.5x Faster    & ***                   \\

            $latency\_p95$  & 2000         & 3.98                                   & 0.0               &                                           & 1.6             & 0.0               & 2.5x Faster    & ***                   \\

            $latency\_p99$  & 2000         & 9.58                                   & 0.0               &                                           & 3.38            & 0.0               & 2.8x Faster    & ***                   \\

            \addlinespace

            $latency\_avg$  & 3000         & 72.22                                  & 0.0               &                                           & 2.29            & 0.0               & 31.6x Faster   & ***                   \\

            $latency\_p50$  & 3000         & 6.12                                   & 0.0               &                                           & 1.53            & 0.0               & 4.0x Faster    & ***                   \\

            $latency\_p95$  & 3000         & 261.79                                 & 0.0               &                                           & 6.22            & 0.0               & 42.1x Faster   & ***                   \\

            $latency\_p99$  & 3000         & 344.01                                 & 0.0               &                                           & 13.62           & 0.0               & 25.3x Faster   & ***                   \\

            \addlinespace

            $latency\_avg$  & 4000         & 79.02                                  & 0.0               &                                           & 8.81            & 0.0               & 9.0x Faster    & ***                   \\

            $latency\_p50$  & 4000         & 6.81                                   & 0.0               &                                           & 3.78            & 0.0               & 1.8x Faster    & ***                   \\

            $latency\_p95$  & 4000         & 263.43                                 & 0.0               &                                           & 39.23           & 0.0               & 6.7x Faster    & ***                   \\

            $latency\_p99$  & 4000         & 350.1                                  & 0.0               &                                           & 55.89           & 0.0               & 6.3x Faster    & ***                   \\

            \addlinespace

            $latency\_avg$  & 5000         & 78.74                                  & 0.0               &                                           & 34.61           & 0.0               & 2.3x Faster    & ***                   \\

            $latency\_p50$  & 5000         & 6.81                                   & 0.0               &                                           & 38.01           & 0.0               & 5.6x Slower    & ***                   \\

            $latency\_p95$  & 5000         & 262.91                                 & 0.0               &                                           & 53.99           & 0.0               & 4.9x Faster    & ***                   \\

            $latency\_p99$  & 5000         & 349.45                                 & 0.0               &                                           & 63.7            & 0.0               & 5.5x Faster    & ***                   \\

            \bottomrule
        \end{tabular}
    }
    \caption[Statistical comparison of latencies for GET /stats/credits (server)]{Statistical comparison of latencies for GET /stats/credits (server), aggregated over at least 25 runs}
    \label{table:run-get-credits_server}
\end{table}
\begin{table}[H]
    \small
    \centering
    \begin{tabular}{rrrcrrcc}
        \toprule
                     & \multicolumn{2}{c}{\textbf{CRUD (ms)}} &                   & \multicolumn{2}{c}{\textbf{ES-CQRS (ms)}} &                 &                                                            \\
        \cmidrule{2-3} \cmidrule{5-6}
        \textbf{RPS} & \textbf{Median}                        & \textbf{CI $\pm$} &                                           & \textbf{Median} & \textbf{CI $\pm$} & \textbf{Ratio} & \textbf{Significance} \\
        \midrule

        25           & 0.01                                   & 0.0006            &                                           & 0.0113          & 0.0               &
        1.1x Higher  & ***                                                                                                                                                                                   \\

        50           & 0.0163                                 & 0.0006            &                                           & 0.0189          & 0.0012            &
        1.2x Higher  & **                                                                                                                                                                                    \\

        100          & 0.0264                                 & 0.0013            &                                           & 0.0266          & 0.0019            &
        1.0x         & *                                                                                                                                                                                     \\

        200          & 0.0418                                 & 0.0025            &                                           & 0.0382          & 0.0007            &
        1.1x Lower   & n.s.                                                                                                                                                                                  \\

        500          & 0.0918                                 & 0.0022            &                                           & 0.0857          & 0.0008            &
        1.1x Lower   & ***                                                                                                                                                                                   \\

        1000         & 0.2069                                 & 0.0023            &                                           & 0.1681          & 0.0015            &
        1.2x Lower   & ***                                                                                                                                                                                   \\

        2000         & 0.4454                                 & 0.0023            &                                           & 0.3535          & 0.0035            &
        1.3x Lower   & ***                                                                                                                                                                                   \\

        3000         & 0.5763                                 & 0.0006            &                                           & 0.5338          & 0.0015            &
        1.1x Lower   & ***                                                                                                                                                                                   \\

        4000         & 0.5608                                 & 0.0007            &                                           & 0.572           & 0.0067            &
        1.0x         & ***                                                                                                                                                                                   \\

        5000         & 0.5703                                 & 0.24              &                                           & 0.6996          & 0.0018            &
        1.2x Higher  & ***                                                                                                                                                                                   \\

        \bottomrule
    \end{tabular}
    \caption[Comparison of $cpu\_usage$ for POST /lectures/create; then /GET/\{lectureId\}]{Statistical comparison of $cpu\_usage$ for POST /lectures/create; then /GET/\{lectureId\}, aggregated over at least 25 runs}
    \label{table:run-get-credits-aggregated-CPU_Usage}
\end{table}
\begin{table}[H]
    \small
    \centering
    \resizebox{\columnwidth}{!}{
        \begin{tabular}{rrrcrrcc}
            \toprule
                         & \multicolumn{2}{c}{\textbf{CRUD (ms)}} &                   & \multicolumn{2}{c}{\textbf{ES-CQRS (ms)}} &                 &                                                            \\
            \cmidrule{2-3} \cmidrule{5-6}
            \textbf{RPS} & \textbf{Median}                        & \textbf{CI $\pm$} &                                           & \textbf{Median} & \textbf{CI $\pm$} & \textbf{Ratio} & \textbf{Significance} \\
            \midrule

            25           & 10.0                                   & 0.0               &                                           & 10.0            & 0.0               &
            1.0x Lower   & n.s.                                                                                                                                                                                  \\

            50           & 10.0                                   & 0.0               &                                           & 10.0            & 0.0               &
            1.0x Lower   & n.s.                                                                                                                                                                                  \\

            100          & 10.0                                   & 0.0               &                                           & 10.0            & 0.0               &
            1.0x Lower   & n.s.                                                                                                                                                                                  \\

            200          & 10.0                                   & 0.0               &                                           & 10.0            & 0.0               &
            1.0x Lower   & n.s.                                                                                                                                                                                  \\

            500          & 10.0                                   & 0.0               &                                           & 10.0            & 0.0               &
            1.0x Lower   & ***                                                                                                                                                                                   \\

            1000         & 11.0                                   & 0.5               &                                           & 16.0            & 0.0               &
            1.5x Higher  & ***                                                                                                                                                                                   \\

            2000         & 36.0                                   & 3.0               &                                           & 31.0            & 0.5               &
            1.2x Lower   & ***                                                                                                                                                                                   \\

            3000         & 200.0                                  & 0.0               &                                           & 123.0           & 10.5              &
            1.6x Lower   & ***                                                                                                                                                                                   \\

            4000         & 200.0                                  & 0.0               &                                           & 200.0           & 0.0               &
            1.0x Lower   & n.s.                                                                                                                                                                                  \\

            \bottomrule
        \end{tabular}
    }
    \caption{Statistical comparison of $tomcat\_threads$ for GET /stats/credits, averaged out over at least 25 runs}
    \label{table:run-get-credits-aggregated-Threadpool_Usage}
\end{table}
\begin{table}[H]
    \small
    \centering
    \begin{tabular}{rrrcrrcc}
        \toprule
                     & \multicolumn{2}{c}{\textbf{CRUD (ms)}} &                   & \multicolumn{2}{c}{\textbf{ES-CQRS (ms)}} &                 &                                                            \\
        \cmidrule{2-3} \cmidrule{5-6}
        \textbf{RPS} & \textbf{Median}                        & \textbf{CI $\pm$} &                                           & \textbf{Median} & \textbf{CI $\pm$} & \textbf{Ratio} & \textbf{Significance} \\
        \midrule

        25           & 0.0                                    & 0.0               &                                           & 0.0             & 0.0               &
        $N/A$        & n.s.                                                                                                                                                                                  \\

        50           & 0.0                                    & 0.0               &                                           & 0.0             & 0.0               &
        $N/A$        & n.s.                                                                                                                                                                                  \\

        100          & 0.0                                    & 0.0               &                                           & 0.0             & 0.0               &
        $N/A$        & ***                                                                                                                                                                                   \\

        200          & 0.0                                    & 0.0               &                                           & 0.0             & 0.0               &
        $N/A$        & ***                                                                                                                                                                                   \\

        500          & 0.0                                    & 0.5               &                                           & 0.0             & 0.0               &
        $N/A$        & ***                                                                                                                                                                                   \\

        1000         & 1.0                                    & 0.0               &                                           & 0.0             & 0.0               &
        $N/A$        & ***                                                                                                                                                                                   \\

        2000         & 3.0                                    & 0.0               &                                           & 0.0             & 0.0               &
        $N/A$        & ***                                                                                                                                                                                   \\

        3000         & 10.0                                   & 0.0               &                                           & 1.0             & 0.0               &
        10.0x Lower  & ***                                                                                                                                                                                   \\

        4000         & 10.0                                   & 0.0               &                                           & 2.0             & 0.5               &
        5.0x Lower   & ***                                                                                                                                                                                   \\

        5000         & 9.0                                    & 1.0               &                                           & 7.0             & 0.5               &
        1.3x Lower   & **                                                                                                                                                                                    \\

        \bottomrule
    \end{tabular}
    \caption{Statistical comparison of $hikari\_connections$ for POST /lectures/create; then /GET/\{lectureId\}, aggregated over at least 25 runs}
    \label{table:run-get-credits-aggregated-Database_Connections}
\end{table}

\newpage
\section{L7: Time to Consistency}
\label{results:l7}

This load test consists of 2 requests per iteration. Therefore, the client-side latencies for both requests are listed separately. In a third table, the dropped iterations are listed.

\begin{table}[H]
    \small \centering
    \resizebox{\columnwidth}{!}{
        \begin{tabular}{lcrrcrrcc}
            \toprule
                            &              & \multicolumn{2}{c}{\textbf{CRUD (ms)}} &                   & \multicolumn{2}{c}{\textbf{ES-CQRS (ms)}} &               &                                                                          \\
            \cmidrule{3-4} \cmidrule{6-7}
            \textbf{Metric} & \textbf{RPS} & \textbf{Mean}                          & \textbf{CI $\pm$} &                                           & \textbf{Mean} & \textbf{CI $\pm$}             & \textbf{Speedup} & \textbf{Significance} \\
            \midrule

            $latency\_avg$  & 10           & 6.96                                   & 0.3               &                                           & 9.3           & 0.06                          & 1.3x Slower      & ***                   \\

            $latency\_p50$  & 10           & 5.69                                   & 0.05              &                                           & 8.31          & 0.03             1.5x Slower  & ***                                      \\

            $latency\_p95$  & 10           & 15.2                                   & 0.08              &                                           & 17.6          & 0.37             1.2x Slower  & ***                                      \\

            $latency\_p99$  & 10           & 17.01                                  & 0.09              &                                           & 21.91         & 0.21              1.3x Slower & ***                                      \\

            $latency\_avg$  & 20           & 5.08                                   & 0.03              &                                           & 7.61          & 0.02                          & 1.5x Slower      & ***                   \\

            $latency\_p50$  & 20           & 4.62                                   & 0.04              &                                           & 6.81          & 0.04                          & 1.5x Slower      & ***                   \\

            $latency\_p95$  & 20           & 7.27                                   & 0.07              &                                           & 11.8          & 0.09                          & 1.6x Slower      & ***                   \\

            $latency\_p99$  & 20           & 15.96                                  & 0.13              &                                           & 20.03         & 0.28                          & 1.3x Slower      & ***                   \\

            $latency\_avg$  & 50           & 3.67                                   & 0.02              &                                           & 5.95          & 0.02                          & 1.6x Slower      & ***                   \\

            $latency\_p50$  & 50           & 3.34                                   & 0.01              &                                           & 5.31          & 0.01                          & 1.6x Slower      & ***                   \\

            $latency\_p95$  & 50           & 5.83                                   & 0.04              &                                           & 10.22         & 0.08                          &
            1.8x Slower     & ***                                                                                                                                                                                                              \\

            $latency\_p99$  & 50           & 7.79                                   & 0.06              &                                           & 13.83         & 0.19                          &
            1.8x Slower     & ***                                                                                                                                                                                                              \\

            $latency\_avg$  & 100          & 2.97                                   & 0.02              &                                           & 4.65          & 0.01                          &
            1.6x Slower     & ***                                                                                                                                                                                                              \\

            $latency\_p50$  & 100          & 2.73                                   & 0.02              &                                           & 4.02          & 0.01                          &
            1.5x Slower     & ***                                                                                                                                                                                                              \\

            $latency\_p95$  & 100          & 4.76                                   & 0.07              &                                           & 8.73          & 0.04                          &
            1.8x Slower     & ***                                                                                                                                                                                                              \\

            $latency\_p99$  & 100          & 6.89                                   & 0.05              &                                           & 13.37         & 0.1                           &
            1.9x Slower     & ***                                                                                                                                                                                                              \\

            $latency\_avg$  & 200          & 2.58                                   & 0.01              &                                           & 4.45          & 0.04                          &
            1.7x Slower     & ***                                                                                                                                                                                                              \\

            $latency\_p50$  & 200          & 2.37                                   & 0.01              &                                           & 3.49          & 0.02                          &
            1.5x Slower     & ***                                                                                                                                                                                                              \\

            $latency\_p95$  & 200          & 3.83                                   & 0.05              &                                           & 9.85          & 0.19                          &
            2.6x Slower     & ***                                                                                                                                                                                                              \\

            $latency\_p99$  & 200          & 6.01                                   & 0.04              &                                           & 16.89         & 0.77                          &
            2.8x Slower     & ***                                                                                                                                                                                                              \\

            $latency\_avg$  & 300          & 2.57                                   & 0.01              &                                           & 6.78          & 0.17                          &
            2.6x Slower     & ***                                                                                                                                                                                                              \\

            $latency\_p50$  & 300          & 2.37                                   & 0.01              &                                           & 3.62          & 0.02                          &
            1.5x Slower     & ***                                                                                                                                                                                                              \\

            $latency\_p95$  & 300          & 3.69                                   & 0.06              &                                           & 22.11         & 0.96                          &
            6.0x Slower     & ***                                                                                                                                                                                                              \\

            $latency\_p99$  & 300          & 6.44                                   & 0.11              &                                           & 40.55         & 1.39                          &
            6.3x Slower     & ***                                                                                                                                                                                                              \\

            $latency\_avg$  & 400          & 2.56                                   & 0.01              &                                           & 17.11         & 1.53                          &
            6.7x Slower     & ***                                                                                                                                                                                                              \\

            $latency\_p50$  & 400          & 2.36                                   & 0.01              &                                           & 4.79          & 0.06                          &
            2.0x Slower     & ***                                                                                                                                                                                                              \\

            $latency\_p95$  & 400          & 3.7                                    & 0.05              &                                           & 82.88         & 12.31                         &
            22.4x Slower    & ***                                                                                                                                                                                                              \\

            $latency\_p99$  & 400          & 6.68                                   & 0.15              &                                           & 151.14        & 25.62                         &
            22.6x Slower    & ***                                                                                                                                                                                                              \\

            $latency\_avg$  & 500          & 2.65                                   & 0.01              &                                           & 178.82        & 6.77                          &
            67.4x Slower    & ***                                                                                                                                                                                                              \\

            $latency\_p50$  & 500          & 2.44                                   & 0.01              &                                           & 6.93          & 0.33                          &
            2.8x Slower     & ***                                                                                                                                                                                                              \\

            $latency\_p95$  & 500          & 3.82                                   & 0.04              &                                           & 957.33        & 38.15                         &
            250.4x Slower   & ***                                                                                                                                                                                                              \\

            $latency\_p99$  & 500          & 7.03                                   & 0.07              &                                           & 1158.17       & 27.85                         &
            164.7x Slower   & ***                                                                                                                                                                                                              \\

            \bottomrule
        \end{tabular}
    }
    \caption{Statistical comparison for POST /lectures/create, averaged out over at least 25 runs}
    \label{table:run-create-lecture_POST_client}
\end{table}

\begin{table}[H]
    \small \centering
    \resizebox{\columnwidth}{!}{
        \begin{tabular}{lcrrcrrcc}
            \toprule
                            &              & \multicolumn{2}{c}{\textbf{CRUD (ms)}} &                   & \multicolumn{2}{c}{\textbf{ES-CQRS (ms)}} &                 &                                                            \\
            \cmidrule{3-4} \cmidrule{6-7}
            \textbf{Metric} & \textbf{RPS} & \textbf{Median}                        & \textbf{CI $\pm$} &                                           & \textbf{Median} & \textbf{CI $\pm$} & \textbf{Ratio} & \textbf{Significance} \\
            \midrule

            $latency\_avg$  & 10           & 3.02                                   & 0.01              &                                           & 3.69            & 0.02              & 1.2x Slower    & ***                   \\

            $latency\_p50$  & 10           & 2.75                                   & 0.02              &                                           & 3.24            & 0.02              & 1.2x Slower    & ***                   \\

            $latency\_p95$  & 10           & 4.8                                    & 0.03              &                                           & 6.43            & 0.06              & 1.3x Slower    & ***                   \\

            $latency\_p99$  & 10           & 6.45                                   & 0.09              &                                           & 7.78            & 0.06              & 1.2x Slower    & ***                   \\

            $latency\_avg$  & 20           & 2.51                                   & 0.01              &                                           & 3.03            & 0.01              & 1.2x Slower    & ***                   \\

            $latency\_p50$  & 20           & 2.28                                   & 0.02              &                                           & 2.58            & 0.01              & 1.1x Slower    & ***                   \\

            $latency\_p95$  & 20           & 4.03                                   & 0.03              &                                           & 5.66            & 0.04              & 1.4x Slower    & ***                   \\

            $latency\_p99$  & 20           & 5.52                                   & 0.11              &                                           & 7.58            & 0.07              & 1.4x Slower    & ***                   \\

            $latency\_avg$  & 50           & 1.87                                   & 0.01              &                                           & 2.34            & 0.01              & 1.2x Slower    & ***                   \\

            $latency\_p50$  & 50           & 1.67                                   & 0.01              &                                           & 1.98            & 0.01              & 1.2x Slower    & ***                   \\

            $latency\_p95$  & 50           & 3.13                                   & 0.01              &                                           & 4.84            & 0.04              & 1.5x Slower    & ***                   \\

            $latency\_p99$  & 50           & 4.55                                   & 0.05              &                                           & 7.04            & 0.08              & 1.5x Slower    & ***                   \\

            $latency\_avg$  & 100          & 1.67                                   & 0.05              &                                           & 1.84            & 0.01              & 1.1x Slower    & ***                   \\

            $latency\_p50$  & 100          & 1.52                                   & 0.08              &                                           & 1.49            & 0.01              & 1.0x Faster    & n.s.                  \\

            $latency\_p95$  & 100          & 2.7                                    & 0.03              &                                           & 4.1             & 0.04              & 1.5x Slower    & ***                   \\

            $latency\_p99$  & 100          & 3.93                                   & 0.05              &                                           & 7.07            & 0.11              & 1.8x Slower    & ***                   \\

            $latency\_avg$  & 200          & 1.56                                   & 0.03              &                                           & 1.82            & 0.02              & 1.2x Slower    & ***                   \\

            $latency\_p50$  & 200          & 1.36                                   & 0.02              &                                           & 1.31            & 0.01              & 1.0x Faster    & ***                   \\

            $latency\_p95$  & 200          & 2.4                                    & 0.02              &                                           & 4.69            & 0.1               & 2.0x Slower    & ***                   \\

            $latency\_p99$  & 200          & 3.5                                    & 0.03              &                                           & 9.02            & 0.36              & 2.6x Slower    & ***                   \\

            $latency\_avg$  & 300          & 1.67                                   & 0.01              &                                           & 3.1             & 0.1               & 1.9x Slower    & ***                   \\

            $latency\_p50$  & 300          & 1.52                                   & 0.01              &                                           & 1.38            & 0.01              & 1.1x Faster    & ***                   \\

            $latency\_p95$  & 300          & 2.56                                   & 0.02              &                                           & 11.46           & 0.5               & 4.5x Slower    & ***                   \\

            $latency\_p99$  & 300          & 3.7                                    & 0.05              &                                           & 23.15           & 0.81              & 6.3x Slower    & ***                   \\

            $latency\_avg$  & 400          & 1.86                                   & 0.03              &                                           & 7.14            & 0.33              & 3.8x Slower    & ***                   \\

            $latency\_p50$  & 400          & 1.74                                   & 0.03              &                                           & 1.71            & 0.03              & 1.0x Faster    & n.s.                  \\

            $latency\_p95$  & 400          & 2.89                                   & 0.02              &                                           & 32.75           & 2.11              & 11.3x Slower   & ***                   \\

            $latency\_p99$  & 400          & 4.02                                   & 0.06              &                                           & 59.97           & 2.87              & 14.9x Slower   & ***                   \\

            $latency\_avg$  & 500          & 2.16                                   & 0.03              &                                           & 94.17           & 3.68              & 43.6x Slower   & ***                   \\

            $latency\_p50$  & 500          & 2.12                                   & 0.04              &                                           & 2.71            & 0.19              & 1.3x Slower    & ***                   \\

            $latency\_p95$  & 500          & 3.32                                   & 0.04              &                                           & 595.97          & 28.16             & 179.7x Slower  & ***                   \\

            $latency\_p99$  & 500          & 4.6                                    & 0.14              &                                           & 737.46          & 18.52             & 160.3x Slower  & ***                   \\

            \bottomrule
        \end{tabular}
    }
    \caption[Statistical comparison of latencies for GET /lectures/\{lectureId\} (client)]{Statistical comparison of latencies for GET /lectures/\{lectureId\} (client), aggregated over at least 25 runs}
    \label{table:run-create-lecture_client_GET}
\end{table}

\begin{table}[H]
    \small
    \centering
    \begin{tabular}{lcrrcrrcc}
        \toprule
                     & \multicolumn{2}{c}{\textbf{CRUD (ms)}} &                   & \multicolumn{2}{c}{\textbf{ES-CQRS (ms)}} &                 &                                                            \\
        \cmidrule{2-3} \cmidrule{5-6}
        \textbf{RPS} & \textbf{Median}                        & \textbf{CI $\pm$} &                                           & \textbf{Median} & \textbf{CI $\pm$} & \textbf{Ratio} & \textbf{Significance} \\
        \midrule

        10           & 0                                      & 0                 &                                           & 0               & 0                 & NaN            & n.s.                  \\
        25           & 0                                      & 0                 &                                           & 0               & 0                 & NaN            & n.s.                  \\
        50           & 0                                      & 0                 &                                           & 0               & 0                 & NaN            & n.s.                  \\
        100          & 0                                      & 0                 &                                           & 0               & 0                 & NaN            & n.s.                  \\
        200          & 0                                      & 0                 &                                           & 0               & 0                 & NaN            & n.s.                  \\
        300          & 0                                      & 0                 &                                           & 0               & 0                 & NaN            & n.s.                  \\
        400          & 0                                      & 0                 &                                           & 0               & 0                 & NaN            & n.s.                  \\
        500          & 0                                      & 0                 &                                           & 8.5             & 1.12              & NaN            & n.s.                  \\

        \bottomrule
    \end{tabular}
    \caption{Statistical comparison of $dropped\_iterations\_rate$ for POST /lectures/create, then GET /lectures/\{lectureId\}, aggregated over at least 25 runs}
    \label{table:run-create-lecture_client_dropped_iterations}
\end{table}


\begin{table}[H]
\small
\centering
\resizebox{\columnwidth}{!}{
\begin{tabular}{rrrcrrcc}
\toprule
& \multicolumn{2}{c}{\textbf{CRUD (ms)}} &                   & \multicolumn{2}{c}{\textbf{ES-CQRS (ms)}} &               &                                                            \\
\cmidrule{2-3} \cmidrule{5-6}
\textbf{RPS} & \textbf{Median}                          & \textbf{CI $\pm$} &                                           & \textbf{Median} & \textbf{CI $\pm$} & \textbf{Ratio} & \textbf{Significance} \\
\midrule

    10 & 0.0201 & 0.0013 & & 0.0391 & 0.0024 & 1.9x Higher & *** \\

    20 & 0.0288 & 0.0021 & & 0.048 & 0.0031 & 1.7x Higher & *** \\

    50 & 0.0328 & 0.0019 & & 0.0658 & 0.0044 & 2.0x Higher & *** \\

    100 & 0.0456 & 0.0059 & & 0.1101 & 0.0049 & 2.4x Higher & *** \\

    200 & 0.0682 & 0.002 & & 0.1779 & 0.0025 & 2.6x Higher & *** \\

    300 & 0.0951 & 0.0018 & & 0.2602 & 0.0018 & 2.7x Higher & *** \\

    400 & 0.1249 & 0.0009 & & 0.3566 & 0.0029 & 2.9x Higher & *** \\

    500 & 0.1601 & 0.001 & & 0.426 & 0.0035 & 2.7x Higher & *** \\

\bottomrule
\end{tabular}
}
\caption{Statistical comparison of $cpu\_usage$ for POST /lectures/create; then /GET/\{lectureId\}, averaged out over at least 25 runs}
\label{table:run-create-lecture-aggregated-CPU_Usage}
\end{table}
\begin{table}[H]
\small
\centering
\resizebox{\columnwidth}{!}{
\begin{tabular}{rrrcrrcc}
\toprule
& \multicolumn{2}{c}{\textbf{CRUD (ms)}} &                   & \multicolumn{2}{c}{\textbf{ES-CQRS (ms)}} &               &                                                            \\
\cmidrule{2-3} \cmidrule{5-6}
\textbf{RPS} & \textbf{Median}                          & \textbf{CI $\pm$} &                                           & \textbf{Median} & \textbf{CI $\pm$} & \textbf{Ratio} & \textbf{Significance} \\
\midrule

    10 & 10.0 & 0.0 & & 10.0 & 0.0 & 1.0x & n.s. \\

    20 & 10.0 & 0.0 & & 10.0 & 0.0 & 1.0x & n.s. \\

    50 & 10.0 & 0.0 & & 10.0 & 0.0 & 1.0x & n.s. \\

    100 & 10.0 & 0.0 & & 10.0 & 0.0 & 1.0x & n.s. \\

    200 & 10.0 & 0.0 & & 11.0 & 0.0 & 1.1x Higher & *** \\

    300 & 11.0 & 0.5 & & 39.0 & 1.0 & 3.5x Higher & *** \\

    400 & 13.0 & 0.0 & & 100.0 & 1.5 & 7.7x Higher & *** \\

    500 & 17.0 & 0.0 & & 200.0 & 0.0 & 11.8x Higher & *** \\

\bottomrule
\end{tabular}
}
\caption{Statistical comparison of $tomcat\_threads$ for POST /lectures/create; then /GET/\{lectureId\}, averaged out over at least 25 runs}
\label{table:run-create-lecture-aggregated-Threadpool_Usage}
\end{table}
\begin{table}[H]
    \small
    \centering
    \resizebox{\columnwidth}{!}{
        \begin{tabular}{rrrcrrcc}
            \toprule
                         & \multicolumn{2}{c}{\textbf{CRUD (ms)}} &                   & \multicolumn{2}{c}{\textbf{ES-CQRS (ms)}} &                 &                                                            \\
            \cmidrule{2-3} \cmidrule{5-6}
            \textbf{RPS} & \textbf{Median}                        & \textbf{CI $\pm$} &                                           & \textbf{Median} & \textbf{CI $\pm$} & \textbf{Ratio} & \textbf{Significance} \\
            \midrule

            10           & 0.0                                    & 0.0               &                                           & 0.0             & 0.0               & $N/A$          & ***                   \\

            20           & 0.0                                    & 0.0               &                                           & 0.0             & 0.0               & $N/A$          & ***                   \\

            50           & 0.0                                    & 0.0               &                                           & 0.0             & 0.0               & $N/A$          & ***                   \\

            100          & 0.0                                    & 0.0               &                                           & 0.0             & 0.0               & $N/A$          & n.s.                  \\

            200          & 1.0                                    & 0.0               &                                           & 1.0             & 0.5               & 1.0x           & ***                   \\

            300          & 1.0                                    & 0.0               &                                           & 3.0             & 0.0               & 3.0x Higher    & ***                   \\

            400          & 1.0                                    & 0.0               &                                           & 4.0             & 0.0               & 4.0x Higher    & ***                   \\

            500          & 2.0                                    & 0.0               &                                           & 5.0             & 0.0               & 2.5x Higher    & ***                   \\

            \bottomrule
        \end{tabular}
    }
    \caption{Statistical comparison of $hikari\_connections$ for POST /lectures/create; then /GET/\{lectureId\}, aggregated over at least 25 runs}
    \label{table:run-create-lecture-aggregated-Database_Connections}
\end{table}
\begin{table}[H]
    \small
    \centering
    \resizebox{\columnwidth}{!}{
        \begin{tabular}{rrrcrrcc}
            \toprule
                         & \multicolumn{2}{c}{\textbf{CRUD (ms)}} &                   & \multicolumn{2}{c}{\textbf{ES-CQRS (ms)}} &                 &                                                            \\
            \cmidrule{2-3} \cmidrule{5-6}
            \textbf{RPS} & \textbf{Median}                        & \textbf{CI $\pm$} &                                           & \textbf{Median} & \textbf{CI $\pm$} & \textbf{Ratio} & \textbf{Significance} \\
            \midrule

            10           & 9.27                                   & 0.01              &                                           & 17.97           & 0.01              & 1.9x Higher    & ***                   \\

            20           & 10.02                                  & 0.02              &                                           & 19.08           & 0.01              & 1.9x Higher    & ***                   \\

            50           & 12.33                                  & 0.02              &                                           & 22.51           & 0.01              & 1.8x Higher    & ***                   \\

            100          & 16.11                                  & 0.01              &                                           & 28.15           & 0.01              & 1.7x Higher    & ***                   \\

            200          & 23.69                                  & 0.01              &                                           & 39.5            & 0.01              & 1.7x Higher    & ***                   \\

            300          & 31.26                                  & 0.01              &                                           & 50.89           & 0.02              & 1.6x Higher    & ***                   \\

            400          & 38.78                                  & 0.01              &                                           & 60.11           & 0.23              & 1.6x Higher    & ***                   \\

            500          & 46.29                                  & 0.02              &                                           & 57.78           & 0.57              & 1.2x Higher    & ***                   \\

            \bottomrule
        \end{tabular}
    }
    \caption{Statistical comparison of Data Store Size (MB) for POST /lectures/create; then /GET/\{lectureId\}, averaged out over at least 25 runs}
    \label{table:run-create-lecture-aggregated-datastore-size}
\end{table}

\begin{table}[H]
    \small \centering
    \resizebox{\columnwidth}{!}{
        \begin{tabular}{lcrrcrrcc}
            \toprule
                                  &              & \multicolumn{2}{c}{\textbf{CRUD (ms)}} &                   & \multicolumn{2}{c}{\textbf{ES-CQRS (ms)}} &                 &                                                              \\
            \cmidrule{3-4} \cmidrule{6-7}
            \textbf{Metric}       & \textbf{RPS} & \textbf{Median}                        & \textbf{CI $\pm$} &                                           & \textbf{Median} & \textbf{CI $\pm$} & \textbf{Speedup} & \textbf{Significance} \\
            \midrule

            $read\_visible\_rate$ & 10           & 1.0                                    & 0.0               &                                           & 1.0             & 0.0               & 1.0x             & n.s.                  \\

            $read\_visible\_rate$ & 20           & 1.0                                    & 0.0               &                                           & 1.0             & 0.0               & 1.0x             & n.s.                  \\

            $read\_visible\_rate$ & 50           & 1.0                                    & 0.0               &                                           & 1.0             & 0.0               & 1.0x             & n.s.                  \\

            $read\_visible\_rate$ & 100          & 1.0                                    & 0.0               &                                           & 1.0             & 0.0               & 1.0x             & n.s.                  \\

            $read\_visible\_rate$ & 200          & 1.0                                    & 0.0               &                                           & 1.0             & 0.0               & 1.0x             & n.s.                  \\

            $read\_visible\_rate$ & 300          & 1.0                                    & 0.0               &                                           & 0.75            & 0.01              & 1.3x Lower       & ***                   \\

            $read\_visible\_rate$ & 400          & 1.0                                    & 0.0               &                                           & 0.02            & 0.0               & 53.3x Lower      & ***                   \\

            $read\_visible\_rate$ & 500          & 1.0                                    & 0.0               &                                           & 0.01            & 0.0               & 126.8x Lower     & ***                   \\

            \bottomrule
        \end{tabular}
    }
    \caption{Statistical comparison of $read\_visible\_rate$ for GET /lectures/\{lectureId\}, averaged out over at least 25 runs}
    \label{table:run-create-lecture_client_read_visible_rate}
\end{table}


\newpage
\section{L8: Reconstruct Grade History}
\label{results:l8}

\begin{table}[H]
    \small \centering
    \resizebox{\columnwidth}{!}{
        \begin{tabular}{lcrrcrrcc}
            \toprule
                                        &              & \multicolumn{2}{c}{\textbf{CRUD (ms)}} &                   & \multicolumn{2}{c}{\textbf{ES-CQRS (ms)}} &                 &                                                            \\
            \cmidrule{3-4} \cmidrule{6-7}
            \textbf{Metric}             & \textbf{RPS} & \textbf{Median}                        & \textbf{CI $\pm$} &                                           & \textbf{Median} & \textbf{CI $\pm$} & \textbf{Ratio} & \textbf{Significance} \\
            \midrule

            $dropped\_iterations\_rate$ & 25           & 0.0                                    & 0.0               &                                           & 0.0             & 0.0               & $N/A$          & n.s.                  \\

            $latency\_avg$              & 25           & 2.05                                   & 0.0               &                                           & 3.47            & 0.0               & 1.7x Slower    & ***                   \\

            $latency\_p50$              & 25           & 1.97                                   & 0.0               &                                           & 3.34            & 0.0               & 1.7x Slower    & ***                   \\

            $latency\_p95$              & 25           & 2.79                                   & 0.0               &                                           & 4.54            & 0.0               & 1.6x Slower    & ***                   \\

            $latency\_p99$              & 25           & 3.83                                   & 0.0               &                                           & 6.03            & 0.0               & 1.6x Slower    & ***                   \\

            $dropped\_iterations\_rate$ & 50           & 0.0                                    & 0.0               &                                           & 0.0             & 0.0               & $N/A$          & n.s.                  \\

            $latency\_avg$              & 50           & 1.67                                   & 0.0               &                                           & 2.87            & 0.0               & 1.7x Slower    & ***                   \\

            $latency\_p50$              & 50           & 1.58                                   & 0.0               &                                           & 2.71            & 0.0               & 1.7x Slower    & ***                   \\

            $latency\_p95$              & 50           & 2.4                                    & 0.0               &                                           & 3.98            & 0.0               & 1.7x Slower    & ***                   \\

            $latency\_p99$              & 50           & 3.36                                   & 0.0               &                                           & 5.57            & 0.0               & 1.7x Slower    & ***                   \\

            $dropped\_iterations\_rate$ & 100          & 0.0                                    & 0.0               &                                           & 0.0             & 0.0               & $N/A$          & n.s.                  \\

            $latency\_avg$              & 100          & 1.36                                   & 0.0               &                                           & 2.43            & 0.0               & 1.8x Slower    & ***                   \\

            $latency\_p50$              & 100          & 1.28                                   & 0.0               &                                           & 2.27            & 0.0               & 1.8x Slower    & ***                   \\

            $latency\_p95$              & 100          & 1.94                                   & 0.0               &                                           & 3.47            & 0.0               & 1.8x Slower    & ***                   \\

            $latency\_p99$              & 100          & 2.95                                   & 0.0               &                                           & 5.12            & 0.0               & 1.7x Slower    & ***                   \\

            $dropped\_iterations\_rate$ & 200          & 0.0                                    & 0.0               &                                           & 0.0             & 0.0               & $N/A$          & n.s.                  \\

            $latency\_avg$              & 200          & 1.1                                    & 0.0               &                                           & 2.02            & 0.0               & 1.8x Slower    & ***                   \\

            $latency\_p50$              & 200          & 1.02                                   & 0.0               &                                           & 1.84            & 0.0               & 1.8x Slower    & ***                   \\

            $latency\_p95$              & 200          & 1.61                                   & 0.0               &                                           & 3.02            & 0.0               & 1.9x Slower    & ***                   \\

            $latency\_p99$              & 200          & 2.47                                   & 0.0               &                                           & 4.76            & 0.0               & 1.9x Slower    & ***                   \\

            $dropped\_iterations\_rate$ & 500          & 0.0                                    & 0.0               &                                           & 0.0             & 0.0               & $N/A$          & n.s.                  \\

            $latency\_avg$              & 500          & 0.95                                   & 0.0               &                                           & 1.88            & 0.0               & 2.0x Slower    & ***                   \\

            $latency\_p50$              & 500          & 0.9                                    & 0.0               &                                           & 1.73            & 0.0               & 1.9x Slower    & ***                   \\

            $latency\_p95$              & 500          & 1.28                                   & 0.0               &                                           & 2.62            & 0.0               & 2.0x Slower    & ***                   \\

            $latency\_p99$              & 500          & 1.86                                   & 0.0               &                                           & 4.92            & 0.0               & 2.6x Slower    & ***                   \\

            $dropped\_iterations\_rate$ & 1000         & 0.0                                    & 0.0               &                                           & 0.0             & 0.0               & $N/A$          & n.s.                  \\

            $latency\_avg$              & 1000         & 0.96                                   & 0.0               &                                           & 2.19            & 0.0               & 2.3x Slower    & ***                   \\

            $latency\_p50$              & 1000         & 0.92                                   & 0.0               &                                           & 1.88            & 0.0               & 2.0x Slower    & ***                   \\

            $latency\_p95$              & 1000         & 1.23                                   & 0.0               &                                           & 3.92            & 0.0               & 3.2x Slower    & ***                   \\

            $latency\_p99$              & 1000         & 1.8                                    & 0.0               &                                           & 7.6             & 0.0               & 4.2x Slower    & ***                   \\

            $dropped\_iterations\_rate$ & 2000         & 0.0                                    & 0.0               &                                           & 19.22           & 10.17             & $N/A$          & ***                   \\

            $latency\_avg$              & 2000         & 1.04                                   & 0.0               &                                           & 1177.93         & 0.25              & 1131.0x Slower & ***                   \\

            $latency\_p50$              & 2000         & 0.98                                   & 0.0               &                                           & 1360.65         & 0.22              & 1395.0x Slower & ***                   \\

            $latency\_p95$              & 2000         & 1.37                                   & 0.0               &                                           & 2022.47         & 0.6               & 1478.6x Slower & ***                   \\

            $latency\_p99$              & 2000         & 2.57                                   & 0.0               &                                           & 2167.05         & 0.61              & 844.6x Slower  & ***                   \\

            \bottomrule
        \end{tabular}
    }
    \caption{Statistical comparison for GET /stats/grades/history (client), aggregated over at least 25 runs}
    \label{table:run-grade-history_client}
\end{table}
\begin{table}[H]
    \small \centering
    \resizebox{\columnwidth}{!}{
        \begin{tabular}{lcrrcrrcc}
            \toprule
                            &              & \multicolumn{2}{c}{\textbf{CRUD (ms)}} &                   & \multicolumn{2}{c}{\textbf{ES-CQRS (ms)}} &                 &                                                              \\
            \cmidrule{3-4} \cmidrule{6-7}
            \textbf{Metric} & \textbf{RPS} & \textbf{Median}                        & \textbf{CI $\pm$} &                                           & \textbf{Median} & \textbf{CI $\pm$} & \textbf{Speedup} & \textbf{Significance} \\
            \midrule

            $latency\_avg$  & 25           & 1.15                                   & 0.0               &                                           & 2.56            & 0.0               & 2.2x Slower      & ***                   \\

            $latency\_p50$  & 25           & 1.08                                   & 0.0               &                                           & 2.43            & 0.0               & 2.2x Slower      & ***                   \\

            $latency\_p95$  & 25           & 1.78                                   & 0.0               &                                           & 3.49            & 0.0               & 2.0x Slower      & ***                   \\

            $latency\_p99$  & 25           & 2.42                                   & 0.0               &                                           & 5.16            & 0.0               & 2.1x Slower      & ***                   \\

            $latency\_avg$  & 50           & 0.94                                   & 0.0               &                                           & 2.08            & 0.0               & 2.2x Slower      & ***                   \\

            $latency\_p50$  & 50           & 0.71                                   & 0.0               &                                           & 1.95            & 0.0               & 2.7x Slower      & ***                   \\

            $latency\_p95$  & 50           & 1.55                                   & 0.0               &                                           & 3.04            & 0.0               & 2.0x Slower      & ***                   \\

            $latency\_p99$  & 50           & 2.07                                   & 0.0               &                                           & 4.42            & 0.0               & 2.1x Slower      & ***                   \\

            $latency\_avg$  & 100          & 0.75                                   & 0.0               &                                           & 1.74            & 0.0               & 2.3x Slower      & ***                   \\

            $latency\_p50$  & 100          & 0.57                                   & 0.0               &                                           & 1.63            & 0.0               & 2.9x Slower      & ***                   \\

            $latency\_p95$  & 100          & 1.27                                   & 0.0               &                                           & 2.64            & 0.0               & 2.1x Slower      & ***                   \\

            $latency\_p99$  & 100          & 1.74                                   & 0.0               &                                           & 3.99            & 0.0               & 2.3x Slower      & ***                   \\

            $latency\_avg$  & 200          & 0.71                                   & 0.0               &                                           & 1.57            & 0.0               & 2.2x Slower      & ***                   \\

            $latency\_p50$  & 200          & 0.54                                   & 0.0               &                                           & 1.44            & 0.0               & 2.7x Slower      & ***                   \\

            $latency\_p95$  & 200          & 1.05                                   & 0.0               &                                           & 2.34            & 0.0               & 2.2x Slower      & ***                   \\

            $latency\_p99$  & 200          & 1.46                                   & 0.0               &                                           & 3.73            & 0.0               & 2.6x Slower      & ***                   \\

            $latency\_avg$  & 500          & 0.63                                   & 0.0               &                                           & 1.53            & 0.0               & 2.4x Slower      & ***                   \\

            $latency\_p50$  & 500          & 0.52                                   & 0.0               &                                           & 1.39            & 0.0               & 2.7x Slower      & ***                   \\

            $latency\_p95$  & 500          & 0.98                                   & 0.0               &                                           & 2.13            & 0.0               & 2.2x Slower      & ***                   \\

            $latency\_p99$  & 500          & 1.32                                   & 0.0               &                                           & 4.13            & 0.0               & 3.1x Slower      & ***                   \\

            $latency\_avg$  & 1000         & 0.67                                   & 0.0               &                                           & 1.86            & 0.0               & 2.8x Slower      & ***                   \\

            $latency\_p50$  & 1000         & 0.52                                   & 0.0               &                                           & 1.61            & 0.0               & 3.1x Slower      & ***                   \\

            $latency\_p95$  & 1000         & 0.98                                   & 0.0               &                                           & 3.43            & 0.0               & 3.5x Slower      & ***                   \\

            $latency\_p99$  & 1000         & 1.31                                   & 0.0               &                                           & 6.76            & 0.0               & 5.2x Slower      & ***                   \\

            $latency\_avg$  & 2000         & 0.75                                   & 0.0               &                                           & 87.01           & 0.0               & 115.5x Slower    & ***                   \\

            $latency\_p50$  & 2000         & 0.54                                   & 0.0               &                                           & 100.67          & 0.0               & 187.5x Slower    & ***                   \\

            $latency\_p95$  & 2000         & 1.05                                   & 0.0               &                                           & 130.56          & 0.0               & 124.6x Slower    & ***                   \\

            $latency\_p99$  & 2000         & 1.71                                   & 0.0               &                                           & 148.77          & 0.0               & 87.0x Slower     & ***                   \\

            \bottomrule
        \end{tabular}
    }
    \caption{Statistical comparison for GET /stats/grades/history (server), aggregated over at least 25 runs}
    \label{table:run-grade-history_server}
\end{table}
\begin{table}[H]
    \small
    \centering
        \begin{tabular}{rrrcrrcc}
            \toprule
                         & \multicolumn{2}{c}{\textbf{CRUD (ms)}} &                   & \multicolumn{2}{c}{\textbf{ES-CQRS (ms)}} &                 &                                                            \\
            \cmidrule{2-3} \cmidrule{5-6}
            \textbf{RPS} & \textbf{Median}                        & \textbf{CI $\pm$} &                                           & \textbf{Median} & \textbf{CI $\pm$} & \textbf{Ratio} & \textbf{Significance} \\
            \midrule

            25           & 0.0087                                 & 0.0006            &                                           & 0.0164          & 0.0006            &
            1.9x Higher  & ***                                                                                                                                                                                   \\

            50           & 0.0125                                 & 0.0006            &                                           & 0.0253          & 0.0013            &
            2.0x Higher  & ***                                                                                                                                                                                   \\

            100          & 0.0177                                 & 0.0012            &                                           & 0.0355          & 0.0018            &
            2.0x Higher  & ***                                                                                                                                                                                   \\

            200          & 0.0239                                 & 0.0007            &                                           & 0.0584          & 0.0023            &
            2.4x Higher  & ***                                                                                                                                                                                   \\

            500          & 0.0546                                 & 0.0009            &                                           & 0.1414          & 0.0018            &
            2.6x Higher  & ***                                                                                                                                                                                   \\

            1000         & 0.1118                                 & 0.0009            &                                           & 0.2892          & 0.0022            &
            2.6x Higher  & ***                                                                                                                                                                                   \\

            2000         & 0.2345                                 & 0.0017            &                                           & 0.5086          & 0.0028            &
            2.2x Higher  & ***                                                                                                                                                                                   \\

            \bottomrule
        \end{tabular}
    \caption{Statistical comparison of $cpu\_usage$ for GET /stats/grades/history, aggregated over at least 25 runs}
    \label{table:run-grade-history-aggregated-CPU_Usage}
\end{table}
\begin{table}[H]
    \small
    \centering
    \begin{tabular}{rrrcrrcc}
        \toprule
                     & \multicolumn{2}{c}{\textbf{CRUD (ms)}} &                   & \multicolumn{2}{c}{\textbf{ES-CQRS (ms)}} &                 &                                                            \\
        \cmidrule{2-3} \cmidrule{5-6}
        \textbf{RPS} & \textbf{Median}                        & \textbf{CI $\pm$} &                                           & \textbf{Median} & \textbf{CI $\pm$} & \textbf{Ratio} & \textbf{Significance} \\
        \midrule

        25           & 10.0                                   & 0.0               &                                           & 10.0            & 0.0               & 1.0x           & n.s.                  \\

        50           & 10.0                                   & 0.0               &                                           & 10.0            & 0.0               & 1.0x           & n.s.                  \\

        100          & 10.0                                   & 0.0               &                                           & 10.0            & 0.0               & 1.0x           & n.s.                  \\

        200          & 10.0                                   & 0.0               &                                           & 10.0            & 0.0               & 1.0x           & n.s.                  \\

        500          & 10.0                                   & 0.0               &                                           & 10.0            & 0.0               & 1.0x           & ***                   \\

        1000         & 15.0                                   & 0.0               &                                           & 19.0            & 0.5               & 1.3x Higher    & ***                   \\

        2000         & 25.0                                   & 0.0               &                                           & 200.0           & 0.0               & 8.0x Higher    & ***                   \\

        \bottomrule
    \end{tabular}
    \caption[Comparison of $tomcat\_threads$ for GET /stats/grades/history]{Statistical comparison of $tomcat\_threads$ for GET /stats/grades/history, aggregated over at least 25 runs}
    \label{table:run-grade-history-aggregated-Threadpool_Usage}
\end{table}
\begin{table}[H]
    \small
    \centering
    \resizebox{\columnwidth}{!}{
        \begin{tabular}{rrrcrrcc}
            \toprule
                         & \multicolumn{2}{c}{\textbf{CRUD (ms)}} &                   & \multicolumn{2}{c}{\textbf{ES-CQRS (ms)}} &                 &                                                            \\
            \cmidrule{2-3} \cmidrule{5-6}
            \textbf{RPS} & \textbf{Median}                        & \textbf{CI $\pm$} &                                           & \textbf{Median} & \textbf{CI $\pm$} & \textbf{Ratio} & \textbf{Significance} \\
            \midrule

            25           & 0.0                                    & 0.0               &                                           & 0.0             & 0.0               & NaN            & ***                   \\

            50           & 0.0                                    & 0.0               &                                           & 0.0             & 0.0               & NaN            & **                    \\

            100          & 0.0                                    & 0.0               &                                           & 0.0             & 0.0               & NaN            & ***                   \\

            200          & 0.0                                    & 0.0               &                                           & 0.0             & 0.0               & NaN            & ***                   \\

            500          & 0.0                                    & 0.0               &                                           & 0.0             & 0.5               & NaN            & ***                   \\

            1000         & 1.0                                    & 0.5               &                                           & 1.0             & 0.0               & 1.0x           & ***                   \\

            2000         & 1.0                                    & 0.5               &                                           & 10.0            & 0.5               & 10.0x Higher   & ***                   \\

            \bottomrule
        \end{tabular}
    }
    \caption{Statistical comparison of $hikari\_connections$ for GET /stats/grades/history, aggregated over at least 25 runs}
    \label{table:run-grade-history-aggregated-Database_Connections}
\end{table}

\chapter{Static Analysis Results}
\label{appendix:static-analysis-results}

This chapter lists tables containing the full results of static analysis metrics. To make the tables more readable, some package names are abbreviated according to \autoref{tab:package-name-abbreviations}.

\begin{table}[H]
    \centering
    \small
    \begin{tabular}{ll}
        \toprule
        \textbf{Package Name} & \textbf{Abbreviation} \\
        \midrule
        karsch.lukas          & k.l                   \\
        features              & f                     \\
        lectures              & le                    \\
        enrollment            & e                     \\
        command               & c                     \\
        \bottomrule
    \end{tabular}
    \caption[Package name abbreviations]{Package name abbreviations used when presenting results.}
    \label{tab:package-name-abbreviations}
\end{table}

\section{Coupling Metrics}
\label{appendix:coupling-results}

\begin{table}[H]
    \footnotesize
    \centering
    \begin{tabular}{lrrrrrr}
        \toprule
        \textbf{Package}          & \textbf{$C_a$} & \textbf{$C_e$} \\
        \midrule

        karsch.lukas              & 0              & 0              \\

        karsch.lukas.audit        & 18             & 23             \\

        karsch.lukas.auth         & 16             & 3              \\

        karsch.lukas.config       & 6              & 0              \\

        karsch.lukas.courses      & 58             & 36             \\

        karsch.lukas.dev          & 0              & 92             \\

        karsch.lukas.exceptions   & 0              & 16             \\

        karsch.lukas.featureflags & 7              & 0              \\

        karsch.lukas.lectures     & 251            & 271            \\

        karsch.lukas.stats        & 3              & 144            \\

        karsch.lukas.time         & 29             & 4              \\

        karsch.lukas.users        & 130            & 19             \\

        karsch.lukas.uuid         & 0              & 0              \\

        \bottomrule
    \end{tabular}
    \caption{Coupling Metrics for CRUD (Packages)}
    \label{table:crud-coupling}
\end{table}
\begin{table}[H]
    \footnotesize
    \centering
    \begin{tabular}{lrrrrrr}
        \toprule
        \textbf{Package}                         & \textbf{$C_a$} & \textbf{$C_e$} \\
        \midrule

        k.l                                      & 0              & 0              \\

        k.l.config                               & 0              & 0              \\

        k.l.config.aspect                        & 0              & 0              \\

        k.l.config.commandInterceptors           & 0              & 2              \\

        k.l.core.auth                            & 7              & 3              \\

        k.l.core.exceptions                      & 74             & 0              \\

        k.l.core.json                            & 6              & 0              \\

        k.l.core.lookup                          & 23             & 0              \\

        k.l.core.queries                         & 10             & 5              \\

        k.l.core.uuid                            & 5              & 2              \\

        k.l.dev                                  & 0              & 6              \\

        k.l.f.course.api                         & 65             & 0              \\

        k.l.f.course.commands                    & 13             & 17             \\

        k.l.f.course.exceptions                  & 4              & 3              \\

        k.l.f.course.queries                     & 0              & 21             \\

        k.l.f.course.web                         & 0              & 33             \\

        k.l.f.enrollment.api                     & 113            & 0              \\

        k.l.f.enrollment.command                 & 7              & 104            \\

        k.l.f.enrollment.command.lookup          & 5              & 6              \\

        k.l.f.enrollment.command.lookup.credits  & 15             & 13             \\

        k.l.f.enrollment.exception               & 5              & 6              \\

        k.l.f.lectures.api                       & 390            & 12             \\

        k.l.f.lectures.command                   & 6              & 177            \\

        k.l.f.lectures.command.lookup.assessment & 10             & 16             \\

        k.l.f.lectures.command.lookup.lecture    & 9              & 14             \\

        k.l.f.lectures.command.lookup.timeSlot   & 6              & 43             \\

        k.l.f.lectures.exceptions                & 4              & 2              \\

        k.l.f.lectures.queries                   & 0              & 277            \\

        k.l.f.lectures.web                       & 0              & 148            \\

        k.l.f.professor.api                      & 18             & 0              \\

        k.l.f.professor.command                  & 8              & 9              \\

        k.l.f.stats.api                          & 23             & 0              \\

        k.l.f.stats.queries.credits              & 0              & 16             \\

        k.l.f.stats.queries.gradeHistory         & 0              & 43             \\

        k.l.f.stats.queries.grades               & 0              & 48             \\

        k.l.f.stats.web                          & 0              & 32             \\

        k.l.f.student.api                        & 21             & 0              \\

        k.l.f.student.command                    & 2              & 9              \\

        k.l.f.student.command.lookup             & 25             & 5              \\

        k.l.f.users.web                          & 0              & 19             \\

        k.l.infra.web                            & 0              & 38             \\

        \bottomrule
    \end{tabular}
    \caption{Coupling Metrics for CRUD architecture (Packages).}
    \label{table:es-cqrs-coupling}
\end{table}

\newpage
\section{Instability and Abstractness Metrics}
\label{appendix:instability-abstractness}

\begin{table}[H]
    \small
    \centering
    \begin{tabular}{lrrrrrr}
        \toprule
        \textbf{Application} & \textbf{Min} & \textbf{P25} & \textbf{Median} & \textbf{P75} & \textbf{Max} & \textbf{Outliers} \\
        \midrule
        CRUD                 & 0.0          & 0.1          & 0.5             & 1.0          & 1.0          & 0                 \\
        ES-CQRS              & 0.0          & 0.3          & 0.6             & 1.0          & 1.0          & 0                 \\
        \bottomrule
    \end{tabular}\caption{Descriptive Statistics for Instability per package $I$}
    \label{table:instability}
\end{table}
\begin{table}
    \small
    \begin{tabular}{lrrrrrr}
        \toprule
        \textbf{Application} & \textbf{Min} & \textbf{P25} & \textbf{Median} & \textbf{P75} & \textbf{Max} & \textbf{Outliers} \\
        \midrule
        CRUD                 & 0.0          & 0.0          & 0.0             & 0.2          & 0.5          & 0                 \\
        ES-CQRS              & 0.0          & 0.0          & 0.0             & 0.4          & 0.5          & 0                 \\
        \bottomrule
    \end{tabular}
    \caption{Descriptive Statistics for Abstractness per package $A$}
    \label{table:abstractness}
\end{table}
\begin{table}
    \small
    \begin{tabular}{lrrrrrr}
        \toprule
        \textbf{Application} & \textbf{Min} & \textbf{P25} & \textbf{Median} & \textbf{P75} & \textbf{Max} & \textbf{Outliers} \\
        \midrule
        CRUD                 & 0.0          & 0.2          & 0.5             & 0.9          & 1.0          & 0                 \\
        ES-CQRS              & 0.0          & 0.1          & 0.4             & 1.0          & 1.0          & 0                 \\
        \bottomrule
    \end{tabular}
    \caption{Descriptive Statistics for Distance from the main sequence per package $D$}
    \label{table:distance-from-the-main-sequence}
\end{table}

\begin{table}[H]
    \footnotesize
    \begin{tabular}{lrrrrrr}
        \toprule
        \textbf{Package}  & \textbf{Abstractness $A$} & \textbf{Instability $I$} & \textbf{Distance from the Main Sequence $D$} \\
        \midrule
        
            karsch.lukas & 0 & 1 & 1 \\
        
            karsch.lukas.audit & 0 & 0 & 0 \\
        
            karsch.lukas.auth & 0 & 0 & 0 \\
        
            karsch.lukas.config & 0 & 1 & 0 \\
        
            karsch.lukas.courses & 0 & 0 & 0 \\
        
            karsch.lukas.dev & 0 & 0 & 1 \\
        
            karsch.lukas.exceptions & 0 & 0 & 1 \\
        
            karsch.lukas.featureflags & 0 & 1 & 0 \\
        
            karsch.lukas.lectures & 0 & 0 & 0 \\
        
            karsch.lukas.stats & 0 & 0 & 0 \\
        
            karsch.lukas.time & 0 & 0 & 0 \\
        
            karsch.lukas.users & 0 & 0 & 0 \\
        
            karsch.lukas.uuid & 0 & 0 & 1 \\
        
        \bottomrule
    \end{tabular}
    \caption{CK Metrics for CRUD architecture (Packages).}
    \label{table:crud-distance}
\end{table}
\begin{table}[H]
     \footnotesize
     \centering
     \begin{tabular}{lrrrrrr}
          \toprule
          \textbf{Package}                         & \textbf{$A$} & \textbf{$I$} & \textbf{$D$} \\
          \midrule

          k.l                                      & 0.0          & 1.0          & 1.0          \\

          k.l.config                               & 0.0          & 1.0          & 1.0          \\

          k.l.config.aspect                        & 0.0          & 1.0          & 1.0          \\

          k.l.config.commandInterceptors           & 0.0          & 0.0          & 1.0          \\

          k.l.core.auth                            & 0.0          & 0.7          & 0.3          \\

          k.l.core.exceptions                      & 0.2          & 0.8          & 0.0          \\

          k.l.core.json                            & 0.0          & 1.0          & 0.0          \\

          k.l.core.lookup                          & 0.0          & 1.0          & 0.0          \\

          k.l.core.queries                         & 0.5          & 0.17         & 0.33         \\

          k.l.core.uuid                            & 0.5          & 0.21         & 0.29         \\

          k.l.dev                                  & 0.0          & 0.0          & 1.0          \\

          k.l.f.course.api                         & 0.0          & 1.0          & 0.0          \\

          k.l.f.course.commands                    & 0.33         & 0.1          & 0.57         \\

          k.l.f.course.exceptions                  & 0.0          & 0.57         & 0.43         \\

          k.l.f.course.queries                     & 0.33         & 0.33         & 1.0          \\

          k.l.f.course.web                         & 0.0          & 0.0          & 1.0          \\

          k.l.f.enrollment.api                     & 0.0          & 1.0          & 0.0          \\

          k.l.f.enrollment.command                 & 0.0          & 0.06         & 0.94         \\

          k.l.f.enrollment.command.lookup          & 0.4          & 0.05         & 0.55         \\

          k.l.f.enrollment.command.lookup.credits  & 0.4          & 0.14         & 0.46         \\

          k.l.f.enrollment.exception               & 0.0          & 0.45         & 0.55         \\

          k.l.f.lectures.api                       & 0.0          & 0.97         & 0.03         \\

          k.l.f.lectures.command                   & 0.0          & 0.03         & 0.97         \\

          k.l.f.lectures.command.lookup.assessment & 0.4          & 0.02         & 0.62         \\

          k.l.f.lectures.command.lookup.lecture    & 0.4          & 0.01         & 0.61         \\

          k.l.f.lectures.command.lookup.timeSlot   & 0.4          & 0.28         & 0.88         \\

          k.l.f.lectures.exceptions                & 0.0          & 0.67         & 0.33         \\

          k.l.f.lectures.queries                   & 0.42         & 0.42         & 1.0          \\

          k.l.f.lectures.web                       & 0.0          & 0.0          & 1.0          \\

          k.l.f.professor.api                      & 0.0          & 1.0          & 0.0          \\

          k.l.f.professor.command                  & 0.33         & 0.14         & 0.53         \\

          k.l.f.stats.api                          & 0.0          & 1.0          & 0.0          \\

          k.l.f.stats.queries.credits              & 0.4          & 0.4          & 1.0          \\

          k.l.f.stats.queries.gradeHistory         & 0.4          & 0.4          & 1.0          \\

          k.l.f.stats.queries.grades               & 0.38         & 0.38         & 1.0          \\

          k.l.f.stats.web                          & 0.0          & 0.0          & 1.0          \\

          k.l.f.student.api                        & 0.0          & 1.0          & 0.0          \\

          k.l.f.student.command                    & 0.0          & 0.18         & 0.82         \\

          k.l.f.student.command.lookup             & 0.4          & 0.43         & 0.17         \\

          k.l.f.users.web                          & 0.0          & 0.0          & 1.0          \\

          k.l.infra.web                            & 0.0          & 0.0          & 1.0          \\

          \bottomrule
     \end{tabular}
     \caption[$A$, $I$ and $D$ metrics for ES-CQRS]{Abstractness $A$, Instability $I$ and Distance from the Main Sequence $D$ metrics for ES-CQRS (Packages)}
     \label{table:es-cqrs-distance}
\end{table}


\newpage
\section{Dependency Metrics}
\label{appendix:dependency}

\footnotesize
\begin{longtable}{lrrrrrr}
    \toprule
    \textbf{Class}                              & \textbf{Dpt} & \textbf{Dpt*} & \textbf{Dcy} & \textbf{Dcy*} & \textbf{PDpt} & \textbf{PDcy} \\
    \midrule

    k.l.CrudApplication                         & 1            & 1             & 0            & 0             & 1             & 0             \\

    k.l.audit.AuditContext                      & 2            & 43            & 0            & 0             & 1             & 0             \\

    k.l.audit.AuditEntityListener               & 1            & 41            & 9            & 12            & 1             & 4             \\

    k.l.audit.AuditHelper                       & 3            & 44            & 0            & 0             & 1             & 0             \\

    k.l.audit.AuditLogEntry                     & 5            & 45            & 0            & 0             & 2             & 0             \\

    k.l.audit.AuditService                      & 2            & 6             & 4            & 4             & 2             & 1             \\

    k.l.audit.AuditableEntity                   & 9            & 41            & 1            & 12            & 4             & 1             \\

    k.l.audit.CollectionWithIdSerializer        & 1            & 44            & 1            & 1             & 1             & 1             \\

    k.l.audit.DynamicIdSerializer               & 2            & 45            & 0            & 0             & 1             & 0             \\

    k.l.audit.IdPropertySerializerModifier      & 1            & 43            & 2            & 2             & 1             & 1             \\

    k.l.audit.IdSerializationModule             & 1            & 42            & 1            & 3             & 1             & 1             \\

    k.l.audit.JpaAuditingConfiguration          & 0            & 0             & 0            & 0             & 0             & 0             \\

    k.l.auth.CustomAuthFilter                   & 0            & 0             & 2            & 3             & 0             & 1             \\

    k.l.auth.NotAuthenticatedException          & 3            & 3             & 0            & 0             & 2             & 0             \\

    k.l.config.SpringContext                    & 1            & 42            & 0            & 0             & 1             & 0             \\

    k.l.courses.CourseDtoMapper                 & 2            & 5             & 4            & 18            & 2             & 3             \\

    k.l.courses.CourseEntity                    & 13           & 39            & 1            & 13            & 7             & 1             \\

    k.l.courses.CoursesController               & 0            & 0             & 8            & 28            & 0             & 5             \\

    k.l.courses.CoursesNotFoundException        & 2            & 4             & 0            & 0             & 2             & 0             \\

    k.l.courses.CoursesService                  & 1            & 1             & 8            & 23            & 1             & 4             \\

    k.l.courses.SimpleCourseDtoMapper           & 2            & 7             & 3            & 16            & 2             & 3             \\

    k.l.dev.SeedDataRunner                      & 0            & 0             & 16           & 30            & 0             & 6             \\

    k.l.exceptions.GlobalExceptionHandler       & 0            & 0             & 1            & 1             & 0             & 1             \\

    k.l.featureflags.Feature                    & 3            & 3             & 0            & 0             & 2             & 0             \\

    k.l.featureflags.FeatureFlagService         & 2            & 2             & 1            & 1             & 2             & 1             \\

    k.l.featureflags.FeaturesEndpoint           & 0            & 0             & 2            & 2             & 0             & 1             \\

    k.l.le.AlreadyEnrolledException             & 1            & 2             & 0            & 0             & 1             & 0             \\

    k.l.le.AssessmentGradeEntity                & 6            & 8             & 3            & 24            & 4             & 3             \\

    k.l.le.EnrollmentEntity                     & 6            & 33            & 3            & 23            & 4             & 3             \\

    k.l.le.LectureAssessmentEntity              & 12           & 33            & 4            & 23            & 5             & 4             \\

    k.l.le.LectureAssessmentMapper              & 1            & 3             & 4            & 28            & 1             & 4             \\

    k.l.le.LectureDetailDtoMapper               & 1            & 2             & 9            & 40            & 1             & 6             \\

    k.l.le.LectureDtoMapper                     & 1            & 2             & 6            & 32            & 1             & 6             \\

    k.l.le.LectureEntity                        & 18           & 33            & 10           & 23            & 7             & 6             \\

    k.l.le.LectureNotFoundException             & 1            & 2             & 0            & 0             & 1             & 0             \\

    k.l.le.LectureNotOpenForEnrollmentException & 1            & 2             & 1            & 1             & 1             & 1             \\

    k.l.le.LectureWaitlistEntryComparator       & 2            & 3             & 2            & 24            & 1             & 2             \\

    k.l.le.LectureWaitlistEntryEntity           & 6            & 33            & 3            & 23            & 1             & 3             \\

    k.l.le.LecturesController                   & 0            & 0             & 17           & 84            & 0             & 5             \\

    k.l.le.LecturesService                      & 1            & 1             & 41           & 79            & 1             & 8             \\

    k.l.le.SimpleLectureDtoMapper               & 2            & 4             & 4            & 26            & 2             & 4             \\

    k.l.le.WaitlistedStudentMapper              & 2            & 3             & 4            & 28            & 1             & 4             \\

    k.l.stats.GradedAssessmentDtoMapper         & 1            & 4             & 4            & 27            & 1             & 3             \\

    k.l.stats.StatsController                   & 0            & 0             & 8            & 45            & 0             & 3             \\

    k.l.stats.StatsService                      & 2            & 3             & 20           & 40            & 2             & 7             \\

    k.l.time.TimeSlotMapper                     & 4            & 5             & 3            & 3             & 1             & 3             \\

    k.l.time.TimeSlotValueObject                & 9            & 37            & 0            & 0             & 5             & 0             \\

    k.l.time.TimeSlotValueObjectComparator      & 2            & 35            & 1            & 1             & 2             & 1             \\

    k.l.users.ProfessorDtoMapper                & 2            & 4             & 3            & 26            & 1             & 3             \\

    k.l.users.ProfessorEntity                   & 9            & 33            & 2            & 23            & 6             & 2             \\

    k.l.users.StudentDtoMapper                  & 2            & 4             & 3            & 26            & 1             & 3             \\

    k.l.users.StudentEntity                     & 16           & 33            & 2            & 23            & 7             & 2             \\

    k.l.users.StudentNotFoundException          & 1            & 4             & 0            & 0             & 1             & 0             \\

    k.l.users.UsersController                   & 0            & 0             & 5            & 31            & 0             & 2             \\

    k.l.users.UsersService                      & 1            & 1             & 6            & 28            & 1             & 1             \\

    k.l.uuid.UuidV7Generator                    & 1            & 1             & 1            & 1             & 1             & 1             \\

    \bottomrule
    \caption{Dependency Metrics for CRUD (Classes)}
    \label{table:crud-dependency}
\end{longtable}
\footnotesize
\begin{longtable}{lrrrrrr}
    \toprule
    \textbf{Class}                                                & \textbf{Dpt} & \textbf{Dpt*} & \textbf{Dcy} & \textbf{Dcy*} & \textbf{PDpt} & \textbf{PDcy} \\
    \midrule

    k.l.EsCqrsApplication                                         & 1            & 1             & 0            & 0             & 1             & 0             \\

    k.l.config.JacksonConfiguration                               & 0            & 0             & 0            & 0             & 0             & 0             \\

    k.l.config.RetryConfig                                        & 0            & 0             & 0            & 0             & 0             & 0             \\

    k.l.config.aspect.LoggingAspect                               & 0            & 0             & 0            & 0             & 0             & 0             \\

    k.l.config.aspect.RetryLoggingAspect                          & 0            & 0             & 0            & 0             & 0             & 0             \\

    k.l.config.commandInterceptors.AggregateNotFoundInterceptor   & 0            & 0             & 1            & 1             & 0             & 1             \\

    k.l.core.auth.CustomAuthFilter                                & 0            & 0             & 2            & 3             & 0             & 1             \\

    k.l.core.auth.NotAuthenticatedException                       & 2            & 2             & 0            & 0             & 2             & 0             \\

    k.l.core.exceptions.DomainException                           & 8            & 16            & 1            & 1             & 3             & 1             \\

    k.l.core.exceptions.ErrorDetails                              & 7            & 31            & 0            & 0             & 5             & 0             \\

    k.l.core.exceptions.MissingResourceException                  & 4            & 20            & 1            & 1             & 3             & 1             \\

    k.l.core.exceptions.NotAllowedException                       & 6            & 13            & 1            & 1             & 3             & 1             \\

    k.l.core.exceptions.QueryException                            & 3            & 3             & 0            & 0             & 3             & 0             \\

    k.l.core.json.Defaults                                        & 2            & 4             & 0            & 0             & 1             & 0             \\

    k.l.core.lookup.TimeSlotEmbeddable                            & 7            & 13            & 0            & 0             & 4             & 0             \\

    k.l.core.lookup.TimeSlotEmbeddableComparator                  & 2            & 4             & 1            & 1             & 1             & 1             \\

    k.l.core.queries.CourseMapper                                 & 2            & 3             & 3            & 3             & 2             & 2             \\

    k.l.core.uuid.UuidProviderImpl                                & 0            & 0             & 2            & 2             & 0             & 2             \\

    k.l.dev.SeedDataRunner                                        & 0            & 0             & 5            & 6             & 0             & 5             \\

    k.l.f.course.api.CourseCreatedEvent                           & 7            & 7             & 0            & 0             & 4             & 0             \\

    k.l.f.course.api.CreateCourseCommand                          & 6            & 7             & 0            & 0             & 5             & 0             \\

    k.l.f.course.api.FindAllCoursesQuery                          & 3            & 3             & 0            & 0             & 2             & 0             \\

    k.l.f.course.api.FindCourseByIdQuery                          & 3            & 3             & 0            & 0             & 2             & 0             \\

    k.l.f.course.commands.CourseAggregate                         & 2            & 2             & 4            & 6             & 1             & 3             \\

    k.l.f.course.commands.CourseLookupProjector                   & 0            & 0             & 4            & 9             & 0             & 2             \\

    k.l.f.course.commands.CourseValidator                         & 0            & 0             & 3            & 3             & 0             & 1             \\

    k.l.f.course.commands.CoursesLookupJpaEntity                  & 3            & 3             & 0            & 0             & 1             & 0             \\

    k.l.f.course.exceptions.MissingCoursesException               & 4            & 14            & 1            & 2             & 2             & 1             \\

    k.l.f.course.queries.CourseProjectionEntity                   & 3            & 3             & 1            & 1             & 1             & 1             \\

    k.l.f.course.queries.CourseProjector                          & 1            & 1             & 8            & 9             & 1             & 4             \\

    k.l.f.course.web.CoursesController                            & 0            & 0             & 10           & 12            & 0             & 7             \\

    k.l.f.e.api.AssignGradeCommand                                & 5            & 7             & 0            & 0             & 4             & 0             \\

    k.l.f.e.api.AwardCreditsCommand                               & 3            & 6             & 0            & 0             & 2             & 0             \\

    k.l.f.e.api.CreditsAwardedEvent                               & 5            & 7             & 0            & 0             & 4             & 0             \\

    k.l.f.e.api.EnrollmentCreatedEvent                            & 7            & 9             & 0            & 0             & 5             & 0             \\

    k.l.f.e.api.GradeAssignedEvent                                & 3            & 6             & 0            & 0             & 3             & 0             \\

    k.l.f.e.api.GradeUpdatedEvent                                 & 3            & 6             & 0            & 0             & 3             & 0             \\

    k.l.f.e.api.UpdateGradeCommand                                & 3            & 5             & 0            & 0             & 2             & 0             \\

    k.l.f.e.c.AwardCreditsSaga                                    & 1            & 1             & 7            & 9             & 1             & 4             \\

    k.l.f.e.c.EnrollmentAggregate                                 & 3            & 3             & 10           & 16            & 3             & 4             \\

    k.l.f.e.c.EnrollmentCommandHandler                            & 0            & 0             & 18           & 31            & 0             & 12            \\

    k.l.f.e.c.lookup.EnrollmentLookupEntity                       & 5            & 6             & 0            & 0             & 2             & 0             \\

    k.l.f.e.c.lookup.EnrollmentLookupProjector                    & 0            & 0             & 4            & 19            & 0             & 3             \\

    k.l.f.e.c.lookup.EnrollmentValidator                          & 0            & 0             & 3            & 3             & 0             & 1             \\

    k.l.f.e.c.lookup.credits.StudentCreditsLookupProjectionEntity & 3            & 3             & 0            & 0             & 1             & 0             \\

    k.l.f.e.c.lookup.credits.StudentCreditsLookupProjector        & 0            & 0             & 5            & 20            & 0             & 4             \\

    k.l.f.e.c.lookup.credits.StudentCreditsValidator              & 0            & 0             & 4            & 4             & 0             & 2             \\

    k.l.f.e.exception.AssessmentNotFoundException                 & 1            & 1             & 1            & 2             & 1             & 1             \\

    k.l.f.e.exception.MissingGradeException                       & 1            & 4             & 1            & 2             & 1             & 1             \\

    k.l.f.e.exception.StudentNotEnrolledException                 & 1            & 1             & 1            & 2             & 1             & 1             \\

    k.l.f.le.api.AddAssessmentCommand                             & 5            & 14            & 2            & 2             & 4             & 2             \\

    k.l.f.le.api.AdvanceLectureLifecycleCommand                   & 6            & 14            & 1            & 1             & 4             & 1             \\

    k.l.f.le.api.AssessmentAddedEvent                             & 7            & 14            & 2            & 2             & 6             & 2             \\

    k.l.f.le.api.AssignTimeSlotsToLectureCommand                  & 3            & 12            & 1            & 1             & 2             & 1             \\

    k.l.f.le.api.ConfirmStudentEnrollmentCommand                  & 4            & 13            & 0            & 0             & 1             & 0             \\

    k.l.f.le.api.CreateEnrollmentCommand                          & 3            & 6             & 0            & 0             & 2             & 0             \\

    k.l.f.le.api.CreateLectureCommand                             & 6            & 15            & 1            & 1             & 5             & 1             \\

    k.l.f.le.api.DisenrollStudentCommand                          & 3            & 12            & 0            & 0             & 2             & 0             \\

    k.l.f.le.api.EnrollStudentCommand                             & 5            & 14            & 0            & 0             & 4             & 0             \\

    k.l.f.le.api.EnrollmentStatusQuery                            & 2            & 2             & 0            & 0             & 2             & 0             \\

    k.l.f.le.api.EnrollmentStatusUpdate                           & 2            & 2             & 1            & 1             & 2             & 1             \\

    k.l.f.le.api.FindLectureByIdQuery                             & 3            & 3             & 0            & 0             & 2             & 0             \\

    k.l.f.le.api.GetLectureWaitlistQuery                          & 2            & 2             & 0            & 0             & 2             & 0             \\

    k.l.f.le.api.GetLecturesForStudentQuery                       & 2            & 2             & 0            & 0             & 2             & 0             \\

    k.l.f.le.api.LectureCreatedEvent                              & 10           & 13            & 2            & 2             & 5             & 1             \\

    k.l.f.le.api.LectureLifecycleAdvancedEvent                    & 8            & 14            & 1            & 1             & 5             & 1             \\

    k.l.f.le.api.StudentDisenrolledEvent                          & 4            & 13            & 0            & 0             & 2             & 0             \\

    k.l.f.le.api.StudentEnrolledEvent                             & 5            & 13            & 0            & 0             & 2             & 0             \\

    k.l.f.le.api.StudentEnrollmentApprovedEvent                   & 5            & 13            & 0            & 0             & 1             & 0             \\

    k.l.f.le.api.StudentRemovedFromWaitlistEvent                  & 4            & 13            & 0            & 0             & 2             & 0             \\

    k.l.f.le.api.StudentWaitlistedEvent                           & 5            & 13            & 0            & 0             & 2             & 0             \\

    k.l.f.le.api.TimeSlotsAssignedEvent                           & 4            & 12            & 1            & 1             & 3             & 1             \\

    k.l.f.le.api.WaitlistClearedEvent                             & 4            & 13            & 0            & 0             & 2             & 0             \\

    k.l.f.le.c.AssessmentValueObject                              & 1            & 11            & 2            & 2             & 1             & 2             \\

    k.l.f.le.c.EnrollmentApprovalSaga                             & 1            & 1             & 5            & 5             & 1             & 3             \\

    k.l.f.le.c.LectureAggregate                                   & 4            & 10            & 32           & 36            & 4             & 11            \\

    k.l.f.le.c.lookup.assessment.AssessmentLookupEntity           & 6            & 7             & 2            & 2             & 2             & 2             \\

    k.l.f.le.c.lookup.assessment.AssessmentLookupProjector        & 0            & 0             & 6            & 40            & 0             & 5             \\

    k.l.f.le.c.lookup.assessment.AssessmentValidator              & 0            & 0             & 3            & 5             & 0             & 1             \\

    k.l.f.le.c.lookup.lecture.LectureLookupEntity                 & 5            & 5             & 1            & 1             & 2             & 1             \\

    k.l.f.le.c.lookup.lecture.LectureLookupProjector              & 0            & 0             & 6            & 39            & 0             & 3             \\

    k.l.f.le.c.lookup.lecture.LectureValidator                    & 0            & 0             & 3            & 4             & 0             & 1             \\

    k.l.f.le.c.lookup.timeSlot.LectureTimeSlotProjector           & 0            & 0             & 8            & 41            & 0             & 5             \\

    k.l.f.le.c.lookup.timeSlot.LectureTimeslotLookupEntity        & 3            & 3             & 2            & 2             & 1             & 1             \\

    k.l.f.le.c.lookup.timeSlot.TimeSlotValidator                  & 0            & 0             & 9            & 12            & 0             & 5             \\

    k.l.f.le.exceptions.LectureNotFoundException                  & 2            & 2             & 1            & 2             & 2             & 1             \\

    k.l.f.le.queries.CourseProjectionEntity                       & 2            & 2             & 1            & 1             & 1             & 1             \\

    k.l.f.le.queries.LectureDetailProjectionEntity                & 2            & 2             & 1            & 1             & 1             & 1             \\

    k.l.f.le.queries.LectureProjector                             & 0            & 0             & 39           & 44            & 0             & 13            \\

    k.l.f.le.queries.ProfessorProjectionEntity                    & 2            & 2             & 0            & 0             & 1             & 0             \\

    k.l.f.le.queries.StudentLecturesProjectionEntity              & 2            & 2             & 1            & 1             & 1             & 1             \\

    k.l.f.le.queries.StudentLecturesProjector                     & 0            & 0             & 14           & 22            & 0             & 4             \\

    k.l.f.le.queries.StudentProjectionEntity                      & 2            & 2             & 0            & 0             & 1             & 0             \\

    k.l.f.le.web.LecturesController                               & 0            & 0             & 31           & 42            & 0             & 8             \\

    k.l.f.professor.api.CreateProfessorCommand                    & 5            & 6             & 0            & 0             & 5             & 0             \\

    k.l.f.professor.api.ProfessorCreatedEvent                     & 3            & 3             & 0            & 0             & 2             & 0             \\

    k.l.f.professor.c.ProfessorAggregate                          & 1            & 1             & 2            & 2             & 1             & 1             \\

    k.l.f.professor.c.ProfessorLookupEntity                       & 2            & 3             & 0            & 0             & 1             & 0             \\

    k.l.f.professor.c.ProfessorLookupTable                        & 0            & 0             & 4            & 5             & 0             & 2             \\

    k.l.f.professor.c.ProfessorValidator                          & 0            & 0             & 2            & 3             & 0             & 1             \\

    k.l.f.stats.api.GetCreditsForStudentQuery                     & 2            & 2             & 0            & 0             & 2             & 0             \\

    k.l.f.stats.api.GetGradeHistoryQuery                          & 2            & 2             & 0            & 0             & 2             & 0             \\

    k.l.f.stats.api.GetGradesForStudentQuery                      & 2            & 2             & 0            & 0             & 2             & 0             \\

    k.l.f.stats.queries.credits.StudentCreditsProjectionEntity    & 2            & 2             & 0            & 0             & 1             & 0             \\

    k.l.f.stats.queries.credits.StudentCreditsProjector           & 0            & 0             & 7            & 8             & 0             & 6             \\

    k.l.f.stats.queries.gradeHistory.AssessmentProjectionEntity   & 2            & 2             & 0            & 0             & 1             & 0             \\

    k.l.f.stats.queries.gradeHistory.EnrollmentProjectionEntity   & 2            & 2             & 0            & 0             & 1             & 0             \\

    k.l.f.stats.queries.gradeHistory.GradeHistoryProjector        & 0            & 0             & 11           & 13            & 0             & 5             \\

    k.l.f.stats.queries.grades.AssessmentProjectionEntity         & 2            & 2             & 1            & 1             & 1             & 1             \\

    k.l.f.stats.queries.grades.GradedAssessmentId                 & 3            & 3             & 0            & 0             & 1             & 0             \\

    k.l.f.stats.queries.grades.GradedAssessmentProjectionEntity   & 2            & 2             & 2            & 2             & 1             & 2             \\

    k.l.f.stats.queries.grades.SimpleCourseProjectionEntity       & 2            & 2             & 0            & 0             & 1             & 0             \\

    k.l.f.stats.queries.grades.SimpleLectureProjectionEntity      & 2            & 2             & 0            & 0             & 1             & 0             \\

    k.l.f.stats.queries.grades.StudentGradesProjectionEntity      & 2            & 2             & 1            & 1             & 1             & 1             \\

    k.l.f.stats.queries.grades.StudentGradesProjectionEntityId    & 3            & 3             & 0            & 0             & 1             & 0             \\

    k.l.f.stats.queries.grades.StudentGradesProjector             & 0            & 0             & 22           & 25            & 0             & 7             \\

    k.l.f.stats.web.StatsController                               & 0            & 0             & 9            & 14            & 0             & 4             \\

    k.l.f.student.api.CreateStudentCommand                        & 5            & 6             & 0            & 0             & 5             & 0             \\

    k.l.f.student.api.StudentCreatedEvent                         & 3            & 3             & 0            & 0             & 3             & 0             \\

    k.l.f.student.c.StudentAggregate                              & 1            & 1             & 2            & 2             & 1             & 1             \\

    k.l.f.student.c.lookup.StudentLookupEntity                    & 6            & 15            & 0            & 0             & 2             & 0             \\

    k.l.f.student.c.lookup.StudentLookupProjector                 & 0            & 0             & 4            & 5             & 0             & 3             \\

    k.l.f.student.c.lookup.StudentValidator                       & 0            & 0             & 3            & 3             & 0             & 1             \\

    k.l.f.users.web.UsersController                               & 0            & 0             & 7            & 7             & 0             & 5             \\

    k.l.infra.web.GlobalExceptionHandler                          & 0            & 0             & 3            & 3             & 0             & 2             \\

    \bottomrule
    \caption{Dependency Metrics for ES-CQRS architecture (Classes).}
    \label{table:es-cqrs-dependency}
\end{longtable}

\newpage
\section{MOOD}
\label{appendix:mood-results}

\begin{table}[H]
    \small
    \centering
    \begin{tabular}{lrr}
        \toprule
        \textbf{Metric} & \textbf{CRUD} & \textbf{ES-CQRS} \\
        \midrule
        AHF             & 97.71\%       & 95.55\%          \\
        AIF             & 28.35\%       & 8.17\%           \\
        CF              & 11.56\%       & 3.18\%           \\
        MHF             & 28.71\%       & 44.53\%          \\
        MIF             & 16.26\%       & 0.76\%           \\
        PF              & 362.50\%      & 185.71\%         \\
        \bottomrule
    \end{tabular}
    \caption{MOOD metrics}
    \label{table:mood}
\end{table}


\newpage
\section{Chidamber Kemerer Metrics}
\label{appendix:ck-results}

\scriptsize
\begin{longtable}{lrrrrrr}
    \toprule
    \textbf{Class}                                    & \textbf{CBO} & \textbf{DIT} & \textbf{LCOM} & \textbf{NOC} & \textbf{RFC} & \textbf{WMC} \\
    \midrule

    k.l.CrudApplication                               & 1            & 1            & 1             & 0            & 2            & 1            \\

    k.l.audit.AuditContext                            & 2            & 1            & 1             & 0            & 12           & 7            \\

    k.l.audit.AuditEntityListener                     & 9            & 1            & 1             & 0            & 41           & 13           \\

    k.l.audit.AuditHelper                             & 3            & 1            & 1             & 0            & 2            & 1            \\

    k.l.audit.AuditLogEntry                           & 5            & 1            & 9             & 0            & 19           & 0            \\

    k.l.audit.AuditLogRepository                      & 0            & 0            & 0             & 0            & 0            & 0            \\

    k.l.audit.AuditService                            & 6            & 1            & 3             & 0            & 20           & 10           \\

    k.l.audit.AuditableEntity                         & 9            & 1            & 1             & 8            & 2            & 0            \\

    k.l.audit.CollectionWithIdSerializer              & 2            & 2            & 2             & 0            & 8            & 6            \\

    k.l.audit.DynamicIdSerializer                     & 2            & 2            & 1             & 0            & 12           & 3            \\

    k.l.audit.IdPropertySerializerModifier            & 3            & 2            & 1             & 0            & 11           & 8            \\

    k.l.audit.IdSerializationModule                   & 2            & 3            & 0             & 0            & 2            & 1            \\

    k.l.audit.JpaAuditingConfiguration                & 0            & 1            & 0             & 0            & 0            & 0            \\

    k.l.auth.CustomAuthFilter                         & 2            & 1            & 1             & 0            & 3            & 1            \\

    k.l.auth.NotAuthenticatedException                & 3            & 7            & 0             & 0            & 3            & 2            \\

    k.l.config.SpringContext                          & 1            & 1            & 1             & 0            & 3            & 2            \\

    k.l.courses.CourseDtoMapper                       & 6            & 1            & 1             & 0            & 10           & 1            \\

    k.l.courses.CourseEntity                          & 14           & 2            & 9             & 0            & 20           & 0            \\

    k.l.courses.CoursesController                     & 8            & 1            & 1             & 0            & 14           & 3            \\

    k.l.courses.CoursesNotFoundException              & 2            & 7            & 0             & 0            & 3            & 1            \\

    k.l.courses.CoursesRepository                     & 0            & 0            & 0             & 0            & 1            & 0            \\

    k.l.courses.CoursesService                        & 9            & 1            & 1             & 0            & 26           & 3            \\

    k.l.courses.SimpleCourseDtoMapper                 & 5            & 1            & 1             & 0            & 6            & 1            \\

    k.l.dev.SeedDataRunner                            & 16           & 1            & 1             & 0            & 55           & 10           \\

    k.l.exceptions.GlobalExceptionHandler             & 1            & 1            & 1             & 0            & 12           & 5            \\

    k.l.featureflags.Feature                          & 3            & 0            & 2             & 0            & 2            & 0            \\

    k.l.featureflags.FeatureFlagService               & 3            & 1            & 1             & 0            & 9            & 4            \\

    k.l.featureflags.FeaturesEndpoint                 & 2            & 1            & 1             & 0            & 8            & 3            \\

    k.l.lectures.AlreadyEnrolledException             & 1            & 7            & 0             & 0            & 3            & 1            \\

    k.l.lectures.AssessmentGradeEntity                & 9            & 2            & 8             & 0            & 18           & 0            \\

    k.l.lectures.AssessmentGradeRepository            & 0            & 0            & 0             & 0            & 3            & 0            \\

    k.l.lectures.EnrollmentEntity                     & 7            & 2            & 4             & 0            & 11           & 1            \\

    k.l.lectures.EnrollmentRepository                 & 0            & 0            & 0             & 0            & 6            & 0            \\

    k.l.lectures.LectureAssessmentEntity              & 15           & 2            & 8             & 0            & 18           & 0            \\

    k.l.lectures.LectureAssessmentMapper              & 5            & 1            & 1             & 0            & 7            & 1            \\

    k.l.lectures.LectureAssessmentRepository          & 0            & 0            & 0             & 0            & 2            & 0            \\

    k.l.lectures.LectureDetailDtoMapper               & 10           & 1            & 1             & 0            & 17           & 1            \\

    k.l.lectures.LectureDtoMapper                     & 7            & 1            & 1             & 0            & 12           & 1            \\

    k.l.lectures.LectureEntity                        & 24           & 2            & 13            & 0            & 38           & 2            \\

    k.l.lectures.LectureNotFoundException             & 1            & 7            & 0             & 0            & 3            & 1            \\

    k.l.lectures.LectureNotOpenForEnrollmentException & 2            & 7            & 0             & 0            & 3            & 1            \\

    k.l.lectures.LectureWaitlistEntryComparator       & 4            & 1            & 1             & 0            & 6            & 2            \\

    k.l.lectures.LectureWaitlistEntryEntity           & 8            & 2            & 4             & 0            & 11           & 1            \\

    k.l.lectures.LectureWaitlistEntryRepository       & 0            & 0            & 0             & 0            & 2            & 0            \\

    k.l.lectures.LecturesController                   & 17           & 1            & 1             & 0            & 35           & 19           \\

    k.l.lectures.LecturesRepository                   & 0            & 0            & 0             & 0            & 4            & 0            \\

    k.l.lectures.LecturesService                      & 42           & 1            & 1             & 0            & 126          & 33           \\

    k.l.lectures.SimpleLectureDtoMapper               & 6            & 1            & 1             & 0            & 6            & 1            \\

    k.l.lectures.WaitlistedStudentMapper              & 6            & 1            & 1             & 0            & 6            & 1            \\

    k.l.stats.GradedAssessmentDtoMapper               & 5            & 1            & 1             & 0            & 8            & 1            \\

    k.l.stats.StatsController                         & 8            & 1            & 1             & 0            & 12           & 5            \\

    k.l.stats.StatsService                            & 22           & 1            & 1             & 0            & 75           & 16           \\

    k.l.time.TimeSlotMapper                           & 7            & 1            & 1             & 0            & 5            & 1            \\

    k.l.time.TimeSlotValueObject                      & 9            & 2            & 3             & 0            & 4            & 0            \\

    k.l.time.TimeSlotValueObjectComparator            & 3            & 1            & 1             & 0            & 6            & 3            \\

    k.l.users.ProfessorDtoMapper                      & 5            & 1            & 1             & 0            & 5            & 1            \\

    k.l.users.ProfessorEntity                         & 10           & 2            & 5             & 0            & 12           & 0            \\

    k.l.users.ProfessorRepository                     & 0            & 0            & 0             & 0            & 0            & 0            \\

    k.l.users.StudentDtoMapper                        & 5            & 1            & 1             & 0            & 5            & 1            \\

    k.l.users.StudentEntity                           & 17           & 2            & 6             & 0            & 15           & 0            \\

    k.l.users.StudentNotFoundException                & 1            & 7            & 0             & 0            & 3            & 1            \\

    k.l.users.StudentRepository                       & 0            & 0            & 0             & 0            & 0            & 0            \\

    k.l.users.UsersController                         & 5            & 1            & 1             & 0            & 7            & 2            \\

    k.l.users.UsersService                            & 7            & 1            & 2             & 0            & 14           & 2            \\

    k.l.uuid.GeneratedUuidV7                          & 0            & 0            & 0             & 0            & 0            & 0            \\

    k.l.uuid.UuidV7Generator                          & 2            & 2            & 1             & 0            & 2            & 1            \\

    \bottomrule
    \caption{CK Metrics for CRUD architecture. \emph{k.l} used as abbreviation for \emph{k.l}}
    \label{table:crud-ck}
\end{longtable}

\scriptsize
\begin{longtable}{lrrrrrr}
    \toprule
    \textbf{Class}                                                & \textbf{CBO} & \textbf{DIT} & \textbf{LCOM} & \textbf{NOC} & \textbf{RFC} & \textbf{WMC} \\
    \midrule

    k.l.EsCqrsApplication                                         & 1            & 1            & 1             & 0            & 2            & 1            \\

    k.l.config.JacksonConfiguration                               & 0            & 1            & 1             & 0            & 9            & 1            \\

    k.l.config.RetryConfig                                        & 0            & 1            & 0             & 0            & 0            & 0            \\

    k.l.config.aspect.LoggingAspect                               & 0            & 1            & 2             & 0            & 18           & 3            \\

    k.l.config.aspect.RetryLoggingAspect                          & 0            & 1            & 2             & 0            & 11           & 4            \\

    k.l.config.commandInterceptors.AggregateNotFoundInterceptor   & 1            & 1            & 1             & 0            & 3            & 1            \\

    k.l.core.auth.CustomAuthFilter                                & 2            & 1            & 1             & 0            & 3            & 1            \\

    k.l.core.auth.NotAuthenticatedException                       & 2            & 7            & 0             & 0            & 3            & 2            \\

    k.l.core.exceptions.DomainException                           & 9            & 7            & 0             & 1            & 2            & 1            \\

    k.l.core.exceptions.ErrorDetails                              & 7            & 0            & 2             & 0            & 2            & 0            \\

    k.l.core.exceptions.MissingResourceException                  & 5            & 7            & 0             & 4            & 2            & 1            \\

    k.l.core.exceptions.NotAllowedException                       & 7            & 7            & 0             & 0            & 2            & 1            \\

    k.l.core.exceptions.QueryException                            & 3            & 4            & 0             & 0            & 2            & 1            \\

    k.l.core.json.Defaults                                        & 2            & 1            & 0             & 0            & 0            & 0            \\

    k.l.core.lookup.TimeSlotEmbeddable                            & 7            & 2            & 3             & 0            & 4            & 0            \\

    k.l.core.lookup.TimeSlotEmbeddableComparator                  & 3            & 1            & 1             & 0            & 6            & 3            \\

    k.l.core.q.CourseMapper                                       & 5            & 1            & 1             & 0            & 13           & 1            \\

    k.l.core.q.ICourseProjectionEntity                            & 0            & 0            & 0             & 0            & 6            & 0            \\

    k.l.core.uuid.UuidProvider                                    & 0            & 0            & 0             & 0            & 1            & 0            \\

    k.l.core.uuid.UuidProviderImpl                                & 2            & 1            & 1             & 0            & 2            & 1            \\

    k.l.dev.SeedDataRunner                                        & 5            & 1            & 1             & 0            & 16           & 1            \\

    k.l.f.course.api.CourseCreatedEvent                           & 7            & 2            & 6             & 0            & 7            & 0            \\

    k.l.f.course.api.CreateCourseCommand                          & 6            & 2            & 6             & 0            & 7            & 0            \\

    k.l.f.course.api.FindAllCoursesQuery                          & 3            & 1            & 0             & 0            & 0            & 0            \\

    k.l.f.course.api.FindCourseByIdQuery                          & 3            & 2            & 1             & 0            & 2            & 0            \\

    k.l.f.course.commands.CourseAggregate                         & 6            & 1            & 1             & 0            & 15           & 4            \\

    k.l.f.course.commands.CourseLookupProjector                   & 4            & 1            & 1             & 0            & 9            & 1            \\

    k.l.f.course.commands.CourseLookupRepository                  & 0            & 0            & 0             & 0            & 2            & 0            \\

    k.l.f.course.commands.CourseValidator                         & 3            & 1            & 1             & 0            & 16           & 5            \\

    k.l.f.course.commands.CoursesLookupJpaEntity                  & 3            & 1            & 5             & 0            & 10           & 0            \\

    k.l.f.course.commands.ICourseValidator                        & 0            & 0            & 0             & 0            & 5            & 0            \\

    k.l.f.course.exceptions.MissingCoursesException               & 5            & 8            & 0             & 0            & 4            & 2            \\

    k.l.f.course.q.CourseProjectionEntity                         & 4            & 1            & 6             & 0            & 12           & 0            \\

    k.l.f.course.q.CourseProjector                                & 9            & 1            & 1             & 0            & 34           & 4            \\

    k.l.f.course.q.CourseRepository                               & 0            & 0            & 0             & 0            & 0            & 0            \\

    k.l.f.course.web.CoursesController                            & 10           & 1            & 1             & 0            & 28           & 3            \\

    k.l.f.e.api.AssignGradeCommand                                & 5            & 2            & 5             & 0            & 7            & 2            \\

    k.l.f.e.api.AwardCreditsCommand                               & 3            & 2            & 2             & 0            & 3            & 0            \\

    k.l.f.e.api.CreditsAwardedEvent                               & 5            & 2            & 4             & 0            & 5            & 0            \\

    k.l.f.e.api.EnrollmentCreatedEvent                            & 7            & 2            & 4             & 0            & 5            & 0            \\

    k.l.f.e.api.GradeAssignedEvent                                & 3            & 2            & 6             & 0            & 7            & 0            \\

    k.l.f.e.api.GradeUpdatedEvent                                 & 3            & 2            & 6             & 0            & 7            & 0            \\

    k.l.f.e.api.UpdateGradeCommand                                & 3            & 2            & 5             & 0            & 7            & 2            \\

    k.l.f.e.c.AwardCreditsSaga                                    & 8            & 1            & 1             & 0            & 20           & 4            \\

    k.l.f.e.c.EnrollmentAggregate                                 & 13           & 1            & 1             & 0            & 46           & 12           \\

    k.l.f.e.c.EnrollmentCommandHandler                            & 18           & 1            & 1             & 0            & 38           & 8            \\

    k.l.f.e.c.lookup.EnrollmentLookupEntity                       & 5            & 1            & 3             & 0            & 8            & 0            \\

    k.l.f.e.c.lookup.EnrollmentLookupProjector                    & 4            & 1            & 1             & 0            & 7            & 1            \\

    k.l.f.e.c.lookup.EnrollmentLookupRepository                   & 0            & 0            & 0             & 0            & 2            & 0            \\

    k.l.f.e.c.lookup.EnrollmentValidator                          & 3            & 1            & 1             & 0            & 6            & 2            \\

    k.l.f.e.c.lookup.IEnrollmentValidator                         & 0            & 0            & 0             & 0            & 2            & 0            \\

    k.l.f.e.c.lookup.credits.IStudentCreditsValidator             & 0            & 0            & 0             & 0            & 2            & 0            \\

    k.l.f.e.c.lookup.credits.StudentCreditsLookupProjectionEntity & 3            & 1            & 3             & 0            & 9            & 0            \\

    k.l.f.e.c.lookup.credits.StudentCreditsLookupProjector        & 5            & 1            & 1             & 0            & 15           & 2            \\

    k.l.f.e.c.lookup.credits.StudentCreditsLookupRepository       & 0            & 0            & 0             & 0            & 0            & 0            \\

    k.l.f.e.c.lookup.credits.StudentCreditsValidator              & 4            & 1            & 1             & 0            & 11           & 2            \\

    k.l.f.e.exception.AssessmentNotFoundException                 & 2            & 8            & 0             & 0            & 3            & 1            \\

    k.l.f.e.exception.MissingGradeException                       & 2            & 8            & 0             & 0            & 2            & 1            \\

    k.l.f.e.exception.StudentNotEnrolledException                 & 2            & 8            & 0             & 0            & 3            & 1            \\

    k.l.f.le.api.AddAssessmentCommand                             & 7            & 2            & 6             & 0            & 7            & 0            \\

    k.l.f.le.api.AdvanceLectureLifecycleCommand                   & 7            & 2            & 3             & 0            & 4            & 0            \\

    k.l.f.le.api.AssessmentAddedEvent                             & 9            & 2            & 6             & 0            & 7            & 0            \\

    k.l.f.le.api.AssignTimeSlotsToLectureCommand                  & 4            & 2            & 3             & 0            & 4            & 0            \\

    k.l.f.le.api.ConfirmStudentEnrollmentCommand                  & 4            & 2            & 2             & 0            & 3            & 0            \\

    k.l.f.le.api.CreateEnrollmentCommand                          & 3            & 2            & 4             & 0            & 5            & 0            \\

    k.l.f.le.api.CreateLectureCommand                             & 7            & 2            & 5             & 0            & 6            & 0            \\

    k.l.f.le.api.DisenrollStudentCommand                          & 3            & 2            & 2             & 0            & 3            & 0            \\

    k.l.f.le.api.EnrollStudentCommand                             & 5            & 2            & 2             & 0            & 3            & 0            \\

    k.l.f.le.api.EnrollmentStatusQuery                            & 2            & 2            & 2             & 0            & 3            & 0            \\

    k.l.f.le.api.EnrollmentStatusUpdate                           & 3            & 2            & 3             & 0            & 4            & 0            \\

    k.l.f.le.api.FindLectureByIdQuery                             & 3            & 2            & 1             & 0            & 2            & 0            \\

    k.l.f.le.api.GetLectureWaitlistQuery                          & 2            & 2            & 1             & 0            & 2            & 0            \\

    k.l.f.le.api.GetLecturesForStudentQuery                       & 2            & 2            & 1             & 0            & 2            & 0            \\

    k.l.f.le.api.LectureCreatedEvent                              & 12           & 2            & 6             & 0            & 7            & 0            \\

    k.l.f.le.api.LectureLifecycleAdvancedEvent                    & 9            & 2            & 3             & 0            & 4            & 0            \\

    k.l.f.le.api.StudentDisenrolledEvent                          & 4            & 2            & 2             & 0            & 3            & 0            \\

    k.l.f.le.api.StudentEnrolledEvent                             & 5            & 2            & 2             & 0            & 3            & 0            \\

    k.l.f.le.api.StudentEnrollmentApprovedEvent                   & 5            & 2            & 3             & 0            & 4            & 0            \\

    k.l.f.le.api.StudentRemovedFromWaitlistEvent                  & 4            & 2            & 2             & 0            & 3            & 0            \\

    k.l.f.le.api.StudentWaitlistedEvent                           & 5            & 2            & 3             & 0            & 4            & 0            \\

    k.l.f.le.api.TimeSlotsAssignedEvent                           & 5            & 2            & 3             & 0            & 4            & 0            \\

    k.l.f.le.api.WaitlistClearedEvent                             & 4            & 2            & 2             & 0            & 3            & 0            \\

    k.l.f.le.c.AssessmentValueObject                              & 3            & 1            & 1             & 0            & 4            & 0            \\

    k.l.f.le.c.EnrollmentApprovalSaga                             & 6            & 1            & 1             & 0            & 12           & 2            \\

    k.l.f.le.c.LectureAggregate                                   & 36           & 1            & 2             & 0            & 114          & 45           \\

    k.l.f.le.c.lookup.assessment.AssessmentLookupEntity           & 8            & 1            & 5             & 0            & 12           & 0            \\

    k.l.f.le.c.lookup.assessment.AssessmentLookupProjector        & 6            & 1            & 1             & 0            & 13           & 1            \\

    k.l.f.le.c.lookup.assessment.AssessmentLookupRepository       & 0            & 0            & 0             & 0            & 1            & 0            \\

    k.l.f.le.c.lookup.assessment.AssessmentValidator              & 3            & 1            & 1             & 0            & 5            & 2            \\

    k.l.f.le.c.lookup.assessment.IAssessmentValidator             & 0            & 0            & 0             & 0            & 2            & 0            \\

    k.l.f.le.c.lookup.lecture.ILectureValidator                   & 0            & 0            & 0             & 0            & 1            & 0            \\

    k.l.f.le.c.lookup.lecture.LectureLookupEntity                 & 6            & 1            & 5             & 0            & 13           & 0            \\

    k.l.f.le.c.lookup.lecture.LectureLookupProjector              & 6            & 1            & 1             & 0            & 21           & 3            \\

    k.l.f.le.c.lookup.lecture.LectureLookupRepository             & 0            & 0            & 0             & 0            & 0            & 0            \\

    k.l.f.le.c.lookup.lecture.LectureValidator                    & 3            & 1            & 1             & 0            & 3            & 1            \\

    k.l.f.le.c.lookup.timeSlot.ITimeSlotValidator                 & 0            & 0            & 0             & 0            & 1            & 0            \\

    k.l.f.le.c.lookup.timeSlot.LectureTimeSlotProjector           & 8            & 1            & 1             & 0            & 23           & 3            \\

    k.l.f.le.c.lookup.timeSlot.LectureTimeslotLookupEntity        & 5            & 1            & 2             & 0            & 7            & 0            \\

    k.l.f.le.c.lookup.timeSlot.LectureTimeslotLookupRepository    & 0            & 0            & 0             & 0            & 0            & 0            \\

    k.l.f.le.c.lookup.timeSlot.TimeSlotValidator                  & 9            & 1            & 1             & 0            & 18           & 2            \\

    k.l.f.le.exceptions.LectureNotFoundException                  & 3            & 8            & 0             & 0            & 2            & 1            \\

    k.l.f.le.q.CourseProjectionEntity                             & 3            & 1            & 6             & 0            & 12           & 0            \\

    k.l.f.le.q.CourseRepository                                   & 0            & 0            & 0             & 0            & 0            & 0            \\

    k.l.f.le.q.LectureDetailProjectionEntity                      & 3            & 1            & 12            & 0            & 25           & 0            \\

    k.l.f.le.q.LectureDetailRepository                            & 0            & 0            & 0             & 0            & 0            & 0            \\

    k.l.f.le.q.LectureProjector                                   & 39           & 1            & 2             & 0            & 133          & 21           \\

    k.l.f.le.q.ProfessorProjectionEntity                          & 2            & 1            & 4             & 0            & 9            & 0            \\

    k.l.f.le.q.ProfessorRepository                                & 0            & 0            & 0             & 0            & 0            & 0            \\

    k.l.f.le.q.StudentLecturesProjectionEntity                    & 3            & 1            & 5             & 0            & 13           & 0            \\

    k.l.f.le.q.StudentLecturesProjector                           & 14           & 1            & 1             & 0            & 64           & 10           \\

    k.l.f.le.q.StudentLecturesRepository                          & 0            & 0            & 0             & 0            & 1            & 0            \\

    k.l.f.le.q.StudentProjectionEntity                            & 2            & 1            & 4             & 0            & 10           & 0            \\

    k.l.f.le.q.StudentRepository                                  & 0            & 0            & 0             & 0            & 0            & 0            \\

    k.l.f.le.web.LecturesController                               & 31           & 1            & 1             & 0            & 66           & 23           \\

    k.l.f.professor.api.CreateProfessorCommand                    & 5            & 2            & 3             & 0            & 4            & 0            \\

    k.l.f.professor.api.ProfessorCreatedEvent                     & 3            & 2            & 3             & 0            & 4            & 0            \\

    k.l.f.professor.c.IProfessorValidator                         & 0            & 0            & 0             & 0            & 1            & 0            \\

    k.l.f.professor.c.ProfessorAggregate                          & 3            & 1            & 1             & 0            & 9            & 2            \\

    k.l.f.professor.c.ProfessorLookupEntity                       & 2            & 1            & 2             & 0            & 6            & 0            \\

    k.l.f.professor.c.ProfessorLookupRepository                   & 0            & 0            & 0             & 0            & 0            & 0            \\

    k.l.f.professor.c.ProfessorLookupTable                        & 4            & 1            & 1             & 0            & 5            & 1            \\

    k.l.f.professor.c.ProfessorValidator                          & 2            & 1            & 1             & 0            & 3            & 1            \\

    k.l.f.stats.api.GetCreditsForStudentQuery                     & 2            & 2            & 1             & 0            & 2            & 0            \\

    k.l.f.stats.api.GetGradeHistoryQuery                          & 2            & 2            & 4             & 0            & 5            & 0            \\

    k.l.f.stats.api.GetGradesForStudentQuery                      & 2            & 2            & 1             & 0            & 2            & 0            \\

    k.l.f.stats.q.credits.StudentCreditsProjectionEntity          & 2            & 1            & 2             & 0            & 6            & 0            \\

    k.l.f.stats.q.credits.StudentCreditsProjectionRepository      & 0            & 0            & 0             & 0            & 0            & 0            \\

    k.l.f.stats.q.credits.StudentCreditsProjector                 & 7            & 1            & 1             & 0            & 23           & 4            \\

    k.l.f.stats.q.gradeHistory.AssessmentProjectionEntity         & 2            & 1            & 2             & 0            & 6            & 0            \\

    k.l.f.stats.q.gradeHistory.AssessmentProjectionRepository     & 0            & 0            & 0             & 0            & 0            & 0            \\

    k.l.f.stats.q.gradeHistory.EnrollmentProjectionEntity         & 2            & 1            & 3             & 0            & 8            & 0            \\

    k.l.f.stats.q.gradeHistory.EnrollmentProjectionRepository     & 0            & 0            & 0             & 0            & 1            & 0            \\

    k.l.f.stats.q.gradeHistory.GradeHistoryProjector              & 11           & 1            & 1             & 0            & 48           & 5            \\

    k.l.f.stats.q.grades.AssessmentProjectionEntity               & 3            & 1            & 4             & 0            & 10           & 0            \\

    k.l.f.stats.q.grades.AssessmentProjectionRepository           & 0            & 0            & 0             & 0            & 1            & 0            \\

    k.l.f.stats.q.grades.GradedAssessmentId                       & 3            & 1            & 3             & 0            & 9            & 0            \\

    k.l.f.stats.q.grades.GradedAssessmentProjectionEntity         & 4            & 1            & 7             & 0            & 16           & 0            \\

    k.l.f.stats.q.grades.GradedAssessmentProjectionRepository     & 0            & 0            & 0             & 0            & 1            & 0            \\

    k.l.f.stats.q.grades.SimpleCourseProjectionEntity             & 2            & 1            & 3             & 0            & 8            & 0            \\

    k.l.f.stats.q.grades.SimpleCourseProjectionRepository         & 0            & 0            & 0             & 0            & 0            & 0            \\

    k.l.f.stats.q.grades.SimpleLectureProjectionEntity            & 2            & 1            & 3             & 0            & 8            & 0            \\

    k.l.f.stats.q.grades.SimpleLectureProjectionRepository        & 0            & 0            & 0             & 0            & 0            & 0            \\

    k.l.f.stats.q.grades.StudentGradesProjectionEntity            & 3            & 1            & 3             & 0            & 8            & 0            \\

    k.l.f.stats.q.grades.StudentGradesProjectionEntityId          & 3            & 1            & 3             & 0            & 9            & 0            \\

    k.l.f.stats.q.grades.StudentGradesProjector                   & 22           & 1            & 1             & 0            & 78           & 16           \\

    k.l.f.stats.q.grades.StudentGradesRepository                  & 0            & 0            & 0             & 0            & 1            & 0            \\

    k.l.f.stats.web.StatsController                               & 9            & 1            & 1             & 0            & 13           & 5            \\

    k.l.f.student.api.CreateStudentCommand                        & 5            & 2            & 4             & 0            & 5            & 0            \\

    k.l.f.student.api.StudentCreatedEvent                         & 3            & 2            & 4             & 0            & 5            & 0            \\

    k.l.f.student.c.StudentAggregate                              & 3            & 1            & 1             & 0            & 11           & 3            \\

    k.l.f.student.c.lookup.IStudentValidator                      & 0            & 0            & 0             & 0            & 3            & 0            \\

    k.l.f.student.c.lookup.StudentLookupEntity                    & 6            & 1            & 2             & 0            & 6            & 0            \\

    k.l.f.student.c.lookup.StudentLookupProjector                 & 4            & 1            & 1             & 0            & 6            & 1            \\

    k.l.f.student.c.lookup.StudentLookupRepository                & 0            & 0            & 0             & 0            & 0            & 0            \\

    k.l.f.student.c.lookup.StudentValidator                       & 3            & 1            & 1             & 0            & 7            & 3            \\

    k.l.f.users.web.UsersController                               & 7            & 1            & 1             & 0            & 13           & 2            \\

    k.l.infra.web.GlobalExceptionHandler                          & 3            & 1            & 2             & 0            & 20           & 13           \\

    \bottomrule
    \caption{CK Metrics for ES-CQRS architecture.}
    \label{table:es-cqrs-ck}
\end{longtable}


%LTeX: language=de-DE
\chapter{Nutzung von KI-Tools}

\begin{table}[htp!]
    \small
    \centering
    \begin{tabularx}{\linewidth}{llX}
        \toprule
        \textbf{Kapitel / Codemodul} & \textbf{Tool(s)} & \textbf{Beschreibung und Begründung Einsatzzweck}                                                     \\ \midrule
        Alle Kapitel                 & Gemini           & LaTeX Syntax-Unterstützung bei Tabellen, figures, listings                                            \\
        \addlinespace
        Alle Kapitel                 & DeepL Write      & Grammatikalische und stilistische Überarbeitung von Textsegmenten                                     \\
        \addlinespace
        Alle Kapitel                 & DeepL Translate  & Übersetzung von Wörtern und Formulierungen                                                            \\
        \addlinespace
        Kapitel 7 (Results)          & Gemini           & Hilfe bei der Visualisierung \& statistischen Auswertung der Ergebnisse (matplotlib, pandas, seaborn) \\
        \bottomrule
    \end{tabularx}
    \caption{AI tools used throughout the thesis}
    \label{table:ki-tools}
\end{table}

\end{document}

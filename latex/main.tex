\documentclass[12pt,a4paper]{article}
\usepackage[utf8]{inputenc}
\usepackage[T1]{fontenc}
\usepackage{graphicx}
\usepackage[ngerman]{babel}

\begin{document}

\title{How does an Event Sourcing architecture compare to CRUD systems with an independent audit log, when it comes to scalability, performance and traceability?}
\author{Lukas Karsch}

\begin{titlepage}
    \centering

    \includegraphics[width=0.3\textwidth]{images/HdM_Logo.svg.png}
    \vspace{1cm}

    {\large Bachelor's Thesis in Computer Science and Media}

    \vspace{1.5cm}

    % Title
    {\LARGE\bfseries How does an Event Sourcing architecture compare to CRUD systems with an independent audit log, when it comes to scalability, performance and traceability?}

    \vspace{0.5cm}
    \rule{\linewidth}{0.5pt} % Horizontal line
    \vspace{0.5cm}

    {\large\bfseries Lukas Karsch}

    \vspace{0.3cm}

    45259

    \vspace{0.8cm}

    {\bfseries Hochschule der Medien Stuttgart}

    \vspace{0.8cm}

    Submitted on 2026/03/02

    \vspace{0.3cm}

    to obtain the degree of Bachelor of Science

    \vfill

    \begin{flushleft}
        \begin{tabular}{ll}
            \textbf{Erstprüfer:}  & Prof. Dr. Tobias Jordine \\[0.3cm]
            \textbf{Zweitprüfer:} & Felix Messner            \\[0.3cm]
        \end{tabular}
    \end{flushleft}

\end{titlepage}

\pagebreak

\section{Hello World!}

\textbf{Hello World!} Today I am learning \LaTeX. %notice how the command will end at the first non-alphabet charecter such as the . after \LaTeX
\LaTeX{} is a great program for writing math. I can write in line math such as $a^2+b^2=c^2$ %$ tells LaTexX to compile as math
. I can also give equations their own space:
\begin{equation} % Creates an equation environment and is compiled as math
    \gamma^2+\theta^2=\omega^2
\end{equation}
If I do not leave any blank lines \LaTeX{} will continue  this text without making it into a new paragraph.  Notice how there was no indentation in the text after equation (1).
Also notice how even though I hit enter after that sentence and here $\downarrow$
\LaTeX{} formats the sentence without any break.  Also   look  how      it   doesn't     matter          how    many  spaces     I put     between       my    words.

For a new paragraph I can leave a blank space in my code.

\end{document} % This is the end of the document

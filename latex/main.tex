\documentclass[12pt,a4paper]{article}
\usepackage[utf8]{inputenc}
\usepackage[T1]{fontenc}
\usepackage{graphicx}
\usepackage[automake]{glossaries}

\usepackage[
  backend=biber,
  style=authoryear, % or numeric, ieee, apa, ...
]{biblatex}

\addbibresource{references.bib}

\makeglossaries

\newglossaryentry{api}
{
    name=API, 
    description={API stands for \emph{Application Programming Interface}. It describes the public interface of a module or service, often exposed over a network}
}

\newglossaryentry{rest}
{
    name=REST,
    description={REST stands for \emph{Representational State Transfer}. It is an architectural style for distributed hypermedia systems.}
}

\newglossaryentry{http}
{
    name=HTTP,
    description={HTTP stands for \emph{Hypertext Transfer Protocol}. It is a protocol used in internet communication and was defined in RFC 2616 \parencite{rfc2616}}
}

\newacronym{www}{WWW}{World Wide Web}

\newacronym{hateoas}{HATEOAS}{Hypermedia as the engine of application state}

\newacronym{html}{HTML}{Hypertext Markup Language}

\newacronym{json}{JSON}{JavaScript Object Notation}

\newacronym{xml}{XML}{Extensible Markup Language}

\begin{document}

\title{How does an Event Sourcing architecture compare to CRUD systems with an independent audit log, when it comes to scalability, performance and traceability?}
\author{Lukas Karsch}

\begin{titlepage}
    \centering

    \includegraphics[width=0.3\textwidth]{images/HdM_Logo.svg.png}
    \vspace{1cm}

    {\large Bachelor's Thesis in Computer Science and Media}

    \vspace{1.5cm}

    % Title
    {\LARGE\bfseries How does an Event Sourcing architecture compare to CRUD systems with an independent audit log, when it comes to scalability, performance and traceability?}

    \vspace{0.5cm}
    \rule{\linewidth}{0.5pt}
    \vspace{0.5cm}

    {\large\bfseries Lukas Karsch}

    \vspace{0.3cm}

    45259

    \vspace{0.8cm}

    %LTeX: language=de-DE
    {\bfseries Hochschule der Medien Stuttgart}
    %LTeX: language=en-US

    \vspace{0.8cm}

    Submitted on 2026/03/02

    \vspace{0.3cm}

    to obtain the degree of Bachelor of Science

    \vfill

    \begin{flushleft}
        \begin{tabular}{ll}
            \textbf{Main Supervisor:}      & Prof. Dr. Tobias Jordine \\[0.3cm]
            \textbf{Secondary Supervisor:} & Felix Messner            \\[0.3cm]
        \end{tabular}
    \end{flushleft}

\end{titlepage}

\newpage

%LTeX: language=de-DE
\section*{Ehrenwörtliche Erklärung}
%LTeX: language=en-US

\newpage

\tableofcontents

\newpage

\section{Introduction}

here i use \gls{api}

\subsection{Motivation}

\subsection{Research question(s)}

\subsection{Goals and non goals}

\subsection{Structure of the paper}

\section{Basics}

\subsection{WWW, Web APIs, REST}

The \acrfull{www} is a connected information network used to exchange data. Resources are can be accessed via URIs which are transferred using formats like JSON or HTML via protocols like \gls{http}. HTTP is a stateless protocol based on a request-response structure. It supports standardized request types (e.g. GET and POST) which convey a semantic meaning \parencite{jacobs_architecture_2004}.

Web APIs are interfaces which enable applications to communicate. They use HTTP as network-based API \parencite[138]{fielding_architectural_2000}. Modern APIs typically follow \gls{rest} principles. REST stands for "Representational State Transfer" and describes an architectural style for distributed hypermedia systems \parencite[76]{fielding_architectural_2000}.

REST APIs follow principles derived from a set of constraints (e.g. imposed by the HTTP protocol). One of them is "stateless communication": Communication between clients and the server must be \emph{stateless}, meaning the client has to attach all necessary information for the server to fully understand the request.

Further, in REST applications, every resource must be addressable via a unique ID, which can then be used to derive URIs to access the resource. Below are some examples for resources and URIs which could be derived from them:

\begin{itemize}
    \item Book; ID=1; URI=\texttt{http://example.com/books/1}
    \item Book; ID=2; URI=\texttt{http://example.com/books/2}
    \item Author; ID=100; URI=\texttt{http://example.com/authors/100}
\end{itemize}

The "\acrfull{hateoas}" principle describes that resources should be linked to each other. Clients should be able to control the application by following a series of links provided by the server \parencite{tilkov_brief_2007}.

Every resource must support the same interface, typically that of HTTP methods (GET, POST, PUT, etc.) where operations on the resource are mapped to one method of the interface. A POST operation on a customer might map to the \texttt{createCustomer()} operation on a service.

Resources are decoupled from their representations. Client can request different representations of a resource, depending on their needs \parencite{tilkov_brief_2007}: a web browser might request \acrshort{html}, while another server or application might request \acrshort{xml} or \acrshort{json}.

\subsection{Layered Architecture Foundations (CRUD)}

\subsection{DDD Architectural Foundations}

\subsection{Traceability and auditing in IT systems}

\subsubsection{Why is traceability a business requirement}

\subsubsection{Audit Logs}

\subsubsection{Event Streams}

\subsubsection{Rebuilding state from an audit log and an event stream}

\subsection{Event Sourcing and event-driven architectures}

\subsection{(Eventual) Consistency}

\subsection{Scalability of systems}

\section{Related Work}

\section{Proposed Method}

\subsection{Project requirements}

\subsection{Performance}

\subsection{Scalability or flexibility (TODO)}

\subsection{Traceability}

\subsection{Tech Stack}

\section{Implementation}

\subsection{CRUD implementation}

\subsection{ES/CQRS implementation}

\subsection{Infrastructure}

\section{Results}

\section{Discussion}

\subsection{Analysis of results}

\subsection{Conclusion \& Further work}

Finally, I'm done.

\printglossaries

\printbibliography

\end{document} % This is the end of the document
